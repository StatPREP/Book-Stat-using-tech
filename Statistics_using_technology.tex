\documentclass[]{book}
\usepackage{lmodern}
\usepackage{amssymb,amsmath}
\usepackage{ifxetex,ifluatex}
\usepackage{fixltx2e} % provides \textsubscript
\ifnum 0\ifxetex 1\fi\ifluatex 1\fi=0 % if pdftex
  \usepackage[T1]{fontenc}
  \usepackage[utf8]{inputenc}
\else % if luatex or xelatex
  \ifxetex
    \usepackage{mathspec}
  \else
    \usepackage{fontspec}
  \fi
  \defaultfontfeatures{Ligatures=TeX,Scale=MatchLowercase}
\fi
% use upquote if available, for straight quotes in verbatim environments
\IfFileExists{upquote.sty}{\usepackage{upquote}}{}
% use microtype if available
\IfFileExists{microtype.sty}{%
\usepackage{microtype}
\UseMicrotypeSet[protrusion]{basicmath} % disable protrusion for tt fonts
}{}
\usepackage{hyperref}
\hypersetup{unicode=true,
            pdftitle={Statistics Using Technology},
            pdfauthor={Kathryn Kozak},
            pdfborder={0 0 0},
            breaklinks=true}
\urlstyle{same}  % don't use monospace font for urls
\usepackage{natbib}
\bibliographystyle{apalike}
\usepackage{longtable,booktabs}
\usepackage{graphicx,grffile}
\makeatletter
\def\maxwidth{\ifdim\Gin@nat@width>\linewidth\linewidth\else\Gin@nat@width\fi}
\def\maxheight{\ifdim\Gin@nat@height>\textheight\textheight\else\Gin@nat@height\fi}
\makeatother
% Scale images if necessary, so that they will not overflow the page
% margins by default, and it is still possible to overwrite the defaults
% using explicit options in \includegraphics[width, height, ...]{}
\setkeys{Gin}{width=\maxwidth,height=\maxheight,keepaspectratio}
\IfFileExists{parskip.sty}{%
\usepackage{parskip}
}{% else
\setlength{\parindent}{0pt}
\setlength{\parskip}{6pt plus 2pt minus 1pt}
}
\setlength{\emergencystretch}{3em}  % prevent overfull lines
\providecommand{\tightlist}{%
  \setlength{\itemsep}{0pt}\setlength{\parskip}{0pt}}
\setcounter{secnumdepth}{5}
% Redefines (sub)paragraphs to behave more like sections
\ifx\paragraph\undefined\else
\let\oldparagraph\paragraph
\renewcommand{\paragraph}[1]{\oldparagraph{#1}\mbox{}}
\fi
\ifx\subparagraph\undefined\else
\let\oldsubparagraph\subparagraph
\renewcommand{\subparagraph}[1]{\oldsubparagraph{#1}\mbox{}}
\fi

%%% Use protect on footnotes to avoid problems with footnotes in titles
\let\rmarkdownfootnote\footnote%
\def\footnote{\protect\rmarkdownfootnote}

%%% Change title format to be more compact
\usepackage{titling}

% Create subtitle command for use in maketitle
\providecommand{\subtitle}[1]{
  \posttitle{
    \begin{center}\large#1\end{center}
    }
}

\setlength{\droptitle}{-2em}

  \title{Statistics Using Technology}
    \pretitle{\vspace{\droptitle}\centering\huge}
  \posttitle{\par}
  \subtitle{Third Edition}
  \author{Kathryn Kozak}
    \preauthor{\centering\large\emph}
  \postauthor{\par}
      \predate{\centering\large\emph}
  \postdate{\par}
    \date{2019-06-17}

\usepackage{booktabs}

\begin{document}
\maketitle

{
\setcounter{tocdepth}{1}
\tableofcontents
}
\hypertarget{preface}{%
\chapter*{Preface}\label{preface}}
\addcontentsline{toc}{chapter}{Preface}

I hope you find this book useful in teaching statistics. When writing this book, I tried to follow the GAISE Standards (GAISE recommendations. (2014, January 05). Retrieved from \url{http://www.amstat.org/education/gaise/GAISECollege_Recommendations.pdf}
):

\begin{itemize}
\tightlist
\item
  Emphasis statistical literacy and develop statistical understanding.
\item
  Use real data.
\item
  Stress conceptual understanding, rather than mere knowledge of procedure.
\item
  Foster active learning in the classroom.
\item
  Use technology for developing concepts and analyzing data.
\end{itemize}

{[}NOTE IN DRAFT: This is \textbf{not} the most recent GAISE report.{]}

To this end, I ask students to interpret the results of their calculations. I incorporated the use of technology for most calculations. Because of that you will not find me using any of the computational formulas for standard deviations or correlation and regression since I prefer students understand the concept of these quantities. Also, because I utilize technology you will not find the standard normal table, Student's t-table, binomial table, chi-square distribution table, and F-distribution table in the book. The only tables I provided were for critical values for confidence intervals since they are more difficult to find using technology. Another difference between this book and other statistics books is the order of hypothesis testing and confidence intervals. Most books present confidence intervals first and then hypothesis tests. I find that presenting hypothesis testing first and then confidence intervals is more understandable for students. Lastly, I have de-emphasized the use of the z-test. In fact, I only use it to introduce hypothesis testing, and never utilize it again. You may also notice that when I introduced hypothesis testing and confidence intervals, proportions were introduced before means. However, when two sample tests and confidence intervals are introduced I switched this order. This is because usually many instructors do not discuss the proportions for two samples. However, you might try assigning problems for proportions without discussing it in class. After doing two samples for means, the proportions are similar. Lastly, to aid student understanding and interest, most of the homework and examples utilize real data. Again, I hope you find this book useful for your introductory statistics class.

I want to make a comment about the mathematical knowledge that I assumed the students possess. The course for which I wrote this book has a higher prerequisite than most introductory statistics books. However, I do feel that students can read and understand this book as long as they have had basic algebra and can substitute numbers into formulas. I do not show how to create most of the graphs, but most students should have been exposed to them in high school. So I hope the mathematical level is appropriate for your course.

The technology that I utilized for creating the graphs was Microsoft Excel, and I utilized the TI-83/84 graphing calculator for most calculations, including hypothesis testing, confidence intervals, and probability distributions. This is because these tools are readily available to my students. Please feel free to use any other technology that is more appropriate for your students. Do make sure that you use some technology.
Acknowledgments:

I would like to thank the following people for taking their valuable time to review the book. Their comments and insights improved this book immensely.

\begin{itemize}
\tightlist
\item
  Jane Tanner, Onondaga Community College
\item
  Rob Farinelli, College of Southern Maryland
\item
  Carrie Kinnison, retired engineer
\item
  Sean Simpson, Westchester Community College
\item
  Kim Sonier, Coconino Community College
\item
  Jim Ham, Delta College
\item
  David Straayer, Tacoma Community College
\item
  Kendra Feinstein, Tacoma Community College
\item
  Students of Coconino Community College
\item
  Students Tacoma Community College
\end{itemize}

I also want to thank Coconino Community College for granting me a sabbatical so that I would have the time to write the book. Lastly, I want to thank my husband Rich and my son Dylan for supporting me in this project. Without their love and support, I would not have been able to complete the book.

\textbf{New to the Second Edition:}

The additions to this edition mostly involve adding the commands to create graphs, compute descriptive statistics, finding probabilities, and computing inferential analysis using the open source software R. Another change involve adding an example at the end of chapter 3 that shows analyzing a data set using graphical and numerical descriptions. Another major change was adding a section 9.4 that gives some insight into which inferential analysis should be completed based on a series of questions that should be asked. Lastly, minor explanations were made and corrections were made where necessary.

On a personal note, I wanted to thank my brother, John Matic, his wife Jenelle, and their children Hannah and Eli for their hospitality when writing the first edition. In addition to allowing my family access to their home, John provided numerous examples and data sets for business applications in this book. I inadvertently left this thank you out of the first edition of the book, and for that I apologize. His help and his family's hospitality were invaluable to me.

\hypertarget{statistical-basics}{%
\chapter{Statistical Basics}\label{statistical-basics}}

\hypertarget{what-is-statistics}{%
\section{What is Statistics?}\label{what-is-statistics}}

You are exposed to statistics regularly. If you are a sports fan, then you have the statistics for your favorite player. If you are interested in politics, then you look at the polls to see how people feel about certain issues or candidates. If you are an environmentalist, then you research arsenic levels in the water of a town or analyze the global temperatures. If you are in the business profession, then you may track the monthly sales of a store or use quality control processes to monitor the number of defective parts manufactured. If you are in the health profession, then you may look at how successful a procedure is or the percentage of people infected with a disease. There are many other examples from other areas. To understand how to collect data and analyze it, you need to understand what the field of statistics is and the basic definitions.

\textbf{Statistics} is the study of how to collect, organize, analyze, and interpret data collected from a group.

There are two branches of statistics. One is called descriptive statistics, which is where you collect and organize data. The other is called inferential statistics, which is where you analyze and interpret data. First you need to look at descriptive statistics since you will use the descriptive statistics when making inferences.

To understand how to create descriptive statistics and then conduct inferences, there are a few definitions that you need to look at. Note, many of the words that are defined have common definitions that are used in non-statistical terminology. In statistics, some have slightly different definitions. It is important that you notice the difference and utilize the statistical definitions.

The first thing to decide in a statistical study is whom you want to measure and what you want to measure. You always want to make sure that you can answer the question of whom you measured and what you measured. The who is known as the individual and the what is the variable.

\textbf{Individual} -- a person or object that you are interested in finding out information about.

\textbf{Variable} -- the measurement or observation of the individual.

If you put the individual and the variable into one statement, then you obtain a population.

\textbf{Population} -- set of all values of the variable for the entire group of individuals.

Notice, the population answers who you want to measure and what you want to measure. Make sure that your population always answers both of these questions. If it doesn't, then you haven't given someone who is reading your study the entire picture. As an example, if you just say that you are going to collect data from the senators in the U.S. Congress, you haven't told your reader want you are going to collect. Do you want to know their income, their highest degree earned, their voting record, their age, their political party, their gender, their marital status, or how they feel about a particular issue? Without telling what you want to measure, your reader has no idea what your study is actually about.

Sometimes the population is very easy to collect. Such as if you are interested in finding the average age of all of the current senators in the U.S. Congress, there are only 100 senators. This wouldn't be hard to find. However, if instead you were interested in knowing the average age that a senator in the U.S. Congress first took office for all senators that ever served in the U.S. Congress, then this would be a bit more work. It is still doable, but it would take a bit of time to collect. But what if you are interested in finding the average diameter of breast height of all of the Ponderosa Pine trees in the Coconino National Forest? This would be impossible to actually collect. What do you do in these cases? Instead of collecting the entire population, you take a smaller group of the population, kind of a snap shot of the population. This smaller group is called a sample.

\textbf{Sample} -- a subset from the population. It looks just like the population, but contains less data.

How you collect your sample can determine how accurate the results of your study are. There are many ways to collect samples. Some of them create better samples than others. No sampling method is perfect, but some are better than others. Sampling techniques will be discussed later. For now, realize that every time you take a sample you will find different data values. The sample is a snapshot of the population, and there is more information than is in the picture. The idea is to try to collect a sample that gives you an accurate picture, but you will never know for sure if your picture is the correct picture. Unlike previous mathematics classes where there was always one right answer, in statistics there can be many answers, and you don't know which are right.

Once you have your data, either from a population or a sample, you need to know how you want to summarize the data. As an example, suppose you are interested in finding the proportion of people who like a candidate, the average height a plant grows to using a new fertilizer, or the variability of the test scores. Understanding how you want to summarize the data helps to determine the type of data you want to collect. Since the population is what we are interested in, then you want to calculate a number from the population. This is known as a parameter. As mentioned already, you can't really collect the entire population. Even though this is the number you are interested in, you can't really calculate it. Instead you use the number calculated from the sample, called a statistic, to estimate the parameter. Since no sample is exactly the same, the statistic values are going to be different from sample to sample. They estimate the value of the parameter, but again, you do not know for sure if your answer is correct.

\textbf{Parameter} -- a number calculated from the population. Usually denoted with a Greek letter. This number is a fixed, unknown number that you want to find.

\textbf{Statistic} -- a number calculated from the sample. Usually denoted with letters from the Latin alphabet, though sometimes there is a Greek letter with a \^{} (called a hat) above it. Since you can find samples, it is readily known, though it changes depending on the sample taken. It is used to estimate the parameter value.

One last concept to mention is that there are two different types of variables -- qualitative and quantitative. Each type of variable has different parameters and statistics that you find. It is important to know the difference between them.

\textbf{Qualitative} \textbf{or categorical variable} -- answer is a word or name that describes a quality of the individual.

\textbf{Quantitative or} \textbf{numerical variable} -- answer is a number, something that can be counted or measured from the individual.

\hypertarget{example-1.1.1-stating-definitions-for-qualitative-variable}{%
\subsection{Example \#1.1.1: Stating Definitions for Qualitative Variable**}\label{example-1.1.1-stating-definitions-for-qualitative-variable}}

\begin{quote}
In 2010, the Pew Research Center questioned 1500 adults in the U.S. to estimate the proportion of the population favoring marijuana use for medical purposes. It was found that 73\% are in favor of using \textgreater{} marijuana for medical purposes. State the individual, variable, population, and sample.
\end{quote}

\textbf{Solution:}

Individual -- a U.S. adult

\begin{quote}
Variable -- the response to the question ``should marijuana be used for medical purposes?'' This is qualitative data since you are recording a \textgreater{} person's response -- yes or no.
\end{quote}

Population -- set of all responses of adults in the U.S.

Sample -- set of 1500 responses of U.S. adults who are questioned.

\begin{quote}
Parameter -- proportion of those who favor marijuana for medical \textgreater{} purposes calculated from population
\end{quote}

\begin{quote}
Statistic-- proportion of those who favor marijuana for medical \textgreater{} purposes calculated from sample
\end{quote}

\hypertarget{example-1.1.2-stating-definitions-for-qualitative-variable}{%
\subsection{Example \#1.1.2: Stating Definitions for Qualitative Variable}\label{example-1.1.2-stating-definitions-for-qualitative-variable}}

\begin{quote}
A parking control officer records the manufacturer of every 5\textsuperscript{th} car in the college parking lot in order to guess the most common manufacturer.
\end{quote}

\textbf{Solution:}

Individual -- a car in the college parking lot

\begin{quote}
Variable -- the name of the manufacturer. This is qualitative data since you are recording a car type.
\end{quote}

Population -- set of all names of the manufacturer of cars in the college parking lot.

\begin{quote}
Sample -- set of recorded names of the manufacturer of the cars in college parking lot
\end{quote}

Parameter -- proportion of each car type calculated from population

Statistic -- proportion of each car type calculated from sample

\hypertarget{example-1.1.3-stating-definitions-for-quantitative-variable}{%
\subsection{Example \#1.1.3: Stating Definitions for Quantitative Variable}\label{example-1.1.3-stating-definitions-for-quantitative-variable}}

\begin{quote}
A biologist wants to estimate the average height of a plant that is \textgreater{} given a new plant food. She gives 10 plants the new plant food. State the individual, variable, population, and sample.
\end{quote}

\textbf{Solution:}

Individual -- a plant given the new plant food

\begin{quote}
Variable -- the height of the plant (Note: it is not the average \textgreater{} height since you cannot measure an average -- it is calculated from data.) This is quantitative data since you will have a number.
\end{quote}

Population -- set of all the heights of plants when the new plant food is used

Sample -- set of 10 heights of plants when the new plant food is used

Parameter -- average height of all plants

Statistic -- average height of 10 plants

\hypertarget{example-1.1.4-stating-definitions-for-quantitative-variable}{%
\subsection{Example \#1.1.4: Stating Definitions for Quantitative Variable}\label{example-1.1.4-stating-definitions-for-quantitative-variable}}

\begin{quote}
A doctor wants to see if a new treatment for cancer extends the life expectancy of a patient versus the old treatment. She gives one group of 25 cancer patients the new treatment and another group of 25 the old treatment. She then measures the life expectancy of each of the patients. State the individuals, variables, populations, and samples.
\end{quote}

\textbf{Solution:}

\begin{quote}
In this example there are two individuals, two variables, two populations, and two samples.
\end{quote}

Individual 1: cancer patient given new treatment

Individual 2: cancer patient given old treatment

\begin{quote}
Variable 1: life expectancy when given new treatment. This is quantitative data since you will have a number.
\end{quote}

\begin{quote}
Variable 2: life expectancy when given old treatment. This is quantitative data since you will have a number.
\end{quote}

Population 1: set of all life expectancies of cancer patients given new treatment

Population 2: set of all life expectancies of cancer patients given old treatment

Sample 1: set of 25 life expectancies of cancer patients given new treatment

Sample 2: set of 25 life expectancies of cancer patients given old treatment

Parameter 1 -- average life expectancy of all cancer patients given new treatment

Parameter 2 -- average life expectancy of all cancer patients given old treatment

Statistic 1 -- average life expectancy of 25 cancer patients given new treatment

Statistic 2 -- average life expectancy of 25 cancer patients given old treatment

There are different types of quantitative variables, called discrete or continuous. The difference is in how many values can the data have. If you can actually count the number of data values (even if you are counting to infinity), then the variable is called discrete. If it is not possible to count the number of data values, then the variable is called continuous.

\textbf{Discrete} data can only take on particular values like integers. Discrete data are usually things you count.

\textbf{Continuous} data can take on any value. Continuous data are usually things you measure.

\hypertarget{example-1.1.5-discrete-or-continuous}{%
\subsection{Example \#1.1.5: Discrete or Continuous}\label{example-1.1.5-discrete-or-continuous}}

Classify the quantitative variable as discrete or continuous.

a.) The weight of a cat.

\textbf{Solution:}

\begin{quote}
This is continuous since it is something you measure.
\end{quote}

b.) The number of fleas on a cat.

\textbf{Solution:}

\begin{quote}
This is discrete since it is something you count.
\end{quote}

c.) The size of a shoe.

\textbf{Solution:}

\begin{quote}
This is discrete since you can only be certain values, such as 7, 7.5, 8, 8.5, 9. You can't buy a 9.73 shoe.
\end{quote}

There are also are four measurement scales for different types of data with each building on the ones below it. They are:

\hypertarget{measurement-scales}{%
\subsection{Measurement Scales:}\label{measurement-scales}}

\textbf{Nominal} -- data is just a name or category. There is no order to any data and since there are no numbers, you cannot do any arithmetic on this level of data. Examples of this are gender, car name, ethnicity, and race.

\textbf{Ordinal} -- data that is nominal, but you can now put the data in order, since one value is more or less than another value. You cannot do arithmetic on this data, but you can now put data values in order. Examples of this are grades (A, B, C, D, F), place value in a race (1st, 2nd, 3rd), and size of a drink (small, medium, large).

\textbf{Interval} -- data that is ordinal, but you can now subtract one value from another and that subtraction makes sense. You can do arithmetic on this data, but only addition and subtraction. Examples of this are temperature and time on a clock.

\textbf{Ratio} -- data that is interval, but you can now divide one value by another and that ratio makes sense. You can now do all arithmetic on this data. Examples of this are height, weight, distance, and time.

Nominal and ordinal data come from qualitative variables. Interval and ratio data come from quantitative variables.

Most people have a hard time deciding if the data are nominal, ordinal, interval, or ratio. First, if the variable is qualitative (words instead of numbers) then it is either nominal or ordinal. Now ask yourself if you can put the data in a particular order. If you can it is ordinal. Otherwise, it is nominal. If the variable is quantitative (numbers), then it is either interval or ratio. For ratio data, a value of 0 means there is no measurement. This is known as the absolute zero. If there is an absolute zero in the data, then it means it is ratio. If there is no absolute zero, then the data are interval. An example of an absolute zero is if you have \$0 in your bank account, then you are without money. The amount of money in your bank account is ratio data. \emph{Word of caution}: sometimes ordinal data is displayed using numbers, such as 5 being strongly agree, and 1 being strongly disagree. These numbers are not really numbers. Instead they are used to assign numerical values to ordinal data. In reality you should not perform any computations on this data, though many people do. If there are numbers, make sure the numbers are inherent numbers, and not numbers that were assigned.

Example \#1.1.6: Measurement Scale

State which measurement scale each is.

a.) Time of first class

\textbf{Solution:}

\begin{quote}
This is interval since it is a number, but 0 o'clock means midnight and not the absence of time.
\end{quote}

b.) Hair color

\textbf{Solution:}

\begin{quote}
This is nominal since it is not a number, and there is no specific order for hair color.
\end{quote}

c.) Length of time to take a test

\textbf{Solution:}

\begin{quote}
This is ratio since it is a number, and if you take 0 minutes to take a test, it means you didn't take any time to complete it.
\end{quote}

d.) Age groupings (baby, toddler, adolescent, teenager, adult, elderly)

\textbf{Solution:}

\begin{quote}
This is ordinal since it is not a number, but you could put the data in order from youngest to oldest or the other way around.
\end{quote}

\hypertarget{homework}{%
\subsection{Homework}\label{homework}}

\begin{enumerate}
\def\labelenumi{\arabic{enumi}.}
\item
  Suppose you want to know how Arizona workers age 16 or older travel to work. To estimate the percentage of people who use the different modes of travel, you take a sample containing 500 Arizona workers age 16 or older. State the individual, variable, population, sample, parameter, and statistic.
\item
  You wish to estimate the mean cholesterol levels of patients two days after they had a heart attack. To estimate the mean you collect data from 28 heart patients. State the individual, variable, population, sample, parameter, and statistic.
\item
  Print-O-Matic would like to estimate their mean salary of all employees. To accomplish this they collect the salary of 19 employees. State the individual, variable, population, sample, parameter, and statistic.
\item
  To estimate the percentage of households in Connecticut which use fuel oil as a heating source, a researcher collects information from 1000 Connecticut households about what fuel is their heating source. State the individual, variable, population, sample, parameter, and statistic.
\item
  The U.S. Census Bureau needs to estimate the median income of males in the U.S., they collect incomes from 2500 males. State the individual, variable, population, sample, parameter, and statistic.
\item
  The U.S. Census Bureau needs to estimate the median income of females in the U.S., they collect incomes from 3500 females. State the individual, variable, population, sample, parameter, and statistic.
\item
  Eyeglassmatic manufactures eyeglasses and they would like to know the percentage of each defect type made. They review 25,891 defects and classify each defect that is made. State the individual, variable, population, sample, parameter, and statistic.
\item
  The World Health Organization wishes to estimate the mean density of people per square kilometer, they collect data on 56 countries. State the individual, variable, population, sample, parameter, and statistic
\item
  State the measurement scale for each.
\end{enumerate}

\begin{enumerate}
\def\labelenumi{\alph{enumi}.}
\item
  Cholesterol level
\item
  Defect type
\item
  Time of first class
\item
  Opinion on a 5 point scale, with 5 being strongly agree and 1 being
  strongly disagree
\end{enumerate}

\begin{enumerate}
\def\labelenumi{\arabic{enumi}.}
\setcounter{enumi}{9}
\tightlist
\item
  State the measurement scale for each.
\end{enumerate}

\begin{enumerate}
\def\labelenumi{\alph{enumi}.}
\item
  Temperature in degrees Celsius
\item
  Ice cream flavors available
\item
  Pain levels on a scale from 1 to 10, 10 being the worst pain ever
\item
  Salary of employees
\end{enumerate}

\hypertarget{sampling-methods}{%
\section{Sampling Methods}\label{sampling-methods}}

As stated before, if you want to know something about a population, it is often impossible or impractical to examine the whole population. It might be too expensive in terms of time or money. It might be impractical -- you can't test all batteries for their length of lifetime because there wouldn't be any batteries left to sell. You need to look at a sample. Hopefully the sample behaves the same as the population.

When you choose a sample you want it to be as similar to the population as possible. If you want to test a new painkiller for adults you would want the sample to include people who are fat, skinny, old, young, healthy, not healthy, male, female, etc.

There are many ways to collect a sample. None are perfect, and you are not guaranteed to collect a representative sample. That is unfortunately the limitations of sampling. However, there are several techniques that can result in samples that give you a semi-accurate picture of the population. Just remember to be aware that the sample may not be representative. As an example, you can take a random sample of a group of people that are equally males and females, yet by chance everyone you choose is female. If this happens, it may be a good idea to collect a new sample if you have the time and money.

There are many sampling techniques, though only four will be presented here. The simplest, and the type that is strived for is a \textbf{simple random sample}. This is where you pick the sample such that every sample has the same chance of being chosen. This type of sample is actually hard to collect, since it is sometimes difficult to obtain a complete list of all individuals. There are many cases where you cannot conduct a truly random sample. However, you can get as close as you can. Now suppose you are interested in what type of music people like. It might not make sense to try to find an answer for everyone in the U.S. You probably don't like the same music as your parents. The answers vary so much you probably couldn't find an answer for everyone all at once. It might make sense to look at people in different age groups, or people of different ethnicities. This is called a \textbf{stratified sample}. The issue with this sample type is that sometimes people subdivide the population too much. It is best to just have one stratification. Also, a stratified sample has similar problems that a simple random sample has. If your population has some order in it, then you could do a \textbf{systematic sample}. This is popular in manufacturing. The problem is that it is possible to miss a manufacturing mistake because of how this sample is taken. If you are collecting polling data based on location, then a \textbf{cluster sample} that divides the population based on geographical means would be the easiest sample to conduct. The problem is that if you are looking for opinions of people, and people who live in the same region may have similar opinions. As you can see each of the sampling techniques have pluses and minuses. Include convenience {[}NOTE IN DRAFT: This sentence is incomplete.{]}

A \textbf{simple random sample (SRS)} of size \emph{n} is a sample that is selected from a population in a way that ensures that every different possible sample of size \emph{n} has the same chance of being selected. Also, every individual associated with the population has the same chance of being selected.
Ways to select a simple random sample:

\begin{quote}
Put all names in a hat and draw a certain number of names out.
\end{quote}

\begin{quote}
Assign each individual a number and use a random number table or a calculator or computer to randomly select the individuals that will be measured.
\#\#\#\# Example \#1.2.1: Choosing a Simple Random Sample
\end{quote}

Describe how to take a simple random sample from a classroom.

\textbf{Solution:}

\begin{quote}
Give each student in the class a number. Using a random number generator you could then pick the number of students you want to pick.
\end{quote}

\hypertarget{example-1.2.2-how-not-to-choose-a-simple-random-sample}{%
\subsubsection{Example \#1.2.2: How Not to Choose a Simple Random Sample}\label{example-1.2.2-how-not-to-choose-a-simple-random-sample}}

\begin{quote}
You want to choose 5 students out of a class of 20. Give some examples of samples that are {not} simple random samples:
\end{quote}

\textbf{Solution:}

\begin{quote}
Choose 5 students from the front row. The people in the last row have no chance of being selected.
\end{quote}

\begin{quote}
Choose the 5 shortest students. The tallest students have no chance of being selected.
\end{quote}

\textbf{Stratified sampling} is where you break the population into groups
called strata, then take a simple random sample from each strata.

For example:

\begin{quote}
If you want to look at musical preference, you could divide the individuals into age groups and then conduct simple random samples inside each group.
\end{quote}

\begin{quote}
If you want to calculate the average price of textbooks, you could divide the individuals into groups by major and then conduct simple random samples inside each group.
\end{quote}

\textbf{Systematic sampling} is where you randomly choose a starting place
then select every \emph{k}th individual to measure.

For example:

You select every 5\textsuperscript{th} item on an assembly line

You select every 10\textsuperscript{th} name on the list

You select every 3\textsuperscript{rd} customer that comes into the store.

\textbf{Cluster sampling} is where you break the population into groups
called clusters. Randomly pick some clusters then poll all individuals
in those clusters.

For example:

\begin{quote}
A large city wants to poll all businesses in the city. They divide the city into sections (clusters), maybe a square block for each section, and use a random number generator to pick some of the clusters. Then they poll all businesses in each chosen cluster.
\end{quote}

\begin{quote}
You want to measure whether a tree in the forest is infected with bark beetles. Instead of having to walk all over the forest, you divide the forest up into sectors, and then randomly pick the sectors that you will travel to. Then record whether a tree is infected or not for every tree in that sector.
\end{quote}

Many people confuse stratified sampling and cluster sampling. In
stratified sampling you use {all} the groups and
{some} of the members in each group. Cluster sampling is the
other way around. It uses {some} of the groups and
{all} the members in each group.

The four sampling techniques that were presented all have advantages and
disadvantages. There is another sampling technique that is sometimes
utilized because either the researcher doesn't know better, or it is
easier to do. This sampling technique is known as a convenience sample.
This sample will not result in a representative sample, and should be
avoided.

\textbf{Convenience sample} is one where the researcher picks individuals to
be included that are easy for the researcher to collect.

An example of a convenience sample is if you want to know the opinion of
people about the criminal justice system, and you stand on a street
corner near the county court house, and questioning the first 10 people
who walk by. The people who walk by the county court house are most
likely involved in some fashion with the criminal justice system, and
their opinion would not represent the opinions of all individuals.

On a rare occasion, you do want to collect the entire population. In
which case you conduct a census.

A \textbf{census} is when every individual of interest is measured.

\hypertarget{example-1.2.3-sampling-type}{%
\subsection{Example \#1.2.3: Sampling type}\label{example-1.2.3-sampling-type}}

\begin{quote}
Banner Health is a several state nonprofit chain of hospitals. Management wants to assess the incident of complications after surgery. They wish to use a sample of surgery patients. Several sampling techniques are described below. Categorize each technique as simple random sample, stratified sample, systematic sample, cluster sample, or convenience sampling.
\end{quote}

\begin{enumerate}
\def\labelenumi{\alph{enumi}.}
\tightlist
\item
  Obtain a list of patients who had surgery at all Banner Health
  facilities. Divide the patients according to type of surgery. Draw
  simple random samples from each group.
\end{enumerate}

\textbf{Solution}

This is a stratified sample since the patients where separated into
different stratum and then random samples were taken from each
strata. The problem with this is that some types of surgeries may
have more chances for complications than others. Of course, the
stratified sample would show you this.

\begin{enumerate}
\def\labelenumi{\alph{enumi}.}
\setcounter{enumi}{1}
\tightlist
\item
  Obtain a list of patients who had surgery at all Banner Health
  facilities. Number these patients, and then use a random number
  table to obtain the sample.
\end{enumerate}

\textbf{Solution}

\begin{quote}
This is a random sample since each patient has the same chance of
being chosen. The problem with this one is that it will take a while
to collect the data.
\end{quote}

\begin{enumerate}
\def\labelenumi{\alph{enumi}.}
\setcounter{enumi}{2}
\tightlist
\item
  Randomly select some Banner Health facilities from each of the seven
  states, and then include all the patients on the surgery lists of
  the states.
\end{enumerate}

\textbf{Solution}

This is a cluster sample since all patients are questioned in each
of the selected hospitals. The problem with this is that you could
have by chance selected hospitals that have no complications.

\begin{enumerate}
\def\labelenumi{\alph{enumi}.}
\setcounter{enumi}{3}
\tightlist
\item
  At the beginning of the year, instruct each Banner Health facility
  to record any complications from every 100\textsuperscript{th} surgery.
\end{enumerate}

\textbf{Solution}

This is a systematic sample since they selected every 100\textsuperscript{th}
surgery. The problem with this is that if every 90\textsuperscript{th} surgery has
complications, you wouldn't see this come up in the data.

\begin{enumerate}
\def\labelenumi{\alph{enumi}.}
\setcounter{enumi}{4}
\tightlist
\item
  Instruct each Banner Health facilities to record any complications
  from 20 surgeries this week and send in the results.
\end{enumerate}

\textbf{Solution}

This is a convenience sample since they left it up to the facility
how to do it. The problem with convenience samples is that the
person collecting the data will probably collect data from surgeries
that had no complications.

\hypertarget{homework-1}{%
\subsection{Homework}\label{homework-1}}

\begin{enumerate}
\def\labelenumi{\arabic{enumi}.}
\tightlist
\item
  Researchers want to collect cholesterol levels of U.S. patients who
  had a heart attack two days prior. The following are different
  sampling techniques that the researcher could use. Classify each as
  simple random sample, stratified sample, systematic sample, cluster
  sample, or convenience sample.
\end{enumerate}

\begin{enumerate}
\def\labelenumi{\alph{enumi}.}
\item
  The researchers randomly select 5 hospitals in the U.S. then measure
  the cholesterol levels of all the heart attack patients in each of
  those hospitals.
\item
  The researchers list all of the heart attack patients and measure
  the cholesterol level of every 25\textsuperscript{th} person on the list.
\item
  The researchers go to one hospital on a given day and measure the
  cholesterol level of the heart attack patients at that time.
\item
  The researchers list all of the heart attack patients. They then
  measure the cholesterol levels of randomly selected patients.
\item
  The researchers divide the heart attack patients based on race, and
  then measure the cholesterol levels of randomly selected patients in
  each race grouping.
\end{enumerate}

\begin{enumerate}
\def\labelenumi{\arabic{enumi}.}
\setcounter{enumi}{1}
\tightlist
\item
  The quality control officer at a manufacturing plant needs to
  determine what percentage of items in a batch are defective. The
  following are different sampling techniques that could be used by
  the officer. Classify each as simple random sample, stratified
  sample, systematic sample, cluster sample, or convenience sample.
\end{enumerate}

\begin{enumerate}
\def\labelenumi{\alph{enumi}.}
\item
  The officer lists all of the batches in a given month. The number of
  defective items is counted in randomly selected batches.
\item
  The officer takes the first 10 batches and counts the number of
  defective items.
\item
  The officer groups the batches made in a month into which shift they
  are made. The number of defective items is counted in randomly
  selected batches in each shift.
\item
  The officer chooses every 15\textsuperscript{th} batch off the line and counts the
  number of defective items in each chosen batch.
\item
  The officer divides the batches made in a month into which day they
  were made. Then certain days are picked and every batch made that
  day is counted to determine the number of defective items.
\end{enumerate}

\begin{enumerate}
\def\labelenumi{\arabic{enumi}.}
\setcounter{enumi}{2}
\item
  You wish to determine the GPA of students at your school. Describe
  what process you would go through to collect a sample if you use a
  simple random sample.
\item
  You wish to determine the GPA of students at your school. Describe
  what process you would go through to collect a sample if you use a
  stratified sample.
\item
  You wish to determine the GPA of students at your school. Describe
  what process you would go through to collect a sample if you use a
  systematic sample.
\item
  You wish to determine the GPA of students at your school. Describe
  what process you would go through to collect a sample if you use a
  cluster sample.
\item
  You wish to determine the GPA of students at your school. Describe
  what process you would go through to collect a sample if you use a
  convenience sample.
\end{enumerate}

\textbf{\\
}

\hypertarget{experimental-design}{%
\section{Experimental Design}\label{experimental-design}}

The section is an introduction to experimental design. This is how to
actually design an experiment or a survey so that they are statistical
sound. Experimental design is a very involved process, so this is just a
small introduction.

\hypertarget{guidelines-for-planning-a-statistical-study}{%
\subsection{Guidelines for planning a statistical study}\label{guidelines-for-planning-a-statistical-study}}

\begin{enumerate}
\def\labelenumi{\arabic{enumi}.}
\item
  Identify the individuals that you are interested in. Realize that
  you can only make conclusions for these individuals. As an example,
  if you use a fertilizer on a certain genus of plant, you can't say
  how the fertilizer will work on any other types of plants. However,
  if you diversify too much, then you may not be able to tell if there
  really is an improvement since you have too many factors to
  consider.
\item
  Specify the variable. You want to make sure this is something that
  you can measure, and make sure that you control for all other
  factors too. As an example, if you are trying to determine if a
  fertilizer works by measuring the height of the plants on a
  particular day, you need to make sure you can control how much
  fertilizer you put on the plants (which would be your treatment),
  and make sure that all the plants receive the same amount of
  sunlight, water, and temperature.
\item
  Specify the population. This is important in order for you know what
  conclusions you can make and what individuals you are making the
  conclusions about.
\item
  Specify the method for taking measurements or making observations.
\item
  Determine if you are taking a census or sample. If taking a sample,
  decide on the sampling method.
\item
  Collect the data.
\item
  Use appropriate descriptive statistics methods and make decisions
  using appropriate inferential statistics methods.
\item
  Note any concerns you might have about your data collection methods
  and list any recommendations for future.
\end{enumerate}

There are two types of studies:

An \textbf{observational study} is when the investigator collects data merely
by watching or asking questions. He doesn't change anything.

An \textbf{experiment} is when the investigator changes a variable or imposes
a treatment to determine its effect.

\hypertarget{example-1.3.1-observational-study-or-experiment}{%
\subsection{Example \#1.3.1: Observational Study or Experiment}\label{example-1.3.1-observational-study-or-experiment}}

\begin{quote}
State if the following is an observational study or an experiment.
\end{quote}

a.) Poll students to see if they favor increasing tuition.

\begin{quote}
\textbf{Solution:}
\end{quote}

This is an observational study. You are only asking a question.

b.) Give some students a tutor to see if grades improve.

\begin{quote}
\textbf{Solution:}
\end{quote}

This is an experiment. The tutor is the treatment.

Many observational studies involve surveys. A \textbf{survey} uses questions
to collect the data and needs to be written so that there is no bias.

In an experiment, there are different options.

\textbf{Randomized two-treatment experiment:} in this experiment, there are
two treatments, and individuals are randomly placed into the two groups.
Either both groups get a treatment, or one group gets a treatment and
the other gets either nothing or a placebo. The group getting either no
treatment or the placebo is called the control group. The group getting
the treatment is called the treatment group. The idea of the placebo is
that a person thinks they are receiving a treatment, but in reality they
are receiving a sugar pill or fake treatment. Doing this helps to
account for the placebo effect, which is where a person's mind makes
their body respond to a treatment because they think they are taking the
treatment when they are not really taking the treatment. Note, not every
experiment needs a placebo, such when using animals or plants. Also, you
can't always use a placebo or no treatment. As an example, if you are
testing a new blood pressure medication you can't give a person with
high blood pressure a placebo or no treatment because of moral reasons.

\textbf{Randomized Block Design:} a block is a group of subjects that are
similar, but the blocks differ from each other. Then randomly assign
treatments to subjects inside each block. An example would be separating
students into full-time versus part-time, and then randomly picking a
certain number full-time students to get the treatment and a certain
number part-time students to get the treatment. This way some of each
type of student gets the treatment and some do not.

\textbf{Rigorously Controlled Design:} carefully assign subjects to different
treatment groups, so that those given each treatment are similar in ways
that are important to the experiment. An example would be if you want to
have a full-time student who is male, takes only night classes, has a
full-time job, and has children in one treatment group, then you need to
have the same type of student getting the other treatment. This type of
design is hard to implement since you don't know how many
differentiations you would use, and should be avoided.

\textbf{Matched Pairs Design:} the treatments are given to two groups that
can be matched up with each other in some ways. One example would be to
measure the effectiveness of a muscle relaxer cream on the right arm and
the left arm of individuals, and then for each individual you can match
up their right arm measurement with their left arm. Another example of
this would be before and after experiments, such as weight before and
weight after a diet.

No matter which experiment type you conduct, you should also consider
the following:

\textbf{Replication}: repetition of an experiment on more than one subject so
you can make sure that the sample is large enough to distinguish true
effects from random effects. It is also the ability for someone else to
duplicate the results of the experiment.

\textbf{Blind study} is where the individual does not know which treatment
they are getting or if they are getting the treatment or a placebo.

\textbf{Double-blind study} is where neither the individual nor the
researcher knows who is getting which treatment or who is getting the
treatment and who is getting the placebo. This is important so that
there can be no bias created by either the individual or the researcher.

One last consideration is the time period that you are collecting the
data over. There are three types of time periods that you can consider.

\textbf{Cross-sectional study}: data observed, measured, or collected at one
point in time.

\textbf{Retrospective} (or \textbf{case-control}) \textbf{study}: data collected from
the past using records, interviews, and other similar artifacts.

\textbf{Prospective} (or \textbf{longitudinal} or \textbf{cohort}) \textbf{study}: data
collected in the future from groups sharing common factors.

\hypertarget{homework-2}{%
\subsection{Homework}\label{homework-2}}

\begin{enumerate}
\def\labelenumi{\arabic{enumi}.}
\item
  You want to determine if cinnamon reduces a person's insulin sensitivity. You give patients who are insulin sensitive a certain amount of cinnamon and then measure their glucose levels. Is this an observation or an experiment? Why?
\item
  You want to determine if eating more fruits reduces a person's chance of developing cancer. You watch people over the years and ask them to tell you how many servings of fruit they eat each day. You then record who develops cancer. Is this an observation or an experiment? Why?
\item
  A researcher wants to evaluate whether countries with lower fertility rates have a higher life expectancy. They collect the fertility rates and the life expectancies of countries around the world. Is this an observation or an experiment? Why?
\item
  To evaluate whether a new fertilizer improves plant growth more than the old fertilizer, the fertilizer developer gives some plants the new fertilizer and others the old fertilizer. Is this an observation or an experiment? Why?
\item
  A researcher designs an experiment to determine if a new drug lowers the blood pressure of patients with high blood pressure. The patients are randomly selected to be in the study and they randomly pick which group to be in. Is this a randomized experiment? Why or why not?
\item
  Doctors trying to see if a new stint works longer for kidney patients, asks patients if they are willing to have one of two different stints put in. During the procedure the doctor decides which stent to put in based on which one is on hand at the time. Is this a randomized experiment? Why or why not?
\item
  A researcher wants to determine if diet and exercise together helps people lose weight over just exercising. The researcher solicits volunteers to be part of the study, randomly picks which volunteers are in the study, and then lets each volunteer decide if they want to be in the diet and exercise group or the exercise only group. Is this a randomized experiment? Why or why not?
\item
  To determine if lack of exercise reduces flexibility in the knee joint, physical therapists ask for volunteers to join their trials. They then randomly select the volunteers to be in the group that exercises and to be in the group that doesn't exercise. Is this a randomized experiment? Why or why not?
\item
  You collect the weights of tagged fish in a tank. You then put an extra protein fish food in water for the fish and then measure their weight a month later. Are the two samples matched pairs or not? Why or why not?
\item
  A mathematics instructor wants to see if a computer homework system improves the scores of the students in the class. The instructor teaches two different sections of the same course. One section utilizes the computer homework system and the other section completes homework with paper and pencil. Are the two samples matched pairs or not? Why or why not?
\item
  A business manager wants to see if a new procedure improves the processing time for a task. The manager measures the processing time of the employees then trains the employees using the new procedure. Then each employee performs the task again and the processing time is measured again. Are the two samples matched pairs or not? Why or why not?
\item
  The prices of generic items are compared to the prices of the equivalent named brand items. Are the two samples matched pairs or not? Why or why not?
\item
  A doctor gives some of the patients a new drug for treating acne and the rest of the patients receive the old drug. Neither the patient nor the doctor knows who is getting which drug. Is this a blind experiment, double blind experiment, or neither? Why?
\item
  One group is told to exercise and one group is told to not exercise. Is this a blind experiment, double blind experiment, or neither? Why?
\item
  The researchers at a hospital want to see if a new surgery procedure has a better recovery time than the old procedure. The patients are not told which procedure that was used on them, but the surgeons obviously did know. Is this a blind experiment, double blind experiment, or neither? Why?
\item
  To determine if a new medication reduces headache pain, some patients are given the new medication and others are given a placebo. Neither the researchers nor the patients know who is taking the real medication and who is taking the placebo. Is this a blind experiment, double blind experiment, or neither? Why?
\item
  A new study is underway to track the eating and exercise patterns of people at different time periods in the future, and see who is afflicted with cancer later in life. Is this a cross-sectional study, a retrospective study, or a prospective study? Why?
\item
  To determine if a new medication reduces headache pain, some patients are given the new medication and others are given a placebo. The pain levels of a patient are then recorded. Is this a cross-sectional study, a retrospective study, or a prospective study? Why?
\item
  To see if there is a link between smoking and bladder cancer, patients with bladder cancer are asked if they currently smoke or if they smoked in the past. Is this a cross-sectional study, a retrospective study, or a prospective study? Why?
\item
  The Nurses Health Survey was a survey where nurses were asked to record their eating habits over a period of time, and their general health was recorded. Is this a cross-sectional study, a retrospective study, or a prospective study? Why?
\item
  Consider a question that you would like to answer. Describe how you would design your own experiment. Make sure you state the question you would like to answer, then determine if an experiment or an observation is to be done, decide if the question needs one or two samples, if two samples are the samples matched, if this is a randomized experiment, if there is any blinding, and if this is a cross-sectional, retrospective, or prospective study.
\end{enumerate}

\hypertarget{how-not-to-do-statistics}{%
\section{How Not to Do Statistics}\label{how-not-to-do-statistics}}

Many studies are conducted and conclusions are made. However, there are
occasions where the study is not conducted in the correct manner or the
conclusion is not correctly made based on the data. There are many
things that you should question when you read a study. There are many
reasons for the study to have bias in it. Bias is where a study may have
a certain slant or preference for a certain result. The following are a
list of some of the questions or issues you should consider to help
decide if there is bias in a study.

One of the first issues you should ask is who funded the study. If the
entity that sponsored the study stands to gain either profits or
notoriety from the results, then you should question the results. It
doesn't mean that the results are wrong, but you should scrutinize them
on your own to make sure they are sound. As an example if a study says
that genetically modified foods are safe, and the study was funded by a
company that sells genetically modified food, then one may question the
validity of the study. Since the company funds the study and their
profits rely on people buying their food, there may be bias.

An experiment could have \textbf{lurking or confounding variables} when you
cannot rule out the possibility that the observed effect is due to some
other variable rather than the factor being studied. An example of this
is when you give fertilizer to some plants and no fertilizer to others,
but the no fertilizer plants also are placed in a location that doesn't
receive direct sunlight. You won't know if the plants that received the
fertilizer grew taller because of the fertilizer or the sunlight. Make
sure you design experiments to eliminate the effects of confounding
variables by controlling all the factors that you can.

\textbf{Overgeneralization} is where you do a study on one group and then try
to say that it will happen on all groups. An example is doing cancer
treatments on rats. Just because the treatment works on rats does not
mean it will work on humans. Another example is that until recently most
FDA medication testing had been done on white males of a particular age.
There is no way to know how the medication affects other genders, ethnic
groups, age groups, and races. The new FDA guidelines stresses using
individuals from different groups.

\textbf{Cause and effect} is where people decide that one variable causes the
other just because the variables are related or correlated. Unless the
study was done as an experiment where a variable was controlled, you
cannot say that one variable caused the other. Most likely there is
another variable that caused both. As an example, there is a
relationship between number of drownings at the beach and ice cream
sales. This does not mean that ice cream sales increasing causes people
to drown. Most likely the cause for both increasing is the heat.

\textbf{Sampling error}: This is the difference between the sample results
and the true population results. This is unavoidable, and results in the
fact that samples are different from each other. As an example, if you
take a sample of 5 people's height in your class, you will get 5
numbers. If you take another sample of 5 people's heights in your class,
you will likely get 5 different numbers.

\textbf{Nonsampling error}: This is where the sample is collected poorly
either through a biased sample or through error in measurements. Care
should be taken to avoid this error.

Lastly, there should be care taken in considering the difference between
\textbf{statistical significance versus practical significance}. This is a
major issue in statistics. Something could be statistically
significance, which means that a statistical test shows there is
evidence to show what you are trying to prove. However, in practice it
doesn't mean much or there are other issues to consider. As an example,
suppose you find that a new drug for high blood pressure does reduce the
blood pressure of patients. When you look at the improvement it actually
doesn't amount to a large difference. Even though statistically there is
a change, it may not be worth marketing the product because it really
isn't that big of a change. Another consideration is that you find the
blood pressure medication does improve a person's blood pressure, but it
has serious side effects or it costs a great deal for a prescription. In
this case, it wouldn't be practical to use it. In both cases, the study
is shown to be statistically significant, but practically you don't want
to use the medication. The main thing to remember in a statistical study
is that the statistics is only part of the process. You also want to
make sure that there is practical significance too.

\textbf{Surveys} have their own areas of bias that can occur. A few of the
issues with surveys are in the wording of the questions, the ordering of
the questions, the manner the survey is conducted, and the response rate
of the survey.

The wording of the questions can cause \textbf{hidden bias}, which is where
the questions are asked in a way that makes a person respond a certain
way. An example is that a poll was done where people were asked if they
believe that there should be an amendment to the constitution protecting
a woman's right to choose. About 60\% of all people questioned said yes.
Another poll was done where people were asked if they believe that there
should be an amendment to the constitution protecting the life of an
unborn child. About 60\% of all people questioned said yes. These two
questions deal with the same issue, though giving opposite results, but
how the question was asked affected the outcome.

The ordering of the question can also cause \textbf{hidden bias}. An example
of this is if you were asked if there should be a fine for texting while
driving, but proceeding that question is the question asking if you text
while drive. By asking a person if they actually partake in the
activity, that person now personalizes the question and that might
affect how they answer the next question of creating the fine.

\textbf{Non-response} is where you send out a survey but not everyone returns
the survey. You can calculate the response rate by dividing the number
of returns by the number of surveys sent. Most response rates are around
30-50\%. A response rate less than 30\% is very poor and the results of
the survey are not valid. To reduce non-response, it is better to
conduct the surveys in person, though these are very expensive. Phones
are the next best way to conduct surveys, emails can be effective, and
physical mailings are the least desirable way to conduct surveys.

\textbf{Voluntary response} is where people are asked to respond via phone,
email or online. The problem with these is that only people who really
care about the topic are likely to call or email. These surveys are not
scientific and the results from these surveys are not valid. Note: all
studies involve volunteers. The difference between a voluntary response
survey and a scientific study is that in a scientific study the
researchers ask the individuals to be involved, while in a voluntary
response survey the individuals become involved on their own choosing.

\hypertarget{example-1.4.1-bias-in-a-study}{%
\subsection{Example \#1.4.1: Bias in a Study**}\label{example-1.4.1-bias-in-a-study}}

\begin{quote}
Suppose a mathematics department at a community college would like to
assess whether computer-based homework improves students' test scores.
They use computer-based homework in one classroom with one teacher and
use traditional paper and pencil homework in a different classroom
with a different teacher. The students using the computer-based
homework had higher test scores. What is wrong with this experiment?
\end{quote}

\textbf{Solution}

\begin{quote}
Since there were different teachers, you do not know if the better
test scores are because of the teacher or the computer-based homework.
A better design would be have the same teacher teach both classes. The
control group would utilize traditional paper and pencil homework and
the treatment group would utilize the computer-based homework. Both
classes would have the same teacher, and the students would be split
between the two classes randomly. The only difference between the two
groups should be the homework method. Of course, there is still
variability between the students, but utilizing the same teacher will
reduce any other confounding variables.
\end{quote}

\hypertarget{example-1.4.2-cause-and-effect}{%
\subsection{Example \#1.4.2: Cause and Effect}\label{example-1.4.2-cause-and-effect}}

Determine if the one variable did cause the change in the other
variable.

\begin{enumerate}
\def\labelenumi{\alph{enumi}.}
\tightlist
\item
  Cinnamon was giving to a group of people who have diabetes, and then
  their blood glucose levels were measured a time period later. All
  other factors for each person were kept the same. Their glucose
  levels went down. Did the cinnamon cause the reduction?
\end{enumerate}

\begin{quote}
\textbf{Solution:}

Since this was a study where the use of cinnamon was controlled, and
all other factors were kept constant from person to person, then any
changes in glucose levels can be attributed to the use of cinnamon.
\end{quote}

\begin{enumerate}
\def\labelenumi{\alph{enumi}.}
\setcounter{enumi}{1}
\tightlist
\item
  There is a link between spray on tanning products and lung cancer.
  Does that mean that spray on tanning products cause lung cancer?
\end{enumerate}

\begin{quote}
\textbf{Solution:}

Since there is only a link, and not a study controlling the use of the
tanning spray, then you cannot say that increased use causes lung
cancer. You can say that there is a link, and that there could be a
cause, but you cannot say for sure that the spray causes the cancer.
\end{quote}

\hypertarget{example-1.4.3-generalizations}{%
\subsection{Example \#1.4.3: Generalizations}\label{example-1.4.3-generalizations}}

\begin{enumerate}
\def\labelenumi{\alph{enumi}.}
\tightlist
\item
  A researcher conducts a study on the use of ibuprofen on humans and
  finds that it is safe. Does that mean that all species can use
  ibuprofen?
\end{enumerate}

\begin{quote}
\textbf{Solution:}

No.~Just because a drug is safe to use on one species doesn't mean it
is safe to use for all species. In fact, ibuprofen is toxic to cats.
\end{quote}

\begin{enumerate}
\def\labelenumi{\alph{enumi}.}
\setcounter{enumi}{1}
\tightlist
\item
  Aspirin has been used for years to bring down fevers in humans.
  Originally it was tested on white males between the ages of 25 and
  40 and found to be safe. Is it safe to give to everyone?
\end{enumerate}

\begin{quote}
\textbf{Solution:}

No.~Just because one age group can use it doesn't mean it is safe to
use for all age groups. In fact, there has been a link between giving
a child under the age of 19 aspirin when they have a fever and Reye's
syndrome.
\end{quote}

\hypertarget{homework-3}{%
\subsection{Homework}\label{homework-3}}

\begin{enumerate}
\def\labelenumi{\arabic{enumi}.}
\item
  Suppose there is a study where a researcher conducts an experiment
  to show that deep breathing exercises helps to lower blood pressure.
  The researcher takes two groups of people and has one group to
  perform deep breathing exercises and a series of aerobic exercises
  every day and the other group was asked to refrain from any
  exercises. The researcher found that the group performing the deep
  breathing exercises and the aerobic exercises had lower blood
  pressure. Discuss any issue with this study.
\item
  Suppose a car dealership offers a low interest rate and a longer
  payoff period to customers or a high interest rate and a shorter
  payoff period to customers, and most customers choose the low
  interest rate and longer payoff period, does that mean that most
  customers want a lower interest rate? Explain.
\item
  Over the years it has been said that coffee is bad for you. When
  looking at the studies that have shown that coffee is linked to poor
  health, you will see that people who tend to drink coffee don't
  sleep much, tend to smoke, don't eat healthy, and tend to not
  exercise. Can you say that the coffee is the reason for the poor
  health or is there a lurking variable that is the actual cause?
  Explain.
\item
  When researchers were trying to figure out what caused polio, they
  saw a connection between ice cream sales and polio. As ice cream
  sales increased so did the incident of polio. Does that mean that
  eating ice cream causes polio? Explain your answer.
\item
  There is a positive correlation between having a discussion of gun
  control, which usually occur after a mass shooting, and the sale of
  guns. Does that mean that the discussion of gun control increases
  the likelihood that people will buy more guns? Explain.
\item
  There is a study that shows that people who are obese have a vitamin
  D deficiency. Does that mean that obesity causes a deficiency in
  vitamin D? Explain.
\item
  A study was conducted that shows that polytetrafluoroethylene (PFOA)
  (Teflon is made from this chemical) has an increase risk of tumors
  in lab mice. Does that mean that PFOA's have an increased risk of
  tumors in humans? Explain.
\item
  Suppose a telephone poll is conducted by contacting U.S. citizens
  via landlines about their view of gay marriage. Suppose over 50\% of
  those called do not support gay marriage. Does that mean that you
  can say over 50\% of all people in the U.S. do not support gay
  marriage? Explain.
\item
  Suppose that it can be shown to be statistically significant that a
  smaller percentage of the people are satisfied with your business.
  The percentage before was 87\% and is now 85\%. Do you change how you
  conduct business? Explain?
\item
  You are testing a new drug for weight loss. You find that the drug
  does in fact statistically show a weight loss. Do you market the new
  drug? Why or why not?
\item
  There was an online poll conducted about whether the mayor of
  Auckland, New Zealand, should resign due to an affair. The majority
  of people participating said he should. Should the mayor resign due
  to the results of this poll? Explain.
\item
  An online poll showed that the majority of Americans believe that
  the government covered up events of 9/11. Does that really mean that
  most Americans believe this? Explain.
\item
  A survey was conducted at a college asking all employees if they
  were satisfied with the level of security provided by the security
  department. Discuss how the results of this question could be
  biased.
\item
  An employee survey says, ``Employees at this institution are very
  satisfied with working here. Please rate your satisfaction with the
  institution.'' Discuss how this question could create bias.
\item
  A survey has a question that says, ``Most people are afraid that they
  will lose their house due to economic collapse. Choose what you
  think is the biggest issue facing the nation today. a) Economic
  collapse, b) Foreign policy issues, c) Environmental concerns.''
  Discuss how this question could create bias.
\item
  A survey says, ``Please rate the career of Roberto Clemente, one of
  the best right field baseball players in the world.'' Discuss how
  this question could create bias.
\end{enumerate}

\hypertarget{graphical-descriptions-of-data}{%
\chapter{Graphical Descriptions of Data}\label{graphical-descriptions-of-data}}

In chapter 1, you were introduced to the concepts of population, which
again is a collection of all the measurements from the individuals of
interest. Remember, in most cases you can't collect the entire
population, so you have to take a sample. Thus, you collect data either
through a sample or a census. Now you have a large number of data
values. What can you do with them? No one likes to look at just a set of
numbers. One thing is to organize the data into a table or graph.
Ultimately though, you want to be able to use that graph to interpret
the data, to describe the distribution of the data set, and to explore
different characteristics of the data. The characteristics that will be
discussed in this chapter and the next chapter are:

\begin{enumerate}
\def\labelenumi{\arabic{enumi}.}
\item
  Center: middle of the data set, also known as the average.
\item
  Variation: how much the data varies.
\item
  Distribution: shape of the data (symmetric, uniform, or skewed).
\item
  Qualitative data: analysis of the data
\item
  Outliers: data values that are far from the majority of the data.
\item
  Time: changing characteristics of the data over time.
\end{enumerate}

This chapter will focus mostly on using the graphs to understand aspects
of the data, and not as much on how to create the graphs. There is
technology that will create most of the graphs, though it is important
for you to understand the basics of how to create them.

\hypertarget{qualitative-data}{%
\section{Qualitative Data}\label{qualitative-data}}

Remember, qualitative data are words describing a characteristic of the
individual. There are several different graphs that are used for
qualitative data. These graphs include bar graphs, Pareto charts, and
pie charts.

Pie charts and bar graphs are the most common ways of displaying
qualitative data. A spreadsheet program like Excel can make both of
them. The first step for either graph is to make a \textbf{frequency or
relative frequency table}. A frequency table is a summary of the data
with counts of how often a data value (or category) occurs.

\textbf{Example \#2.1.1: Creating a Frequency Table}

\begin{quote}
Suppose you have the following data for which type of car students at
a college drive?

Ford, Chevy, Honda, Toyota, Toyota, Nissan, Kia, Nissan, Chevy,
Toyota, Honda, Chevy, Toyota, Nissan, Ford, Toyota, Nissan, Mercedes,
Chevy, Ford, Nissan, Toyota, Nissan, Ford, Chevy, Toyota, Nissan,
Honda, Porsche, Hyundai, Chevy, Chevy, Honda, Toyota, Chevy, Ford,
Nissan, Toyota, Chevy, Honda, Chevy, Saturn, Toyota, Chevy, Chevy,
Nissan, Honda, Toyota, Toyota, Nissan

A listing of data is too hard to look at and analyze, so you need to
summarize it. First you need to decide the categories. In this case it
is relatively easy; just use the car type. However, there are several
cars that only have one car in the list. In that case it is easier to
make a category called other for the ones with low values. Now just
count how many of each type of cars there are. For example, there are
5 Fords, 12 Chevys, and 6 Hondas. This can be put in a frequency
distribution:
\end{quote}

\textbf{Table \#2.1.1: Frequency Table for Type of Car Data}

\begin{longtable}[]{@{}ll@{}}
\toprule
\endhead
Category & Frequency\tabularnewline
Ford & 5\tabularnewline
Chevy & 12\tabularnewline
Honda & 6\tabularnewline
Toyota & 12\tabularnewline
Nissan & 10\tabularnewline
Other & 5\tabularnewline
Total & 50\tabularnewline
\bottomrule
\end{longtable}

\begin{quote}
The total of the frequency column should be the number of observations
in the data.

Since raw numbers are not as useful to tell other people it is better
to create a third column that gives the relative frequency of each
category. This is just the frequency divided by the total. As an
example for Ford category:

This can be written as a decimal, fraction, or percent. You now have a
relative frequency distribution:
\end{quote}

\textbf{Table \#2.1.2: Relative Frequency Table for Type of Car Data}

\begin{longtable}[]{@{}lll@{}}
\toprule
\endhead
Category & Frequency & Relative Frequency\tabularnewline
Ford & 5 & 0.10\tabularnewline
Chevy & 12 & 0.24\tabularnewline
Honda & 6 & 0.12\tabularnewline
Toyota & 12 & 0.24\tabularnewline
Nissan & 10 & 0.20\tabularnewline
Other & 5 & 0.10\tabularnewline
Total & 50 & 1.00\tabularnewline
\bottomrule
\end{longtable}

\begin{quote}
The relative frequency column should add up to 1.00. It might be off a little due to rounding errors.
\end{quote}

Now that you have the frequency and relative frequency table, it would be good to display this data using a graph. There are several different types of graphs that can be used: bar chart, pie chart, and Pareto charts.

\textbf{Bar graphs or charts} consist of the frequencies on one axis and the categories on the other axis. Then you draw rectangles for each category with a height (if frequency is on the vertical axis) or length (if frequency is on the horizontal axis) that is equal to the frequency. All of the rectangles should be the same width, and there should be equally width gaps between each bar.

\textbf{Example \#2.1.2: Drawing a Bar Graph}

Draw a bar graph of the data in example \#2.1.1.

\textbf{Table \#2.1.2: Frequency Table for Type of Car Data}

\begin{longtable}[]{@{}lll@{}}
\toprule
\endhead
Category & Frequency & Relative Frequency\tabularnewline
Ford & 5 & 0.10\tabularnewline
Chevy & 12 & 0.24\tabularnewline
Honda & 6 & 0.12\tabularnewline
Toyota & 12 & 0.24\tabularnewline
Nissan & 10 & 0.20\tabularnewline
Other & 5 & 0.10\tabularnewline
Total & 50 & 1.00\tabularnewline
\bottomrule
\end{longtable}

\begin{quote}
Put the frequency on the vertical axis and the category on the horizontal axis. Then just draw a box above each category whose height is the frequency.
\end{quote}

\begin{quote}
All graphs are drawn using R. The command in R to create a bar graph is:
\end{quote}

\begin{quote}
variable\textless{}-c(type in percentages or frequencies for each class with commas in between values)
\end{quote}

\begin{quote}
barplot(variable,names.arg=c("type in name of 1\textsuperscript{st} category", "type in name of 2\textsuperscript{nd} category",\ldots{},"type in name of last category"), ylim=c(0,number over max), xlab="type in label for x-axis", ylab="type in label for y-axis",ylim=c(0,number above maximum y value), main="type in title", col="type in a color") -- creates a bar graph of the data in a color if you want.
\end{quote}

\begin{quote}
For this example the command would be:

car\textless{}-c(5, 12, 6, 12, 10, 5)

barplot(car, names.arg=c("Ford", "Chevy", "Honda", "Toyota",
"Nissan", "Other"), xlab="Type of Car", ylab="Frequency",
ylim=c(0,12), main="Type of Car Driven by College Students",
col="blue")
\end{quote}

\textbf{Graph \#2.1.1: Bar Graph for Type of Car Data}

\begin{quote}
\includegraphics[width=4.15278in,height=4.15278in]{media/image2.emf}

Notice from the graph, you can see that Toyota and Chevy are the more
popular car, with Nissan not far behind. Ford seems to be the type of
car that you can tell was the least liked, though the cars in the
other category would be liked less than a Ford.
\end{quote}

\textbf{Some key features of a bar graph:}

\begin{itemize}
\item
  Equal spacing on each axis.
\item
  Bars are the same width.
\item
  There should be labels on each axis and a title for the graph.
\item
  There should be a scaling on the frequency axis and the categories
  should be listed on the category axis.
\item
  The bars don't touch.
\end{itemize}

You can also draw a bar graph using relative frequency on the vertical
axis. This is useful when you want to compare two samples with different
sample sizes. The relative frequency graph and the frequency graph
should look the same, except for the scaling on the frequency axis.

Using R, the command would be:

\begin{quote}
car\textless{}-c(0.1, 0.24, 0.12, 0.24, 0.2, 0.1)

barplot(car, names.arg=c("Ford", "Chevy", "Honda", "Toyota",
"Nissan", "Other"), xlab="Type of Car", ylab="Relative
Frequency", main="Type of Car Driven by College Students",
col="blue", ylim=c(0,.25))
\end{quote}

\textbf{Graph \#2.1.2: Relative Frequency Bar Graph for Type of Car Data}

\includegraphics[width=4.65278in,height=4.65278in]{media/image3.emf}

Another type of graph for qualitative data is a pie chart. A pie chart
is where you have a circle and you divide pieces of the circle into pie
shapes that are proportional to the size of the relative frequency.
There are 360 degrees in a full circle. Relative frequency is just the
percentage as a decimal. All you have to do to find the angle by
multiplying the relative frequency by 360 degrees. Remember that 180
degrees is half a circle and 90 degrees is a quarter of a circle.

\textbf{Example \#2.1.3: Drawing a Pie Chart}

Draw a pie chart of the data in example \#2.1.1.

First you need the relative frequencies.

\textbf{Table \#2.1.2: Frequency Table for Type of Car Data}

\begin{longtable}[]{@{}lll@{}}
\toprule
\endhead
Category & Frequency & Relative Frequency\tabularnewline
Ford & 5 & 0.10\tabularnewline
Chevy & 12 & 0.24\tabularnewline
Honda & 6 & 0.12\tabularnewline
Toyota & 12 & 0.24\tabularnewline
Nissan & 10 & 0.20\tabularnewline
Other & 5 & 0.10\tabularnewline
Total & 50 & 1.00\tabularnewline
\bottomrule
\end{longtable}

\begin{quote}
Then you multiply each relative frequency by 360° to obtain the angle
measure for each category.
\end{quote}

\textbf{Table \#2.1.3: Pie Chart Angles for Type of Car Data}

\begin{longtable}[]{@{}lll@{}}
\toprule
\endhead
Category & Relative Frequency & Angle (in degrees (°))\tabularnewline
Ford & 0.10 & 36.0\tabularnewline
Chevy & 0.24 & 86.4\tabularnewline
Honda & 0.12 & 43.2\tabularnewline
Toyota & 0.24 & 86.4\tabularnewline
Nissan & 0.20 & 72.0\tabularnewline
Other & 0.10 & 36.0\tabularnewline
Total & 1.00 & 360.0\tabularnewline
\bottomrule
\end{longtable}

\begin{quote}
Now draw the pie chart using a compass, protractor, and straight edge.
Technology is preferred. If you use technology, there is no need for
the relative frequencies or the angles.

You can use R to graph the pie chart. In R, the commands would be:

pie(variable,labels=c("type in name of 1\textsuperscript{st} category", "type in
name of 2\textsuperscript{nd} category",\ldots{},"type in name of last
category"),main="type in title", col=rainbow(number of categories))
-- creates a pie chart with a title and rainbow of colors for each
category.

For this example, the commands would be:

car\textless{}-c(5, 12, 6, 12, 10, 5)

pie(car, labels=c("Ford, 10\%", "Chevy, 24\%", "Honda, 12\%",
"Toyota, 24\%", "Nissan, 20\%", "Other, 10\%"), main="Type of Car
Driven by College Students", col=rainbow(6))
\end{quote}

\textbf{Graph \#2.1.3: Pie Chart for Type of Car Data}

\begin{quote}
\includegraphics[width=4.77778in,height=4.77778in]{media/image4.emf}

As you can see from the graph, Toyota and Chevy are more popular,
while the cars in the other category are liked the least. Of the cars
that you can determine from the graph, Ford is liked less than the
others.
\end{quote}

Pie charts are useful for comparing sizes of categories. Bar charts show
similar information. It really doesn't matter which one you use. It
really is a personal preference and also what information you are trying
to address. However, pie charts are best when you only have a few
categories and the data can be expressed as a percentage. The data
doesn't have to be percentages to draw the pie chart, but if a data
value can fit into multiple categories, you cannot use a pie chart. As
an example, if you asking people about what their favorite national park
is, and you say to pick the top three choices, then the total number of
answers can add up to more than 100\% of the people involved. So you
cannot use a pie chart to display the favorite national park.

A third type of qualitative data graph is a \textbf{Pareto chart,} which is
just a bar chart with the bars sorted with the highest frequencies on
the left. Here is the Pareto chart for the data in Example \#2.1.1.

\textbf{Graph \#2.1.4: Pareto Chart for Type of Car Data}

\includegraphics[width=5.36111in,height=5.36111in]{media/image5.emf}

The advantage of Pareto charts is that you can visually see the more
popular answer to the least popular. This is especially useful in
business applications, where you want to know what services your
customers like the most, what processes result in more injuries, which
issues employees find more important, and other type of questions like
these.

There are many other types of graphs that can be used on qualitative
data. There are spreadsheet software packages that will create most of
them, and it is better to look at them to see what can be done. It
depends on your data as to which may be useful. The next example
illustrates one of these types known as a multiple bar graph.

\textbf{Example \#2.1.4: Multiple Bar Graph}

\begin{quote}
In the Wii Fit game, you can do four different types if exercises:
yoga, strength, aerobic, and balance. The Wii system keeps track of
how many minutes you spend on each of the exercises everyday. The
following graph is the data for Dylan over one week time period.
Discuss any indication you can infer from the graph.
\end{quote}

\textbf{Graph \#2.1.5: Multiple Bar Chart for Wii Fit Data}

\begin{quote}
\includegraphics[width=4.97222in,height=3.05556in]{media/image6.png}

\textbf{Solution:}

It appears that Dylan spends more time on balance exercises than on
any other exercises on any given day. He seems to spend less time on
strength exercises on a given day. There are several days when the
amount of exercise in the different categories is almost equal.
\end{quote}

The usefulness of a multiple bar graph is the ability to compare several
different categories over another variable, in example \#2.1.4 the
variable would be time. This allows a person to interpret the data with
a little more ease.

\hypertarget{homework-4}{%
\subsection{Homework}\label{homework-4}}

\begin{enumerate}
\def\labelenumi{\arabic{enumi}.}
\tightlist
\item
  Eyeglassomatic manufactures eyeglasses for different retailers. The
  number of lenses for different activities is in table \#2.1.4.
\end{enumerate}

\begin{quote}
\textbf{Table \#2.1.4: Data for Eyeglassomatic}
\end{quote}

\begin{longtable}[]{@{}lllllll@{}}
\toprule
Activity & Grind & Multicoat & Assemble & Make frames & Receive finished & Unknown\tabularnewline
\midrule
\endhead
Number of lenses & 18872 & 12105 & 4333 & 25880 & 26991 & 1508\tabularnewline
\bottomrule
\end{longtable}

\begin{quote}
Grind means that they ground the lenses and put them in frames,
multicoat means that they put tinting or scratch resistance coatings
on lenses and then put them in frames, assemble means that they
receive frames and lenses from other sources and put them together,
make frames means that they make the frames and put lenses in from
other sources, receive finished means that they received glasses from
other source, and unknown means they do not know where the lenses came
from. Make a bar chart and a pie chart of this data. State any
findings you can see from the graphs.
\end{quote}

\begin{enumerate}
\def\labelenumi{\arabic{enumi}.}
\setcounter{enumi}{1}
\tightlist
\item
  To analyze how Arizona workers ages 16 or older travel to work the
  percentage of workers using carpool, private vehicle (alone), and
  public transportation was collected. Create a bar chart and pie
  chart of the data in table \#2.1.5. State any findings you can see
  from the graphs.
\end{enumerate}

\textbf{Table \#2.1.5: Data of Travel Mode for Arizona Workers}

\begin{longtable}[]{@{}ll@{}}
\toprule
Transportation type & Percentage\tabularnewline
\midrule
\endhead
Carpool & 11.6\%\tabularnewline
Private Vehicle (Alone) & 75.8\%\tabularnewline
Public Transportation & 2.0\%\tabularnewline
Other & 10.6\%\tabularnewline
\bottomrule
\end{longtable}

\begin{enumerate}
\def\labelenumi{\arabic{enumi}.}
\setcounter{enumi}{2}
\tightlist
\item
  The number of deaths in the US due to carbon monoxide (CO) poisoning
  from generators from the years 1999 to 2011 are in table \#2.1.6
  (Hinatov, 2012). Create a bar chart and pie chart of this data.
  State any findings you see from the graphs.
\end{enumerate}

\textbf{Table \#2.1.6: Data of Number of Deaths Due to CO Poisoning}

\begin{longtable}[]{@{}ll@{}}
\toprule
Region & Number of deaths from CO while using a generator\tabularnewline
\midrule
\endhead
Urban Core & 401\tabularnewline
Sub-Urban & 97\tabularnewline
Large Rural & 86\tabularnewline
Small Rural/Isolated & 111\tabularnewline
\bottomrule
\end{longtable}

\begin{enumerate}
\def\labelenumi{\arabic{enumi}.}
\setcounter{enumi}{3}
\tightlist
\item
  In Connecticut households use gas, fuel oil, or electricity as a
  heating source. Table \#2.1.7 shows the percentage of households
  that use one of these as their principle heating sources
  ("Electricity usage," 2013), ("Fuel oil usage," 2013), ("Gas
  usage," 2013). Create a bar chart and pie chart of this data. State
  any findings you see from the graphs.
\end{enumerate}

\textbf{Table \#2.1.7: Data of Household Heating Sources}

\begin{longtable}[]{@{}ll@{}}
\toprule
Heating Source & Percentage\tabularnewline
\midrule
\endhead
Electricity & 15.3\%\tabularnewline
Fuel Oil & 46.3\%\tabularnewline
Gas & 35.6\%\tabularnewline
Other & 2.8\%\tabularnewline
\bottomrule
\end{longtable}

\begin{enumerate}
\def\labelenumi{\arabic{enumi}.}
\setcounter{enumi}{4}
\tightlist
\item
  Eyeglassomatic manufactures eyeglasses for different retailers. They
  test to see how many defective lenses they made during the time
  period of January 1 to March 31. Table \#2.1.8 gives the defect and
  the number of defects. Create a Pareto chart of the data and then
  describe what this tells you about what causes the most defects.
\end{enumerate}

\textbf{Table \#2.1.8: Data of Defect Type}

\begin{longtable}[]{@{}ll@{}}
\toprule
Defect type & Number of defects\tabularnewline
\midrule
\endhead
Scratch & 5865\tabularnewline
Right shaped -- small & 4613\tabularnewline
Flaked & 1992\tabularnewline
Wrong axis & 1838\tabularnewline
Chamfer wrong & 1596\tabularnewline
Crazing, cracks & 1546\tabularnewline
Wrong shape & 1485\tabularnewline
Wrong PD & 1398\tabularnewline
Spots and bubbles & 1371\tabularnewline
Wrong height & 1130\tabularnewline
Right shape -- big & 1105\tabularnewline
Lost in lab & 976\tabularnewline
Spots/bubble -- intern & 976\tabularnewline
\bottomrule
\end{longtable}

\begin{enumerate}
\def\labelenumi{\arabic{enumi}.}
\setcounter{enumi}{5}
\tightlist
\item
  People in Bangladesh were asked to state what type of birth control
  method they use. The percentages are given in table \#2.1.9
  ("Contraceptive use," 2013). Create a Pareto chart of the data and
  then state any findings you can from the graph.
\end{enumerate}

\textbf{Table \#2.1.9: Data of Birth Control Type}

\begin{longtable}[]{@{}ll@{}}
\toprule
Method & Percentage\tabularnewline
\midrule
\endhead
Condom & 4.50\%\tabularnewline
Pill & 28.50\%\tabularnewline
Periodic Abstinence & 4.90\%\tabularnewline
Injection & 7.00\%\tabularnewline
Female Sterilization & 5.00\%\tabularnewline
IUD & 0.90\%\tabularnewline
Male Sterilization & 0.70\%\tabularnewline
Withdrawal & 2.90\%\tabularnewline
Other Modern Methods & 0.70\%\tabularnewline
Other Traditional Methods & 0.60\%\tabularnewline
\bottomrule
\end{longtable}

\begin{enumerate}
\def\labelenumi{\arabic{enumi}.}
\setcounter{enumi}{6}
\tightlist
\item
  The percentages of people who use certain contraceptives in Central
  American countries are displayed in graph \#2.1.6 ("Contraceptive
  use," 2013). State any findings you can from the graph.
\end{enumerate}

\textbf{Graph \#2.1.6: Multiple Bar Chart for Contraceptive Types}

\includegraphics[width=6in,height=5.45238in]{media/image7.png}

\hypertarget{quantitative-data}{%
\section{Quantitative Data}\label{quantitative-data}}

The graph for quantitative data looks similar to a bar graph, except there are some major differences. First, in a bar graph the categories can be put in any order on the horizontal axis. There is no set order for these data values. You can't say how the data is distributed based on the shape, since the shape can change just by putting the categories in different orders. With quantitative data, the data are in specific orders, since you are dealing with numbers. With quantitative data, you can talk about a distribution, since the shape only changes a little bit depending on how many categories you set up. This is called a \textbf{frequency distribution}.

This leads to the second difference from bar graphs. In a bar graph, the categories that you made in the frequency table were determined by you. In quantitative data, the categories are numerical categories, and the numbers are determined by how many categories (or what are called classes) you choose. If two people have the same number of categories, then they will have the same frequency distribution. Whereas in qualitative data, there can be many different categories depending on the point of view of the author.

The third difference is that the categories touch with quantitative data, and there will be no gaps in the graph. The reason that bar graphs have gaps is to show that the categories do not continue on, like they do in quantitative data. Since the graph for quantitative data is different from qualitative data, it is given a new name. The name of the graph is a \textbf{histogram}. To create a histogram, you must first create the frequency distribution. The idea of a frequency distribution is to take the interval that the data spans and divide it up into equal subintervals called classes.

\textbf{Summary of the steps involved in making a frequency distribution:}

\begin{enumerate}
\def\labelenumi{\arabic{enumi}.}
\item
  Find the range = largest value -- smallest value
\item
  Pick the number of classes to use. Usually the number of classes is between five and twenty. Five classes are used if there are a small number of data points and twenty classes if there are a large number of data points (over 1000 data points). (Note: categories will now be called classes from now on.)
\item
  . Always round up to the next integer (even if the answer is already a whole number go to the next integer). If you don't do this, your last class will not contain your largest data value, and you would have to add another class just for it. If you round up, then your largest data value will fall in the last class, and there are no issues.
\item
  Create the classes. Each class has limits that determine which values fall in each class. To find the class limits, set the smallest value as the lower class limit for the first class. Then add the class width to the lower class limit to get the next lower class limit. Repeat until you get all the classes. The upper class limit for a class is one less than the lower limit for the next class.
\item
  In order for the classes to actually touch, then one class needs to start where the previous one ends. This is known as the class boundary. To find the class boundaries, subtract 0.5 from the lower class limit and add 0.5 to the upper class limit.
\item
  Sometimes it is useful to find the class midpoint. The process is
\item
  To figure out the number of data points that fall in each class, go through each data value and see which class boundaries it is between. Utilizing tally marks may be helpful in counting the data values. The frequency for a class is the number of data values that fall in the class.
\end{enumerate}

Note: the above description is for data values that are whole numbers. If you data value has decimal places, then your class width should be rounded up to the nearest value with the same number of decimal places as the original data. In addition, your class boundaries should have one more decimal place than the original data. As an example, if your data have one decimal place, then the class width would have one decimal place, and the class boundaries are formed by adding and subtracting 0.05 from each class limit.

\textbf{Example \#2.2.1: Creating a Frequency Table}

\begin{quote}
Table \#2.21 contains the amount of rent paid every month for 24
students from a statistics course. Make a relative frequency
distribution using 7 classes.
\end{quote}

\textbf{Table \#2.2.1: Data of Monthly Rent}

\begin{longtable}[]{@{}llllll@{}}
\toprule
1500 & 1350 & 350 & 1200 & 850 & 900\tabularnewline
\midrule
\endhead
1500 & 1150 & 1500 & 900 & 1400 & 1100\tabularnewline
1250 & 600 & 610 & 960 & 890 & 1325\tabularnewline
900 & 800 & 2550 & 495 & 1200 & 690\tabularnewline
\bottomrule
\end{longtable}

\begin{quote}
\textbf{Solution:}

1) Find the range:

2) Pick the number of classes:

The directions say to use 7 classes.

3) Find the class width:

Round up to 315.

Always round up to the next integer even if the width is already an
integer.

4) Find the class limits:

Start at the smallest value. This is the lower class limit for the
first class. Add the width to get the lower limit of the next class.
Keep adding the width to get all the lower limits.

The upper limit is one less than the next lower limit: so for the
first class the upper class limit would be .

When you have all 7 classes, make sure the last number, in this case
the 2550, is at least as large as the largest value in the data. If
not, you made a mistake somewhere.

5) Find the class boundaries:

Subtract 0.5 from the lower class limit to get the class boundaries.
Add 0.5 to the upper class limit for the last class's boundary.

Every value in the data should fall into exactly one of the classes.
No data values should fall right on the boundary of two classes.

6) Find the class midpoints:

7) Tally and find the frequency of the data:

Go through the data and put a tally mark in the appropriate class for
each piece of data by looking to see which class boundaries the data
value is between. Fill in the frequency by changing each of the
tallies into a number.

\textbf{Table \#2.2.2: Frequency Distribution for Monthly Rent}
\end{quote}

\begin{longtable}[]{@{}lllll@{}}
\toprule
\endhead
\begin{minipage}[t]{0.18\columnwidth}\raggedright
~

Class Limits\strut
\end{minipage} & \begin{minipage}[t]{0.23\columnwidth}\raggedright
Class

Boundaries\strut
\end{minipage} & \begin{minipage}[t]{0.13\columnwidth}\raggedright
Class

Midpoint\strut
\end{minipage} & \begin{minipage}[t]{0.10\columnwidth}\raggedright
~

Tally\strut
\end{minipage} & \begin{minipage}[t]{0.14\columnwidth}\raggedright
~

Frequency\strut
\end{minipage}\tabularnewline
\begin{minipage}[t]{0.18\columnwidth}\raggedright
350 -- 664\strut
\end{minipage} & \begin{minipage}[t]{0.23\columnwidth}\raggedright
349.5 -- 664.5\strut
\end{minipage} & \begin{minipage}[t]{0.13\columnwidth}\raggedright
507\strut
\end{minipage} & \begin{minipage}[t]{0.10\columnwidth}\raggedright
~\strut
\end{minipage} & \begin{minipage}[t]{0.14\columnwidth}\raggedright
~4\strut
\end{minipage}\tabularnewline
\begin{minipage}[t]{0.18\columnwidth}\raggedright
665 -- 979\strut
\end{minipage} & \begin{minipage}[t]{0.23\columnwidth}\raggedright
664.5 -- 979.5\strut
\end{minipage} & \begin{minipage}[t]{0.13\columnwidth}\raggedright
822\strut
\end{minipage} & \begin{minipage}[t]{0.10\columnwidth}\raggedright
\strut
\end{minipage} & \begin{minipage}[t]{0.14\columnwidth}\raggedright
8\strut
\end{minipage}\tabularnewline
\begin{minipage}[t]{0.18\columnwidth}\raggedright
980 -- 1294\strut
\end{minipage} & \begin{minipage}[t]{0.23\columnwidth}\raggedright
979.5 -- 1294.5\strut
\end{minipage} & \begin{minipage}[t]{0.13\columnwidth}\raggedright
1137\strut
\end{minipage} & \begin{minipage}[t]{0.10\columnwidth}\raggedright
~\strut
\end{minipage} & \begin{minipage}[t]{0.14\columnwidth}\raggedright
~5\strut
\end{minipage}\tabularnewline
\begin{minipage}[t]{0.18\columnwidth}\raggedright
1295 -- 1609\strut
\end{minipage} & \begin{minipage}[t]{0.23\columnwidth}\raggedright
1294.5 -- 1609.5\strut
\end{minipage} & \begin{minipage}[t]{0.13\columnwidth}\raggedright
1452\strut
\end{minipage} & \begin{minipage}[t]{0.10\columnwidth}\raggedright
~\strut
\end{minipage} & \begin{minipage}[t]{0.14\columnwidth}\raggedright
6\strut
\end{minipage}\tabularnewline
\begin{minipage}[t]{0.18\columnwidth}\raggedright
1610 -- 1924\strut
\end{minipage} & \begin{minipage}[t]{0.23\columnwidth}\raggedright
1609.5 -- 1924.5\strut
\end{minipage} & \begin{minipage}[t]{0.13\columnwidth}\raggedright
1767\strut
\end{minipage} & \begin{minipage}[t]{0.10\columnwidth}\raggedright
~\strut
\end{minipage} & \begin{minipage}[t]{0.14\columnwidth}\raggedright
~0\strut
\end{minipage}\tabularnewline
\begin{minipage}[t]{0.18\columnwidth}\raggedright
1925 -- 2239\strut
\end{minipage} & \begin{minipage}[t]{0.23\columnwidth}\raggedright
1924.5 -- 2239.5\strut
\end{minipage} & \begin{minipage}[t]{0.13\columnwidth}\raggedright
2082\strut
\end{minipage} & \begin{minipage}[t]{0.10\columnwidth}\raggedright
\strut
\end{minipage} & \begin{minipage}[t]{0.14\columnwidth}\raggedright
~0\strut
\end{minipage}\tabularnewline
\begin{minipage}[t]{0.18\columnwidth}\raggedright
2240 -- 2554\strut
\end{minipage} & \begin{minipage}[t]{0.23\columnwidth}\raggedright
2239.5 -- 2554.5\strut
\end{minipage} & \begin{minipage}[t]{0.13\columnwidth}\raggedright
2397\strut
\end{minipage} & \begin{minipage}[t]{0.10\columnwidth}\raggedright
~~\strut
\end{minipage} & \begin{minipage}[t]{0.14\columnwidth}\raggedright
~1\strut
\end{minipage}\tabularnewline
\bottomrule
\end{longtable}

\begin{quote}
Make sure the total of the frequencies is the same as the number of
data points.
\end{quote}

R command for a frequency distribution:

\textbf{To create a frequency distribution:}

summary(variable) -- so you can find out the minimum and maximum.

breaks = seq(min, number above max, by = class width)

breaks -- so you can see the breaks that R made.

variable.cut=cut(variable, breaks, right=FALSE) -- this will cut up the
data into the classes.

variable.freq=table(variable.cut) -- this will create the frequency
table.

variable.freq -- this will display the frequency table.

For the data in Example \#2.2.1, the R command would be:

\begin{quote}
rent\textless{}-c(1500, 1350, 350, 1200, 850, 900, 1500, 1150, 1500, 900, 1400,
1100, 1250, 600, 610, 960, 890, 1325, 900, 800, 2550, 495, 1200, 690)

summary(rent)

Output:

Min. 1st Qu. Median Mean 3rd Qu. Max.

350.0 837.5 1030.0 1082.0 1331.0 2550.0

breaks=seq(350, 3000, by = 315)

breaks

Output:

\[1\] 350 665 980 1295 1610 1925 2240 2555 2870

These are your lower limits of the frequency distribution. You can now
write your own table.

rent.cut=cut(rent, breaks, right=FALSE)

rent.freq=table(rent.cut)

rent.freq

Output:

rent.cut

{[}350,665) {[}665,980) {[}980,1.3e+03) {[}1.3e+03,1.61e+03)

4 8 5 6

{[}1.61e+03,1.92e+03) {[}1.92e+03,2.24e+03) {[}2.24e+03,2.56e+03)
{[}2.56e+03,2.87e+03)

0 0 1 0
\end{quote}

It is difficult to determine the basic shape of the distribution by
looking at the frequency distribution. It would be easier to look at a
graph. The graph of a frequency distribution for quantitative data is
called a \textbf{frequency} \textbf{histogram} or just histogram for short.

\textbf{Histogram}: a graph of the frequencies on the vertical axis and the
class boundaries on the horizontal axis. Rectangles where the height is
the frequency and the width is the class width are draw for each class.

\textbf{Example \#2.2.2: Drawing a Histogram }

Draw a histogram for the distribution from example \#2.2.1.

\begin{quote}
\textbf{Solution:}

The class boundaries are plotted on the horizontal axis and the
frequencies are plotted on the vertical axis. You can plot the
midpoints of the classes instead of the class boundaries. Graph
\#2.2.1 was created using the midpoints because it was easier to do
with the software that created the graph. On R, the command is

hist(variable, col="type in what color you want", breaks,
main="type the title you want", xlab="type the label you want for
the horizontal axis", ylim=c(0, number above maximum frequency) --
produces histogram with specified color and using the breaks you made
for the frequency distribution.

For this example, the command in R would be (assuming you created a
frequency distribution in R as described previously):

hist(rent, col="blue", breaks, right=FALSE, main="Monthly Rent Paid
by Students", ylim=c(0,8) xlab="Monthly Rent (\$)")
\end{quote}

\textbf{Graph \#2.2.1: Histogram for Monthly Rent}

\begin{quote}
\includegraphics[width=4.18056in,height=4.18056in]{media/image22.emf}

If no frequency distribution was created before the histogram, then
the command would be:

hist(variable, col="type in what color you want", number of classes,
main="type the title you want", xlab="type the label you want for
the horizontal axis") -- produces histogram with specified color and
number of classes (though the number of classes is an estimate and R
will create the number of classes near this value).

For this example, the R command without a frequency distribution
created first would be:

hist(rent, col="blue", 7, main="Monthly Rent Paid by Students",
xlab="Monthly Rent (\$)")

Notice the graph has the axes labeled, the tick marks are labeled on
each axis, and there is a title.

Reviewing the graph you can see that most of the students pay around
\$750 per month for rent, with about \$1500 being the other common
value. You can see from the graph, that most students pay between
\$600 and \$1600 per month for rent. Of course, these values are just
estimates from the graph. There is a large gap between the \$1500
class and the highest data value. This seems to say that one student
is paying a great deal more than everyone else. This value could be
considered an outlier. An \textbf{outlier} is a data value that is far from
the rest of the values. It may be an unusual value or a mistake. It is
a data value that should be investigated. In this case, the student
lives in a very expensive part of town, thus the value is not a
mistake, and is just very unusual. There are other aspects that can be
discussed, but first some other concepts need to be introduced.
\end{quote}

Frequencies are helpful, but understanding the relative size each class
is to the total is also useful. To find this you can divide the
frequency by the total to create a relative frequency. If you have the
relative frequencies for all of the classes, then you have a relative
frequency distribution.

\textbf{Relative Frequency Distribution }

A variation on a frequency distribution is a relative frequency
distribution. Instead of giving the frequencies for each class, the
relative frequencies are calculated.

This gives you percentages of data that fall in each class.

\textbf{Example \#2.2.3: Creating a Relative Frequency Table}

\begin{quote}
Find the relative frequency for the grade data.

\textbf{Solution:}

From example \#2.2.1, the frequency distribution is reproduced in
table \#2.2.2.
\end{quote}

\textbf{Table \#2.2.2: Frequency Distribution for Monthly Rent}

\begin{longtable}[]{@{}llll@{}}
\toprule
\endhead
\begin{minipage}[t]{0.18\columnwidth}\raggedright
~

Class Limits\strut
\end{minipage} & \begin{minipage}[t]{0.23\columnwidth}\raggedright
Class

Boundaries\strut
\end{minipage} & \begin{minipage}[t]{0.14\columnwidth}\raggedright
Class

Midpoint\strut
\end{minipage} & \begin{minipage}[t]{0.15\columnwidth}\raggedright
~

Frequency\strut
\end{minipage}\tabularnewline
\begin{minipage}[t]{0.18\columnwidth}\raggedright
350 -- 664\strut
\end{minipage} & \begin{minipage}[t]{0.23\columnwidth}\raggedright
349.5 -- 664.5\strut
\end{minipage} & \begin{minipage}[t]{0.14\columnwidth}\raggedright
507\strut
\end{minipage} & \begin{minipage}[t]{0.15\columnwidth}\raggedright
4\strut
\end{minipage}\tabularnewline
\begin{minipage}[t]{0.18\columnwidth}\raggedright
665 -- 979\strut
\end{minipage} & \begin{minipage}[t]{0.23\columnwidth}\raggedright
664.5 -- 979.5\strut
\end{minipage} & \begin{minipage}[t]{0.14\columnwidth}\raggedright
822\strut
\end{minipage} & \begin{minipage}[t]{0.15\columnwidth}\raggedright
8\strut
\end{minipage}\tabularnewline
\begin{minipage}[t]{0.18\columnwidth}\raggedright
980 -- 1294\strut
\end{minipage} & \begin{minipage}[t]{0.23\columnwidth}\raggedright
979.5 -- 1294.5\strut
\end{minipage} & \begin{minipage}[t]{0.14\columnwidth}\raggedright
1137\strut
\end{minipage} & \begin{minipage}[t]{0.15\columnwidth}\raggedright
5\strut
\end{minipage}\tabularnewline
\begin{minipage}[t]{0.18\columnwidth}\raggedright
1295 -- 1609\strut
\end{minipage} & \begin{minipage}[t]{0.23\columnwidth}\raggedright
1294.5 -- 1609.5\strut
\end{minipage} & \begin{minipage}[t]{0.14\columnwidth}\raggedright
1452\strut
\end{minipage} & \begin{minipage}[t]{0.15\columnwidth}\raggedright
6\strut
\end{minipage}\tabularnewline
\begin{minipage}[t]{0.18\columnwidth}\raggedright
1610 -- 1924\strut
\end{minipage} & \begin{minipage}[t]{0.23\columnwidth}\raggedright
1609.5 -- 1924.5\strut
\end{minipage} & \begin{minipage}[t]{0.14\columnwidth}\raggedright
1767\strut
\end{minipage} & \begin{minipage}[t]{0.15\columnwidth}\raggedright
0\strut
\end{minipage}\tabularnewline
\begin{minipage}[t]{0.18\columnwidth}\raggedright
1925 -- 2239\strut
\end{minipage} & \begin{minipage}[t]{0.23\columnwidth}\raggedright
1924.5 -- 2239.5\strut
\end{minipage} & \begin{minipage}[t]{0.14\columnwidth}\raggedright
2082\strut
\end{minipage} & \begin{minipage}[t]{0.15\columnwidth}\raggedright
0\strut
\end{minipage}\tabularnewline
\begin{minipage}[t]{0.18\columnwidth}\raggedright
2240 -- 2554\strut
\end{minipage} & \begin{minipage}[t]{0.23\columnwidth}\raggedright
2239.5 -- 2554.5\strut
\end{minipage} & \begin{minipage}[t]{0.14\columnwidth}\raggedright
2397\strut
\end{minipage} & \begin{minipage}[t]{0.15\columnwidth}\raggedright
1\strut
\end{minipage}\tabularnewline
\bottomrule
\end{longtable}

\begin{quote}
Divide each frequency by the number of data points.
\end{quote}

\textbf{Table \#2.2.3: Relative Frequency Distribution for Monthly Rent}

\begin{longtable}[]{@{}lllll@{}}
\toprule
\endhead
\begin{minipage}[t]{0.18\columnwidth}\raggedright
~

Class Limits\strut
\end{minipage} & \begin{minipage}[t]{0.23\columnwidth}\raggedright
Class

Boundaries\strut
\end{minipage} & \begin{minipage}[t]{0.13\columnwidth}\raggedright
Class

Midpoint\strut
\end{minipage} & \begin{minipage}[t]{0.14\columnwidth}\raggedright
~

Frequency\strut
\end{minipage} & \begin{minipage}[t]{0.14\columnwidth}\raggedright
Relative~

Frequency\strut
\end{minipage}\tabularnewline
\begin{minipage}[t]{0.18\columnwidth}\raggedright
350 -- 664\strut
\end{minipage} & \begin{minipage}[t]{0.23\columnwidth}\raggedright
349.5 -- 664.5\strut
\end{minipage} & \begin{minipage}[t]{0.13\columnwidth}\raggedright
507\strut
\end{minipage} & \begin{minipage}[t]{0.14\columnwidth}\raggedright
4\strut
\end{minipage} & \begin{minipage}[t]{0.14\columnwidth}\raggedright
~0.17\strut
\end{minipage}\tabularnewline
\begin{minipage}[t]{0.18\columnwidth}\raggedright
665 -- 979\strut
\end{minipage} & \begin{minipage}[t]{0.23\columnwidth}\raggedright
664.5 -- 979.5\strut
\end{minipage} & \begin{minipage}[t]{0.13\columnwidth}\raggedright
822\strut
\end{minipage} & \begin{minipage}[t]{0.14\columnwidth}\raggedright
8\strut
\end{minipage} & \begin{minipage}[t]{0.14\columnwidth}\raggedright
0.33\strut
\end{minipage}\tabularnewline
\begin{minipage}[t]{0.18\columnwidth}\raggedright
980 -- 1294\strut
\end{minipage} & \begin{minipage}[t]{0.23\columnwidth}\raggedright
979.5 -- 1294.5\strut
\end{minipage} & \begin{minipage}[t]{0.13\columnwidth}\raggedright
1137\strut
\end{minipage} & \begin{minipage}[t]{0.14\columnwidth}\raggedright
5\strut
\end{minipage} & \begin{minipage}[t]{0.14\columnwidth}\raggedright
~0.21\strut
\end{minipage}\tabularnewline
\begin{minipage}[t]{0.18\columnwidth}\raggedright
1295 -- 1609\strut
\end{minipage} & \begin{minipage}[t]{0.23\columnwidth}\raggedright
1294.5 -- 1609.5\strut
\end{minipage} & \begin{minipage}[t]{0.13\columnwidth}\raggedright
1452\strut
\end{minipage} & \begin{minipage}[t]{0.14\columnwidth}\raggedright
6\strut
\end{minipage} & \begin{minipage}[t]{0.14\columnwidth}\raggedright
0.25\strut
\end{minipage}\tabularnewline
\begin{minipage}[t]{0.18\columnwidth}\raggedright
1610 -- 1924\strut
\end{minipage} & \begin{minipage}[t]{0.23\columnwidth}\raggedright
1609.5 -- 1924.5\strut
\end{minipage} & \begin{minipage}[t]{0.13\columnwidth}\raggedright
1767\strut
\end{minipage} & \begin{minipage}[t]{0.14\columnwidth}\raggedright
0\strut
\end{minipage} & \begin{minipage}[t]{0.14\columnwidth}\raggedright
0\strut
\end{minipage}\tabularnewline
\begin{minipage}[t]{0.18\columnwidth}\raggedright
1925 -- 2239\strut
\end{minipage} & \begin{minipage}[t]{0.23\columnwidth}\raggedright
1924.5 -- 2239.5\strut
\end{minipage} & \begin{minipage}[t]{0.13\columnwidth}\raggedright
2082\strut
\end{minipage} & \begin{minipage}[t]{0.14\columnwidth}\raggedright
0\strut
\end{minipage} & \begin{minipage}[t]{0.14\columnwidth}\raggedright
0\strut
\end{minipage}\tabularnewline
\begin{minipage}[t]{0.18\columnwidth}\raggedright
2240 -- 2554\strut
\end{minipage} & \begin{minipage}[t]{0.23\columnwidth}\raggedright
2239.5 -- 2554.5\strut
\end{minipage} & \begin{minipage}[t]{0.13\columnwidth}\raggedright
2397\strut
\end{minipage} & \begin{minipage}[t]{0.14\columnwidth}\raggedright
1\strut
\end{minipage} & \begin{minipage}[t]{0.14\columnwidth}\raggedright
~0.04\strut
\end{minipage}\tabularnewline
\begin{minipage}[t]{0.18\columnwidth}\raggedright
Total\strut
\end{minipage} & \begin{minipage}[t]{0.23\columnwidth}\raggedright
\strut
\end{minipage} & \begin{minipage}[t]{0.13\columnwidth}\raggedright
\strut
\end{minipage} & \begin{minipage}[t]{0.14\columnwidth}\raggedright
24\strut
\end{minipage} & \begin{minipage}[t]{0.14\columnwidth}\raggedright
1\strut
\end{minipage}\tabularnewline
\bottomrule
\end{longtable}

\begin{quote}
The relative frequencies should add up to 1 or 100\%. (This might be
off a little due to rounding errors.)
\end{quote}

The graph of the relative frequency is known as a relative frequency
histogram. It looks identical to the frequency histogram, but the
vertical axis is relative frequency instead of just frequencies.

\textbf{Example \#2.2.4: Drawing a Relative Frequency Histogram}

\begin{quote}
Draw a relative frequency histogram for the grade distribution from
example \#2.2.1.

\textbf{Solution:}

The class boundaries are plotted on the horizontal axis and the
relative frequencies are plotted on the vertical axis. (This is not
easy to do in R, so use another technology to graph a relative
frequency histogram.)
\end{quote}

\textbf{Graph \#2.2.2: Relative Frequency Histogram for Monthly Rent}

\begin{quote}
\includegraphics[width=3.26389in,height=2.32099in]{media/image25.png}

Notice the shape is the same as the frequency distribution.
\end{quote}

Another useful piece of information is how many data points fall below a
particular class boundary. As an example, a teacher may want to know how
many students received below an 80\%, a doctor may want to know how many
adults have cholesterol below 160, or a manager may want to know how
many stores gross less than \$2000 per day. This is known as a
\textbf{cumulative frequency}. If you want to know what percent of the data
falls below a certain class boundary, then this would be a \textbf{cumulative
relative frequency}. For cumulative frequencies you are finding how
many data values fall below the upper class limit.

To create a \textbf{cumulative frequency distribution}, count the number of
data points that are below the upper class boundary, starting with the
first class and working up to the top class. The last upper class
boundary should have all of the data points below it. Also include the
number of data points below the lowest class boundary, which is zero.

\textbf{Example \#2.2.5: Creating a Cumulative Frequency Distribution}

Create a cumulative frequency distribution for the data in example
\#2.2.1.

\begin{quote}
\textbf{Solution:}
\end{quote}

The frequency distribution for the data is in table \#2.2.2.

\textbf{Table \#2.2.2: Frequency Distribution for Monthly Rent}

\begin{longtable}[]{@{}llll@{}}
\toprule
\endhead
\begin{minipage}[t]{0.18\columnwidth}\raggedright
~

Class Limits\strut
\end{minipage} & \begin{minipage}[t]{0.23\columnwidth}\raggedright
Class

Boundaries\strut
\end{minipage} & \begin{minipage}[t]{0.14\columnwidth}\raggedright
Class

Midpoint\strut
\end{minipage} & \begin{minipage}[t]{0.15\columnwidth}\raggedright
~

Frequency\strut
\end{minipage}\tabularnewline
\begin{minipage}[t]{0.18\columnwidth}\raggedright
350 -- 664\strut
\end{minipage} & \begin{minipage}[t]{0.23\columnwidth}\raggedright
349.5 -- 664.5\strut
\end{minipage} & \begin{minipage}[t]{0.14\columnwidth}\raggedright
507\strut
\end{minipage} & \begin{minipage}[t]{0.15\columnwidth}\raggedright
4\strut
\end{minipage}\tabularnewline
\begin{minipage}[t]{0.18\columnwidth}\raggedright
665 -- 979\strut
\end{minipage} & \begin{minipage}[t]{0.23\columnwidth}\raggedright
664.5 -- 979.5\strut
\end{minipage} & \begin{minipage}[t]{0.14\columnwidth}\raggedright
822\strut
\end{minipage} & \begin{minipage}[t]{0.15\columnwidth}\raggedright
8\strut
\end{minipage}\tabularnewline
\begin{minipage}[t]{0.18\columnwidth}\raggedright
980 -- 1294\strut
\end{minipage} & \begin{minipage}[t]{0.23\columnwidth}\raggedright
979.5 -- 1294.5\strut
\end{minipage} & \begin{minipage}[t]{0.14\columnwidth}\raggedright
1137\strut
\end{minipage} & \begin{minipage}[t]{0.15\columnwidth}\raggedright
5\strut
\end{minipage}\tabularnewline
\begin{minipage}[t]{0.18\columnwidth}\raggedright
1295 -- 1609\strut
\end{minipage} & \begin{minipage}[t]{0.23\columnwidth}\raggedright
1294.5 -- 1609.5\strut
\end{minipage} & \begin{minipage}[t]{0.14\columnwidth}\raggedright
1452\strut
\end{minipage} & \begin{minipage}[t]{0.15\columnwidth}\raggedright
6\strut
\end{minipage}\tabularnewline
\begin{minipage}[t]{0.18\columnwidth}\raggedright
1610 -- 1924\strut
\end{minipage} & \begin{minipage}[t]{0.23\columnwidth}\raggedright
1609.5 -- 1924.5\strut
\end{minipage} & \begin{minipage}[t]{0.14\columnwidth}\raggedright
1767\strut
\end{minipage} & \begin{minipage}[t]{0.15\columnwidth}\raggedright
0\strut
\end{minipage}\tabularnewline
\begin{minipage}[t]{0.18\columnwidth}\raggedright
1925 -- 2239\strut
\end{minipage} & \begin{minipage}[t]{0.23\columnwidth}\raggedright
1924.5 -- 2239.5\strut
\end{minipage} & \begin{minipage}[t]{0.14\columnwidth}\raggedright
2082\strut
\end{minipage} & \begin{minipage}[t]{0.15\columnwidth}\raggedright
0\strut
\end{minipage}\tabularnewline
\begin{minipage}[t]{0.18\columnwidth}\raggedright
2240 -- 2554\strut
\end{minipage} & \begin{minipage}[t]{0.23\columnwidth}\raggedright
2239.5 -- 2554.5\strut
\end{minipage} & \begin{minipage}[t]{0.14\columnwidth}\raggedright
2397\strut
\end{minipage} & \begin{minipage}[t]{0.15\columnwidth}\raggedright
1\strut
\end{minipage}\tabularnewline
\bottomrule
\end{longtable}

\begin{quote}
Now ask yourself how many data points fall below each class boundary.
Below 349.5, there are 0 data points. Below 664.5 there are 4 data
points, below 979.5, there are 4 + 8 = 12 data points, below 1294.5
there are 4 + 8 + 5 = 17 data points, and continue this process until
you reach the upper class boundary. This is summarized in Table
\#2.2.4.

To produce cumulative frequencies in R, you need to have performed the
commands for the frequency distribution. Once you have complete that,
then use variable.cumfreq=cumsum(variable.freq) -- creates the
cumulative frequencies for the variable

cumfreq0=c(0,variable.cumfreq) -- creates a cumulative frequency table
for the variable.

cumfreq0 -- displays the cumulative frequency table.

For this example the command would be:

rent.cumfreq=cumsum(rent.freq)

cumfreq0=c(0,rent.cumfreq)

cumfreq0

Output:

{[}350,665) {[}665,980) {[}980,1.3e+03)

0 4 12 17

{[}1.3e+03,1.61e+03) {[}1.61e+03,1.92e+03) {[}1.92e+03,2.24e+03)
{[}2.24e+03,2.56e+03)

23 23 23 24

{[}2.56e+03,2.87e+03)

24

Now type this into a table. See Table \#2.2.4.
\end{quote}

\textbf{Table \#2.2.4: Cumulative Distribution for Monthly Rent}

\begin{longtable}[]{@{}lllll@{}}
\toprule
\endhead
\begin{minipage}[t]{0.18\columnwidth}\raggedright
~

Class Limits\strut
\end{minipage} & \begin{minipage}[t]{0.23\columnwidth}\raggedright
Class

Boundaries\strut
\end{minipage} & \begin{minipage}[t]{0.13\columnwidth}\raggedright
Class

Midpoint\strut
\end{minipage} & \begin{minipage}[t]{0.14\columnwidth}\raggedright
~

Frequency\strut
\end{minipage} & \begin{minipage}[t]{0.17\columnwidth}\raggedright
~Cumulative

Frequency\strut
\end{minipage}\tabularnewline
\begin{minipage}[t]{0.18\columnwidth}\raggedright
350 -- 664\strut
\end{minipage} & \begin{minipage}[t]{0.23\columnwidth}\raggedright
349.5 -- 664.5\strut
\end{minipage} & \begin{minipage}[t]{0.13\columnwidth}\raggedright
~507\strut
\end{minipage} & \begin{minipage}[t]{0.14\columnwidth}\raggedright
4\strut
\end{minipage} & \begin{minipage}[t]{0.17\columnwidth}\raggedright
~ 4\strut
\end{minipage}\tabularnewline
\begin{minipage}[t]{0.18\columnwidth}\raggedright
665 -- 979\strut
\end{minipage} & \begin{minipage}[t]{0.23\columnwidth}\raggedright
664.5 -- 979.5\strut
\end{minipage} & \begin{minipage}[t]{0.13\columnwidth}\raggedright
~822\strut
\end{minipage} & \begin{minipage}[t]{0.14\columnwidth}\raggedright
8\strut
\end{minipage} & \begin{minipage}[t]{0.17\columnwidth}\raggedright
12\strut
\end{minipage}\tabularnewline
\begin{minipage}[t]{0.18\columnwidth}\raggedright
980 -- 1294\strut
\end{minipage} & \begin{minipage}[t]{0.23\columnwidth}\raggedright
979.5 -- 1294.5\strut
\end{minipage} & \begin{minipage}[t]{0.13\columnwidth}\raggedright
~1137\strut
\end{minipage} & \begin{minipage}[t]{0.14\columnwidth}\raggedright
5\strut
\end{minipage} & \begin{minipage}[t]{0.17\columnwidth}\raggedright
~17\strut
\end{minipage}\tabularnewline
\begin{minipage}[t]{0.18\columnwidth}\raggedright
1295 -- 1609\strut
\end{minipage} & \begin{minipage}[t]{0.23\columnwidth}\raggedright
1294.5 -- 1609.5\strut
\end{minipage} & \begin{minipage}[t]{0.13\columnwidth}\raggedright
~1452\strut
\end{minipage} & \begin{minipage}[t]{0.14\columnwidth}\raggedright
6\strut
\end{minipage} & \begin{minipage}[t]{0.17\columnwidth}\raggedright
23\strut
\end{minipage}\tabularnewline
\begin{minipage}[t]{0.18\columnwidth}\raggedright
1610 -- 1924\strut
\end{minipage} & \begin{minipage}[t]{0.23\columnwidth}\raggedright
1609.5 -- 1924.5\strut
\end{minipage} & \begin{minipage}[t]{0.13\columnwidth}\raggedright
~1767\strut
\end{minipage} & \begin{minipage}[t]{0.14\columnwidth}\raggedright
0\strut
\end{minipage} & \begin{minipage}[t]{0.17\columnwidth}\raggedright
~23\strut
\end{minipage}\tabularnewline
\begin{minipage}[t]{0.18\columnwidth}\raggedright
1925 -- 2239\strut
\end{minipage} & \begin{minipage}[t]{0.23\columnwidth}\raggedright
1924.5 -- 2239.5\strut
\end{minipage} & \begin{minipage}[t]{0.13\columnwidth}\raggedright
~2082\strut
\end{minipage} & \begin{minipage}[t]{0.14\columnwidth}\raggedright
0\strut
\end{minipage} & \begin{minipage}[t]{0.17\columnwidth}\raggedright
~23\strut
\end{minipage}\tabularnewline
\begin{minipage}[t]{0.18\columnwidth}\raggedright
2240 -- 2554\strut
\end{minipage} & \begin{minipage}[t]{0.23\columnwidth}\raggedright
2239.5 -- 2554.5\strut
\end{minipage} & \begin{minipage}[t]{0.13\columnwidth}\raggedright
2397\strut
\end{minipage} & \begin{minipage}[t]{0.14\columnwidth}\raggedright
1\strut
\end{minipage} & \begin{minipage}[t]{0.17\columnwidth}\raggedright
~24\strut
\end{minipage}\tabularnewline
\bottomrule
\end{longtable}

Again, it is hard to look at the data the way it is. A graph would be
useful. The graph for cumulative frequency is called an \textbf{ogive}
(o-jive). To create an ogive, first create a scale on both the
horizontal and vertical axes that will fit the data. Then plot the
points of the class upper class boundary versus the cumulative
frequency. Make sure you include the point with the lowest class
boundary and the 0 cumulative frequency. Then just connect the dots.

\textbf{Example \#2.2.6: Drawing an Ogive}

Draw an ogive for the data in example \#2.2.1.

\begin{quote}
\textbf{Solution:}

In R, the commands would be:

plot(breaks,cumfreq0, main="title you want to use", xlab="label you
want to use", ylab="label you want to use", ylim=c(0, number above
maximum cumulative frequency) -- plots the ogive

lines(breaks,cumfreq0) -- connects the dots on the ogive

For this example, the commands would be:

Plot(breaks,cumfreq0, main=``Cumulative Frequency for Monthly Rent'',
xlab=``Monthly Rent (\$)'', ylab=``Cumulative Frequency'', ylim=c(0,25))

lines(breaks,cumfreq0)
\end{quote}

\textbf{Graph \#2.2.3: Ogive for Monthly Rent}

\begin{quote}
\includegraphics[width=4.66667in,height=4.66667in]{media/image26.emf}
\end{quote}

The usefulness of a ogive is to allow the reader to find out how many
students pay less than a certain value, and also what amount of monthly
rent is paid by a certain number of students. As an example, suppose you
want to know how many students pay less than \$1500 a month in rent,
then you can go up from the \$1500 until you hit the graph and then you
go over to the cumulative frequency axes to see what value corresponds
to this value. It appears that around 20 students pay less than \$1500.
(See graph \#2.2.4.)

\textbf{\\
}

\begin{quote}
\textbf{Graph \#2.2.4: Ogive for Monthly Rent with Example}
\end{quote}

\includegraphics[width=3.625in,height=3.625in]{media/image27.emf}

Also, if you want to know the amount that 15 students pay less than,
then you start at 15 on the vertical axis and then go over to the graph
and down to the horizontal axis where the line intersects the graph. You
can see that 15 students pay less than about \$1200 a month. (See graph
\#2.2.5.)

\textbf{Graph \#2.2.5: Ogive for Monthly Rent with Example}

\includegraphics[width=3.65278in,height=3.65278in]{media/image28.emf}

If you graph the cumulative relative frequency then you can find out
what percentage is below a certain number instead of just the number of
people below a certain value.

Shapes of the distribution:

When you look at a distribution, look at the basic shape. There are some
basic shapes that are seen in histograms. Realize though that some
distributions have no shape. The common shapes are symmetric, skewed,
and uniform. Another interest is how many peaks a graph may have. This
is known as modal.

Symmetric means that you can fold the graph in half down the middle and
the two sides will line up. You can think of the two sides as being
mirror images of each other. Skewed means one ``tail'' of the graph is
longer than the other. The graph is skewed in the direction of the
longer tail (backwards from what you would expect). A uniform graph has
all the bars the same height.

Modal refers to the number of peaks. Unimodal has one peak and bimodal
has two peaks. Usually if a graph has more than two peaks, the modal
information is not longer of interest.

Other important features to consider are gaps between bars, a repetitive
pattern, how spread out is the data, and where the center of the graph
is.

\textbf{Examples of graphs:}

This graph is roughly symmetric and unimodal:

\textbf{Graph \#2.2.6: Symmetric, Unimodal Graph}

\begin{quote}
\includegraphics[width=5.01389in,height=3.01389in]{media/image29.png}
\end{quote}

This graph is symmetric and bimodal:

\textbf{Graph \#2.2.7: Symmetric, Bimodal Graph}

\includegraphics[width=5.04167in,height=3.43056in]{media/image30.png}

This graph is skewed to the right:

\textbf{Graph \#2.2.8: Skewed Right Graph}

\begin{quote}
\includegraphics[width=5.01389in,height=3.01389in]{media/image31.png}

This graph is skewed to the left and has a gap:
\end{quote}

\textbf{Graph \#2.2.9: Skewed Left Graph}

\begin{quote}
\includegraphics[width=5.01389in,height=3.01389in]{media/image32.png}
\end{quote}

This graph is uniform since all the bars are the same height:

\textbf{Graph \#2.2.10: Uniform Graph}

\begin{quote}
\includegraphics[width=5.01389in,height=3.01389in]{media/image33.png}
\end{quote}

\textbf{Example \#2.2.7: Creating a Frequency Distribution, Histogram, and
Ogive}

\begin{quote}
The following data represents the percent change in tuition levels at
public, four-year colleges (inflation adjusted) from 2008 to 2013
(Weissmann, 2013). Create a frequency distribution, histogram, and
ogive for the data.
\end{quote}

\textbf{Table \#2.2.5: Data of Tuition Levels at Public, Four-Year Colleges}

\begin{longtable}[]{@{}llllllll@{}}
\toprule
19.5\% & 40.8\% & 57.0\% & 15.1\% & 17.4\% & 5.2\% & 13.0\% & 15.6\%\tabularnewline
\midrule
\endhead
51.5\% & 15.6\% & 14.5\% & 22.4\% & 19.5\% & 31.3\% & 21.7\% & 27.0\%\tabularnewline
13.1\% & 26.8\% & 24.3\% & 38.0\% & 21.1\% & 9.3\% & 46.7\% & 14.5\%\tabularnewline
78.4\% & 67.3\% & 21.1\% & 22.4\% & 5.3\% & 17.3\% & 17.5\% & 36.6\%\tabularnewline
72.0\% & 63.2\% & 15.1\% & 2.2\% & 17.5\% & 36.7\% & 2.8\% & 16.2\%\tabularnewline
20.5\% & 17.8\% & 30.1\% & 63.6\% & 17.8\% & 23.2\% & 25.3\% & 21.4\%\tabularnewline
28.5\% & 9.4\% & & & & & &\tabularnewline
\bottomrule
\end{longtable}

\textbf{Solution:}

\begin{quote}
1) Find the range:

2) Pick the number of classes:

Since there are 50 data points, then around 6 to 8 classes should be
used. Let's use 8.

3) Find the class width:

Since the data has one decimal place, then the class width should
round to one decimal place. Make sure you round up.

width = 9.6\%

4) Find the class limits:

5) Find the class boundaries:

Since the data has one decimal place, the class boundaries should have
two decimal places, so subtract 0.05 from the lower class limit to get
the class boundaries. Add 0.05 to the upper class limit for the last
class's boundary.

Every value in the data should fall into exactly one of the classes.
No data values should fall right on the boundary of two classes.

6) Find the class midpoints:

7) Tally and find the frequency of the data:

\textbf{Table \#2.2.6: Frequency Distribution for Tuition Levels at Public,
Four-Year Colleges}
\end{quote}

\begin{longtable}[]{@{}lllllll@{}}
\toprule
\endhead
\begin{minipage}[t]{0.11\columnwidth}\raggedright
Class
Limits\strut
\end{minipage} & \begin{minipage}[t]{0.11\columnwidth}\raggedright
Class

Boundar
ies\strut
\end{minipage} & \begin{minipage}[t]{0.11\columnwidth}\raggedright
Class

Midpoin
t\strut
\end{minipage} & \begin{minipage}[t]{0.11\columnwidth}\raggedright
Tally\strut
\end{minipage} & \begin{minipage}[t]{0.11\columnwidth}\raggedright
Frequen
cy\strut
\end{minipage} & \begin{minipage}[t]{0.11\columnwidth}\raggedright
Relativ
e
Frequen
cy\strut
\end{minipage} & \begin{minipage}[t]{0.11\columnwidth}\raggedright
Cumulat
ive
Frequen
cy\strut
\end{minipage}\tabularnewline
\begin{minipage}[t]{0.11\columnwidth}\raggedright
2.2 --
11.7\strut
\end{minipage} & \begin{minipage}[t]{0.11\columnwidth}\raggedright
2.15 --
11.75\strut
\end{minipage} & \begin{minipage}[t]{0.11\columnwidth}\raggedright
6.95\strut
\end{minipage} & \begin{minipage}[t]{0.11\columnwidth}\raggedright
~\strut
\end{minipage} & \begin{minipage}[t]{0.11\columnwidth}\raggedright
6\strut
\end{minipage} & \begin{minipage}[t]{0.11\columnwidth}\raggedright
0.12\strut
\end{minipage} & \begin{minipage}[t]{0.11\columnwidth}\raggedright
6\strut
\end{minipage}\tabularnewline
\begin{minipage}[t]{0.11\columnwidth}\raggedright
11.8 --
21.3\strut
\end{minipage} & \begin{minipage}[t]{0.11\columnwidth}\raggedright
\hypertarget{section}{%
\section{11.75}\label{section}}

21.35\strut
\end{minipage} & \begin{minipage}[t]{0.11\columnwidth}\raggedright
16.55\strut
\end{minipage} & \begin{minipage}[t]{0.11\columnwidth}\raggedright
\strut
\end{minipage} & \begin{minipage}[t]{0.11\columnwidth}\raggedright
20\strut
\end{minipage} & \begin{minipage}[t]{0.11\columnwidth}\raggedright
0.40\strut
\end{minipage} & \begin{minipage}[t]{0.11\columnwidth}\raggedright
26\strut
\end{minipage}\tabularnewline
\begin{minipage}[t]{0.11\columnwidth}\raggedright
21.4 --
30.9\strut
\end{minipage} & \begin{minipage}[t]{0.11\columnwidth}\raggedright
\hypertarget{section-1}{%
\section{21.35}\label{section-1}}

30.95\strut
\end{minipage} & \begin{minipage}[t]{0.11\columnwidth}\raggedright
26.15\strut
\end{minipage} & \begin{minipage}[t]{0.11\columnwidth}\raggedright
~\strut
\end{minipage} & \begin{minipage}[t]{0.11\columnwidth}\raggedright
11\strut
\end{minipage} & \begin{minipage}[t]{0.11\columnwidth}\raggedright
0.22\strut
\end{minipage} & \begin{minipage}[t]{0.11\columnwidth}\raggedright
37\strut
\end{minipage}\tabularnewline
\begin{minipage}[t]{0.11\columnwidth}\raggedright
31.0 --
40.5\strut
\end{minipage} & \begin{minipage}[t]{0.11\columnwidth}\raggedright
\hypertarget{section-2}{%
\section{30.95}\label{section-2}}

40.55\strut
\end{minipage} & \begin{minipage}[t]{0.11\columnwidth}\raggedright
35.75\strut
\end{minipage} & \begin{minipage}[t]{0.11\columnwidth}\raggedright
~\strut
\end{minipage} & \begin{minipage}[t]{0.11\columnwidth}\raggedright
4\strut
\end{minipage} & \begin{minipage}[t]{0.11\columnwidth}\raggedright
0.08\strut
\end{minipage} & \begin{minipage}[t]{0.11\columnwidth}\raggedright
41\strut
\end{minipage}\tabularnewline
\begin{minipage}[t]{0.11\columnwidth}\raggedright
40.6 --
50.1\strut
\end{minipage} & \begin{minipage}[t]{0.11\columnwidth}\raggedright
\hypertarget{section-3}{%
\section{40.55}\label{section-3}}

50.15\strut
\end{minipage} & \begin{minipage}[t]{0.11\columnwidth}\raggedright
45.35\strut
\end{minipage} & \begin{minipage}[t]{0.11\columnwidth}\raggedright
\strut
\end{minipage} & \begin{minipage}[t]{0.11\columnwidth}\raggedright
2\strut
\end{minipage} & \begin{minipage}[t]{0.11\columnwidth}\raggedright
0.04\strut
\end{minipage} & \begin{minipage}[t]{0.11\columnwidth}\raggedright
43\strut
\end{minipage}\tabularnewline
\begin{minipage}[t]{0.11\columnwidth}\raggedright
50.2 --
59.7\strut
\end{minipage} & \begin{minipage}[t]{0.11\columnwidth}\raggedright
\hypertarget{section-4}{%
\section{50.15}\label{section-4}}

59.75\strut
\end{minipage} & \begin{minipage}[t]{0.11\columnwidth}\raggedright
54.95\strut
\end{minipage} & \begin{minipage}[t]{0.11\columnwidth}\raggedright
\strut
\end{minipage} & \begin{minipage}[t]{0.11\columnwidth}\raggedright
2\strut
\end{minipage} & \begin{minipage}[t]{0.11\columnwidth}\raggedright
0.04\strut
\end{minipage} & \begin{minipage}[t]{0.11\columnwidth}\raggedright
45\strut
\end{minipage}\tabularnewline
\begin{minipage}[t]{0.11\columnwidth}\raggedright
59.8 --
69.3\strut
\end{minipage} & \begin{minipage}[t]{0.11\columnwidth}\raggedright
\hypertarget{section-5}{%
\section{59.75}\label{section-5}}

69.35\strut
\end{minipage} & \begin{minipage}[t]{0.11\columnwidth}\raggedright
64.55\strut
\end{minipage} & \begin{minipage}[t]{0.11\columnwidth}\raggedright
~\strut
\end{minipage} & \begin{minipage}[t]{0.11\columnwidth}\raggedright
3\strut
\end{minipage} & \begin{minipage}[t]{0.11\columnwidth}\raggedright
0.06\strut
\end{minipage} & \begin{minipage}[t]{0.11\columnwidth}\raggedright
48\strut
\end{minipage}\tabularnewline
\begin{minipage}[t]{0.11\columnwidth}\raggedright
69.4 --
78.9\strut
\end{minipage} & \begin{minipage}[t]{0.11\columnwidth}\raggedright
\hypertarget{section-6}{%
\section{69.35}\label{section-6}}

78.95\strut
\end{minipage} & \begin{minipage}[t]{0.11\columnwidth}\raggedright
74.15\strut
\end{minipage} & \begin{minipage}[t]{0.11\columnwidth}\raggedright
\strut
\end{minipage} & \begin{minipage}[t]{0.11\columnwidth}\raggedright
2\strut
\end{minipage} & \begin{minipage}[t]{0.11\columnwidth}\raggedright
0.04\strut
\end{minipage} & \begin{minipage}[t]{0.11\columnwidth}\raggedright
50\strut
\end{minipage}\tabularnewline
\bottomrule
\end{longtable}

\begin{quote}
Make sure the total of the frequencies is the same as the number of
data points.
\end{quote}

\textbf{\\
}

\begin{quote}
\textbf{Graph \#2.2.11: Histogram for Tuition Levels at Public, Four-Year
Colleges}

\includegraphics[width=3.5in,height=3.5in]{media/image48.emf}

This graph is skewed right, with no gaps. This says that most percent
increases in tuition were around 16.55\%, with very few states having a
percent increase greater than 45.35\%.

\textbf{Graph \#2.2.11: Ogive for Tuition Levels at Public, Four-Year
Colleges}

\includegraphics[width=3.76389in,height=3.76389in]{media/image49.emf}

Looking at the ogive, you can see that 30 states had a percent change
in tuition levels of about 25\% or less.
\end{quote}

There are occasions where the class limits in the frequency distribution
are predetermined. Example \#2.2.8 demonstrates this situation.

\textbf{Example \#2.2.8: Creating a Frequency Distribution and Histogram}

\begin{quote}
The following are the percentage grades of 25 students from a
statistics course. Make a frequency distribution and histogram.

\textbf{Table \#2.2.7: Data of Test Grades}
\end{quote}

\begin{longtable}[]{@{}llllllllll@{}}
\toprule
62 & 87 & 81 & 69 & 87 & 62 & 45 & 95 & 76 & 76\tabularnewline
\midrule
\endhead
62 & 71 & 65 & 67 & 72 & 80 & 40 & 77 & 87 & 58\tabularnewline
84 & 73 & 93 & 64 & 89 & & & & &\tabularnewline
\bottomrule
\end{longtable}

\textbf{Solution:}

\begin{quote}
Since this data is percent grades, it makes more sense to make the
classes in multiples of 10, since grades are usually 90 to 100\%, 80 to
90\%, and so forth. It is easier to not use the class boundaries, but
instead use the class limits and think of the upper class limit being
up to but not including the next classes lower limit. As an example
the class 80 -- 90 means a grade of 80\% up to but not including a 90\%.
A student with an 89.9\% would be in the 80-90 class.

\textbf{Table \#2.2.8: Frequency Distribution for Test Grades}
\end{quote}

\begin{longtable}[]{@{}llll@{}}
\toprule
Class Limit & Class Midpoint & Tally & Frequency\tabularnewline
\midrule
\endhead
40 -- 50 & 45 & & 2\tabularnewline
50 -- 60 & 55 & & 1\tabularnewline
60 -- 70 & 65 & & 7\tabularnewline
70 -- 80 & 75 & & 6\tabularnewline
80 -- 90 & 85 & & 7\tabularnewline
90 -- 100 & 95 & & 2\tabularnewline
\bottomrule
\end{longtable}

\begin{quote}
\textbf{Graph \#2.2.12: Histogram for Test Grades}

\includegraphics[width=4.01389in,height=4.01389in]{media/image56.emf}

It appears that most of the students had between 60 to 90\%. This graph
looks somewhat symmetric and also bimodal. The same number of students
earned between 60 to 70\% and 80 to 90\%.
\end{quote}

There are other types of graphs for quantitative data. They will be
explored in the next section.

\hypertarget{homework-5}{%
\subsection{Homework}\label{homework-5}}

\begin{enumerate}
\def\labelenumi{\arabic{enumi}.}
\tightlist
\item
  The median incomes of males in each state of the United States,
  including the District of Columbia and Puerto Rico, are given in
  table \#2.2.9 ("Median income of," 2013). Create a frequency
  distribution, relative frequency distribution, and cumulative
  frequency distribution using 7 classes.
\end{enumerate}

\begin{quote}
\textbf{Table \#2.2.9: Data of Median Income for Males}
\end{quote}

\begin{longtable}[]{@{}lllllll@{}}
\toprule
\$42,951 & \$52,379 & \$42,544 & \$37,488 & \$49,281 & \$50,987 & \$60,705\tabularnewline
\midrule
\endhead
\$50,411 & \$66,760 & \$40,951 & \$43,902 & \$45,494 & \$41,528 & \$50,746\tabularnewline
\$45,183 & \$43,624 & \$43,993 & \$41,612 & \$46,313 & \$43,944 & \$56,708\tabularnewline
\$60,264 & \$50,053 & \$50,580 & \$40,202 & \$43,146 & \$41,635 & \$42,182\tabularnewline
\$41,803 & \$53,033 & \$60,568 & \$41,037 & \$50,388 & \$41,950 & \$44,660\tabularnewline
\$46,176 & \$41,420 & \$45,976 & \$47,956 & \$22,529 & \$48,842 & \$41,464\tabularnewline
\$40,285 & \$41,309 & \$43,160 & \$47,573 & \$44,057 & \$52,805 & \$53,046\tabularnewline
\$42,125 & \$46,214 & \$51,630 & & & &\tabularnewline
\bottomrule
\end{longtable}

\begin{enumerate}
\def\labelenumi{\arabic{enumi}.}
\setcounter{enumi}{1}
\tightlist
\item
  The median incomes of females in each state of the United States,
  including the District of Columbia and Puerto Rico, are given in
  table \#2.2.10 ("Median income of," 2013). Create a frequency
  distribution, relative frequency distribution, and cumulative
  frequency distribution using 7 classes.
\end{enumerate}

\begin{quote}
\textbf{Table \#2.2.10: Data of Median Income for Females}
\end{quote}

\begin{longtable}[]{@{}llllllll@{}}
\toprule
\$31,862 & \$40,550 & \$36,048 & \$30,752 & \$41,817 & \$40,236 & \$47,476 & \$40,500\tabularnewline
\midrule
\endhead
\$60,332 & \$33,823 & \$35,438 & \$37,242 & \$31,238 & \$39,150 & \$34,023 & \$33,745\tabularnewline
\$33,269 & \$32,684 & \$31,844 & \$34,599 & \$48,748 & \$46,185 & \$36,931 & \$40,416\tabularnewline
\$29,548 & \$33,865 & \$31,067 & \$33,424 & \$35,484 & \$41,021 & \$47,155 & \$32,316\tabularnewline
\$42,113 & \$33,459 & \$32,462 & \$35,746 & \$31,274 & \$36,027 & \$37,089 & \$22,117\tabularnewline
\$41,412 & \$31,330 & \$31,329 & \$33,184 & \$35,301 & \$32,843 & \$38,177 & \$40,969\tabularnewline
\$40,993 & \$29,688 & \$35,890 & \$34,381 & & & &\tabularnewline
\bottomrule
\end{longtable}

\begin{enumerate}
\def\labelenumi{\arabic{enumi}.}
\setcounter{enumi}{2}
\tightlist
\item
  The density of people per square kilometer for African countries is
  in table \#2.2.11 ("Density of people," 2013). Create a frequency
  distribution, relative frequency distribution, and cumulative
  frequency distribution using 8 classes.
\end{enumerate}

\begin{quote}
\textbf{Table \#2.2.11: Data of Density of People per Square Kilometer}
\end{quote}

\begin{longtable}[]{@{}llllllll@{}}
\toprule
15 & 16 & 81 & 3 & 62 & 367 & 42 & 123\tabularnewline
\midrule
\endhead
8 & 9 & 337 & 12 & 29 & 70 & 39 & 83\tabularnewline
26 & 51 & 79 & 6 & 157 & 105 & 42 & 45\tabularnewline
72 & 72 & 37 & 4 & 36 & 134 & 12 & 3\tabularnewline
630 & 563 & 72 & 29 & 3 & 13 & 176 & 341\tabularnewline
415 & 187 & 65 & 194 & 75 & 16 & 41 & 18\tabularnewline
69 & 49 & 103 & 65 & 143 & 2 & 18 & 31\tabularnewline
\bottomrule
\end{longtable}

\begin{enumerate}
\def\labelenumi{\arabic{enumi}.}
\setcounter{enumi}{3}
\tightlist
\item
  The Affordable Care Act created a market place for individuals to
  purchase health care plans. In 2014, the premiums for a 27 year old
  for the bronze level health insurance are given in table \#2.2.12
  ("Health insurance marketplace," 2013). Create a frequency
  distribution, relative frequency distribution, and cumulative
  frequency distribution using 5 classes.
\end{enumerate}

\begin{quote}
\textbf{Table \#2.2.12: Data of Health Insurance Premiums}
\end{quote}

\begin{longtable}[]{@{}llllll@{}}
\toprule
\$114 & \$119 & \$121 & \$125 & \$132 & \$139\tabularnewline
\midrule
\endhead
\$139 & \$141 & \$143 & \$145 & \$151 & \$153\tabularnewline
\$156 & \$159 & \$162 & \$163 & \$165 & \$166\tabularnewline
\$170 & \$170 & \$176 & \$177 & \$181 & \$185\tabularnewline
\$185 & \$186 & \$186 & \$189 & \$190 & \$192\tabularnewline
\$196 & \$203 & \$204 & \$219 & \$254 & \$286\tabularnewline
\bottomrule
\end{longtable}

\begin{enumerate}
\def\labelenumi{\arabic{enumi}.}
\setcounter{enumi}{4}
\item
  Create a histogram and relative frequency histogram for the data in
  table \#2.2.9. Describe the shape and any findings you can from the
  graph.
\item
  Create a histogram and relative frequency histogram for the data in
  table \#2.2.10. Describe the shape and any findings you can from the
  graph.
\item
  Create a histogram and relative frequency histogram for the data in
  table \#2.2.11. Describe the shape and any findings you can from the
  graph.
\item
  Create a histogram and relative frequency histogram for the data in
  table \#2.2.12. Describe the shape and any findings you can from the
  graph.
\item
  Create an ogive for the data in table \#2.2.9. Describe any findings
  you can from the graph.
\item
  Create an ogive for the data in table \#2.2.10. Describe any
  findings you can from the graph.
\item
  Create an ogive for the data in table \#2.2.11. Describe any
  findings you can from the graph.
\item
  Create an ogive for the data in table \#2.2.12. Describe any
  findings you can from the graph.
\item
  Students in a statistics class took their first test. The following
  are the scores they earned. Create a frequency distribution and
  histogram for the data using class limits that make sense for grade
  data. Describe the shape of the distribution.
\end{enumerate}

\begin{quote}
\textbf{Table \#2.2.13: Data of Test 1 Grades}
\end{quote}

\begin{longtable}[]{@{}lllllll@{}}
\toprule
80 & 79 & 89 & 74 & 73 & 67 & 79\tabularnewline
\midrule
\endhead
93 & 70 & 70 & 76 & 88 & 83 & 73\tabularnewline
81 & 79 & 80 & 85 & 79 & 80 & 79\tabularnewline
58 & 93 & 94 & 74 & & &\tabularnewline
\bottomrule
\end{longtable}

\begin{enumerate}
\def\labelenumi{\arabic{enumi}.}
\setcounter{enumi}{13}
\tightlist
\item
  Students in a statistics class took their first test. The following
  are the scores they earned. Create a frequency distribution and
  histogram for the data using class limits that make sense for grade
  data. Describe the shape of the distribution. Compare to the graph
  in question 13.
\end{enumerate}

\begin{quote}
\textbf{Table \#2.2.14: Data of Test 1 Grades}
\end{quote}

\begin{longtable}[]{@{}llllll@{}}
\toprule
67 & 67 & 76 & 47 & 85 & 70\tabularnewline
\midrule
\endhead
87 & 76 & 80 & 72 & 84 & 98\tabularnewline
84 & 64 & 65 & 82 & 81 & 81\tabularnewline
88 & 74 & 87 & 83 & &\tabularnewline
\bottomrule
\end{longtable}

\textbf{\\
}

\hypertarget{other-graphical-representations-of-data}{%
\section{Other Graphical Representations of Data}\label{other-graphical-representations-of-data}}

There are many other types of graphs. Some of the more common ones are the frequency polygon, the dot plot, the stem plot, scatter plot, and a time-series plot. There are also many different graphs that have emerged lately for qualitative data. Many are found in publications and websites. The following is a description of the stem plot, the scatter plot, and the time-series plot.

\textbf{Stem Plots}

Stem plots are a quick and easy way to look at small samples of numerical data. You can look for any patterns or any strange data values. It is easy to compare two samples using stem plots.

The first step is to divide each number into 2 parts, the stem (such as the leftmost digit) and the leaf (such as the rightmost digit). There are no set rules, you just have to look at the data and see what makes sense.

\textbf{Example \#2.3.1: Stem Plot for Grade Distribution}

\begin{quote}
The following are the percentage grades of 25 students from a
statistics course. Draw a stem plot of the data.

\textbf{Table \#2.3.1: Data of Test Grades}
\end{quote}

\begin{longtable}[]{@{}llllllllll@{}}
\toprule
62 & 87 & 81 & 69 & 87 & 62 & 45 & 95 & 76 & 76\tabularnewline
\midrule
\endhead
62 & 71 & 65 & 67 & 72 & 80 & 40 & 77 & 87 & 58\tabularnewline
84 & 73 & 93 & 64 & 89 & & & & &\tabularnewline
\bottomrule
\end{longtable}

\begin{quote}
\textbf{Solution:}

Divide each number so that the tens digit is the stem and the ones
digit is the leaf. 62 becomes 6\textbar{}2.

Make a vertical chart with the stems on the left of a vertical bar. Be
sure to fill in any missing stems. In other words, the stems should
have equal spacing (for example, count by ones or count by tens). The
graph \#2.3.1 shows the stems for this example.

\textbf{Graph \#2.3.1: Stem plot for Test Grades Step 1}
\end{quote}

\begin{longtable}[]{@{}llllllll@{}}
\toprule
\endhead
4 & & & & & & &\tabularnewline
5 & & & & & & &\tabularnewline
6 & & & & & & &\tabularnewline
7 & & & & & & &\tabularnewline
8 & & & & & & &\tabularnewline
9 & & & & & & &\tabularnewline
\bottomrule
\end{longtable}

\begin{quote}
Now go through the list of data and add the leaves. Put each leaf next
to its corresponding stem. Don't worry about order yet just get all
the leaves down.

When the data value 62 is placed on the plot it looks like the plot in
graph \#2.3.2.

\textbf{Graph \#2.3.2: Stem plot for Test Grades Step 2}
\end{quote}

\begin{longtable}[]{@{}llllllll@{}}
\toprule
\endhead
4 & & & & & & &\tabularnewline
5 & & & & & & &\tabularnewline
6 & 2 & & & & & &\tabularnewline
7 & & & & & & &\tabularnewline
8 & & & & & & &\tabularnewline
9 & & & & & & &\tabularnewline
\bottomrule
\end{longtable}

\begin{quote}
When the data value 87 is placed on the plot it looks like the plot in
graph \#2.3.3.

\textbf{Graph \#2.3.3: Stem plot for Test Grades Step 3}
\end{quote}

\begin{longtable}[]{@{}llllllll@{}}
\toprule
\endhead
4 & & & & & & &\tabularnewline
5 & & & & & & &\tabularnewline
6 & 2 & & & & & &\tabularnewline
7 & & & & & & &\tabularnewline
8 & 7 & & & & & &\tabularnewline
9 & & & & & & &\tabularnewline
\bottomrule
\end{longtable}

\begin{quote}
Filling in the rest of the leaves to obtain the plot in graph \#2.3.4.

\textbf{Graph \#2.3.4: Stem plot for Test Grades Step 4}
\end{quote}

\begin{longtable}[]{@{}llllllll@{}}
\toprule
\endhead
4 & 5 & 0 & & & & &\tabularnewline
5 & 8 & & & & & &\tabularnewline
6 & 2 & 9 & 2 & 2 & 5 & 7 & 4\tabularnewline
7 & 6 & 6 & 1 & 2 & 7 & 3 &\tabularnewline
8 & 7 & 1 & 7 & 0 & 7 & 4 & 9\tabularnewline
9 & 5 & 3 & & & & &\tabularnewline
\bottomrule
\end{longtable}

\begin{quote}
Now you have to add labels and make the graph look pretty. You need to
add a label and sort the leaves into increasing order. You also need
to tell people what the stems and leaves mean by inserting a legend.
\textbf{Be careful to line the leaves up in columns.} You need to be able
to compare the lengths of the rows when you interpret the graph. The
final stem plot for the test grade data is in graph \#2.3.5.
\end{quote}

\textbf{\\
}

\begin{quote}
\textbf{Graph \#2.3.5: Stem plot for Test Grades}
\end{quote}

\begin{longtable}[]{@{}llllllll@{}}
\toprule
Test Scores & & & & & & &\tabularnewline
\midrule
\endhead
4 & 0 = 40\% & & & & & &\tabularnewline
4 & 0 & 5 & & & & &\tabularnewline
5 & 8 & & & & & &\tabularnewline
6 & 2 & 2 & 2 & 4 & 5 & 7 & 9\tabularnewline
7 & 1 & 2 & 3 & 6 & 6 & 7 &\tabularnewline
8 & 0 & 1 & 4 & 7 & 7 & 7 & 9\tabularnewline
9 & 3 & 5 & & & & &\tabularnewline
\bottomrule
\end{longtable}

\begin{quote}
Now you can interpret the stem-and-leaf display. The data is bimodal
and somewhat symmetric. There are no gaps in the data. The center of
the distribution is around 70.
\end{quote}

You can create a stem and leaf plot on R. the command is:

stem(variable) -- creates a stem and leaf plot, if you do not get a stem
plot that shows all of the stems then use scale = a number. Adjust the
number until you see all of the stems. So you would have stem(variable,
scale = a number)

For Example \#2.3.1, the command would be

\begin{quote}
grades\textless{}-c(62, 87, 81, 69, 87, 62, 45, 95, 76, 76, 62, 71, 65, 67, 72,
80, 40, 77, 87, 58, 84, 73, 93, 64, 89)

stem(grades, scale = 2)

Output:

The decimal point is 1 digit(s) to the right of the \textbar{}

4 \textbar{} 05

5 \textbar{} 8

6 \textbar{} 2224579

7 \textbar{} 123667

8 \textbar{} 0147779

9 \textbar{} 35

Now just put a title on the stem plot
\end{quote}

\textbf{Scatter Plot}

Sometimes you have two different variables and you want to see if they
are related in any way. A scatter plot helps you to see what the
relationship would look like. A scatter plot is just a plotting of the
ordered pairs.

\textbf{Example \#2.3.2: Scatter Plot}

\begin{quote}
Is there any relationship between elevation and high temperature on a
given day? The following data are the high temperatures at various
cities on a single day and the elevation of the city.

\textbf{Table \#2.3.2: Data of Temperature versus Elevation}
\end{quote}

\begin{longtable}[]{@{}llllllll@{}}
\toprule
Elevation (in feet) & 7000 & 4000 & 6000 & 3000 & 7000 & 4500 & 5000\tabularnewline
\midrule
\endhead
Temperature (°F) & 50 & 60 & 48 & 70 & 55 & 55 & 60\tabularnewline
\bottomrule
\end{longtable}

\begin{quote}
\textbf{Solution:}

Preliminary: State the random variables

Let x = altitude

y = high temperature

Now plot the x values on the horizontal axis, and the y values on the
vertical axis. Then set up a scale that fits the data on each axes.
Once that is done, then just plot the x and y values as an ordered
pair. In R, the command is:

independent variable\textless{}-c(type in data with commas in between values)

dependent variable\textless{}-c(type in data with commas in between values)

plot(independent variable, dependent variable, main="type in a title
you want", xlab="type in a label for the horizontal axis",
ylab="type in a label for the vertical axis", ylim=c(0, number above
maximum y value)

For this example, that would be:

elevation\textless{}-c(7000, 4000, 6000, 3000, 7000, 4500, 5000)

temperature\textless{}-c(50, 60, 48, 70, 55, 55, 60)

plot(elevation, temperature, main="Temperature versus Elevation",
xlab="Elevation (in feet)", ylab="Temperature (in degrees F)",
ylim=c(0, 80))

\textbf{Graph \#2.3.6: Scatter Plot of Temperature versus Elevation}

\includegraphics[width=3.70833in,height=3.70833in]{media/image57.emf}

Looking at the graph, it appears that there is a linear relationship
between temperature and elevation. It also appears to be a negative
relationship, thus as elevation increases, the temperature decreases.
\end{quote}

\textbf{Time-Series }

A time-series plot is a graph showing the data measurements in
chronological order, the data being quantitative data. For example, a
time-series plot is used to show profits over the last 5 years. To
create a time-series plot, the time always goes on the horizontal axis,
and the other variable goes on the vertical axis. Then plot the ordered
pairs and connect the dots. The purpose of a time-series graph is to
look for trends over time. Caution, you must realize that the trend may
not continue. Just because you see an increase, doesn't mean the
increase will continue forever. As an example, prior to 2007, many
people noticed that housing prices were increasing. The belief at the
time was that housing prices would continue to increase. However, the
housing bubble burst in 2007, and many houses lost value, and haven't
recovered.

\textbf{Example \#2.3.3: Time-Series Plot}

\begin{quote}
The following table tracks the weight of a dieter, where the time in
months is measuring how long since the person started the diet.

\textbf{Table \#2.3.3: Data of Weights versus Time}
\end{quote}

\begin{longtable}[]{@{}lllllll@{}}
\toprule
\endhead
Time (months) & 0 & 1 & 2 & 3 & 4 & 5\tabularnewline
Weight (pounds) & 200 & 195 & 192 & 193 & 190 & 187\tabularnewline
\bottomrule
\end{longtable}

\begin{quote}
Make a time-series plot of this data

\textbf{Solution:}

In R, the command would be:

variable1\textless{}-c(type in data with commas in between values, this should
be the time variable)

variable2\textless{}-c(type in data with commas in between values)

plot(variable1, variable2, ylim=c(0,number over max), main="type in a
title you want", xlab="type in a label for the horizontal axis",
ylab="type in a label for the vertical axis")

lines(variable1, variable2) -- connects the dots

For this example:

time\textless{}-c(0, 1, 2, 3, 4, 5)

weight\textless{}-c(200, 195, 192, 193, 190, 187)

plot(time, weight, ylim=c(0,250), main="Weight over Time",
xlab="Time (Months) ", ylab="Weight (pounds)")

lines(time, weight)

\textbf{Graph \#2.3.7: Time-Series Graph of Weight versus Time}
\end{quote}

\includegraphics[width=2.95833in,height=2.95833in]{media/image58.emf}

\begin{quote}
Notice, that over the 5 months, the weight appears to be decreasing.
Though it doesn't look like there is a large decrease.
\end{quote}

Be careful when making a graph. If you don't start the vertical axis at
0, then the change can look much more dramatic than it really is. As an
example, graph \#2.3.8 shows the graph \#2.3.7 with a different scaling
on the vertical axis. Notice the decrease in weight looks much larger
than it really is.

\begin{quote}
\textbf{Graph \#2.3.8: Example of a Poor Graph}
\end{quote}

\includegraphics[width=3.27778in,height=3.27778in]{media/image59.emf}

\hypertarget{homework-6}{%
\subsection{Homework}\label{homework-6}}

\begin{enumerate}
\def\labelenumi{\arabic{enumi}.}
\tightlist
\item
  Students in a statistics class took their first test. The data in
  table \#2.3.4 are the scores they earned. Create a stem plot.
\end{enumerate}

\begin{quote}
\textbf{Table \#2.3.4: Data of Test 1 Grades}
\end{quote}

\begin{longtable}[]{@{}lllllll@{}}
\toprule
80 & 79 & 89 & 74 & 73 & 67 & 79\tabularnewline
\midrule
\endhead
93 & 70 & 70 & 76 & 88 & 83 & 73\tabularnewline
81 & 79 & 80 & 85 & 79 & 80 & 79\tabularnewline
58 & 93 & 94 & 74 & & &\tabularnewline
\bottomrule
\end{longtable}

\begin{enumerate}
\def\labelenumi{\arabic{enumi}.}
\setcounter{enumi}{1}
\tightlist
\item
  Students in a statistics class took their first test. The data in
  table \#2.3.5 are the scores they earned. Create a stem plot.
  Compare to the graph in question 1.
\end{enumerate}

\begin{quote}
\textbf{Table \#2.3.5: Data of Test 1 Grades}
\end{quote}

\begin{longtable}[]{@{}llllll@{}}
\toprule
67 & 67 & 76 & 47 & 85 & 70\tabularnewline
\midrule
\endhead
87 & 76 & 80 & 72 & 84 & 98\tabularnewline
84 & 64 & 65 & 82 & 81 & 81\tabularnewline
88 & 74 & 87 & 83 & &\tabularnewline
\bottomrule
\end{longtable}

\begin{enumerate}
\def\labelenumi{\arabic{enumi}.}
\setcounter{enumi}{2}
\tightlist
\item
  When an anthropologist finds skeletal remains, they need to figure
  out the height of the person. The height of a person (in cm) and the
  length of one of their metacarpal bone (in cm) were collected and
  are in table \#2.4.6 ("Prediction of height," 2013). Create a
  scatter plot and state if there is a relationship between the height
  of a person and the length of their metacarpal.
\end{enumerate}

\begin{quote}
\textbf{Table \#2.3.6: Data of Metacarpal versus Height}
\end{quote}

\begin{longtable}[]{@{}ll@{}}
\toprule
Length of Metacarpal & Height of Person\tabularnewline
\midrule
\endhead
45 & 171\tabularnewline
51 & 178\tabularnewline
39 & 157\tabularnewline
41 & 163\tabularnewline
48 & 172\tabularnewline
49 & 183\tabularnewline
46 & 173\tabularnewline
43 & 175\tabularnewline
47 & 173\tabularnewline
\bottomrule
\end{longtable}

\begin{enumerate}
\def\labelenumi{\arabic{enumi}.}
\setcounter{enumi}{3}
\tightlist
\item
  Table \#2.3.7 contains the value of the house and the amount of
  rental income in a year that the house brings in ("Capital and
  rental," 2013). Create a scatter plot and state if there is a
  relationship between the value of the house and the annual rental
  income.
\end{enumerate}

\begin{quote}
\textbf{Table \#2.3.7: Data of House Value versus Rental}
\end{quote}

\begin{longtable}[]{@{}llllllll@{}}
\toprule
Value & Rental & Value & Rental & Value & Rental & Value & Rental\tabularnewline
\midrule
\endhead
81000 & 6656 & 77000 & 4576 & 75000 & 7280 & 67500 & 6864\tabularnewline
95000 & 7904 & 94000 & 8736 & 90000 & 6240 & 85000 & 7072\tabularnewline
121000 & 12064 & 115000 & 7904 & 110000 & 7072 & 104000 & 7904\tabularnewline
135000 & 8320 & 130000 & 9776 & 126000 & 6240 & 125000 & 7904\tabularnewline
145000 & 8320 & 140000 & 9568 & 140000 & 9152 & 135000 & 7488\tabularnewline
165000 & 13312 & 165000 & 8528 & 155000 & 7488 & 148000 & 8320\tabularnewline
178000 & 11856 & 174000 & 10400 & 170000 & 9568 & 170000 & 12688\tabularnewline
200000 & 12272 & 200000 & 10608 & 194000 & 11232 & 190000 & 8320\tabularnewline
214000 & 8528 & 208000 & 10400 & 200000 & 10400 & 200000 & 8320\tabularnewline
240000 & 10192 & 240000 & 12064 & 240000 & 11648 & 225000 & 12480\tabularnewline
289000 & 11648 & 270000 & 12896 & 262000 & 10192 & 244500 & 11232\tabularnewline
325000 & 12480 & 310000 & 12480 & 303000 & 12272 & 300000 & 12480\tabularnewline
\bottomrule
\end{longtable}

\begin{enumerate}
\def\labelenumi{\arabic{enumi}.}
\setcounter{enumi}{4}
\tightlist
\item
  The World Bank collects information on the life expectancy of a
  person in each country ("Life expectancy at," 2013) and the
  fertility rate per woman in the country ("Fertility rate," 2013).
  The data for 24 randomly selected countries for the year 2011 are in
  table \#2.3.8. Create a scatter plot of the data and state if there
  appears to be a relationship between life expectancy and the number
  of births per woman.
\end{enumerate}

\begin{quote}
\textbf{Table \#2.3.8: Data of Life Expectancy versus Fertility Rate}
\end{quote}

\begin{longtable}[]{@{}llll@{}}
\toprule
Life Expectancy & Fertility Rate & Life Expectancy & Fertility Rate\tabularnewline
\midrule
\endhead
77.2 & 1.7 & 72.3 & 3.9\tabularnewline
55.4 & 5.8 & 76.0 & 1.5\tabularnewline
69.9 & 2.2 & 66.0 & 4.2\tabularnewline
76.4 & 2.1 & 55.9 & 5.2\tabularnewline
75.0 & 1.8 & 54.4 & 6.8\tabularnewline
78.2 & 2.0 & 62.9 & 4.7\tabularnewline
73.0 & 2.6 & 78.3 & 2.1\tabularnewline
70.8 & 2.8 & 72.1 & 2.9\tabularnewline
82.6 & 1.4 & 80.7 & 1.4\tabularnewline
68.9 & 2.6 & 74.2 & 2.5\tabularnewline
81.0 & 1.5 & 73.3 & 1.5\tabularnewline
54.2 & 6.9 & 67.1 & 2.4\tabularnewline
\bottomrule
\end{longtable}

\begin{enumerate}
\def\labelenumi{\arabic{enumi}.}
\setcounter{enumi}{5}
\tightlist
\item
  The World Bank collected data on the percentage of gross domestic
  product (GDP) that a country spends on health expenditures ("Health
  expenditure," 2013) and the percentage of woman receiving prenatal
  care ("Pregnant woman receiving," 2013). The data for the
  countries where this information is available for the year 2011 is
  in table \#2.3.9. Create a scatter plot of the data and state if
  there appears to be a relationship between percentage spent on
  health expenditure and the percentage of woman receiving prenatal
  care.
\end{enumerate}

\begin{quote}
\textbf{Table \#2.3.9: Data of Prenatal Care versus Health Expenditure}
\end{quote}

\begin{longtable}[]{@{}ll@{}}
\toprule
Prenatal Care (\%) & Health Expenditure (\% of GDP)\tabularnewline
\midrule
\endhead
47.9 & 9.6\tabularnewline
54.6 & 3.7\tabularnewline
93.7 & 5.2\tabularnewline
84.7 & 5.2\tabularnewline
100.0 & 10.0\tabularnewline
42.5 & 4.7\tabularnewline
96.4 & 4.8\tabularnewline
77.1 & 6.0\tabularnewline
58.3 & 5.4\tabularnewline
95.4 & 4.8\tabularnewline
78.0 & 4.1\tabularnewline
93.3 & 6.0\tabularnewline
93.3 & 9.5\tabularnewline
93.7 & 6.8\tabularnewline
89.8 & 6.1\tabularnewline
\bottomrule
\end{longtable}

\begin{enumerate}
\def\labelenumi{\arabic{enumi}.}
\setcounter{enumi}{6}
\tightlist
\item
  The Australian Institute of Criminology gathered data on the number
  of deaths (per 100,000 people) due to firearms during the period
  1983 to 1997 ("Deaths from firearms," 2013). The data is in table
  \#2.3.10. Create a time-series plot of the data and state any
  findings you can from the graph.
\end{enumerate}

\begin{quote}
\textbf{Table \#2.3.10: Data of Year versus Number of Deaths due to
Firearms}
\end{quote}

\begin{longtable}[]{@{}lllllllll@{}}
\toprule
Year & 1983 & 1984 & 1985 & 1986 & 1987 & 1988 & 1989 & 1990\tabularnewline
\midrule
\endhead
Rate & 4.31 & 4.42 & 4.52 & 4.35 & 4.39 & 4.21 & 3.40 & 3.61\tabularnewline
Year & 1991 & 1992 & 1993 & 1994 & 1995 & 1996 & 1997 &\tabularnewline
Rate & 3.67 & 3.61 & 2.98 & 2.95 & 2.72 & 2.95 & 2.3 &\tabularnewline
\bottomrule
\end{longtable}

\begin{enumerate}
\def\labelenumi{\arabic{enumi}.}
\setcounter{enumi}{7}
\tightlist
\item
  The economic crisis of 2008 affected many countries, though some
  more than others. Some people in Australia have claimed that
  Australia wasn't hurt that badly from the crisis. The bank assets
  (in billions of Australia dollars (AUD)) of the Reserve Bank of
  Australia (RBA) for the time period of March 2007 through March 2013
  are contained in table \#2.3.11 ("B1 assets of," 2013). Create a
  time-series plot and interpret any findings.
\end{enumerate}

\begin{quote}
\textbf{Table \#2.3.11: Data of Date versus RBA Assets}
\end{quote}

\begin{longtable}[]{@{}ll@{}}
\toprule
Date & Assets in billions of AUD\tabularnewline
\midrule
\endhead
Mar-2006 & 96.9\tabularnewline
Jun-2006 & 107.4\tabularnewline
Sep-2006 & 107.2\tabularnewline
Dec-2006 & 116.2\tabularnewline
Mar-2007 & 123.7\tabularnewline
Jun-2007 & 134.0\tabularnewline
Sep-2007 & 123.0\tabularnewline
Dec-2007 & 93.2\tabularnewline
Mar-2008 & 93.7\tabularnewline
Jun-2008 & 105.6\tabularnewline
Sep-2008 & 101.5\tabularnewline
Dec-2008 & 158.8\tabularnewline
Mar-2009 & 118.7\tabularnewline
Jun-2009 & 111.9\tabularnewline
Sep-2009 & 87.0\tabularnewline
Dec-2009 & 86.1\tabularnewline
Mar-2010 & 83.4\tabularnewline
Jun-2010 & 85.7\tabularnewline
Sep-2010 & 74.8\tabularnewline
Dec-2010 & 76.0\tabularnewline
Mar-2011 & 75.7\tabularnewline
Jun-2011 & 75.9\tabularnewline
Sep-2011 & 75.2\tabularnewline
Dec-2011 & 87.9\tabularnewline
Mar-2012 & 91.0\tabularnewline
Jun-2012 & 90.1\tabularnewline
Sep-2012 & 83.9\tabularnewline
Dec-2012 & 95.8\tabularnewline
Mar-2013 & 90.5\tabularnewline
\bottomrule
\end{longtable}

\begin{enumerate}
\def\labelenumi{\arabic{enumi}.}
\setcounter{enumi}{8}
\tightlist
\item
  The consumer price index (CPI) is a measure used by the U.S.
  government to describe the cost of living. Table \#2.3.12 gives the
  cost of living for the U.S. from the years 1947 through 2011, with
  the year 1977 being used as the year that all others are compared
  (DeNavas-Walt, Proctor \& Smith, 2012). Create a time-series plot and
  interpret.
\end{enumerate}

\begin{quote}
\textbf{Table \#2.3.12: Data of Time versus CPI}
\end{quote}

\begin{longtable}[]{@{}llll@{}}
\toprule
Year & CPI-U-RS1 index (December 1977=100) & Year & CPI-U-RS1 index (December 1977=100)\tabularnewline
\midrule
\endhead
1947 & 37.5 & 1980 & 127.1\tabularnewline
1948 & 40.5 & 1981 & 139.2\tabularnewline
1949 & 40.0 & 1982 & 147.6\tabularnewline
1950 & 40.5 & 1983 & 153.9\tabularnewline
1951 & 43.7 & 1984 & 160.2\tabularnewline
1952 & 44.5 & 1985 & 165.7\tabularnewline
1953 & 44.8 & 1986 & 168.7\tabularnewline
1954 & 45.2 & 1987 & 174.4\tabularnewline
1955 & 45.0 & 1988 & 180.8\tabularnewline
1956 & 45.7 & 1989 & 188.6\tabularnewline
1957 & 47.2 & 1990 & 198.0\tabularnewline
1958 & 48.5 & 1991 & 205.1\tabularnewline
1959 & 48.9 & 1992 & 210.3\tabularnewline
1960 & 49.7 & 1993 & 215.5\tabularnewline
1961 & 50.2 & 1994 & 220.1\tabularnewline
1962 & 50.7 & 1995 & 225.4\tabularnewline
1963 & 51.4 & 1996 & 231.4\tabularnewline
1964 & 52.1 & 1997 & 236.4\tabularnewline
1965 & 52.9 & 1998 & 239.7\tabularnewline
1966 & 54.4 & 1999 & 244.7\tabularnewline
1967 & 56.1 & 2000 & 252.9\tabularnewline
1968 & 58.3 & 2001 & 260.0\tabularnewline
1969 & 60.9 & 2002 & 264.2\tabularnewline
1970 & 63.9 & 2003 & 270.1\tabularnewline
1971 & 66.7 & 2004 & 277.4\tabularnewline
1972 & 68.7 & 2005 & 286.7\tabularnewline
1973 & 73.0 & 2006 & 296.1\tabularnewline
1974 & 80.3 & 2007 & 304.5\tabularnewline
1975 & 86.9 & 2008 & 316.2\tabularnewline
1976 & 91.9 & 2009 & 315.0\tabularnewline
1977 & 97.7 & 2010 & 320.2\tabularnewline
1978 & 104.4 & 2011 & 330.3\tabularnewline
1979 & 114.4 & &\tabularnewline
\bottomrule
\end{longtable}

\begin{enumerate}
\def\labelenumi{\arabic{enumi}.}
\setcounter{enumi}{9}
\tightlist
\item
  The median incomes for all households in the U.S. for the years 1967
  to 2011 are given in table \#2.3.13 (DeNavas-Walt, Proctor \& Smith,
  2012). Create a time-series plot and interpret.
\end{enumerate}

\begin{quote}
\textbf{Table \#2.3.13: Data of Time versus Median Income}
\end{quote}

\begin{longtable}[]{@{}llll@{}}
\toprule
Year & Median Income & Year & Median Income\tabularnewline
\midrule
\endhead
1967 & 42,056 & 1990 & 49,950\tabularnewline
1968 & 43,868 & 1991 & 48,516\tabularnewline
1969 & 45,499 & 1992 & 48,117\tabularnewline
1970 & 45,146 & 1993 & 47,884\tabularnewline
1971 & 44,707 & 1994 & 48,418\tabularnewline
1972 & 46,622 & 1995 & 49,935\tabularnewline
1973 & 47,563 & 1996 & 50,661\tabularnewline
1974 & 46,057 & 1997 & 51,704\tabularnewline
1975 & 44,851 & 1998 & 53,582\tabularnewline
1976 & 45,595 & 1999 & 54,932\tabularnewline
1977 & 45,884 & 2000 & 54,841\tabularnewline
1978 & 47,659 & 2001 & 53,646\tabularnewline
1979 & 47,527 & 2002 & 53,019\tabularnewline
1980 & 46,024 & 2003 & 52,973\tabularnewline
1981 & 45,260 & 2004 & 52,788\tabularnewline
1982 & 45,139 & 2005 & 53,371\tabularnewline
1983 & 44,823 & 2006 & 53,768\tabularnewline
1984 & 46,215 & 2007 & 54,489\tabularnewline
1985 & 47,079 & 2008 & 52,546\tabularnewline
1986 & 48,746 & 2009 & 52,195\tabularnewline
1987 & 49,358 & 2010 & 50,831\tabularnewline
1988 & 49,737 & 2011 & 50,054\tabularnewline
1989 & 50,624 & &\tabularnewline
\bottomrule
\end{longtable}

\begin{enumerate}
\def\labelenumi{\arabic{enumi}.}
\setcounter{enumi}{10}
\tightlist
\item
  State everything that makes graph \#2.3.9 a misleading or poor
  graph.
\end{enumerate}

\begin{quote}
\textbf{Graph \#2.3.9: Example of a Poor Graph}
\end{quote}

\includegraphics[width=5.01389in,height=3.01389in]{media/image60.png}

\begin{enumerate}
\def\labelenumi{\arabic{enumi}.}
\setcounter{enumi}{11}
\tightlist
\item
  State everything that makes graph \#2.3.10 a misleading or poor
  graph (Benen, 2011).
\end{enumerate}

\begin{quote}
\textbf{Graph \#2.3.10: Example of a Poor Graph}
\end{quote}

\includegraphics[width=5.01389in,height=3.01389in]{media/image61.png}

\begin{enumerate}
\def\labelenumi{\arabic{enumi}.}
\setcounter{enumi}{12}
\tightlist
\item
  State everything that makes graph \#2.3.11 a misleading or poor
  graph ("United States unemployment," 2013).
\end{enumerate}

\begin{quote}
\textbf{Graph \#2.3.11: Example of a Poor Graph}
\end{quote}

\includegraphics[width=5.67383in,height=2.59722in]{media/image62.png}

\begin{enumerate}
\def\labelenumi{\arabic{enumi}.}
\setcounter{enumi}{13}
\tightlist
\item
  State everything that makes graph \#2.3.12 a misleading or poor
  graph.
\end{enumerate}

\begin{quote}
\textbf{Graph \#2.3.12: Example of a Poor Graph}
\end{quote}

\includegraphics[width=5.01389in,height=3.01389in]{media/image63.png}

Data Sources:

\emph{B1 assets of financial institutions}. (2013, June 27). Retrieved from
\href{http://www.rba.gov.au/statistics/tables/xls/b01hist.xls}{www.rba.gov.au/statistics/tables/xls/b01hist.xls}

Benen, S. (2011, September 02). \[Web log message\]. Retrieved from
\url{http://www.washingtonmonthly.com/political-animal/2011_09/gop_leaders_stop_taking_credit031960.php}

\emph{Capital and rental values of Auckland properties}. (2013, September
26). Retrieved from \url{http://www.statsci.org/data/oz/rentcap.html}

\emph{Contraceptive use}. (2013, October 9). Retrieved from
\url{http://www.prb.org/DataFinder/Topic/Rankings.aspx?ind=35}

\emph{Deaths from firearms}. (2013, September 26). Retrieved from
\url{http://www.statsci.org/data/oz/firearms.html}

DeNavas-Walt, C., Proctor, B., \& Smith, J. U.S. Department of Commerce,
U.S. Census Bureau. (2012). \emph{Income, poverty, and health insurance
coverage in the United States: 2011} (P60-243). Retrieved from website:
www.census.gov/prod/2012pubs/p60-243.pdf‎

\emph{Density of people in Africa}. (2013, October 9). Retrieved from
\href{http://www.prb.org/DataFinder/Topic/Rankings.aspx?ind=30\&loc=249,250,251,252,253,254,34227,255,257,258,259,260,261,262,263,264,265,266,267,268,269,270,271,272,274,275,276,277,278,279,280,281,282,283,284,285,286,287,288,289,290,291,292,294,295,296,297,298,2}{http://www.prb.org/DataFinder/Topic/Rankings.aspx?ind=30\&loc=249,250,251,252,253,254,34227,255,257,258,259,260,261,262,263,264,265,266,267,268,269,270,271,272,274,275,276,277,278,279,280,281,282,283,284,285,286,287,288,289,290,291,292,294,295,296,297,298,299,300,301,302,304,305,306,307,308}

Department of Health and Human Services, ASPE. (2013). \emph{Health insurance
marketplace premiums for 2014}. Retrieved from website:
\url{http://aspe.hhs.gov/health/reports/2013/marketplacepremiums/ib_premiumslandscape.pdf}

\emph{Electricity usage}. (2013, October 9). Retrieved from
\url{http://www.prb.org/DataFinder/Topic/Rankings.aspx?ind=162}

\emph{Fertility rate}. (2013, October 14). Retrieved from
\url{http://data.worldbank.org/indicator/SP.DYN.TFRT.IN}

\emph{Fuel oil usage}. (2013, October 9). Retrieved from
\url{http://www.prb.org/DataFinder/Topic/Rankings.aspx?ind=164}

\emph{Gas usage}. (2013, October 9). Retrieved from
\url{http://www.prb.org/DataFinder/Topic/Rankings.aspx?ind=165}

\emph{Health expenditure}. (2013, October 14). Retrieved from
\url{http://data.worldbank.org/indicator/SH.XPD.TOTL.ZS}

Hinatov, M. U.S. Consumer Product Safety Commission, Directorate of
Epidemiology. (2012). \emph{Incidents, deaths, and in-depth investigations
associated with non-fire carbon monoxide from engine-driven generators
and other engine-driven tools, 1999-2011}. Retrieved from website:
\url{http://www.cpsc.gov/PageFiles/129857/cogenerators.pdf}

\emph{Life expectancy at birth}. (2013, October 14). Retrieved from
\url{http://data.worldbank.org/indicator/SP.DYN.LE00.IN}

\emph{Median income of males}. (2013, October 9). Retrieved from
\url{http://www.prb.org/DataFinder/Topic/Rankings.aspx?ind=137}

\emph{Median income of males}. (2013, October 9). Retrieved from
\url{http://www.prb.org/DataFinder/Topic/Rankings.aspx?ind=136}

\emph{Prediction of height from metacarpal bone length}. (2013, September
26). Retrieved from \url{http://www.statsci.org/data/general/stature.html}

\emph{Pregnant woman receiving prenatal care}. (2013, October 14). Retrieved
from \url{http://data.worldbank.org/indicator/SH.STA.ANVC.ZS}

\emph{United States unemployment}. (2013, October 14). Retrieved from
\url{http://www.tradingeconomics.com/united-states/unemployment-rate}

Weissmann, J. (2013, March 20). A truly devastating graph on state
higher education spending. \emph{The Atlantic}. Retrieved from
\url{http://www.theatlantic.com/business/archive/2013/03/a-truly-devastating-graph-on-state-higher-education-spending/274199/}

\hypertarget{numerical-descriptions-of-data}{%
\chapter{Numerical Descriptions of Data}\label{numerical-descriptions-of-data}}

Chapter 1 discussed what a population, sample, parameter, and statistic are, and how to take different types of samples. Chapter 2 discussed ways to graphically display data. There was also a discussion of important characteristics: center, variations, distribution, outliers, and changing characteristics of the data over time. Distributions and outliers can be answered using graphical means. Finding the center and variation can be done using numerical methods that will be discussed in this chapter. Both graphical and numerical methods are part of a branch of statistics known as \textbf{descriptive statistics}. Later descriptive statistics will be used to make decisions and/or estimate population parameters using methods that are part of the branch called \textbf{inferential statistics}.

\hypertarget{measures-of-center}{%
\section{Measures of Center}\label{measures-of-center}}

This section focuses on measures of central tendency. Many times you are asking what to expect on average. Such as when you pick a major, you would probably ask how much you expect to earn in that field. If you are thinking of relocating to a new town, you might ask how much you can expect to pay for housing. If you are planting vegetables in the spring, you might want to know how long it will be until you can harvest. These questions, and many more, can be answered by knowing the center of the data set. There are three measures of the ``center'' of the data. They are the mode, median, and mean. Any of the values can be referred to as the ``average.''

The \textbf{mode} is the data value that occurs the most frequently in the data. To find it, you count how often each data value occurs, and then determine which data value occurs most often.

The \textbf{median} is the data value in the middle of a sorted list of data. To find it, you put the data in order, and then determine which data value is in the middle of the data set.

The \textbf{mean} is the arithmetic average of the numbers. This is the center that most people call the average, though all three -- mean, median, and mode -- really are averages.

There are no symbols for the mode and the median, but the mean is used a great deal, and statisticians gave it a symbol. There are actually two symbols, one for the population parameter and one for the sample statistic. In most cases you cannot find the population parameter, so you use the sample statistic to estimate the population parameter.

\textbf{Population Mean:}

, pronounced mu

\emph{N} is the size of the population.

\emph{x} represents a data value.

means to add up all of the data values.

\textbf{Sample Mean}:

, pronounced x bar.

\emph{n} is the size of the sample.

\emph{x} represents a data value.

means to add up all of the data values.

The value for is used to estimate since can't be calculated in most
situations.

\textbf{Example \#3.1.1: Finding the Mean, Median, and Mode}

\begin{quote}
Suppose a vet wants to find the average weight of cats. The weights
(in pounds) of five cats are in table \#3.1.1.

\textbf{Table \#3.1.1: Weights of cats in pounds}
\end{quote}

\begin{longtable}[]{@{}lllll@{}}
\toprule
\endhead
6.8 & 8.2 & 7.5 & 9.4 & 8.2\tabularnewline
\bottomrule
\end{longtable}

\begin{quote}
Find the mean, median, and mode of the weight of a cat.

\textbf{Solution:}

Before starting any mathematics problem, it is always a good idea to
define the unknown in the problem. In this case, you want to define
the variable. The symbol for the variable is \emph{x}.

The variable is \emph{x} = weight of a cat

Mean:

Median:

You need to sort the list for both the median and mode. The sorted
list is in table \#3.1.2.

\textbf{Table \#3.1.2: Sorted List of Cats' Weights}
\end{quote}

\begin{longtable}[]{@{}lllll@{}}
\toprule
\endhead
6.8 & 7.5 & 8.2 & 8.2 & 9.4\tabularnewline
\bottomrule
\end{longtable}

\begin{quote}
There are 5 data points so the middle of the list would be the 3\textsuperscript{rd}
number. (Just put a finger at each end of the list and move them
toward the center one number at a time. Where your fingers meet is the
median.)

\textbf{Table \#3.1.3: Sorted List of Cats' Weights with Median Marked}
\end{quote}

\begin{longtable}[]{@{}lllll@{}}
\toprule
\endhead
6.8 & 7.5 & 8.2 & 8.2 & 9.4\tabularnewline
\bottomrule
\end{longtable}

\begin{quote}
The median is therefore 8.2 pounds.

Mode:

This is easiest to do from the sorted list that is in table \#3.1.2.
Which value appears the most number of times? The number 8.2 appears
twice, while all other numbers appear once.

Mode = 8.2 pounds.
\end{quote}

A data set can have more than one mode. If there is a tie between two
values for the most number of times then both values are the mode and
the data is called bimodal (two modes). If every data point occurs the
same number of times, there is no mode. If there are more than two
numbers that appear the most times, then usually there is no mode.

In example \#3.1.1, there were an odd number of data points. In that
case, the median was just the middle number. What happens if there is an
even number of data points? What would you do?

\textbf{Example \#3.1.2: Finding the Median with an Even Number of Data
Points}

\begin{quote}
Suppose a vet wants to find the median weight of cats. The weights (in
pounds) of six cats are in table \#3.1.4. Find the median

\textbf{Table \#3.1.4: Weights of Six Cats}
\end{quote}

\begin{longtable}[]{@{}llllll@{}}
\toprule
\endhead
6.8 & 8.2 & 7.5 & 9.4 & 8.2 & 6.3\tabularnewline
\bottomrule
\end{longtable}

\begin{quote}
\textbf{Solution:}

Variable: \emph{x} = weight of a cat

First sort the list if it is not already sorted.

There are 6 numbers in the list so the number in the middle is between
the 3\textsuperscript{rd} and 4\textsuperscript{th} number. Use your fingers starting at each end of
the list in table \#3.1.5 and move toward the center until they meet.
There are two numbers there.

\textbf{Table \#3.1.5: Sorted List of Weights of Six Cats}
\end{quote}

\begin{longtable}[]{@{}llllll@{}}
\toprule
\endhead
6.3 & 6.8 & 7.5 & 8.2 & 8.2 & 9.4\tabularnewline
\bottomrule
\end{longtable}

\begin{quote}
To find the median, just average the two numbers.

The median is 7.85 pounds.
\end{quote}

\textbf{Example \#3.1.3: Finding Mean and Median using Technology}

\begin{quote}
Suppose a vet wants to find the median weight of cats. The weights (in
pounds) of six cats are in table \#3.1.4. Find the median

\textbf{Solution:}

Variable: \emph{x} = weight of a cat
\end{quote}

You can do the calculations for the mean and median using the
technology.

The procedure for calculating the sample mean () and the sample median
(Med) on the TI-83/84 is in figures 3.1.1 through 3.1.4. First you need
to go into the STAT menu, and then Edit. This will allow you to type in
your data (see figure \#3.1.1).

\begin{quote}
\textbf{Figure \#3.1.1: TI-83/84 Calculator Edit Setup}

\includegraphics[width=2.94444in,height=2.21291in]{media/image11.png}
\end{quote}

Once you have the data into the calculator, you then go back to the STAT
menu, move over to CALC, and then choose 1-Var Stats (see figure
\#3.1.2). The calculator will now put 1-Var Stats on the main screen.
Now type in L1 (2nd button and 1) and then press ENTER. (Note if you
have the newer operating system on the TI-84, then the procedure is
slightly different.) If you press the down arrow, you will see the rest
of the output from the calculator. The results from the calculator are
in figure \#3.1.3.

\begin{quote}
\textbf{Figure \#3.1.2: TI-83/84 Calculator CALC Menu}

\includegraphics[width=2.75in,height=1.86111in]{media/image12.gif}
\end{quote}

\textbf{\\
}

\begin{quote}
\textbf{Figure \#3.1.3: TI-83/84 Calculator Input for Example \#3.1.3
Variable}

\includegraphics[width=2.61111in,height=1.96239in]{media/image13.png}

\textbf{Figure \#3.1.4: TI-83/84 Calculator Results for Example \#3.1.3
Variable}
\end{quote}

\includegraphics[width=2.79167in,height=2.09809in]{media/image14.png}\includegraphics[width=2.83333in,height=2.1294in]{media/image15.png}

The commands for finding the mean and median using R are as follows:

\begin{quote}
variable\textless{}-c(type in your data with commas in between)

To find the mean, use mean(variable)

To find the median, use median(variable)
\end{quote}

So for this example, the commands would be

\begin{quote}
weights\textless{}-c(6.8, 8.2, 7.5, 9.4, 8.2, 6.3)

mean(weights)

\[1\] 7.733333

median(weights)

\[1\] 7.85
\end{quote}

\textbf{Example \#3.1.4: Affect of Extreme Values on Mean and Median}

\begin{quote}
Suppose you have the same set of cats from example 3.1.1 but one
additional cat was added to the data set. Table \#3.1.6 contains the
six cats' weights, in pounds.

\textbf{Table \#3.1.6: Weights of Six Cats}
\end{quote}

\begin{longtable}[]{@{}llllll@{}}
\toprule
\endhead
6.8 & 7.5 & 8.2 & 8.2 & 9.4 & 22.1\tabularnewline
\bottomrule
\end{longtable}

\begin{quote}
Find the mean and the median.
\end{quote}

\textbf{\\
}

\begin{quote}
\textbf{Solution:}
\end{quote}

Variable: \emph{x} = weight of a cat

\begin{quote}
The data is already in order, thus the median is between 8.2 and 8.2.

The mean is much higher than the median. Why is this? Notice that when
the value of 22.1 was added, the mean went from 8.02 to 10.37, but the
median did not change at all. This is because the mean is affected by
extreme values, while the median is not. The very heavy cat brought
the mean weight up. In this case, the median is a much better measure
of the center.
\end{quote}

An outlier is a data value that is very different from the rest of the
data. It can be really high or really low. Extreme values may be an
outlier if the extreme value is far enough from the center. In example
\#3.1.4, the data value 22.1 pounds is an extreme value and it may be an
outlier.

If there are extreme values in the data, the median is a better measure
of the center than the mean. If there are no extreme values, the mean
and the median will be similar so most people use the mean.

The mean is not a resistant measure because it is affected by extreme
values. The median and the mode are resistant measures because they are
not affected by extreme values.

As a consumer you need to be aware that people choose the measure of
center that best supports their claim. When you read an article in the
newspaper and it talks about the ``average'' it usually means the mean but
sometimes it refers to the median. Some articles will use the word
``median'' instead of ``average'' to be more specific. If you need to make
an important decision and the information says ``average'', it would be
wise to ask if the ``average'' is the mean or the median before you
decide.

As an example, suppose that a company wants to use the mean salary as
the average salary for the company. This is because the high salaries of
the administration will pull the mean higher. The company can say that
the employees are paid well because the average is high. However, the
employees want to use the median since it discounts the extreme values
of the administration and will give a lower value of the average. This
will make the salaries seem lower and that a raise is in order.

Why use the mean instead of the median? The reason is because when
multiple samples are taken from the same population, the sample means
tend to be more consistent than other measures of the center. The sample
mean is the more reliable measure of center.

To understand how the different measures of center related to skewed or
symmetric distributions, see figure \#3.1.5. As you can see sometimes
the mean is smaller than the median and mode, sometimes the mean is
larger than the median and mode, and sometimes they are the same values.

\textbf{Figure \#3.1.5: Mean, Median, Mode as Related to a Distribution}

\includegraphics[width=5.98667in,height=2.15333in]{media/image18.jpg}

One last type of average is a weighted average. Weighted averages are
used quite often in real life. Some teachers use them in calculating
your grade in the course, or your grade on a project. Some employers use
them in employee evaluations. The idea is that some activities are more
important than others. As an example, a fulltime teacher at a community
college may be evaluated on their service to the college, their service
to the community, whether their paperwork is turned in on time, and
their teaching. However, teaching is much more important than whether
their paperwork is turned in on time. When the evaluation is completed,
more weight needs to be given to the teaching and less to the paperwork.
This is a weighted average.

\textbf{Weighted Average}

where \emph{w} is the weight of the data value, \emph{x}.

\textbf{Example \#3.1.5: Weighted Average}

\begin{quote}
In your biology class, your final grade is based on several things: a
lab score, scores on two major tests, and your score on the final
exam. There are 100 points available for each score. The lab score is
worth 15\% of the course, the two exams are worth 25\% of the course
each, and the final exam is worth 35\% of the course. Suppose you
earned scores of 95 on the labs, 83 and 76 on the two exams, and 84 on
the final exam. Compute your weighted average for the course.

\textbf{Solution:}
\end{quote}

Variable: \emph{x} = score

The weighted average is

\begin{quote}
A weighted average can be found using technology.

The procedure for calculating the weighted average on the TI-83/84 is
in figures 3.1.6 through 3.1.9. First you need to go into the STAT
menu, and then Edit. This will allow you to type in the scores into L1
and the weights into L2 (see figure \#3.1.6).

\textbf{Figure \#3.1.6: TI-83/84 Calculator Edit Setup}

\includegraphics[width=3.09722in,height=2.32773in]{media/image22.png}

Once you have the data into the calculator, you then go back to the
STAT menu, move over to CALC, and then choose 1-Var Stats (see figure
\#3.1.7). The calculator will now put 1-Var Stats on the main screen.
Now type in L1 (2nd button and 1), then a comma (button above the 7
button), and then L2 (2nd button and 2) and then press ENTER. (Note if
you have the newer operating system on the TI-84, then the procedure
is slightly different.) The results from the calculator are in figure
\#3.1.9. The is the weighted average.

\textbf{Figure \#3.1.7: TI-83/84 Calculator CALC Menu}

\includegraphics[width=2.75in,height=1.86111in]{media/image12.gif}
\end{quote}

\textbf{\\
}

\begin{quote}
\textbf{Figure \#3.1.8: TI-83/84 Calculator Input for Weighted Average}

\includegraphics[width=2.72222in,height=2.04589in]{media/image24.png}

\textbf{Figure \#3.1.9: TI-83/84 Calculator Results for Weighted Average}

\includegraphics[width=2.69812in,height=2.02778in]{media/image25.png}
\end{quote}

The commands for finding the mean and median using R are as follows:

\begin{quote}
x\textless{}-c(type in your data with commas in between)

w\textless{}-c(type in your weights with commas in between

weighted.mean(x,w)
\end{quote}

So for this example, the commands would be

\begin{quote}
x\textless{}-c(95, 83, 76, 84)

w\textless{}-c(.15, .25, .25, .35)

weighted.mean(x,w)

\[1\] 83.4
\end{quote}

\textbf{Example \#3.1.6: Weighted Average}

\begin{quote}
The faculty evaluation process at John Jingle University rates a
faculty member on the following activities: teaching, publishing,
committee service, community service, and submitting paperwork in a
timely manner. The process involves reviewing student evaluations,
peer evaluations, and supervisor evaluation for each teacher and
awarding him/her a score on a scale from 1 to 10 (with 10 being the
best). The weights for each activity are 20 for teaching, 18 for
publishing, 6 for committee service, 4 for community service, and 2
for paperwork.
\end{quote}

\begin{enumerate}
\def\labelenumi{\alph{enumi})}
\tightlist
\item
  One faculty member had the following ratings: 8 for teaching, 9 for
  publishing, 2 for committee work, 1 for community service, and 8 for
  paperwork. Compute the weighted average of the evaluation.
\end{enumerate}

\begin{quote}
\textbf{Solution:}

Variable: \emph{x} = rating

The weighted average is .
\end{quote}

\begin{enumerate}
\def\labelenumi{\alph{enumi})}
\setcounter{enumi}{1}
\tightlist
\item
  Another faculty member had ratings of 6 for teaching, 8 for
  publishing, 9 for committee work, 10 for community service, and 10
  for paperwork. Compute the weighted average of the evaluation.
\end{enumerate}

\begin{quote}
\textbf{Solution:}
\end{quote}

\begin{enumerate}
\def\labelenumi{\alph{enumi})}
\setcounter{enumi}{2}
\tightlist
\item
  Which faculty member had the higher average evaluation?
\end{enumerate}

\begin{quote}
\textbf{Solution:}

The second faculty member has a higher average evaluation.
\end{quote}

You can find a weighted average using technology. On the

The last thing to mention is which average is used on which type of
data.

\begin{quote}
Mode can be found on nominal, ordinal, interval, and ratio data, since
the mode is just the data value that occurs most often. You are just
counting the data values.

Median can be found on ordinal, interval, and ratio data, since you
need to put the data in order. As long as there is order to the data
you can find the median.

Mean can be found on interval and ratio data, since you must have
numbers to add together.
\end{quote}

\hypertarget{homework-7}{%
\subsection{Homework}\label{homework-7}}

\begin{enumerate}
\def\labelenumi{\arabic{enumi}.}
\tightlist
\item
  Cholesterol levels were collected from patients two days after they had a heart attack (Ryan, Joiner \& Ryan, Jr, 1985) and are in table \#3.1.7. Find the mean, median, and mode.
\end{enumerate}

\begin{quote}
\textbf{Table \#3.1.7: Cholesterol Levels}
\end{quote}

\begin{longtable}[]{@{}lllllll@{}}
\toprule
270 & 236 & 210 & 142 & 280 & 272 & 160\tabularnewline
\midrule
\endhead
220 & 226 & 242 & 186 & 266 & 206 & 318\tabularnewline
294 & 282 & 234 & 224 & 276 & 282 & 360\tabularnewline
310 & 280 & 278 & 288 & 288 & 244 & 236\tabularnewline
\bottomrule
\end{longtable}

\begin{enumerate}
\def\labelenumi{\arabic{enumi}.}
\setcounter{enumi}{1}
\tightlist
\item
  The lengths (in kilometers) of rivers on the South Island of New
  Zealand that flow to the Pacific Ocean are listed in table \#3.1.8
  (Lee, 1994). Find the mean, median, and mode.
\end{enumerate}

\begin{quote}
\textbf{Table \#3.1.8: Lengths of Rivers (km) Flowing to Pacific Ocean}
\end{quote}

\begin{longtable}[]{@{}llll@{}}
\toprule
River & Length (km) & River & Length (km)\tabularnewline
\midrule
\endhead
Clarence & 209 & Clutha & 322\tabularnewline
Conway & 48 & Taieri & 288\tabularnewline
Waiau & 169 & Shag & 72\tabularnewline
Hurunui & 138 & Kakanui & 64\tabularnewline
Waipara & 64 & Rangitata & 121\tabularnewline
Ashley & 97 & Ophi & 80\tabularnewline
Waimakariri & 161 & Pareora & 56\tabularnewline
Selwyn & 95 & Waihao & 64\tabularnewline
Rakaia & 145 & Waitaki & 209\tabularnewline
Ashburton & 90 & &\tabularnewline
\bottomrule
\end{longtable}

\begin{enumerate}
\def\labelenumi{\arabic{enumi}.}
\setcounter{enumi}{2}
\tightlist
\item
  The lengths (in kilometers) of rivers on the South Island of New
  Zealand that flow to the Tasman Sea are listed in table \#3.1.9
  (Lee, 1994). Find the mean, median, and mode.
\end{enumerate}

\begin{quote}
\textbf{Table \#3.1.9: Lengths of Rivers (km) Flowing to Tasman Sea}
\end{quote}

\begin{longtable}[]{@{}llll@{}}
\toprule
River & Length (km) & River & Length (km)\tabularnewline
\midrule
\endhead
Hollyford & 76 & Waimea & 48\tabularnewline
Cascade & 64 & Motueka & 108\tabularnewline
Arawhata & 68 & Takaka & 72\tabularnewline
Haast & 64 & Aorere & 72\tabularnewline
Karangarua & 37 & Heaphy & 35\tabularnewline
Cook & 32 & Karamea & 80\tabularnewline
Waiho & 32 & Mokihinui & 56\tabularnewline
Whataroa & 51 & Buller & 177\tabularnewline
Wanganui & 56 & Grey & 121\tabularnewline
Waitaha & 40 & Taramakau & 80\tabularnewline
Hokitika & 64 & Arahura & 56\tabularnewline
\bottomrule
\end{longtable}

\begin{enumerate}
\def\labelenumi{\arabic{enumi}.}
\setcounter{enumi}{3}
\tightlist
\item
  Eyeglassmatic manufactures eyeglasses for their retailers. They
  research to see how many defective lenses they made during the time
  period of January 1 to March 31. Table \#3.1.10 contains the defect
  and the number of defects. Find the mean, median, and mode.
\end{enumerate}

\begin{quote}
\textbf{Table \#3.1.10: Number of Defective Lenses}
\end{quote}

\begin{longtable}[]{@{}ll@{}}
\toprule
Defect type & Number of defects\tabularnewline
\midrule
\endhead
Scratch & 5865\tabularnewline
Right shaped -- small & 4613\tabularnewline
Flaked & 1992\tabularnewline
Wrong axis & 1838\tabularnewline
Chamfer wrong & 1596\tabularnewline
Crazing, cracks & 1546\tabularnewline
Wrong shape & 1485\tabularnewline
Wrong PD & 1398\tabularnewline
Spots and bubbles & 1371\tabularnewline
Wrong height & 1130\tabularnewline
Right shape -- big & 1105\tabularnewline
Lost in lab & 976\tabularnewline
Spots/bubble -- intern & 976\tabularnewline
\bottomrule
\end{longtable}

\begin{enumerate}
\def\labelenumi{\arabic{enumi}.}
\setcounter{enumi}{4}
\tightlist
\item
  Print-O-Matic printing company's employees have salaries that are
  contained in table \#3.1.1.
\end{enumerate}

\begin{quote}
\textbf{Table \#3.1.11: Salaries of Print-O-Matic Printing Company
Employees}
\end{quote}

\begin{longtable}[]{@{}ll@{}}
\toprule
Employee & Salary (\$)\tabularnewline
\midrule
\endhead
CEO & 272,500\tabularnewline
Driver & 58,456\tabularnewline
CD74 & 100,702\tabularnewline
CD65 & 57,380\tabularnewline
Embellisher & 73,877\tabularnewline
Folder & 65,270\tabularnewline
GTO & 74,235\tabularnewline
Handwork & 52,718\tabularnewline
Horizon & 76,029\tabularnewline
ITEK & 64,553\tabularnewline
Mgmt & 108,448\tabularnewline
Platens & 69,573\tabularnewline
Polar & 75,526\tabularnewline
Pre Press Manager & 108,448\tabularnewline
Pre Press Manager/ IT & 98,837\tabularnewline
Pre Press/ Graphic Artist & 75,311\tabularnewline
Designer & 90,090\tabularnewline
Sales & 109,739\tabularnewline
Administration & 66,346\tabularnewline
\bottomrule
\end{longtable}

\begin{enumerate}
\def\labelenumi{\alph{enumi}.}
\item
  Find the mean and median.
\item
  Find the mean and median with the CEO's salary removed.
\item
  What happened to the mean and median when the CEO's salary was
  removed? Why?
\item
  If you were the CEO, who is answering concerns from the union that
  employees are underpaid, which average of the complete data set
  would you prefer? Why?
\item
  If you were a platen worker, who believes that the employees need a
  raise, which average would you prefer? Why?
\end{enumerate}

\begin{enumerate}
\def\labelenumi{\arabic{enumi}.}
\setcounter{enumi}{5}
\tightlist
\item
  Print-O-Matic printing company spends specific amounts on fixed
  costs every month. The costs of those fixed costs are in table
  \#3.1.12.
\end{enumerate}

\begin{quote}
\textbf{Table \#3.1.12: Fixed Costs for Print-O-Matic Printing Company}
\end{quote}

\begin{longtable}[]{@{}ll@{}}
\toprule
Monthly charges & Monthly cost (\$)\tabularnewline
\midrule
\endhead
Bank charges & 482\tabularnewline
Cleaning & 2208\tabularnewline
Computer expensive & 2471\tabularnewline
Lease payments & 2656\tabularnewline
Postage & 2117\tabularnewline
Uniforms & 2600\tabularnewline
\bottomrule
\end{longtable}

\begin{enumerate}
\def\labelenumi{\alph{enumi}.}
\item
  Find the mean and median.
\item
  Find the mean and median with the bank charges removed.
\item
  What happened to the mean and median when the bank charges was
  removed? Why?
\item
  If it is your job to oversee the fixed costs, which average using
  the complete data set would you prefer to use when submitting a
  report to administration to show that costs are low? Why?
\item
  If it is your job to find places in the budget to reduce costs,
  which average using the complete data set would you prefer to use
  when submitting a report to administration to show that fixed costs
  need to be reduced? Why?
\end{enumerate}

\begin{enumerate}
\def\labelenumi{\arabic{enumi}.}
\setcounter{enumi}{6}
\tightlist
\item
  State which type of measurement scale each represents, and then
  which center measures can be use for the variable?
\end{enumerate}

\begin{enumerate}
\def\labelenumi{\alph{enumi}.}
\item
  You collect data on people's likelihood (very likely, likely,
  neutral, unlikely, very unlikely) to vote for a candidate.
\item
  You collect data on the diameter at breast height of trees in the
  Coconino National Forest.
\item
  You collect data on the year wineries were started.
\item
  You collect the drink types that people in Sydney, Australia drink.
\end{enumerate}

\begin{enumerate}
\def\labelenumi{\arabic{enumi}.}
\setcounter{enumi}{7}
\tightlist
\item
  State which type of measurement scale each represents, and then
  which center measures can be use for the variable?
\end{enumerate}

\begin{enumerate}
\def\labelenumi{\alph{enumi}.}
\item
  You collect data on the height of plants using a new fertilizer.
\item
  You collect data on the cars that people drive in Campbelltown,
  Australia.
\item
  You collect data on the temperature at different locations in
  Antarctica.
\item
  You collect data on the first, second, and third winner in a beer
  competition.
\end{enumerate}

\begin{enumerate}
\def\labelenumi{\arabic{enumi}.}
\setcounter{enumi}{8}
\tightlist
\item
  Looking at graph \#3.1.1, state if the graph is skewed left, skewed
  right, or symmetric and then state which is larger, the mean or the
  median?
\end{enumerate}

\begin{quote}
\textbf{Graph \#3.1.1: Skewed or Symmetric Graph}

\includegraphics[width=5.01389in,height=3.01389in]{media/image29.png}
\end{quote}

\begin{enumerate}
\def\labelenumi{\arabic{enumi}.}
\setcounter{enumi}{9}
\tightlist
\item
  Looking at graph \#3.1.2, state if the graph is skewed left, skewed
  right, or symmetric and then state which is larger, the mean or the
  median?
\end{enumerate}

\begin{quote}
\textbf{Graph \#3.1.2: Skewed or Symmetric Graph}

\includegraphics[width=5.01389in,height=3.01389in]{media/image30.png}
\end{quote}

\begin{enumerate}
\def\labelenumi{\arabic{enumi}.}
\setcounter{enumi}{10}
\item
  An employee at Coconino Community College (CCC) is evaluated based
  on goal setting and accomplishments toward the goals, job
  effectiveness, competencies, and CCC core values. Suppose for a
  specific employee, goal 1 has a weight of 30\%, goal 2 has a weight
  of 20\%, job effectiveness has a weight of 25\%, competency 1 has a
  goal of 4\%, competency 2 has a goal has a weight of 3\%, competency 3
  has a weight of 3\%, competency 4 has a weight of 3\%, competency 5
  has a weight of 2\%, and core values has a weight of 10\%. Suppose the
  employee has scores of 3.0 for goal 1, 3.0 for goal 2, 2.0 for job
  effectiveness, 3.0 for competency 1, 2.0 for competency 2, 2.0 for
  competency 3, 3.0 for competency 4, 4.0 for competency 5, and 3.0
  for core values. Find the weighted average score for this employee.
  If an employee has a score less than 2.5, they must have a
  Performance Enhancement Plan written. Does this employee need a
  plan?
\item
  An employee at Coconino Community College (CCC) is evaluated based
  on goal setting and accomplishments toward goals, job effectiveness,
  competencies, CCC core values. Suppose for a specific employee, goal
  1 has a weight of 20\%, goal 2 has a weight of 20\%, goal 3 has a
  weight of 10\%, job effectiveness has a weight of 25\%, competency 1
  has a goal of 4\%, competency 2 has a goal has a weight of 3\%,
  competency 3 has a weight of 3\%, competency 4 has a weight of 5\%,
  and core values has a weight of 10\%. Suppose the employee has scores
  of 2.0 for goal 1, 2.0 for goal 2, 4.0 for goal 3, 3.0 for job
  effectiveness, 2.0 for competency 1, 3.0 for competency 2, 2.0 for
  competency 3, 3.0 for competency 4, and 4.0 for core values. Find
  the weighted average score for this employee. If an employee that
  has a score less than 2.5, they must have a Performance Enhancement
  Plan written. Does this employee need a plan?
\item
  A statistics class has the following activities and weights for
  determining a grade in the course: test 1 worth 15\% of the grade,
  test 2 worth 15\% of the grade, test 3 worth 15\% of the grade,
  homework worth 10\% of the grade, semester project worth 20\% of the
  grade, and the final exam worth 25\% of the grade. If a student
  receives an 85 on test 1, a 76 on test 2, an 83 on test 3, a 74 on
  the homework, a 65 on the project, and a 79 on the final, what grade
  did the student earn in the course?
\item
  A statistics class has the following activities and weights for
  determining a grade in the course: test 1 worth 15\% of the grade,
  test 2 worth 15\% of the grade, test 3 worth 15\% of the grade,
  homework worth 10\% of the grade, semester project worth 20\% of the
  grade, and the final exam worth 25\% of the grade. If a student
  receives a 92 on test 1, an 85 on test 2, a 95 on test 3, a 92 on
  the homework, a 55 on the project, and an 83 on the final, what
  grade did the student earn in the course?
\end{enumerate}

\hypertarget{measures-of-spread}{%
\section{Measures of Spread}\label{measures-of-spread}}

Variability is an important idea in statistics. If you were to measure the height of everyone in your classroom, every observation gives you a different value. That means not every student has the same height. Thus there is variability in people's heights. If you were to take a sample of the income level of people in a town, every sample gives you different information. There is variability between samples too. Variability describes how the data are spread out. If the data are very close to each other, then there is low variability. If the data are very spread out, then there is high variability. How do you measure variability? It would be good to have a number that measures it. This section will describe some of the different measures of variability, also known as variation.

In example \#3.1.1, the average weight of a cat was calculated to be 8.02 pounds. How much does this tell you about the weight of all cats? Can you tell if most of the weights were close to 8.02 or were the weights really spread out? What are the highest weight and the lowest weight? All you know is that the center of the weights is 8.02 pounds.

You need more information.

The \textbf{range} of a set of data is the difference between the highest and the lowest data values (or maximum and minimum values).

\textbf{Example \#3.2.1: Finding the Range}

\begin{quote}
Look at the following three sets of data. Find the range of each of
these.
\end{quote}

\begin{enumerate}
\def\labelenumi{\alph{enumi})}
\tightlist
\item
  10, 20, 30, 40, 50
\end{enumerate}

\begin{quote}
\textbf{Solution:}

\textbf{Graph \#3.2.1: Dot Plot for Example \#3.2.1a}

\includegraphics[width=2.56944in,height=0.70833in]{media/image32.png}

mean = 30, median = 30,
\end{quote}

\begin{enumerate}
\def\labelenumi{\alph{enumi})}
\setcounter{enumi}{1}
\tightlist
\item
  10, 29, 30, 31, 50
\end{enumerate}

\begin{quote}
\textbf{Solution:}

\textbf{Graph \#3.2.2: Dot Plot for Example \#3.2.1b}

\includegraphics[width=2.54167in,height=0.59722in]{media/image34.png}

mean = 30, median = 30,
\end{quote}

\begin{enumerate}
\def\labelenumi{\alph{enumi})}
\setcounter{enumi}{2}
\tightlist
\item
  28, 29, 30, 31, 32
\end{enumerate}

\begin{quote}
\textbf{Solution:}

\textbf{Graph \#3.2.3: Dot Plot for Example \#3.2.1}

\includegraphics[width=2.56944in,height=0.66667in]{media/image36.png}

mean = 30, median = 30,

Based on the mean, median, and range in example \#3.2.1, the first two
distributions are the same, but you can see from the graphs that they
are different. In example \#3.2.1a the data are spread out equally. In
example \#3.2.1b the data has a clump in the middle and a single value
at each end. The mean and median are the same for example \#3.2.1c but
the range is very different. All the data is clumped together in the
middle.
\end{quote}

The range doesn't really provide a very accurate picture of the
variability. A better way to describe how the data is spread out is
needed. Instead of looking at the distance the highest value is from the
lowest how about looking at the distance each value is from the mean.
This distance is called the \textbf{deviation}.

\textbf{Example \#3.2.2: Finding the Deviations}

\begin{quote}
Suppose a vet wants to analyze the weights of cats. The weights (in
pounds) of five cats are 6.8, 8.2, 7.5, 9.4, and 8.2. Find the
deviation for each of the data values.

\textbf{Solution:}
\end{quote}

Variable: \emph{x} = weight of a cat

\begin{quote}
The mean for this data set is .

\textbf{Table \#3.2.1: Deviations of Weights of Cats}
\end{quote}

\begin{longtable}[]{@{}ll@{}}
\toprule
\emph{x} &\tabularnewline
\midrule
\endhead
6.8 & 6.8 -- 8.02 =\tabularnewline
8.2 & 8.2 -- 8.02 = 0.18\tabularnewline
7.5 & 7.5 -- 8.02 =\tabularnewline
9.4 & 9.4 -- 8.02 = 1.38\tabularnewline
8.2 & 8.2 -- 8.02 = 0.18\tabularnewline
\bottomrule
\end{longtable}

\begin{quote}
Now you might want to average the deviation, so you need to add the
deviations together.

\textbf{~\\
Table \#3.2.2: Sum of Deviations of Weights of Cats }
\end{quote}

\begin{longtable}[]{@{}ll@{}}
\toprule
\emph{x} &\tabularnewline
\midrule
\endhead
6.8 & 6.8 -- 8.02 =\tabularnewline
8.2 & 8.2 -- 8.02 = .018\tabularnewline
7.5 & 7.5 -- 8.02 =\tabularnewline
9.4 & 9.4 -- 8.02 = 1.38\tabularnewline
8.2 & 8.2 -- 8.02 = 0.18\tabularnewline
Total & 0\tabularnewline
\bottomrule
\end{longtable}

\begin{quote}
This can't be right. The average distance from the mean cannot be 0.
The reason it adds to 0 is because there are some positive and
negative values. You need to get rid of the negative signs. How can
you do that? You could square each deviation.

\textbf{Table \#3.2.3: Squared Deviations of Weights of Cats}
\end{quote}

\begin{longtable}[]{@{}lll@{}}
\toprule
\emph{x} & &\tabularnewline
\midrule
\endhead
6.8 & 6.8 -- 8.02 = & 1.4884\tabularnewline
8.2 & 8.2 -- 8.02 = .018 & 0.0324\tabularnewline
7.5 & 7.5 -- 8.02 = & 0.2704\tabularnewline
9.4 & 9.4 -- 8.02 = 1.38 & 1.9044\tabularnewline
8.2 & 8.2 -- 8.02 = 0.18 & 0.0324\tabularnewline
Total & 0 & 3.728\tabularnewline
\bottomrule
\end{longtable}

\begin{quote}
Now average the total of the squared deviations. The only thing is
that in statistics there is a strange average here. Instead of
dividing by the number of data values you divide by the number of data
values minus 1. In this case you would have

Notice that this is denoted as . This is called the variance and it is
a measure of the average squared distance from the mean. If you now
take the square root, you will get the average distance from the mean.
This is called the standard deviation, and is denoted with the letter
\emph{s}.
\end{quote}

The standard deviation is the average (mean) distance from a data point
to the mean. It can be thought of as how much a typical data point
differs from the mean.

The \textbf{sample variance} formula:

where is the sample mean, \emph{n} is the sample size, and means to find the
sum

The \textbf{sample standard deviation} formula:

The on the bottom has to do with a concept called degrees of freedom.
Basically, it makes the sample standard deviation a better approximation
of the population standard deviation.

The \textbf{population variance} formula:

where is the Greek letter sigma and represents the population variance,
is the population mean, and N is the size of the population.

The \textbf{population standard deviation} formula:

Note: the sum of the deviations should always be 0. If it isn't, then it
is because you rounded, you used the median instead of the mean, or you
made an error. Try not to round too much in the calculations for
standard deviation since each rounding causes a slight error.

\textbf{Example \#3.2.3: Finding the Standard Deviation}

\begin{quote}
Suppose that a manager wants to test two new training programs. He
randomly selects 5 people for each training type and measures the time
it takes to complete a task after the training. The times for both
trainings are in table \#3.2.4. Which training method is better?

\textbf{Table \#3.2.4: Time to Finish Task in Minutes}
\end{quote}

\begin{longtable}[]{@{}llllll@{}}
\toprule
Training 1 & 56 & 75 & 48 & 63 & 59\tabularnewline
\midrule
\endhead
Training 2 & 60 & 58 & 66 & 59 & 58\tabularnewline
\bottomrule
\end{longtable}

\begin{quote}
\textbf{Solution:}

It is important that you define what each variable is since there are
two of them.

Variable 1: = productivity from training 1

Variable 2: = productivity from training 2

To answer which training method better, first you need some
descriptive statistics. Start with the mean for each sample.

Since both means are the same values, you cannot answer the question
about which is better. Now calculate the standard deviation for each
sample.

\textbf{Table \#3.2.5: Squared Deviations for Training 1}
\end{quote}

\begin{longtable}[]{@{}lll@{}}
\toprule
\endhead
56 & & 17.64\tabularnewline
75 & 14.8 & 219.04\tabularnewline
48 & & 148.84\tabularnewline
63 & 2.8 & 7.84\tabularnewline
59 & & 1.44\tabularnewline
Total & 0 & 394.8\tabularnewline
\bottomrule
\end{longtable}

\begin{quote}
\textbf{Table \#3.2.6: Squared Deviations for Training 2}
\end{quote}

\begin{longtable}[]{@{}lll@{}}
\toprule
\endhead
60 & & 0.04\tabularnewline
58 & & 4.84\tabularnewline
66 & 5.8 & 33.64\tabularnewline
59 & & 1.44\tabularnewline
58 & & 4.84\tabularnewline
Total & 0 & 44.8\tabularnewline
\bottomrule
\end{longtable}

\begin{quote}
The variance for each sample is:

The standard deviations are:

From the standard deviations, the second training seemed to be the
better training since the data is less spread out. This means it is
more consistent. It would be better for the managers in this case to
have a training program that produces more consistent results so they
know what to expect for the time it takes to complete the task.
\end{quote}

You can do the calculations for the descriptive statistics using the
technology. The procedure for calculating the sample mean () and the
sample standard deviation () for in example \#3.2.3 on the TI-83/84 is
in figures 3.2.1 through 3.2.4 (the procedure is the same for ). Note
the calculator gives you the population standard deviation () because it
doesn't know whether the data you input is a population or a sample. You
need to decide which value you need to use, based on whether you have a
population or sample. In almost all cases you have a sample and will be
using . Also, the calculator uses the notation of instead of just \emph{s}.
It is just a way for it to denote the information. First you need to go
into the STAT menu, and then Edit. This will allow you to type in your
data (see figure \#3.2.1).

\begin{quote}
\textbf{Figure \#3.2.1: TI-83/84 Calculator Edit Setup}

\includegraphics[width=2.75in,height=1.86111in]{media/image90.gif}
\end{quote}

Once you have the data into the calculator, you then go back to the STAT
menu, move over to CALC, and then choose 1-Var Stats (see figure
\#3.2.2). The calculator will now put 1-Var Stats on the main screen.
Now type in L2 (2nd button and 2) and then press ENTER. (Note if you
have the newer operating system on the TI-84, then the procedure is
slightly different.) The results from the calculator are in figure
\#3.2.4.

\begin{quote}
\textbf{Figure \#3.2.2: TI-83/84 Calculator CALC Menu}

\includegraphics[width=2.75in,height=1.86111in]{media/image12.gif}
\end{quote}

\textbf{\\
}

\begin{quote}
\textbf{Figure \#3.2.3: TI-83/84 Calculator Input for Example \#3.2.3
Variable }

\includegraphics[width=2.75in,height=1.86111in]{media/image92.png}

\textbf{Figure \#3.2.4: TI-83/84 Calculator Results for Example \#3.2.3
Variable }

\includegraphics[width=2.75in,height=1.86111in]{media/image94.gif}
\end{quote}

The processes for finding the mean, median, range, standard deviation,
and variance on R are as follows:

\begin{quote}
variable\textless{}-c(type in your data)

To find the mean, use mean(variable)

To find the median, use median(variable)

To find the range, use range(variable). Then find maximum -- minimum.

To find the standard deviation, use sd(variable)

To find the variance, use var(variable)
\end{quote}

For the second data set in example \#3.2.3, the commands and results
would be

\begin{quote}
productivity\_2\textless{}-c(60, 58, 66, 59, 58)

mean(productivity\_2)

\[1\] 60.2

median(productivity\_2)

\[1\] 59

range(productivity\_2)

\[1\] 58 66

sd(productivity\_2)

\[1\] 3.34664

var(productivity\_2)

\[1\] 11.2
\end{quote}

In general a ``small'' standard deviation means the data is close together
(more consistent) and a ``large'' standard deviation means the data is
spread out (less consistent). Sometimes you want consistent data and
sometimes you don't. As an example if you are making bolts, you want to
lengths to be very consistent so you want a small standard deviation. If
you are administering a test to see who can be a pilot, you want a large
standard deviation so you can tell who are the good pilots and who are
the bad ones.

What do ``small'' and ``large'' mean? To a bicyclist whose average speed is
20 mph, \emph{s} = 20 mph is huge. To an airplane whose average speed is 500
mph, \emph{s} = 20 mph is nothing. The ``size'' of the variation depends on the
size of the numbers in the problem and the mean. Another situation where
you can determine whether a standard deviation is small or large is when
you are comparing two different samples such as in example \#3.2.3. A
sample with a smaller standard deviation is more consistent than a
sample with a larger standard deviation.

Many other books and authors stress that there is a computational
formula for calculating the standard deviation. However, this formula
doesn't give you an idea of what standard deviation is and what you are
doing. It is only good for doing the calculations quickly. It goes back
to the days when standard deviations were calculated by hand, and the
person needed a quick way to calculate the standard deviation. It is an
archaic formula that this author is trying to eradicate it. It is not
necessary anymore, since most calculators and computers will do the
calculations for you with as much meaning as this formula gives. It is
suggested that you never use it. If you want to understand what the
standard deviation is doing, then you should use the definition formula.
If you want an answer quickly, use a computer or calculator.

\textbf{Use of Standard Deviation}

One of the uses of the standard deviation is to describe how a
population is distributed by using Chebyshev's Theorem. This theorem
works for any distribution, whether it is skewed, symmetric, bimodal, or
any other shape. It gives you an idea of how much data is a certain
distance on either side of the mean.

Chebyshev's Theorem

For {any} set of data:

At least 75\% of the data fall in the interval from.

At least 88.9\% of the data fall in the interval from.

At least 93.8\% of the data fall in the interval from.

\textbf{Example \#3.2.4: Using Chebyshev's Theorem}

\begin{quote}
The U.S. Weather Bureau has provided the information in table \#3.2.7
about the total annual number of reported strong to violent (F3+)
tornados in the United States for the years 1954 to 2012. ("U.S.
tornado climatology," 17)

\textbf{Table \#3.2.7: Annual Number of Violent Tornados in the U.S.}
\end{quote}

\begin{longtable}[]{@{}llllllllll@{}}
\toprule
46 & 47 & 31 & 41 & 24 & 56 & 56 & 23 & 31 & 59\tabularnewline
\midrule
\endhead
39 & 70 & 73 & 85 & 33 & 38 & 45 & 39 & 35 & 22\tabularnewline
51 & 39 & 51 & 131 & 37 & 24 & 57 & 42 & 28 & 45\tabularnewline
98 & 35 & 54 & 45 & 30 & 15 & 35 & 64 & 21 & 84\tabularnewline
40 & 51 & 44 & 62 & 65 & 27 & 34 & 23 & 32 & 28\tabularnewline
41 & 98 & 82 & 47 & 62 & 21 & 31 & 29 & 32 &\tabularnewline
\bottomrule
\end{longtable}

\begin{enumerate}
\def\labelenumi{\alph{enumi}.}
\tightlist
\item
  Use Chebyshev's theorem to find an interval centered about the mean
  annual number of strong to violent (F3+) tornados in which you would
  expect at least 75\% of the years to fall.
\end{enumerate}

\begin{quote}
\textbf{Solution:}

Variable: \emph{x} = number of strong or violent (F3+) tornadoes

Chebyshev's theorem says that at least 75\% of the data will fall in
the interval from to .

You do not have the population, so you need to estimate the population
mean and standard deviation using the sample mean and standard
deviation. You can find the sample mean and standard deviation using
technology:

.

Since you can't have fractional number of tornados, round to the
nearest whole number.

At least 75\% of the years have between 2 and 91 strong to violent
(F3+) tornados.

(Actually, all but three years' values fall in this interval, that
means that actually fall in the interval.)
\end{quote}

\begin{enumerate}
\def\labelenumi{\alph{enumi}.}
\setcounter{enumi}{1}
\tightlist
\item
  Use Chebyshev's theorem to find an interval centered about the mean
  annual number of strong to violent (F3+) tornados in which you would
  expect at least 88.9\% of the years to fall.
\end{enumerate}

\begin{quote}
\textbf{Solution:}

Variable: \emph{x} = number of strong or violent (F3+) tornadoes

Chebyshev's theorem says that at least 88.9\% of the data will fall in
the interval from to .

Since you can't have negative number of tornados, the lower limit is
actually 0. Since you can't have fractional number of tornados, round
to the nearest whole number.

At least 88.9\% of the years have between 0 and 113 strong to violent
(F3+) tornados.

(Actually, all but one year falls in this interval, that means that
actually fall in the interval.)
\end{quote}

Chebyshev's Theorem says that at least 75\% of the data is within two
standard deviations of the mean. That percentage is fairly high. There
isn't much data outside two standard deviations. A rule that can be
followed is that if a data value is within two standard deviations, then
that value is a common data value. If the data value is outside two
standard deviations of the mean, either above or below, then the number
is uncommon. It could even be called unusual. An easy calculation that
you can do to figure it out is to find the difference between the data
point and the mean, and then divide that answer by the standard
deviation. As a formula this would be

.

If you don't know the population mean, , and the population standard
deviation, , then use the sample mean, , and the sample standard
deviation, \emph{s}, to estimate the population parameter values. However,
realize that using the sample standard deviation may not actually be
very accurate.

\textbf{Example \#3.2.5: Determining If a Value Is Unusual}

\begin{enumerate}
\def\labelenumi{\alph{enumi}.}
\tightlist
\item
  In 1974, there were 131 strong or violent (F3+) tornados in the
  United States. Is this value unusual? Why or why not?
\end{enumerate}

\begin{quote}
\textbf{Solution:}

Variable: \emph{x} = number of strong or violent (F3+) tornadoes

To answer this question, first find how many standard deviations 131
is from the mean. From example \#3.2.4, we know and . For \emph{x} = 131,

Since this value is more than 2, then it is unusual to have 131 strong
or violent (F3+) tornados in a year.
\end{quote}

\begin{enumerate}
\def\labelenumi{\alph{enumi}.}
\setcounter{enumi}{1}
\tightlist
\item
  In 1987, there were 15 strong or violent (F3+) tornados in the
  United States. Is this value unusual? Why or why not?
\end{enumerate}

\begin{quote}
\textbf{Solution:}

Variable: \emph{x} = number of strong or violent (F3+) tornadoes

For this question the \emph{x} = 15,

Since this value is between and 2, then it is not unusual to have only
15 strong or violent (F3+) tornados in a year.
\end{quote}

\hypertarget{homework-8}{%
\subsection{Homework}\label{homework-8}}

\begin{enumerate}
\def\labelenumi{\arabic{enumi}.}
\tightlist
\item
  Cholesterol levels were collected from patients two days after they
  had a heart attack (Ryan, Joiner \& Ryan, Jr, 1985) and are in table
  \#3.2.8.
\end{enumerate}

\begin{quote}
\textbf{Table \#3.2.8: Cholesterol Levels}
\end{quote}

\begin{longtable}[]{@{}lllllll@{}}
\toprule
270 & 236 & 210 & 142 & 280 & 272 & 160\tabularnewline
\midrule
\endhead
220 & 226 & 242 & 186 & 266 & 206 & 318\tabularnewline
294 & 282 & 234 & 224 & 276 & 282 & 360\tabularnewline
310 & 280 & 278 & 288 & 288 & 244 & 236\tabularnewline
\bottomrule
\end{longtable}

\begin{quote}
Find the mean, median, range, variance, and standard deviation using
technology.
\end{quote}

\begin{enumerate}
\def\labelenumi{\arabic{enumi}.}
\setcounter{enumi}{1}
\tightlist
\item
  The lengths (in kilometers) of rivers on the South Island of New
  Zealand that flow to the Pacific Ocean are listed in table \#3.2.9
  (Lee, 1994).
\end{enumerate}

\begin{quote}
\textbf{Table \#3.2.9: Lengths of Rivers (km) Flowing to Pacific Ocean}
\end{quote}

\begin{longtable}[]{@{}llll@{}}
\toprule
River & Length (km) & River & Length (km)\tabularnewline
\midrule
\endhead
Clarence & 209 & Clutha & 322\tabularnewline
Conway & 48 & Taieri & 288\tabularnewline
Waiau & 169 & Shag & 72\tabularnewline
Hurunui & 138 & Kakanui & 64\tabularnewline
Waipara & 64 & Waitaki & 209\tabularnewline
Ashley & 97 & Waihao & 64\tabularnewline
Waimakariri & 161 & Pareora & 56\tabularnewline
Selwyn & 95 & Rangitata & 121\tabularnewline
Rakaia & 145 & Ophi & 80\tabularnewline
Ashburton & 90 & &\tabularnewline
\bottomrule
\end{longtable}

\begin{enumerate}
\def\labelenumi{\alph{enumi}.}
\item
  Find the mean and median.
\item
  Find the range.
\item
  Find the variance and standard deviation.
\end{enumerate}

\begin{enumerate}
\def\labelenumi{\arabic{enumi}.}
\setcounter{enumi}{2}
\tightlist
\item
  The lengths (in kilometers) of rivers on the South Island of New
  Zealand that flow to the Tasman Sea are listed in table \#3.2.10
  (Lee, 1994).
\end{enumerate}

\begin{quote}
\textbf{Table \#3.2.10: Lengths of Rivers (km) Flowing to Tasman Sea}
\end{quote}

\begin{longtable}[]{@{}llll@{}}
\toprule
River & Length (km) & River & Length (km)\tabularnewline
\midrule
\endhead
Hollyford & 76 & Waimea & 48\tabularnewline
Cascade & 64 & Motueka & 108\tabularnewline
Arawhata & 68 & Takaka & 72\tabularnewline
Haast & 64 & Aorere & 72\tabularnewline
Karangarua & 37 & Heaphy & 35\tabularnewline
Cook & 32 & Karamea & 80\tabularnewline
Waiho & 32 & Mokihinui & 56\tabularnewline
Whataroa & 51 & Buller & 177\tabularnewline
Wanganui & 56 & Grey & 121\tabularnewline
Waitaha & 40 & Taramakau & 80\tabularnewline
Hokitika & 64 & Arahura & 56\tabularnewline
\bottomrule
\end{longtable}

\begin{enumerate}
\def\labelenumi{\alph{enumi}.}
\item
  Find the mean and median.
\item
  Find the range.
\item
  Find the variance and standard deviation.
\end{enumerate}

\begin{enumerate}
\def\labelenumi{\arabic{enumi}.}
\setcounter{enumi}{3}
\tightlist
\item
  Eyeglassmatic manufactures eyeglasses for their retailers. They test
  to see how many defective lenses they made the time period of
  January 1 to March 31. Table \#3.2.11 gives the defect and the
  number of defects.
\end{enumerate}

\begin{quote}
\textbf{Table \#3.2.11: Number of Defective Lenses}
\end{quote}

\begin{longtable}[]{@{}ll@{}}
\toprule
Defect type & Number of defects\tabularnewline
\midrule
\endhead
Scratch & 5865\tabularnewline
Right shaped -- small & 4613\tabularnewline
Flaked & 1992\tabularnewline
Wrong axis & 1838\tabularnewline
Chamfer wrong & 1596\tabularnewline
Crazing, cracks & 1546\tabularnewline
Wrong shape & 1485\tabularnewline
Wrong PD & 1398\tabularnewline
Spots and bubbles & 1371\tabularnewline
Wrong height & 1130\tabularnewline
Right shape -- big & 1105\tabularnewline
Lost in lab & 976\tabularnewline
Spots/bubble -- intern & 976\tabularnewline
\bottomrule
\end{longtable}

\begin{enumerate}
\def\labelenumi{\alph{enumi}.}
\item
  Find the mean and median.
\item
  Find the range.
\item
  Find the variance and standard deviation.
\end{enumerate}

\begin{enumerate}
\def\labelenumi{\arabic{enumi}.}
\setcounter{enumi}{4}
\tightlist
\item
  Print-O-Matic printing company's employees have salaries that are
  contained in table \#3.2.12.
\end{enumerate}

\begin{quote}
\textbf{Table \#3.2.12: Salaries of Print-O-Matic Printing Company
Employees}
\end{quote}

\begin{longtable}[]{@{}llll@{}}
\toprule
Employee & Salary (\$) & Employee & Salary (\$)\tabularnewline
\midrule
\endhead
CEO & 272,500 & Administration & 66,346\tabularnewline
Driver & 58,456 & Sales & 109,739\tabularnewline
CD74 & 100,702 & Designer & 90,090\tabularnewline
CD65 & 57,380 & Platens & 69,573\tabularnewline
Embellisher & 73,877 & Polar & 75,526\tabularnewline
Folder & 65,270 & ITEK & 64,553\tabularnewline
GTO & 74,235 & Mgmt & 108,448\tabularnewline
Pre Press Manager & 108,448 & Handwork & 52,718\tabularnewline
Pre Press Manager/ IT & 98,837 & Horizon & 76,029\tabularnewline
Pre Press/ Graphic Artist & 75,311 & &\tabularnewline
\bottomrule
\end{longtable}

\begin{quote}
Find the mean, median, range, variance, and standard deviation using
technology.
\end{quote}

\begin{enumerate}
\def\labelenumi{\arabic{enumi}.}
\setcounter{enumi}{5}
\tightlist
\item
  Print-O-Matic printing company spends specific amounts on fixed
  costs every month. The costs of those fixed costs are in table
  \#3.2.13.
\end{enumerate}

\begin{quote}
\textbf{Table \#3.2.13: Fixed Costs for Print-O-Matic Printing Company}
\end{quote}

\begin{longtable}[]{@{}ll@{}}
\toprule
Monthly charges & Monthly cost (\$)\tabularnewline
\midrule
\endhead
Bank charges & 482\tabularnewline
Cleaning & 2208\tabularnewline
Computer expensive & 2471\tabularnewline
Lease payments & 2656\tabularnewline
Postage & 2117\tabularnewline
Uniforms & 2600\tabularnewline
\bottomrule
\end{longtable}

\begin{enumerate}
\def\labelenumi{\alph{enumi}.}
\item
  Find the mean and median.
\item
  Find the range.
\item
  Find the variance and standard deviation.
\end{enumerate}

\begin{enumerate}
\def\labelenumi{\arabic{enumi}.}
\setcounter{enumi}{6}
\item
  Compare the two data sets in problems 2 and 3 using the mean and
  standard deviation. Discuss which mean is higher and which has a
  larger spread of the data.
\item
  Table \#3.2.14 contains pulse rates collected from males, who are
  non-smokers but do drink alcohol ("Pulse rates before," 2013). The
  before pulse rate is before they exercised, and the after pulse rate
  was taken after the subject ran in place for one minute.
\end{enumerate}

\begin{quote}
\textbf{Table \#3.2.14: Pulse Rates of Males Before and After Exercise}
\end{quote}

\begin{longtable}[]{@{}llll@{}}
\toprule
Pulse before & Pulse after & Pulse before & Pulse after\tabularnewline
\midrule
\endhead
76 & 88 & 59 & 92\tabularnewline
56 & 110 & 60 & 104\tabularnewline
64 & 126 & 65 & 82\tabularnewline
50 & 90 & 76 & 150\tabularnewline
49 & 83 & 145 & 155\tabularnewline
68 & 136 & 84 & 140\tabularnewline
68 & 125 & 78 & 141\tabularnewline
88 & 150 & 85 & 131\tabularnewline
80 & 146 & 78 & 132\tabularnewline
78 & 168 & &\tabularnewline
\bottomrule
\end{longtable}

\begin{quote}
Compare the two data sets using the mean and standard deviation.
Discuss which mean is higher and which has a larger spread of the
data.
\end{quote}

\begin{enumerate}
\def\labelenumi{\arabic{enumi}.}
\setcounter{enumi}{8}
\tightlist
\item
  Table \#3.2.15 contains pulse rates collected from females, who are
  non-smokers but do drink alcohol ("Pulse rates before," 2013). The
  before pulse rate is before they exercised, and the after pulse rate
  was taken after the subject ran in place for one minute.
\end{enumerate}

\begin{quote}
\textbf{Table \#3.2.15: Pulse Rates of Females Before and After Exercise}
\end{quote}

\begin{longtable}[]{@{}llll@{}}
\toprule
Pulse before & Pulse after & Pulse before & Pulse after\tabularnewline
\midrule
\endhead
96 & 176 & 92 & 120\tabularnewline
82 & 150 & 70 & 96\tabularnewline
86 & 150 & 75 & 130\tabularnewline
72 & 115 & 70 & 119\tabularnewline
78 & 129 & 70 & 95\tabularnewline
90 & 160 & 68 & 84\tabularnewline
88 & 120 & 47 & 136\tabularnewline
71 & 125 & 64 & 120\tabularnewline
66 & 89 & 70 & 98\tabularnewline
76 & 132 & 74 & 168\tabularnewline
70 & 120 & 85 & 130\tabularnewline
\bottomrule
\end{longtable}

\begin{quote}
Compare the two data sets using the mean and standard deviation.
Discuss which mean is higher and which has a larger spread of the
data.
\end{quote}

\begin{enumerate}
\def\labelenumi{\arabic{enumi}.}
\setcounter{enumi}{9}
\tightlist
\item
  To determine if Reiki is an effective method for treating pain, a
  pilot study was carried out where a certified second-degree Reiki
  therapist provided treatment on volunteers. Pain was measured using
  a visual analogue scale (VAS) immediately before and after the Reiki
  treatment (Olson \& Hanson, 1997) and the data is in table \#3.2.16.
\end{enumerate}

\begin{quote}
\textbf{Table \#3.2.16: Pain Measurements Before and After Reiki Treatment}
\end{quote}

\begin{longtable}[]{@{}llll@{}}
\toprule
VAS before & VAS after & VAS before & VAS after\tabularnewline
\midrule
\endhead
6 & 3 & 5 & 1\tabularnewline
2 & 1 & 1 & 0\tabularnewline
2 & 0 & 6 & 4\tabularnewline
9 & 1 & 6 & 1\tabularnewline
3 & 0 & 4 & 4\tabularnewline
3 & 2 & 4 & 1\tabularnewline
4 & 1 & 7 & 6\tabularnewline
5 & 2 & 2 & 1\tabularnewline
2 & 2 & 4 & 3\tabularnewline
3 & 0 & 8 & 8\tabularnewline
\bottomrule
\end{longtable}

\begin{quote}
Compare the two data sets using the mean and standard deviation.
Discuss which mean is higher and which has a larger spread of the
data.
\end{quote}

\begin{enumerate}
\def\labelenumi{\arabic{enumi}.}
\setcounter{enumi}{10}
\tightlist
\item
  Table \#3.2.17 contains data collected on the time it takes in
  seconds of each passage of play in a game of rugby. ("Time of
  passages," 2013)
\end{enumerate}

\begin{quote}
\textbf{Table \#3.2.17: Times (in seconds) of rugby plays}
\end{quote}

\begin{longtable}[]{@{}llllllllll@{}}
\toprule
39.2 & 2.7 & 9.2 & 14.6 & 1.9 & 17.8 & 15.5 & 53.8 & 17.5 & 27.5\tabularnewline
\midrule
\endhead
4.8 & 8.6 & 22.1 & 29.8 & 10.4 & 9.8 & 27.7 & 32.7 & 32 & 34.3\tabularnewline
29.1 & 6.5 & 2.8 & 10.8 & 9.2 & 12.9 & 7.1 & 23.8 & 7.6 & 36.4\tabularnewline
35.6 & 28.4 & 37.2 & 16.8 & 21.2 & 14.7 & 44.5 & 24.7 & 36.2 & 20.9\tabularnewline
19.9 & 24.4 & 7.9 & 2.8 & 2.7 & 3.9 & 14.1 & 28.4 & 45.5 & 38\tabularnewline
18.5 & 8.3 & 56.2 & 10.2 & 5.5 & 2.5 & 46.8 & 23.1 & 9.2 & 10.3\tabularnewline
10.2 & 22 & 28.5 & 24 & 17.3 & 12.7 & 15.5 & 4 & 5.6 & 3.8\tabularnewline
21.6 & 49.3 & 52.4 & 50.1 & 30.5 & 37.2 & 15 & 38.7 & 3.1 & 11\tabularnewline
10 & 5 & 48.8 & 3.6 & 12.6 & 9.9 & 58.6 & 37.9 & 19.4 & 29.2\tabularnewline
12.3 & 39.2 & 22.2 & 39.7 & 6.4 & 2.5 & 34 & & &\tabularnewline
\bottomrule
\end{longtable}

\begin{enumerate}
\def\labelenumi{\alph{enumi}.}
\item
  Using technology, find the mean and standard deviation.
\item
  Use Chebyshev's theorem to find an interval centered about the mean
  times of each passage of play in the game of rugby in which you
  would expect at least 75\% of the times to fall.
\end{enumerate}

\begin{enumerate}
\def\labelenumi{\alph{enumi}.}
\setcounter{enumi}{2}
\tightlist
\item
  Use Chebyshev's theorem to find an interval centered about the mean
  times of each passage of play in the game of rugby in which you
  would expect at least 88.9\% of the times to fall.
\end{enumerate}

\begin{enumerate}
\def\labelenumi{\arabic{enumi}.}
\setcounter{enumi}{11}
\tightlist
\item
  Yearly rainfall amounts (in millimeters) in Sydney, Australia, are
  in table \#3.2.18 ("Annual maximums of," 2013).
\end{enumerate}

\begin{quote}
\textbf{Table \#3.2.18: Yearly Rainfall Amounts in Sydney, Australia}
\end{quote}

\begin{longtable}[]{@{}llllllll@{}}
\toprule
146.8 & 383 & 90.9 & 178.1 & 267.5 & 95.5 & 156.5 & 180\tabularnewline
\midrule
\endhead
90.9 & 139.7 & 200.2 & 171.7 & 187.2 & 184.9 & 70.1 & 58\tabularnewline
84.1 & 55.6 & 133.1 & 271.8 & 135.9 & 71.9 & 99.4 & 110.6\tabularnewline
47.5 & 97.8 & 122.7 & 58.4 & 154.4 & 173.7 & 118.8 & 88\tabularnewline
84.6 & 171.5 & 254.3 & 185.9 & 137.2 & 138.9 & 96.2 & 85\tabularnewline
45.2 & 74.7 & 264.9 & 113.8 & 133.4 & 68.1 & 156.4 &\tabularnewline
\bottomrule
\end{longtable}

\begin{enumerate}
\def\labelenumi{\alph{enumi}.}
\item
  Using technology, find the mean and standard deviation.
\item
  Use Chebyshev's theorem to find an interval centered about the mean
  yearly rainfall amounts in Sydney, Australia, in which you would
  expect at least 75\% of the amounts to fall.
\item
  Use Chebyshev's theorem to find an interval centered about the mean
  yearly rainfall amounts in Sydney, Australia, in which you would
  expect at least 88.9\% of the amounts to fall.
\end{enumerate}

\begin{enumerate}
\def\labelenumi{\arabic{enumi}.}
\setcounter{enumi}{12}
\tightlist
\item
  The number of deaths attributed to UV radiation in African countries
  in the year 2002 is given in table \#3.2.19 ("UV radiation:
  Burden," 2013).
\end{enumerate}

\begin{quote}
\textbf{Table \#3.2.19: Number of Deaths from UV Radiation}
\end{quote}

\begin{longtable}[]{@{}lllllllll@{}}
\toprule
50 & 84 & 31 & 338 & 6 & 504 & 40 & 7 & 58\tabularnewline
\midrule
\endhead
204 & 15 & 27 & 39 & 1 & 45 & 174 & 98 & 94\tabularnewline
199 & 9 & 27 & 58 & 356 & 5 & 45 & 5 & 94\tabularnewline
26 & 171 & 13 & 57 & 138 & 39 & 3 & 171 & 41\tabularnewline
1177 & 102 & 123 & 433 & 35 & 40 & 456 & 125 &\tabularnewline
\bottomrule
\end{longtable}

\begin{enumerate}
\def\labelenumi{\alph{enumi}.}
\item
  Using technology, find the mean and standard deviation.
\item
  Use Chebyshev's theorem to find an interval centered about the mean
  number of deaths from UV radiation in which you would expect at
  least 75\% of the numbers to fall.
\item
  Use Chebyshev's theorem to find an interval centered about the mean
  number of deaths from UV radiation in which you would expect at
  least 88.9\% of the numbers to fall.
\end{enumerate}

\begin{enumerate}
\def\labelenumi{\arabic{enumi}.}
\setcounter{enumi}{13}
\tightlist
\item
  The time (in 1/50 seconds) between successive pulses along a nerve
  fiber ("Time between nerve," 2013) are given in table \#3.2.20.
\end{enumerate}

\begin{quote}
\textbf{Table 3.2.20: Time (in 1/50 seconds) Between Successive Pulses}
\end{quote}

\begin{longtable}[]{@{}llllllllll@{}}
\toprule
10.5 & 1.5 & 2.5 & 5.5 & 29.5 & 3 & 9 & 27.5 & 18.5 & 4.5\tabularnewline
\midrule
\endhead
7 & 9.5 & 1 & 7 & 4.5 & 2.5 & 7.5 & 11.5 & 7.5 & 4\tabularnewline
12 & 8 & 3 & 5.5 & 7.5 & 4.5 & 1.5 & 10.5 & 1 & 7\tabularnewline
12 & 14.5 & 8 & 3.5 & 3.5 & 2 & 1 & 7.5 & 6 & 13\tabularnewline
7.5 & 16.5 & 3 & 25.5 & 5.5 & 14 & 18 & 7 & 27.5 & 14\tabularnewline
\bottomrule
\end{longtable}

\begin{enumerate}
\def\labelenumi{\alph{enumi}.}
\item
  Using technology, find the mean and standard deviation.
\item
  Use Chebyshev's theorem to find an interval centered about the mean
  time between successive pulses along a nerve fiber in which you
  would expect at least 75\% of the times to fall.
\item
  Use Chebyshev's theorem to find an interval centered about the mean
  time between successive pulses along a nerve fiber in which you
  would expect at least 88.9\% of the times to fall.
\end{enumerate}

\begin{enumerate}
\def\labelenumi{\arabic{enumi}.}
\setcounter{enumi}{14}
\item
  Suppose a passage of play in a rugby game takes 75.1 seconds. Would
  it be unusual for this to happen? Use the mean and standard
  deviation that you calculated in problem 11.
\item
  Suppose Sydney, Australia received 300 mm of rainfall in a year.
  Would this be unusual? Use the mean and standard deviation that you
  calculated in problem 12.
\item
  Suppose in a given year there were 2257 deaths attributed to UV
  radiation in an African country. Is this value unusual? Use the mean
  and standard deviation that you calculated in problem 13.
\item
  Suppose it only takes 2 (1/50 seconds) for successive pulses along a
  nerve fiber. Is this value unusual? Use the mean and standard
  deviation that you calculated in problem 14.
\end{enumerate}

\hypertarget{ranking}{%
\section{Ranking}\label{ranking}}

Along with the center and the variability, another useful numerical
measure is the ranking of a number. A \textbf{percentile} is a measure of
ranking. It represents a location measurement of a data value to the
rest of the values. Many standardized tests give the results as a
percentile. Doctors also use percentiles to track a child's growth.

The \textbf{\emph{k}th percentile} is the data value that has k\% of the data at or
below that value.

\textbf{Example \#3.3.1: Interpreting Percentile}

\begin{enumerate}
\def\labelenumi{\alph{enumi}.}
\tightlist
\item
  What does a score of the 90\textsuperscript{th} percentile mean?
\end{enumerate}

\begin{quote}
\textbf{Solution:}

This means that 90\% of the scores were at or below this score. (A
person did the same as or better than 90\% of the test takers.)
\end{quote}

\begin{enumerate}
\def\labelenumi{\alph{enumi}.}
\setcounter{enumi}{1}
\tightlist
\item
  What does a score of the 70\textsuperscript{th} percentile mean?
\end{enumerate}

\begin{quote}
\textbf{Solution:}

This means that 70\% of the scores were at or below this score.
\end{quote}

\textbf{Example \#3.3.2: Percentile Versus Score}

\begin{quote}
If the test was out of 100 points and you scored at the 80\textsuperscript{th}
percentile, what was your score on the test?

\textbf{Solution:}

You don't know! All you know is that you scored the same as or better
than 80\% of the people who took the test. If all the scores were
really low, you could have still failed the test. On the other hand,
if many of the scores were high you could have gotten a 95\% or so.
\end{quote}

There are special percentiles called quartiles. Quartiles are numbers
that divide the data into fourths. One fourth (or a quarter) of the data
falls between consecutive quartiles.

\textbf{To find the quartiles:}

1) Sort the data in increasing order.

2) Find the median, this divides the data list into 2 halves.

3) Find the median of the data below the median. This value is \emph{Q1}.

4) Find the median of the data above the median. This value is \emph{Q3}.

Ignore the median in both calculations for \emph{Q1} and \emph{Q3}

If you record the quartiles together with the maximum and minimum you
have five numbers. This is known as the five-number summary. The
five-number summary consists of the minimum, the first quartile (\emph{Q1}),
the median, the third quartile (\emph{Q3}), and the maximum (in that order).

The interquartile range, \emph{IQR}, is the difference between the first and
third quartiles, \emph{Q1} and \emph{Q3}. Half of the data (50\%) falls in the
interquartile range. If the \emph{IQR} is ``large'' the data is spread out and
if the \emph{IQR} is ``small'' the data is closer together.

Interquartile Range (\emph{IQR})

\textbf{Determining probable outliers from IQR: \emph{fences}}

A value that is less than (this value is often referred to as a
\textbf{\emph{low}} \textbf{\emph{fence}}) is considered an outlier.

Similarly, a value that is more than (the \textbf{\emph{high}} \textbf{\emph{fence}}) is
considered an outlier.

A box plot (or box-and-whisker plot) is a graphical display of the
five-number summary. It can be drawn vertically or horizontally. The
basic format is a box from \emph{Q1} to \emph{Q3}, a vertical line across the box
for the median and horizontal lines as whiskers extending out each end
to the minimum and maximum. The minimum and maximum can be represented
with dots. Don't forget to label the tick marks on the number line and
give the graph a title.

An alternate form of a Box-and-Whiskers Plot, known as a modified box
plot, only extends the left line to the smallest value greater than the
\emph{low fence}, and extends the left line to the largest value less than
the \emph{high fence}, and displays markers (dots, circles or asterisks) for
each outlier.

~

If the data are \emph{symmetrical}, then the box plot will be visibly
symmetrical. If the data distribution has a left skew or a right skew,
the line on that side of the box plot will be visibly long.~ If the plot
is symmetrical, and the four quartiles are all about the same length,
then the data are likely a near \emph{uniform} distribution. If a box plot is
symmetrical, and both outside lines are noticeably longer than the Q1 to
median and median to Q3 distance, the distribution is then probably
\emph{bell-shaped.}

\textbf{Figure \#3.3.1: Typical Box Plot}

\begin{quote}
\includegraphics[width=4.11111in,height=1.58333in]{media/image119.png}
\end{quote}

\textbf{Example \#3.3.3: Five-number Summary for an Even Number of Data
Points}

\begin{quote}
The total assets in billions of Australian dollars (AUD) of Australian
banks for the year 2012 are given in table \#3.3.1 ("Reserve bank
of," 2013). Find the five-number summary and the interquartile range
(\emph{IQR}), and draw a box-and-whiskers plot.

\textbf{Table \#3.3.1: Total Assets (in billions of AUD) of Australian
Banks}
\end{quote}

\begin{longtable}[]{@{}llllll@{}}
\toprule
2855 & 2862 & 2861 & 2884 & 3014 & 2965\tabularnewline
\midrule
\endhead
2971 & 3002 & 3032 & 2950 & 2967 & 2964\tabularnewline
\bottomrule
\end{longtable}

\begin{quote}
\textbf{Solution:}

Variable: \emph{x} = total assets of Australian banks

First sort the data.

\textbf{Table \#3.3.2: Sorted Data for Total Assets}
\end{quote}

\begin{longtable}[]{@{}llllllllllll@{}}
\toprule
\endhead
2855 & 2861 & 2862 & 2884 & 2950 & 2964 & 2965 & 2967 & 2971 & 3002 & 3014 & 3032\tabularnewline
\bottomrule
\end{longtable}

\begin{quote}
The minimum is 2855 billion AUD and the maximum is 3032 billion AUD.

There are 12 data points so the median is the average of the 6\textsuperscript{th} and
7\textsuperscript{th} numbers.

\textbf{Table \#3.3.3: Sorted Data for Total Assets with Median}
\end{quote}

\begin{longtable}[]{@{}llllllllllll@{}}
\toprule
\endhead
2855 & 2861 & 2862 & 2884 & 2950 & 2964 & 2965 & 2967 & 2971 & 3002 & 3014 & 3032\tabularnewline
\bottomrule
\end{longtable}

\begin{quote}
To find \emph{Q1}, find the median of the first half of the list.

\textbf{Table \#3.3.4: Finding \emph{Q1}}
\end{quote}

\begin{longtable}[]{@{}llllll@{}}
\toprule
\endhead
2855 & 2861 & 2862 & 2884 & 2950 & 2964\tabularnewline
\bottomrule
\end{longtable}

\begin{quote}
To find \emph{Q3}, find the median of the second half of the list.

\textbf{Table \#3.3.5: Finding \emph{Q3}}
\end{quote}

\begin{longtable}[]{@{}llllll@{}}
\toprule
\endhead
2965 & 2967 & 2971 & 3002 & 3014 & 3032\tabularnewline
\bottomrule
\end{longtable}

\begin{quote}
The five-number summary is (all numbers in billion AUD)

Minimum: 2855

\emph{Q1}: 2873

Median: 2964.5

\emph{Q3}: 2986.5

Maximum: 3032

To find the interquartile range, \emph{IQR}, find .

This tells you the middle 50\% of assets were within 113.5 billion AUD
of each other.

You can use the five-number summary to draw the box-and-whiskers plot:

\textbf{Graph \#3.3.1: Box Plot of Total Assets of Australian Banks}

\includegraphics[width=5.40278in,height=2.25in]{media/image125.emf}

The distribution is skewed right because the right tail is longer.
\end{quote}

\textbf{\\
}

\textbf{Example \#3.3.4: Five-number Summary for an Odd Number of Data Points
}

\begin{quote}
The life expectancy for a person living in one of 11 countries in the
region of South East Asia in 2012 is given below ("Life expectancy
in," 2013). Find the five-number summary for the data and the \emph{IQR},
then draw a box-and-whiskers plot.

\textbf{Table \#3.3.6: Life Expectancy of a Person Living in South-East
Asia}
\end{quote}

\begin{longtable}[]{@{}llllll@{}}
\toprule
70 & 67 & 69 & 65 & 69 & 77\tabularnewline
\midrule
\endhead
65 & 68 & 75 & 74 & 64 &\tabularnewline
\bottomrule
\end{longtable}

\begin{quote}
\textbf{Solution:}

Variable: \emph{x} = life expectancy of a person

Sort the data first.

\textbf{Table \#3.3.7: Sorted Life Expectancies}
\end{quote}

\begin{longtable}[]{@{}lllllllllll@{}}
\toprule
\endhead
64 & 65 & 65 & 67 & 68 & 69 & 69 & 70 & 74 & 75 & 77\tabularnewline
\bottomrule
\end{longtable}

\begin{quote}
The minimum is 64 years and the maximum is 77 years.

There are 11 data points so the median is the 6\textsuperscript{th} number in the
list.

\textbf{Table \#3.3.8: Finding the Median of Life Expectancies}
\end{quote}

\begin{longtable}[]{@{}lllllllllll@{}}
\toprule
\endhead
64 & 65 & 65 & 67 & 68 & 69 & 69 & 70 & 74 & 75 & 77\tabularnewline
\bottomrule
\end{longtable}

\begin{quote}
Median = 69 years

Finding the \emph{Q1} and \emph{Q3} you need to find the median of the numbers
below the median and above the median. The median is not included in
either calculation.

\textbf{Table \#3.3.9: Finding \emph{Q1}}
\end{quote}

\begin{longtable}[]{@{}lllll@{}}
\toprule
\endhead
64 & 65 & 65 & 67 & 68\tabularnewline
\bottomrule
\end{longtable}

\begin{quote}
\textbf{Table \#3.3.10: Finding \emph{Q3}}
\end{quote}

\begin{longtable}[]{@{}lllll@{}}
\toprule
\endhead
69 & 70 & 74 & 75 & 77\tabularnewline
\bottomrule
\end{longtable}

\begin{quote}
\emph{Q1} = 65 years and \emph{Q3} = 74 years.

The five-number summary is (in years)

Minimum: 64

Q1: 65

Median: 69

Q3: 74

Maximum: 77
\end{quote}

To find the interquartile range (\emph{IQR})

The middle 50\% of life expectancies are within 9 years.

\begin{quote}
\textbf{Graph \#3.3.2: Box Plot of Life Expectancy}

\includegraphics[width=5.65278in,height=1.90278in]{media/image127.emf}

This distribution looks somewhat skewed right, since the whisker is
longer on the right. However, it could be considered almost symmetric
too since the box looks somewhat symmetric.
\end{quote}

You can draw 2 box plots side by side (or one above the other) to
compare 2 samples. Since you want to compare the two data sets, make
sure the box plots are on the same axes. As an example, suppose you look
at the box-and-whiskers plot for life expectancy for European countries
and Southeast Asian countries.

\begin{quote}
\textbf{Graph \#3.3.3: Box Plot of Life Expectancy of Two Regions}
\end{quote}

\includegraphics[width=6in,height=2.48031in]{media/image128.emf}

Looking at the box-and-whiskers plot, you will notice that the three
quartiles for life expectancy are all higher for the European countries,
yet the minimum life expectancy for the European countries is less than
that for the Southeast Asian countries. The life expectancy for the
European countries appears to be skewed left, while the life
expectancies for the Southeast Asian countries appear to be more
symmetric. There are of course more qualities that can be compared
between the two graphs.

To find the five-number summary using R, the command is:

\begin{quote}
variable\textless{}-c(type in data with commas)\\
summary(variable)
\end{quote}

This command will give you the five number summary and the mean.

For example \#3.3.4, the commands would be

\begin{quote}
expectancy\textless{}-c(70, 67, 69, 65, 69, 77, 65, 68, 75, 74, 64)

summary(expectancy)

The output would be:

Min. 1st Qu. Median Mean 3rd Qu. Max.

64.00 66.00 69.00 69.36 72.00 77.00
\end{quote}

To draw the box plot the command is boxplot(variable, main="title you
want", xlab="label you want", horizontal = TRUE). The horizontal =
TRUE orients the box plot to be horizontal. If you leave that part off,
the box plot will be vertical by default.~

For example \#3.3.4, the command is~

boxplot(expectancy, main="Life Expectancy of Southeast Asian Countries
in 2011",horizontal=TRUE, xlab="Life Expectancy")

You should get the box plot in graph \#3.3.4.

\textbf{Graph \#3.3.4: Box plot for Life Expectance in Southeast Asian
Countries}

\includegraphics[width=3.125in,height=3.125in]{media/image129.emf}

This is known as a modified box plot. Instead of plotting the maximum
and minimum, the box plot has as a lower line , and as an upper line,~.
Any values below the lower line or above the upper line are considered
outliers. Outliers are plotted as dots on the modified box plot. This
data set does not have any outliers.

\textbf{Example \#3.3.5: Putting it all together}

A random sample was collected on the health expenditures (as a \% of GDP)
of countries around the world. The data is in Table \#3.3.11. Using
graphical and numerical descriptive statistics, analyze the data and use
it to predict the health expenditures of all countries in the world.

\textbf{Table \#3.3.11: Health Expenditures as a Percentage of GDP}

\begin{longtable}[]{@{}llllllllll@{}}
\toprule
3.35 & 5.94 & 10.64 & 5.24 & 3.79 & 5.65 & 7.66 & 7.38 & 5.87 & 11.15\tabularnewline
\midrule
\endhead
5.96 & 4.78 & 7.75 & 2.72 & 9.50 & 7.69 & 10.05 & 11.96 & 8.18 & 6.74\tabularnewline
5.89 & 6.20 & 5.98 & 8.83 & 6.78 & 6.66 & 9.45 & 5.41 & 5.16 & 8.55\tabularnewline
\bottomrule
\end{longtable}

\textbf{Solution:}

\begin{quote}
First, it might be useful to look at a visualization of the data, so
create a histogram.

\textbf{Graph \#3.3.5: Histogram of Health Expenditure}

\includegraphics[width=5.05556in,height=5.05556in]{media/image132.emf}

From the graph, the data appears to be somewhat skewed right. So there
are some countries that spend more on health based on a percentage of
GDP than other countries, but the majority of countries appear to
spend around 4 to 8\% of their GDP on health.

Numerical descriptions might also be useful. Using technology, the
mean is 7.03\%, the standard deviation is 2.27\%, and the five-number
summary is minimum = 2.72\%, Q1 = 5.71\%, median = 6.70\%, Q3 = 8.46\%,
and maximum = 11.96\%. To visualize the five-number summary, create a
box plot.

\textbf{Graph \#3.3.6: Box Plot of Health Expenditure}

\includegraphics[width=4.72222in,height=4.72222in]{media/image133.emf}

So it appears that countries spend on average about 7\% of their GPD on
health. The spread is somewhat low, since the standard deviation is
fairly small, which means that the data is fairly consistent. The
five-number summary confirms that the data is slightly skewed right.
The box plot shows that there are no outliers. So from all of this
information, one could say that countries spend a small percentage of
their GDP on health and that most countries spend around the same
amount. There doesn't appear to be any country that spends much more
than other countries or much less than other countries.
\end{quote}

\hypertarget{homework-9}{%
\subsection{Homework}\label{homework-9}}

\begin{enumerate}
\def\labelenumi{\arabic{enumi}.}
\item
  Suppose you take a standardized test and you are in the 10\textsuperscript{th}
  percentile. What does this percentile mean? Can you say that you
  failed the test? Explain.
\item
  Suppose your child takes a standardized test in mathematics and
  scores in the 96\textsuperscript{th} percentile. What does this percentile mean? Can
  you say your child passed the test? Explain.
\item
  Suppose your child is in the 83\textsuperscript{rd} percentile in height and 24\textsuperscript{th}
  percentile in weight. Describe what this tells you about your
  child's stature.
\item
  Suppose your work evaluates the employees and places them on a
  percentile ranking. If your evaluation is in the 65\textsuperscript{th} percentile,
  do you think you are working hard enough? Explain.
\item
  Cholesterol levels were collected from patients two days after they
  had a heart attack (Ryan, Joiner \& Ryan, Jr, 1985) and are in table
  \#3.3.12.
\end{enumerate}

\begin{quote}
\textbf{Table \#3.3.12: Cholesterol Levels}
\end{quote}

\begin{longtable}[]{@{}lllllll@{}}
\toprule
270 & 236 & 210 & 142 & 280 & 272 & 160\tabularnewline
\midrule
\endhead
220 & 226 & 242 & 186 & 266 & 206 & 318\tabularnewline
294 & 282 & 234 & 224 & 276 & 282 & 360\tabularnewline
310 & 280 & 278 & 288 & 288 & 244 & 236\tabularnewline
\bottomrule
\end{longtable}

\begin{quote}
Find the five-number summary and interquartile range (IQR), and draw a
box-and-whiskers plot
\end{quote}

\begin{enumerate}
\def\labelenumi{\arabic{enumi}.}
\setcounter{enumi}{5}
\tightlist
\item
  The lengths (in kilometers) of rivers on the South Island of New
  Zealand that flow to the Pacific Ocean are listed in table \#3.3.13
  (Lee, 1994).
\end{enumerate}

\begin{quote}
\textbf{Table \#3.3.13: Lengths of Rivers (km) Flowing to Pacific Ocean}
\end{quote}

\begin{longtable}[]{@{}llll@{}}
\toprule
River & Length (km) & River & Length (km)\tabularnewline
\midrule
\endhead
Clarence & 209 & Clutha & 322\tabularnewline
Conway & 48 & Taieri & 288\tabularnewline
Waiau & 169 & Shag & 72\tabularnewline
Hurunui & 138 & Kakanui & 64\tabularnewline
Waipara & 64 & Waitaki & 209\tabularnewline
Ashley & 97 & Waihao & 64\tabularnewline
Waimakariri & 161 & Pareora & 56\tabularnewline
Selwyn & 95 & Rangitata & 121\tabularnewline
Rakaia & 145 & Ophi & 80\tabularnewline
Ashburton & 90 & &\tabularnewline
\bottomrule
\end{longtable}

\begin{quote}
Find the five-number summary and interquartile range (IQR), and draw a
box-and-whiskers plot
\end{quote}

\begin{enumerate}
\def\labelenumi{\arabic{enumi}.}
\setcounter{enumi}{6}
\tightlist
\item
  The lengths (in kilometers) of rivers on the South Island of New
  Zealand that flow to the Tasman Sea are listed in table \#3.3.14
  (Lee, 1994).
\end{enumerate}

\begin{quote}
\textbf{Table \#3.3.14: Lengths of Rivers (km) Flowing to Tasman Sea}
\end{quote}

\begin{longtable}[]{@{}llll@{}}
\toprule
River & Length (km) & River & Length (km)\tabularnewline
\midrule
\endhead
Hollyford & 76 & Waimea & 48\tabularnewline
Cascade & 64 & Motueka & 108\tabularnewline
Arawhata & 68 & Takaka & 72\tabularnewline
Haast & 64 & Aorere & 72\tabularnewline
Karangarua & 37 & Heaphy & 35\tabularnewline
Cook & 32 & Karamea & 80\tabularnewline
Waiho & 32 & Mokihinui & 56\tabularnewline
Whataroa & 51 & Buller & 177\tabularnewline
Wanganui & 56 & Grey & 121\tabularnewline
Waitaha & 40 & Taramakau & 80\tabularnewline
Hokitika & 64 & Arahura & 56\tabularnewline
\bottomrule
\end{longtable}

\begin{quote}
Find the five-number summary and interquartile range (IQR), and draw a
box-and-whiskers plot
\end{quote}

\begin{enumerate}
\def\labelenumi{\arabic{enumi}.}
\setcounter{enumi}{7}
\tightlist
\item
  Eyeglassmatic manufactures eyeglasses for their retailers. They test
  to see how many defective lenses they made the time period of
  January 1 to March 31. Table \#3.3.15 gives the defect and the
  number of defects.
\end{enumerate}

\begin{quote}
\textbf{Table \#3.3.15: Number of Defective Lenses}
\end{quote}

\begin{longtable}[]{@{}ll@{}}
\toprule
Defect type & Number of defects\tabularnewline
\midrule
\endhead
Scratch & 5865\tabularnewline
Right shaped -- small & 4613\tabularnewline
Flaked & 1992\tabularnewline
Wrong axis & 1838\tabularnewline
Chamfer wrong & 1596\tabularnewline
Crazing, cracks & 1546\tabularnewline
Wrong shape & 1485\tabularnewline
Wrong PD & 1398\tabularnewline
Spots and bubbles & 1371\tabularnewline
Wrong height & 1130\tabularnewline
Right shape -- big & 1105\tabularnewline
Lost in lab & 976\tabularnewline
Spots/bubble -- intern & 976\tabularnewline
\bottomrule
\end{longtable}

\begin{quote}
Find the five-number summary and interquartile range (IQR), and draw a
box-and-whiskers plot
\end{quote}

\begin{enumerate}
\def\labelenumi{\arabic{enumi}.}
\setcounter{enumi}{8}
\tightlist
\item
  A study was conducted to see the effect of exercise on pulse rate.
  Male subjects were taken who do not smoke, but do drink. Their pulse
  rates were measured ("Pulse rates before," 2013). Then they ran in
  place for one minute and then measured their pulse rate again. Graph
  \#3.3.7 is of box-and-whiskers plots that were created of the before
  and after pulse rates. Discuss any conclusions you can make from the
  graphs.
\end{enumerate}

\begin{quote}
\textbf{Graph \#3.3.7: Box-and-Whiskers Plot of Pulse Rates for Males}

\includegraphics[width=5.77073in,height=2.25in]{media/image134.emf}
\end{quote}

\begin{enumerate}
\def\labelenumi{\arabic{enumi}.}
\setcounter{enumi}{9}
\tightlist
\item
  A study was conducted to see the effect of exercise on pulse rate.
  Female subjects were taken who do not smoke, but do drink. Their
  pulse rates were measured ("Pulse rates before," 2013). Then they
  ran in place for one minute, and after measured their pulse rate
  again. Graph \#3.3.8 is of box-and-whiskers plots that were created
  of the before and after pulse rates. Discuss any conclusions you can
  make from the graphs.
\end{enumerate}

\begin{quote}
\textbf{Graph \#3.3.8: Box-and-Whiskers Plot of Pulse Rates for Females}

\includegraphics[width=5.84067in,height=2.25in]{media/image135.emf}
\end{quote}

\begin{enumerate}
\def\labelenumi{\arabic{enumi}.}
\setcounter{enumi}{10}
\tightlist
\item
  To determine if Reiki is an effective method for treating pain, a
  pilot study was carried out where a certified second-degree Reiki
  therapist provided treatment on volunteers. Pain was measured using
  a visual analogue scale (VAS) immediately before and after the Reiki
  treatment (Olson \& Hanson, 1997). Graph \#3.3.9 is of
  box-and-whiskers plots that were created of the before and after VAS
  ratings. Discuss any conclusions you can make from the graphs.
\end{enumerate}

\begin{quote}
\textbf{Graph \#3.3.9: Box-and-Whiskers Plot of Pain Using Reiki}

\includegraphics[width=5.24245in,height=2.125in]{media/image136.emf}
\end{quote}

\begin{enumerate}
\def\labelenumi{\arabic{enumi}.}
\setcounter{enumi}{11}
\tightlist
\item
  The number of deaths attributed to UV radiation in African countries
  and Middle Eastern countries in the year 2002 were collected by the
  World Health Organization ("UV radiation: Burden," 2013). Graph
  \#3.3.10 is of box-and-whiskers plots that were created of the
  deaths in African countries and deaths in Middle Eastern countries.
  Discuss any conclusions you can make from the graphs.
\end{enumerate}

\begin{quote}
\textbf{Table \#3.3.10: Box-and-Whiskers Plot of UV Radiation Deaths in
Different Regions}

\includegraphics[width=5.67361in,height=1.79167in]{media/image137.emf}
\end{quote}

Data Sources:

\emph{Annual maximums of daily rainfall in Sydney}. (2013, September 25).
Retrieved from \url{http://www.statsci.org/data/oz/sydrain.html}

Lee, A. (1994). \emph{Data analysis: An introduction based on r. Auckland}.
Retrieved from \url{http://www.statsci.org/data/oz/nzrivers.html}

\emph{Life expectancy in southeast Asia}. (2013, September 23). Retrieved
from \url{http://apps.who.int/gho/data/node.main.688}

Olson, K., \& Hanson, J. (1997). Using reiki to manage pain: a
preliminary report. \emph{Cancer Prev Control}, \emph{1}(2), 108-13. Retrieved
from \url{http://www.ncbi.nlm.nih.gov/pubmed/9765732}

\emph{Pulse rates before and after exercise}. (2013, September 25). Retrieved
from \url{http://www.statsci.org/data/oz/ms212.html}

\emph{Reserve bank of Australia}. (2013, September 23). Retrieved from
\url{http://data.gov.au/dataset/banks-assets}

Ryan, B. F., Joiner, B. L., \& Ryan, Jr, T. A. (1985). \emph{Cholesterol
levels after heart attack}. Retrieved from
\url{http://www.statsci.org/data/general/cholest.html}

\emph{Time between nerve pulses}. (2013, September 25). Retrieved from
\url{http://www.statsci.org/data/general/nerve.html}

\emph{Time of passages of play in rugby}. (2013, September 25). Retrieved
from \url{http://www.statsci.org/data/oz/rugby.html}

\emph{U.S. tornado climatology}. (17, May 2013). Retrieved from
\url{http://www.ncdc.noaa.gov/oa/climate/severeweather/tornadoes.html}

\emph{UV radiation: Burden of disease by country}. (2013, September 4).
Retrieved from \url{http://apps.who.int/gho/data/node.main.165?lang=en}

\hypertarget{probability}{%
\chapter{Probability}\label{probability}}

{[}NOTE IN DRAFT: Often you'll want to have an introductory paragraph before breaking into sub-sections{]}

\hypertarget{empirical-probability}{%
\section{Empirical Probability}\label{empirical-probability}}

One story about how probability theory was developed is that a gambler wanted to know when to bet more and when to bet less. He talked to a couple of friends of his that happened to be mathematicians. Their names were Pierre de Fermat and Blaise Pascal. Since then many other mathematicians have worked to develop probability theory.

Understanding probabilities are important in life. Examples of mundane questions that probability can answer for you are if you need to carry an umbrella or wear a heavy coat on a given day. More important questions that probability can help with are your chances that the car you are buying will need more maintenance, your chances of passing a class, your chances of winning the lottery, your chances of being in a car accident, and the chances that the U.S. will be attacked by terrorists. Most people do not have a very good understanding of probability, so they worry about being attacked by a terrorist but not about being in a car accident. The probability of being in a terrorist attack is much smaller than the probability of being in a car accident, thus it actually would make more sense to worry about driving. Also, the chance of you winning the lottery is very small, yet many people will spend the money on lottery tickets. Yet, if instead they saved the money that they spend on the lottery, they would have more money. In general, events that have a low probability (under 5\%) are unlikely to occur. Whereas if an event has a high probability of happening (over 80\%), then there is a good chance that the event will happen. This chapter will present some of the theory that you need to help make a determination of whether an event is likely to happen or not.

First you need some definitions.

\textbf{Experiment:} an activity that has specific results that can occur, but it is unknown which results will occur.

\textbf{Outcomes:} the results of an experiment

\textbf{Event:} a set of certain outcomes of an experiment that you want to have happen

\textbf{Sample Space:} collection of all possible outcomes of the experiment. Usually denoted as SS.

\textbf{Event space:} the set of outcomes that make up an event. The symbol is usually a capital letter.

Start with an experiment. Suppose that the experiment is rolling a die. The sample space is \{1, 2, 3, 4, 5, 6\}. The event that you want is to get a 6, and the event space is \{6\}. To do this, roll a die 10 times. When you do that, you get a 6 two times. Based on this experiment, the probability of getting a 6 is 2 out of 10 or 1/5. To get more accuracy, repeat the experiment more times. It is easiest to put this in a table, where \emph{n} represents the number of times the experiment is repeated. When you put the number of 6s found over the number of times you repeat the experiment, this is the relative frequency.

\textbf{Table \#4.1.1: Trials for Die Experiment}

\begin{longtable}[]{@{}lll@{}}
\toprule
\emph{n}Numb & er of 6s Rela & tive Frequency\tabularnewline
\midrule
\endhead
10 2 0 & .2 &\tabularnewline
50 6 0 & .12 &\tabularnewline
10018 0 & .18 &\tabularnewline
50081 0 & .162 &\tabularnewline
1000 & 1630.163 &\tabularnewline
\bottomrule
\end{longtable}

Notice that as \emph{n} increased, the relative frequency seems to approach a
number. It looks like it is approaching 0.163. You can say that the
probability of getting a 6 is approximately 0.163. If you want more
accuracy, then increase \emph{n} even more.

These probabilities are called \textbf{experimental probabilities} since they
are found by actually doing the experiment. They come about from the
relative frequencies and give an approximation of the true probability.
The approximate probability of an event \emph{A}, \emph{,} is

\textbf{Experimental Probabilities}

For the event of getting a 6, the probability would be .

You must do experimental probabilities whenever it is not possible to
calculate probabilities using other means. An example is if you want to
find the probability that a family has 5 children, you would have to
actually look at many families, and count how many have 5 children. Then
you could calculate the probability. Another example is if you want to
figure out if a die is fair. You would have to roll the die many times
and count how often each side comes up. Make sure you repeat an
experiment many times, because otherwise you will not be able to
estimate the true probability. This is due to the law of large numbers.

\textbf{Law of large numbers}: as \emph{n} increases, the relative frequency tends
towards the actual probability value.

Note: probability, relative frequency, percentage, and proportion are all different words for the same concept. Also, probabilities can be given as percentages, decimals, or fractions.

\hypertarget{homework-10}{%
\subsection{Homework}\label{homework-10}}

\begin{enumerate}
\def\labelenumi{\arabic{enumi}.}
\tightlist
\item
  Table \#4.1.2 contains the number of M\&M's of each color that were
  found in a case (Madison, 2013).
\end{enumerate}

\textbf{Table \#4.1.2: M\&M Distribution}

\begin{longtable}[]{@{}lllllll@{}}
\toprule
Blue & Brown & Green & Orange & Red & Yellow & Total\tabularnewline
\midrule
\endhead
481371 & 483 544 & 372 3 & 69 2620 & & &\tabularnewline
\bottomrule
\end{longtable}

Find the probability of choosing each color based on this experiment.

\begin{enumerate}
\def\labelenumi{\arabic{enumi}.}
\setcounter{enumi}{1}
\tightlist
\item
  Eyeglassomatic manufactures eyeglasses for different retailers. They
  test to see how many defective lenses they made the time period of
  January 1 to March 31. Table \#4.1.3 gives the defect and the number
  of defects.
\end{enumerate}

\textbf{Table \#4.1.3: Number of Defective Lenses}

\begin{longtable}[]{@{}ll@{}}
\toprule
Defect type Number of de & fects\tabularnewline
\midrule
\endhead
Scratch 5865 &\tabularnewline
Right shaped -- small4613 &\tabularnewline
Flaked 1992 &\tabularnewline
Wrong axis 1838 &\tabularnewline
Chamfer wrong1596 &\tabularnewline
Crazing, cracks 1546 &\tabularnewline
Wrong shape 1485 &\tabularnewline
Wrong PD 1398 &\tabularnewline
Spots and bubbles1371 &\tabularnewline
Wrong height 1130 &\tabularnewline
Right shape -- big 1105 &\tabularnewline
Lost in lab 976 &\tabularnewline
Spots/bubble -- intern & 976\tabularnewline
\bottomrule
\end{longtable}

Find the probability of each defect type based on this data.

\begin{enumerate}
\def\labelenumi{\arabic{enumi}.}
\setcounter{enumi}{2}
\item
  In Australia in 1995, of the 2907 indigenous people in prison 17 of
  them died. In that same year, of the 14501 non-indigenous people in
  prison 42 of them died ("Aboriginal deaths in," 2013). Find the
  probability that an indigenous person dies in prison and the
  probability that a non-indigenous person dies in prison. Compare
  these numbers and discuss what the numbers may mean.
\item
  A project conducted by the Australian Federal Office of Road Safety
  asked people many questions about their cars. One question was the
  reason that a person chooses a given car, and that data is in table
  \#4.1.4 ("Car preferences," 2013).
\end{enumerate}

\begin{quote}
\textbf{Table \#4.1.4: Reason for Choosing a Car}
\end{quote}

\begin{longtable}[]{@{}llllll@{}}
\toprule
Safety & Reliability & Cost & Performance & Comfort & Looks\tabularnewline
\midrule
\endhead
84 6246 & 344727 & & & &\tabularnewline
\bottomrule
\end{longtable}

Find the probability a person chooses a car for each of the given
reasons.

\textbf{\\
}

\hypertarget{theoretical-probability}{%
\section{Theoretical Probability}\label{theoretical-probability}}

It is not always feasible to conduct an experiment over and over again, so it would be better to be able to find the probabilities without conducting the experiment. These probabilities are called \textbf{Theoretical Probabilities.}

To be able to do theoretical probabilities, there is an assumption that you need to consider. It is that all of the outcomes in the sample space need to be \textbf{equally likely outcomes}. This means that every outcome of the experiment needs to have the same chance of happening.

\textbf{Example \#4.2.1: Equally Likely Outcomes}

\begin{quote}
Which of the following experiments have equally likely outcomes?
\end{quote}

\begin{enumerate}
\def\labelenumi{\alph{enumi}.}
\tightlist
\item
  Rolling a fair die.
\end{enumerate}

\textbf{Solution:}

\begin{quote}
Since the die is fair, every side of the die has the same chance of
coming up. The outcomes are the different sides, so each outcome is
equally likely
\end{quote}

\begin{enumerate}
\def\labelenumi{\alph{enumi}.}
\setcounter{enumi}{1}
\tightlist
\item
  Flip a coin that is weighted so one side comes up more often than
  the other.
\end{enumerate}

\textbf{Solution:}

Since the coin is weighted, one side is more likely to come up than
the other side. The outcomes are the different sides, so each
outcome is not equally likely

\begin{enumerate}
\def\labelenumi{\alph{enumi}.}
\setcounter{enumi}{2}
\tightlist
\item
  Pull a ball out of a can containing 6 red balls and 8 green balls.
  All balls are the same size.
\end{enumerate}

\textbf{Solution:}

Since each ball is the same size, then each ball has the same chance
of being chosen. The outcomes of this experiment are the individual
balls, so each outcome is equally likely. Don't assume that because
the chances of pulling a red ball are less than pulling a green ball
that the outcomes are not equally likely. The outcomes are the
individual balls and they are equally likely.

\begin{enumerate}
\def\labelenumi{\alph{enumi}.}
\setcounter{enumi}{3}
\tightlist
\item
  Picking a card from a deck.
\end{enumerate}

\textbf{Solution:}

If you assume that the deck is fair, then each card has the same
chance of being chosen. Thus the outcomes are equally likely
outcomes. You do have to make this assumption. For many of the
experiments you will do, you do have to make this kind of
assumption.

\begin{enumerate}
\def\labelenumi{\alph{enumi}.}
\setcounter{enumi}{4}
\tightlist
\item
  Rolling a die to see if it is fair.
\end{enumerate}

\textbf{Solution:}

In this case you are not sure the die is fair. The only way to
determine if it is fair is to actually conduct the experiment, since
you don't know if the outcomes are equally likely. If the
experimental probabilities are fairly close to the theoretical
probabilities, then the die is fair.

If the outcomes are not equally likely, then you must do experimental
probabilities. If the outcomes are equally likely, then you can do
theoretical probabilities.

\textbf{Theoretical Probabilities:} If the outcomes of an experiment are
equally likely, then the probability of event A happening is

\textbf{Example \#4.2.2: Calculating Theoretical Probabilities}

\begin{quote}
Suppose you conduct an experiment where you flip a fair coin twice
\end{quote}

\begin{enumerate}
\def\labelenumi{\alph{enumi}.}
\tightlist
\item
  What is the sample space?
\end{enumerate}

\begin{quote}
\textbf{Solution:}

There are several different sample spaces you can do. One is

SS=\{0, 1, 2\} where you are counting the number of heads. However, the
outcomes are not equally likely since you can get one head by getting
a head on the first flip and a tail on the second or a tail on the
first flip and a head on the second. There are 2 ways to get that
outcome and only one way to get the other outcomes. Instead it might
be better to give the sample space as listing what can happen on each
flip. Let H = head and T = tail, and list which can happen on each
flip.

SS=\{HH, HT, TH, TT\}
\end{quote}

\begin{enumerate}
\def\labelenumi{\alph{enumi}.}
\setcounter{enumi}{1}
\tightlist
\item
  What is the probability of getting exactly one head?
\end{enumerate}

\textbf{Solution:}

\begin{quote}
Let A = getting exactly one head. The event space is A = \{HT, TH\}. So

.

It may not be advantageous to reduce the fractions to lowest terms,
since it is easier to compare fractions if they have the same
denominator.
\end{quote}

\begin{enumerate}
\def\labelenumi{\alph{enumi}.}
\setcounter{enumi}{2}
\tightlist
\item
  What is the probability of getting at least one head?
\end{enumerate}

\textbf{Solution:}

Let B = getting at least one head. At least one head means get one
or more. The event space is B = \{HT, TH, HH\} and

Since P(B) is greater than the P(A), then event B is more likely to
happen than event A.

\begin{enumerate}
\def\labelenumi{\alph{enumi}.}
\setcounter{enumi}{3}
\tightlist
\item
  What is the probability of getting a head and a tail?
\end{enumerate}

\textbf{Solution:}

Let C = getting a head and a tail = \{HT, TH\} and

This is the same event space as event A, but it is a different
event. Sometimes two different events can give the same event space.

\begin{enumerate}
\def\labelenumi{\alph{enumi}.}
\setcounter{enumi}{4}
\tightlist
\item
  What is the probability of getting a head or a tail?
\end{enumerate}

\textbf{Solution:}

\begin{quote}
Let D = getting a head or a tail. Since or means one or the other or
both and it doesn't specify the number of heads or tails, then D =
\{HH, HT, TH, TT\} and
\end{quote}

\begin{enumerate}
\def\labelenumi{\alph{enumi}.}
\setcounter{enumi}{5}
\tightlist
\item
  What is the probability of getting a foot?
\end{enumerate}

\textbf{Solution:}

Let E = getting a foot. Since you can't get a foot, E = \{\} or the
empty set and

\begin{enumerate}
\def\labelenumi{\alph{enumi}.}
\setcounter{enumi}{6}
\tightlist
\item
  What is the probability of each outcome? What is the sum of these
  probabilities?
\end{enumerate}

\textbf{Solution:}

. If you add all of these probabilities together you get 1.

This example had some results in it that are important concepts. They
are summarized below:

\begin{quote}
\textbf{Probability Properties}

\begin{enumerate}
\def\labelenumi{\arabic{enumi}.}
\item
\end{enumerate}

2. If the P(event) = 1, then it will happen and is called the certain
event

3. If the P(event) = 0, then it cannot happen and is called the
impossible event

\begin{enumerate}
\def\labelenumi{\arabic{enumi}.}
\setcounter{enumi}{3}
\item
\end{enumerate}
\end{quote}

\textbf{Example \#4.2.3: Calculating Theoretical Probabilities}

\begin{quote}
Suppose you conduct an experiment where you pull a card from a
standard deck.
\end{quote}

\begin{enumerate}
\def\labelenumi{\alph{enumi}.}
\tightlist
\item
  What is the sample space?
\end{enumerate}

\textbf{Solution:}

SS = \{2S, 3S, 4S, 5S, 6S, 7S, 8S, 9S, 10S, JS, QS, KS, AS, 2C, 3C,
4C, 5C, 6C, 7C, 8C, 9C, 10C, JC, QC, KC, AC, 2D, 3D, 4D, 5D, 6D, 7D,
8D, 9D, 10D, JD, QD, KD, AD, 2H, 3H, 4H, 5H, 6H, 7H, 8H, 9H, 10H,
JH, QH, KH, AH\}

\begin{enumerate}
\def\labelenumi{\alph{enumi}.}
\setcounter{enumi}{1}
\tightlist
\item
  What is the probability of getting a Spade?
\end{enumerate}

\textbf{Solution:}

\begin{quote}
Let A = getting a spade = \{2S, 3S, 4S, 5S, 6S, 7S, 8S, 9S, 10S, JS,
QS, KS, AS\} so
\end{quote}

\begin{enumerate}
\def\labelenumi{\alph{enumi}.}
\setcounter{enumi}{2}
\tightlist
\item
  What is the probability of getting a Jack?
\end{enumerate}

\textbf{Solution:}

Let B = getting a Jack = \{JS, JC, JH, JD\} so

\begin{enumerate}
\def\labelenumi{\alph{enumi}.}
\setcounter{enumi}{3}
\tightlist
\item
  What is the probability of getting an Ace?
\end{enumerate}

\textbf{Solution:}

\begin{quote}
Let C = getting an Ace = \{AS, AC, AH, AD\} so
\end{quote}

\begin{enumerate}
\def\labelenumi{\alph{enumi}.}
\setcounter{enumi}{4}
\tightlist
\item
  What is the probability of not getting an Ace?
\end{enumerate}

\textbf{Solution:}

Let D = not getting an Ace = \{2S, 3S, 4S, 5S, 6S, 7S, 8S, 9S, 10S,
JS, QS, KS, 2C, 3C, 4C, 5C, 6C, 7C, 8C, 9C, 10C, JC, QC, KC, 2D, 3D,
4D, 5D, 6D, 7D, 8D, 9D, 10D, JD, QD, KD, 2H, 3H, 4H, 5H, 6H, 7H, 8H,
9H, 10H, JH, QH, KH\} so

Notice, , so you could have found the probability of D by doing 1
minus the probability of C

\begin{enumerate}
\def\labelenumi{\alph{enumi}.}
\setcounter{enumi}{5}
\tightlist
\item
  What is the probability of getting a Spade and an Ace?
\end{enumerate}

\textbf{Solution:}

Let E = getting a Spade and an Ace = \{AS\} so

\begin{enumerate}
\def\labelenumi{\alph{enumi}.}
\setcounter{enumi}{6}
\tightlist
\item
  What is the probability of getting a Spade or an Ace?
\end{enumerate}

\textbf{Solution:}

Let F = getting a Spade and an Ace =\{2S, 3S, 4S, 5S, 6S, 7S, 8S, 9S,
10S, JS, QS, KS, AS, AC, AD, AH\} so

\begin{enumerate}
\def\labelenumi{\alph{enumi}.}
\setcounter{enumi}{7}
\tightlist
\item
  What is the probability of getting a Jack and an Ace?
\end{enumerate}

\textbf{Solution:}

Let G = getting a Jack and an Ace = \{ \} since you can't do that with
one card. So

\begin{enumerate}
\def\labelenumi{\roman{enumi}.}
\tightlist
\item
  What is the probability of getting a Jack or an Ace?
\end{enumerate}

\textbf{Solution:}

\begin{quote}
Let H = getting a Jack or an Ace = \{JS, JC, JD, JH, AS, AC, AD, AH\} so
\end{quote}

\textbf{\\
}

\textbf{Example \#4.2.4: Calculating Theoretical Probabilities}

\begin{quote}
Suppose you have an iPod Shuffle with the following songs on it: 5
Rolling Stones songs, 7 Beatles songs, 9 Bob Dylan songs, 4 Faith Hill
songs, 2 Taylor Swift songs, 7 U2 songs, 4 Mariah Carey songs, 7 Bob
Marley songs, 6 Bunny Wailer songs, 7 Elton John songs, 5 Led Zeppelin
songs, and 4 Dave Mathews Band songs. The different genre that you
have are rock from the 60s which includes Rolling Stones, Beatles, and
Bob Dylan; country includes Faith Hill and Taylor Swift; rock of the
90s includes U2 and Mariah Carey; Reggae includes Bob Marley and Bunny
Wailer; rock of the 70s includes Elton John and Led Zeppelin; and
bluegrasss/rock includes Dave Mathews Band.

The way an iPod Shuffle works, is it randomly picks the next song so
you have no idea what the next song will be. Now you would like to
calculate the probability that you will hear the type of music or the
artist that you are interested in. The sample set is too difficult to
write out, but you can figure it from looking at the number in each
set and the total number. The total number of songs you have is 67.
\end{quote}

\begin{enumerate}
\def\labelenumi{\alph{enumi}.}
\tightlist
\item
  What is the probability that you will hear a Faith Hill song?
\end{enumerate}

\textbf{Solution:}

There are 4 Faith Hill songs out of the 67 songs, so

\begin{enumerate}
\def\labelenumi{\alph{enumi}.}
\setcounter{enumi}{1}
\tightlist
\item
  What is the probability that you will hear a Bunny Wailer song?
\end{enumerate}

\textbf{Solution:}

There are 6 Bunny Wailer songs, so

\begin{enumerate}
\def\labelenumi{\alph{enumi}.}
\setcounter{enumi}{2}
\tightlist
\item
  What is the probability that you will hear a song from the 60s?
\end{enumerate}

\textbf{Solution:}

\begin{quote}
There are 5, 7, and 9 songs that are classified as rock from the 60s,
which is 21 total, so
\end{quote}

\begin{enumerate}
\def\labelenumi{\alph{enumi}.}
\setcounter{enumi}{3}
\tightlist
\item
  What is the probability that you will hear a Reggae song?
\end{enumerate}

\textbf{Solution:}

There are 6 and 7 songs that are classified as Reggae, which is 13
total, so

\begin{enumerate}
\def\labelenumi{\alph{enumi}.}
\setcounter{enumi}{4}
\tightlist
\item
  What is the probability that you will hear a song from the 90s or a
  bluegrass/rock song?
\end{enumerate}

\textbf{Solution:}

There are 7 and 4 songs that are songs from the 90s and 4 songs that
are bluegrass/rock, for a total of 15, so

\begin{enumerate}
\def\labelenumi{\alph{enumi}.}
\setcounter{enumi}{5}
\tightlist
\item
  What is the probability that you will hear an Elton John or a Taylor
  Swift song?
\end{enumerate}

\textbf{Solution:}

There are 7 Elton John songs and 2 Taylor Swift songs, for a total
of 9, so

\begin{enumerate}
\def\labelenumi{\alph{enumi}.}
\setcounter{enumi}{6}
\tightlist
\item
  What is the probability that you will hear a country song or a U2
  song?
\end{enumerate}

\textbf{Solution:}

There are 6 country songs and 7 U2 songs, for a total of 13, so

\begin{quote}
Of course you can do any other combinations you would like.
\end{quote}

Notice in example \#4.2.3 part e, it was mentioned that the probability
of event D plus the probability of event C was 1. This is because these
two events have no outcomes in common, and together they make up the
entire sample space. Events that have this property are called
\textbf{complementary events}.

If two events are \textbf{complementary events} then to find the probability
of one just subtract the probability of the other from one. Notation
used for complement of A is not A or , or .

\textbf{\\
}

\textbf{Example \#4.2.5: Complementary Events}

\begin{enumerate}
\def\labelenumi{\alph{enumi}.}
\tightlist
\item
  Suppose you know that the probability of it raining today is 0.45.
  What is the probability of it not raining?
\end{enumerate}

\textbf{Solution:}

\begin{quote}
Since not raining is the complement of raining, then
\end{quote}

\begin{enumerate}
\def\labelenumi{\alph{enumi}.}
\setcounter{enumi}{1}
\tightlist
\item
  Suppose you know the probability of not getting the flu is 0.24.
  What is the probability of getting the flu?
\end{enumerate}

\textbf{Solution:}

\begin{quote}
Since getting the flu is the complement of not getting the flu, then
\end{quote}

\begin{enumerate}
\def\labelenumi{\alph{enumi}.}
\setcounter{enumi}{2}
\tightlist
\item
  In an experiment of picking a card from a deck, what is the
  probability of not getting a card that is a Queen?
\end{enumerate}

\textbf{Solution:}

You could do this problem by listing all the ways to not get a
queen, but that set is fairly large. One advantage of the complement
is that it reduces the workload. You use the complement in many
situations to make the work shorter and easier. In this case it is
easier to list all the ways to get a Queen, find the probability of
the Queen, and then subtract from one.

Queen = \{QS, QC, QD, QH\} so

and

The complement is useful when you are trying to find the probability of
an event that involves the words at least or an event that involves the
words at most. As an example of an at least event is suppose you want to
find the probability of making at least \$50,000 when you graduate from
college. That means you want the probability of your salary being
greater than or equal to \$50,000. An example of an at most event is
suppose you want to find the probability of rolling a die and getting at
most a 4. That means that you want to get less than or equal to a 4 on
the die. The reason to use the complement is that sometimes it is easier
to find the probability of the complement and then subtract from 1.
Example \#4.2.6 demonstrates how to do this.

\textbf{\\
}

\textbf{Example \#4.2.6: Using the Complement to Find Probabilities}

\begin{enumerate}
\def\labelenumi{\alph{enumi}.}
\tightlist
\item
  In an experiment of rolling a fair die one time, find the
  probability of rolling at most a 4 on the die.
\end{enumerate}

\textbf{Solution:}

The sample space for this experiment is \{1, 2, 3, 4, 5, 6\}. You want
the event of getting at most a 4, which is the same as thinking of
getting 4 or less. The event space is \{1, 2, 3, 4\}. The probability
is

Or you could have used the complement. The complement of rolling at
most a 4 would be rolling number bigger than 4. The event space for
the complement is \{5, 6\}. The probability of the complement is . The
probability of at most 4 would be

Notice you have the same answer, but the event space was easier to
write out. On this example it probability wasn't that useful, but in
the future there will be events where it is much easier to use the
complement.

\begin{enumerate}
\def\labelenumi{\alph{enumi}.}
\setcounter{enumi}{1}
\tightlist
\item
  In an experiment of pulling a card from a fair deck, find the
  probability of pulling at least a 5 (ace is a high card in this
  example).
\end{enumerate}

\textbf{Solution:}

The sample space for this experiment is

SS = \{2S, 3S, 4S, 5S, 6S, 7S, 8S, 9S, 10S, JS, QS, KS, AS, 2C, 3C,
4C, 5C, 6C, 7C, 8C, 9C, 10C, JC, QC, KC, AC, 2D, 3D, 4D, 5D, 6D, 7D,
8D, 9D, 10D, JD, QD, KD, AD, 2H, 3H, 4H, 5H, 6H, 7H, 8H, 9H, 10H,
JH, QH, KH, AH\}

Pulling a card that is at least a 5 would involve listing all of the
cards that are a 5 or more. It would be much easier to list the
outcomes that make up the complement. The complement of at least a 5
is less than a 5. That would be the event of 4 or less. The event
space for the complement would be \{2S, 3S, 4S, 2C, 3C, 4C, 2D, 3D,
4D, 2H, 3H, 4H\}. The probability of the complement would be . The
probability of at least a 5 would be

Another concept was show in example \#4.2.3 parts g and i. The problems
were looking for the probability of one event or another. In part g, it
was looking for the probability of getting a Spade or an Ace. That was
equal to . In part i, it was looking for the probability of getting a
Jack or an Ace. That was equal to . If you look back at the parts b, c,
and d, you might notice the following result:

\begin{quote}
but .
\end{quote}

Why does adding two individual probabilities together work in one
situation to give the probability of one or another event and not give
the correct probability in the other?

The reason this is true in the case of the Jack and the Ace is that
these two events cannot happen together. There is no overlap between the
two events, and in fact the \includegraphics{media/image46.emf}. However, in the case
of the Spade and Ace, they can happen together. There is overlap, mainly
the ace of spades. The .

When two events cannot happen at the same time, they are called
\textbf{mutually exclusive}. In the above situation, the events Jack and Ace
are mutually exclusive, while the events Spade and Ace are not mutually
exclusive.

\textbf{Addition Rules:}

If two events A and B are mutually exclusive, then

and

If two events A and B are not mutually exclusive, then

\textbf{Example \#4.2.7: Using Addition Rules}

\begin{quote}
Suppose your experiment is to roll two fair dice.
\end{quote}

\begin{enumerate}
\def\labelenumi{\alph{enumi}.}
\tightlist
\item
  What is the sample space?
\end{enumerate}

\textbf{Solution:}

As with the other examples you need to come up with a sample space
that has equally likely outcomes. One sample space is to list the
sums possible on each roll. That sample space would look like: SS =
\{2, 3, 4, 5, 6, 7, 8, 9, 10, 11, 12\}. However, there are more ways
to get a sum of 7 then there are to get a sum of 2, so these
outcomes are not equally likely. Another thought is to list the
possibilities on each roll. As an example you could roll the dice
and on the first die you could get a 1. The other die could be any
number between 1 and 6, but say it is a 1 also. Then this outcome
would look like (1,1). Similarly, you could get (1, 2), (1, 3),
(1,4), (1, 5), or (1, 6). Also, you could get a 2, 3, 4, 5, or 6 on
the first die instead. Putting this all together, you get the sample
space:

SS = \{(1,1), (1,2), (1,3), (1,4), (1,5), (1,6),

(2,1), (2,2), (2,3), (2,4), (2,5), (2,6),

(3,1), (3,2), (3,3), (3,4), (3,5), (3,6),

(4,1), (4,2), (4,3), (4,4), (4,5), (4,6),

(5,1), (5,2), (5,3), (5,4), (5,5), (5,6),

(6,1), (6,2), (6,3), (6,4), (6,5), (6,6)\}

Notice that a (2,3) is different from a (3,2), since the order that
you roll the die is important and you can tell the difference
between these two outcomes. You don't need any of the doubles twice,
since these are not distinguishable from each other in either order.

This will always be the sample space for rolling two dice.

\begin{enumerate}
\def\labelenumi{\alph{enumi}.}
\setcounter{enumi}{1}
\tightlist
\item
  What is the probability of getting a sum of 5?
\end{enumerate}

\textbf{Solution:}

\begin{quote}
Let A = getting a sum of 5 = \{(4,1), (3,2), (2,3), (1,4)\} so
\end{quote}

\begin{enumerate}
\def\labelenumi{\alph{enumi}.}
\setcounter{enumi}{2}
\tightlist
\item
  What is the probability of getting the first die a 2?
\end{enumerate}

\textbf{Solution:}

Let B = getting first die a 2 = \{(2,1), (2,2), (2,3), (2,4), (2,5),
(2,6)\} so

\begin{enumerate}
\def\labelenumi{\alph{enumi}.}
\setcounter{enumi}{3}
\tightlist
\item
  What is the probability of getting a sum of 7?
\end{enumerate}

\textbf{Solution:}

Let C = getting a sum of 7 = \{(6,1), (5,2), (4,3), (3,4), (2,5),
(1,6)\} so

\begin{enumerate}
\def\labelenumi{\alph{enumi}.}
\setcounter{enumi}{4}
\tightlist
\item
  What is the probability of getting a sum of 5 and the first die a 2?
\end{enumerate}

\textbf{Solution:}

This is events A and B which contains the outcome \{(2,3)\} so

\begin{enumerate}
\def\labelenumi{\alph{enumi}.}
\setcounter{enumi}{5}
\tightlist
\item
  What is the probability of getting a sum of 5 or the first die a 2?
\end{enumerate}

\textbf{Solution:}

Notice from part e, that these two events are not mutually
exclusive, so

\begin{enumerate}
\def\labelenumi{\alph{enumi}.}
\setcounter{enumi}{6}
\tightlist
\item
  What is the probability of getting a sum of 5 and sum of 7?
\end{enumerate}

\textbf{Solution:}

These are the events A and C, which have no outcomes in common. Thus
A and C = \{ \} so

\begin{enumerate}
\def\labelenumi{\alph{enumi}.}
\setcounter{enumi}{7}
\tightlist
\item
  What is the probability of getting a sum of 5 or sum of 7?
\end{enumerate}

\textbf{Solution:}

From part g, these two events are mutually exclusive, so

\textbf{Odds}

Many people like to talk about the odds of something happening or not
happening. Mathematicians, statisticians, and scientists prefer to deal
with probabilities since odds are difficult to work with, but gamblers
prefer to work in odds for figuring out how much they are paid if they
win.

The \textbf{actual odds against} event \emph{A} occurring are the ratio , usually
expressed in the form \emph{a:b} or \emph{a} to \emph{b}, where \emph{a} and \emph{b} are
integers with no common factors.

The \textbf{actual odds in favor} event \emph{A} occurring are the ratio , which
is the reciprocal of the odds against. If the odds against event \emph{A} are
\emph{a:b}, then the odds in favor event \emph{A} are \emph{b:a}.

The \textbf{payoff odds} against event \emph{A} occurring are the ratio of the net
profit (if you win) to the amount bet.

payoff odds against event \emph{A} = (net profit) : (amount bet)

\textbf{\\
}

\textbf{Example \#4.2.8: Odds Against and Payoff Odds}

In the game of Craps, if a shooter has a come-out roll of a 7 or an 11,
it is called a natural and the pass line wins. The payoff odds are given
by a casino as 1:1.

\begin{enumerate}
\def\labelenumi{\alph{enumi}.}
\tightlist
\item
  Find the probability of a natural.
\end{enumerate}

\textbf{Solution:}

\begin{quote}
A natural is a 7 or 11. The sample space is
\end{quote}

SS = \{(1,1), (1,2), (1,3), (1,4), (1,5), (1,6),

(2,1), (2,2), (2,3), (2,4), (2,5), (2,6),

(3,1), (3,2), (3,3), (3,4), (3,5), (3,6),

(4,1), (4,2), (4,3), (4,4), (4,5), (4,6),

(5,1), (5,2), (5,3), (5,4), (5,5), (5,6),

\begin{quote}
(6,1), (6,2), (6,3), (6,4), (6,5), (6,6)\}
\end{quote}

The event space is \{(1,6), (2,5), (3,4), (4,3), (5,2), (6,1), (5,6),
(6,5)\}

\begin{quote}
So
\end{quote}

\begin{enumerate}
\def\labelenumi{\alph{enumi}.}
\setcounter{enumi}{1}
\tightlist
\item
  Find the actual odds for a natural.
\end{enumerate}

\textbf{Solution:}

\begin{enumerate}
\def\labelenumi{\alph{enumi}.}
\setcounter{enumi}{2}
\tightlist
\item
  Find the actual odds against a natural.
\end{enumerate}

\textbf{Solution:}

\begin{enumerate}
\def\labelenumi{\alph{enumi}.}
\setcounter{enumi}{3}
\tightlist
\item
  If the casino pays 1:1, how much profit does the casino make on a
  \$10 bet?
\end{enumerate}

\textbf{Solution:}

\begin{quote}
The actual odds are 3.5 to 1 while the payoff odds are 1 to 1. The
casino pays you \$10 for your \$10 bet. If the casino paid you the
actual odds, they would pay \$3.50 on every \$1 bet, and on \$10, they
pay . Their profit is .
\end{quote}

\hypertarget{homework-11}{%
\subsection{Homework}\label{homework-11}}

\begin{enumerate}
\def\labelenumi{\arabic{enumi}.}
\tightlist
\item
  Table \#4.2.1 contains the number of M\&M's of each color that were
  found in a case (Madison, 2013).
\end{enumerate}

\textbf{Table \#4.2.1: M\&M Distribution}

\begin{longtable}[]{@{}lllllll@{}}
\toprule
Blue & Brown & Green & Orange & Red & Yellow & Total\tabularnewline
\midrule
\endhead
481371 & 483 544 & 372 3 & 69 2620 & & &\tabularnewline
\bottomrule
\end{longtable}

\begin{enumerate}
\def\labelenumi{\alph{enumi}.}
\item
  Find the probability of choosing a green or red M\&M.
\item
  Find the probability of choosing a blue, red, or yellow M\&M.
\item
  Find the probability of not choosing a brown M\&M.
\item
  Find the probability of not choosing a green M\&M.
\end{enumerate}

\begin{enumerate}
\def\labelenumi{\arabic{enumi}.}
\setcounter{enumi}{1}
\tightlist
\item
  Eyeglassomatic manufactures eyeglasses for different retailers. They
  test to see how many defective lenses they made in a time period.
  Table \#4.2.2 gives the defect and the number of defects.
\end{enumerate}

\textbf{Table \#4.2.2: Number of Defective Lenses}

\begin{longtable}[]{@{}ll@{}}
\toprule
Defect type Number of de & fects\tabularnewline
\midrule
\endhead
Scratch 5865 &\tabularnewline
Right shaped -- small4613 &\tabularnewline
Flaked 1992 &\tabularnewline
Wrong axis 1838 &\tabularnewline
Chamfer wrong1596 &\tabularnewline
Crazing, cracks 1546 &\tabularnewline
Wrong shape 1485 &\tabularnewline
Wrong PD 1398 &\tabularnewline
Spots and bubbles1371 &\tabularnewline
Wrong height 1130 &\tabularnewline
Right shape -- big 1105 &\tabularnewline
Lost in lab 976 &\tabularnewline
Spots/bubble -- intern & 976\tabularnewline
\bottomrule
\end{longtable}

\begin{enumerate}
\def\labelenumi{\alph{enumi}.}
\item
  Find the probability of picking a lens that is scratched or flaked.
\item
  Find the probability of picking a lens that is the wrong PD or was
  lost in lab.
\item
  Find the probability of picking a lens that is not scratched.
\item
  Find the probability of picking a lens that is not the wrong shape.
\end{enumerate}

\begin{enumerate}
\def\labelenumi{\arabic{enumi}.}
\setcounter{enumi}{2}
\tightlist
\item
  An experiment is to flip a fair coin three times.
\end{enumerate}

\begin{enumerate}
\def\labelenumi{\alph{enumi}.}
\item
  State the sample space.
\item
  Find the probability of getting exactly two heads. Make sure you
  state the event space.
\item
  Find the probability of getting at least two heads. Make sure you
  state the event space.
\item
  Find the probability of getting an odd number of heads. Make sure
  you state the event space.
\item
  Find the probability of getting all heads or all tails. Make sure
  you state the event space.
\item
  Find the probability of getting exactly two heads or exactly two
  tails.
\item
  Find the probability of not getting an odd number of heads.
\end{enumerate}

\begin{enumerate}
\def\labelenumi{\arabic{enumi}.}
\setcounter{enumi}{3}
\tightlist
\item
  An experiment is rolling a fair die and then flipping a fair coin.
\end{enumerate}

\begin{enumerate}
\def\labelenumi{\alph{enumi}.}
\item
  State the sample space.
\item
  Find the probability of getting a head. Make sure you state the
  event space.
\item
  Find the probability of getting a 6. Make sure you state the event
  space.
\item
  Find the probability of getting a 6 or a head.
\item
  Find the probability of getting a 3 and a tail.
\end{enumerate}

\begin{enumerate}
\def\labelenumi{\arabic{enumi}.}
\setcounter{enumi}{4}
\tightlist
\item
  An experiment is rolling two fair dice.
\end{enumerate}

\begin{enumerate}
\def\labelenumi{\alph{enumi}.}
\item
  State the sample space.
\item
  Find the probability of getting a sum of 3. Make sure you state the
  event space.
\item
  Find the probability of getting the first die is a 4. Make sure you
  state the event space.
\item
  Find the probability of getting a sum of 8. Make sure you state the
  event space.
\item
  Find the probability of getting a sum of 3 or sum of 8.
\item
  Find the probability of getting a sum of 3 or the first die is a 4.
\item
  Find the probability of getting a sum of 8 or the first die is a 4.
\item
  Find the probability of not getting a sum of 8.
\end{enumerate}

\begin{enumerate}
\def\labelenumi{\arabic{enumi}.}
\setcounter{enumi}{5}
\tightlist
\item
  An experiment is pulling one card from a fair deck.
\end{enumerate}

\begin{enumerate}
\def\labelenumi{\alph{enumi}.}
\item
  State the sample space.
\item
  Find the probability of getting a Ten. Make sure you state the event
  space.
\item
  Find the probability of getting a Diamond. Make sure you state the
  event space.
\item
  Find the probability of getting a Club. Make sure you state the
  event space.
\item
  Find the probability of getting a Diamond or a Club.
\item
  Find the probability of getting a Ten or a Diamond.
\end{enumerate}

\begin{enumerate}
\def\labelenumi{\arabic{enumi}.}
\setcounter{enumi}{6}
\tightlist
\item
  An experiment is pulling a ball from an urn that contains 3 blue
  balls and 5 red balls.
\end{enumerate}

\begin{enumerate}
\def\labelenumi{\alph{enumi}.}
\item
  Find the probability of getting a red ball.
\item
  Find the probability of getting a blue ball.
\item
  Find the odds for getting a red ball.
\item
  Find the odds for getting a blue ball.
\end{enumerate}

\begin{enumerate}
\def\labelenumi{\arabic{enumi}.}
\setcounter{enumi}{7}
\tightlist
\item
  In the game of roulette, there is a wheel with spaces marked 0
  through 36 and a space marked 00.
\end{enumerate}

\begin{enumerate}
\def\labelenumi{\alph{enumi}.}
\item
  Find the probability of winning if you pick the number 7 and it
  comes up on the wheel.
\item
  Find the odds against winning if you pick the number 7.
\item
  The casino will pay you \$20 for every dollar you bet if your number
  comes up. How much profit is the casino making on the bet?
\end{enumerate}

\hypertarget{conditional-probability}{%
\section{Conditional Probability}\label{conditional-probability}}

Suppose you want to figure out if you should buy a new car. When you first go and look, you find two cars that you like the most. In your mind they are equal, and so each has a 50\% chance that you will pick it. Then you start to look at the reviews of the cars and realize that the first car has had 40\% of them needing to be repaired in the first year, while the second car only has 10\% of the cars needing to be repaired in the first year. You could use this information to help you decide which car you want to actually purchase. Both cars no longer have a 50\% chance of being the car you choose. You could actually calculate the probability you will buy each car, which is a conditional probability. You probably wouldn't do this, but it gives you an example of what a conditional probability is.

\textbf{Conditional probabilities} are probabilities calculated after information is given. This is where you want to find the probability of event A happening after you know that event B has happened. If you know that B has happened, then you don't need to consider the rest of the sample space. You only need the outcomes that make up event B. Event B becomes the new sample space, which is called the \textbf{restricted sample space, R}. If you always write a restricted sample space when doing conditional probabilities and use this as your sample space, you will have no trouble with conditional probabilities. The notation for conditional probabilities is . The event following the vertical line is always the restricted sample space.

\textbf{Example \#4.3.1: Conditional Probabilities}

\begin{enumerate}
\def\labelenumi{\alph{enumi}.}
\tightlist
\item
  Suppose you roll two dice. What is the probability of getting a sum
  of 5, given that the first die is a 2?
\end{enumerate}

\textbf{Solution:}

Since you know that the first die is a 2, then this is your
restricted sample space, so

R = \{(2,1), (2,2), (2,3), (2,4), (2,5), (2,6)\}

Out of this restricted sample space, the way to get a sum of 5 is
\{(2,3)\}. Thus

\begin{enumerate}
\def\labelenumi{\alph{enumi}.}
\setcounter{enumi}{1}
\tightlist
\item
  Suppose you roll two dice. What is the probability of getting a sum
  of 7, given the first die is a 4?
\end{enumerate}

\textbf{Solution:}

Since you know that the first die is a 4, this is your restricted
sample space, so

R = \{(4,1), (4,2), (4,3), (4,4), (4,5), (4,6)\}

Out of this restricted sample space, the way to get a sum of 7 is
\{(4,3)\}.

Thus

\begin{enumerate}
\def\labelenumi{\alph{enumi}.}
\setcounter{enumi}{2}
\tightlist
\item
  Suppose you roll two dice. What is the probability of getting the
  second die a 2, given the sum is a 9?
\end{enumerate}

\textbf{Solution:}

Since you know the sum is a 9, this is your restricted sample space,
so

R = \{(3,6), (4,5), (5,4), (6,3)\}

Out of this restricted sample space there is no way to get the
second die a 2. Thus

\begin{enumerate}
\def\labelenumi{\alph{enumi}.}
\setcounter{enumi}{3}
\tightlist
\item
  Suppose you pick a card from a deck. What is the probability of
  getting a Spade, given that the card is a Jack?
\end{enumerate}

\textbf{Solution:}

Since you know that the card is a Jack, this is your restricted
sample space, so

R = \{JS, JC, JD, JH\}

Out of this restricted sample space, the way to get a Spade is \{JS\}.
Thus

\begin{enumerate}
\def\labelenumi{\alph{enumi}.}
\setcounter{enumi}{4}
\tightlist
\item
  Suppose you pick a card from a deck. What is the probability of
  getting an Ace, given the card is a Queen?
\end{enumerate}

\textbf{Solution:}

Since you know that the card is a Queen, then this is your
restricted sample space, so

R = \{QS, QC, QD, QH\}

Out of this restricted sample space, there is no way to get an Ace,
thus

If you look at the results of example \#4.2.7 part d and example \#4.3.1
part b, you will notice that you get the same answer. This means that
knowing that the first die is a 4 did not change the probability that
the sum is a 7. This added knowledge did not help you in any way. It is
as if that information was not given at all. However, if you compare
example \#4.2.7 part b and example \#4.3.1 part a, you will notice that
they are not the same answer. In this case, knowing that the first die
is a 2 did change the probability of getting a sum of 5. In the first
case, the events sum of 7 and first die is a 4 are called \textbf{independent
events.} In the second case, the events sum of 5 and first die is a 2
are called \textbf{dependent events.}

Events A and B are considered \textbf{independent events} if the fact that
one event happens does not change the probability of the other event
happening. In other words, events A and B are independent if the fact
that B has happened does not affect the probability of event A happening
and the fact that A has happened does not affect the probability of
event B happening. Otherwise, the two events are dependent.

In symbols, A and B are independent if

\begin{quote}
or
\end{quote}

\textbf{Example \#4.3.2: Independent Events}

\begin{enumerate}
\def\labelenumi{\alph{enumi}.}
\tightlist
\item
  Suppose you roll two dice. Are the events ``sum of 7'' and ``first die
  is a 3'' independent?
\end{enumerate}

\textbf{Solution:}

To determine if they are independent, you need to see if .

It doesn't matter which event is A or B, so just assign one as A and
one as B.

Let A = sum of 7 = \{(1,6), (2,5), (3,4), (4,3), (5,2), (6,1)\} and B
= first die is a 3 = \{(3,1), (3,2), (3,3), (3,4), (3,5), (3,6)\}

means that you assume that B has happened. The restricted sample
space is B,

R = \{(3,1), (3,2), (3,3), (3,4), (3,5), (3,6)\}

In this restricted sample space, the way for A to happen is \{(3,4)\},
so

The

. Thus ``sum of 7'' and ``first die is a 3'' are independent events.

\begin{enumerate}
\def\labelenumi{\alph{enumi}.}
\setcounter{enumi}{1}
\tightlist
\item
  Suppose you roll two dice. Are the events ``sum of 6'' and ``first die
  is a 4'' independent?
\end{enumerate}

\textbf{Solution:}

To determine if they are independent, you need to see if .

It doesn't matter which event is A or B, so just assign one as A and
one as B.

Let A = sum of 6 = \{(1,5), (2,4), (3,3), (4,2), (5,1)\} and B = first
die is a 4 = \{(4,1), (4,2), (4,3), (4,4), (4,5), (4,6)\}, so

For \includegraphics{media/image80.emf}, the restricted sample space is B,

R = \{(4,1), (4,2), (4,3), (4,4), (4,5), (4,6)\}

In this restricted sample space, the way for A to happen is \{(4,2)\},
so

.

In this case, ``sum of 6'' and ``first die is a 4'' are dependent since
.

\begin{enumerate}
\def\labelenumi{\alph{enumi}.}
\setcounter{enumi}{2}
\tightlist
\item
  Suppose you pick a card from a deck. Are the events ``Jack'' and
  ``Spade'' independent?
\end{enumerate}

\textbf{Solution:}

To determine if they are independent, you need to see if .

It doesn't matter which event is A or B, so just assign one as A and
one as B.

Let A = Jack = \{JS, JC, JD, JH\} and B = Spade \{2S, 3S, 4S, 5S, 6S,
7S, 8S, 9S, 10S, JS, QS, KS, AS\}

For \includegraphics{media/image85.emf}, the restricted sample space is B,

R = \{2S, 3S, 4S, 5S, 6S, 7S, 8S, 9S, 10S, JS, QS, KS, AS\}

In this restricted sample space, the way A happens is \{JS\}, so

In this case, ``Jack'' and ``Spade'' are independent since

.

\begin{enumerate}
\def\labelenumi{\alph{enumi}.}
\setcounter{enumi}{3}
\tightlist
\item
  Suppose you pick a card from a deck. Are the events ``Heart'' and
  ``Red'' card independent?
\end{enumerate}

\textbf{Solution:}

To determine if they are independent, you need to see if .

It doesn't matter which event is A or B, so just assign one as A and
one as B.

Let A = Heart = \{2H, 3H, 4H, 5H, 6H, 7H, 8H, 9H, 10H, JH, QH, KH,
AH\} and B = Red card = \{2D, 3D, 4D, 5D, 6D, 7D, 8D, 9D, 10D, JD, QD,
KD, AD, 2H, 3H, 4H, 5H, 6H, 7H, 8H, 9H, 10H, JH, QH, KH, AH\}, so

For , the restricted sample space is B,

R = \{2D, 3D, 4D, 5D, 6D, 7D, 8D, 9D, 10D, JD, QD, KD, AD, 2H, 3H,
4H, 5H, 6H, 7H, 8H, 9H, 10H, JH, QH, KH, AH\}

In this restricted sample space, the way A can happen is 13,

.

In this case, ``Heart'' and ``Red'' card are dependent, since .

\begin{enumerate}
\def\labelenumi{\alph{enumi}.}
\setcounter{enumi}{4}
\tightlist
\item
  Suppose you have two children via separate births. Are the events
  ``the first is a boy'' and ``the second is a girl'' independent?
\end{enumerate}

\textbf{Solution:}

In this case, you actually don't need to do any calculations. The
gender of one child does not affect the gender of the second child,
the events are independent.

\begin{enumerate}
\def\labelenumi{\alph{enumi}.}
\setcounter{enumi}{5}
\tightlist
\item
  Suppose you flip a coin 50 times and get a head every time, what is
  the probability of getting a head on the next flip?
\end{enumerate}

\textbf{Solution:}

\begin{quote}
Since one flip of the coin does not affect the next flip (the coin
does not remember what it did the time before), the probability of
getting a head on the next flip is still one-half.
\end{quote}

\textbf{Multiplication Rule:}

Two more useful formulas:

If two events are dependent, then

If two events are independent, then

If you solve the first equation for , you obtain , which is a formula to
calculate a conditional probability. However, it is easier to find a
conditional probability by using the restricted sample space and
counting unless the sample space is large.

\textbf{\\
}

\textbf{Example \#4.3.3: Multiplication Rule}

\begin{enumerate}
\def\labelenumi{\alph{enumi}.}
\tightlist
\item
  Suppose you pick three cards from a deck, what is the probability
  that they are all Queens if the cards are not replaced after they
  are picked?
\end{enumerate}

\textbf{Solution:}

This sample space is too large to write out, so using the
multiplication rule makes sense. Since the cards are not replaced,
then the probability will change for the second and third cards.
They are dependent events. This means that on the second draw there
is one less Queen and one less card, and on the third draw there are
two less Queens and 2 less cards.

\begin{enumerate}
\def\labelenumi{\alph{enumi}.}
\setcounter{enumi}{1}
\tightlist
\item
  Suppose you pick three cards from a deck, what is the probability
  that they are all Queens if the cards are replaced after they are
  picked and before the next card is picked?
\end{enumerate}

\textbf{Solution:}

\begin{quote}
Again, the sample space is too large to write out, so using the
multiplication rule makes sense. Since the cards are put back, one
draw has no affect on the next draw and they are all independent.
\end{quote}

\textbf{\\
}

\textbf{Example \#4.3.4: Application Problem}

\begin{quote}
The World Health Organization (WHO) keeps track of how many incidents
of leprosy there are in the world. Using the WHO regions and the World
Banks income groups, one can ask if an income level and a WHO region
are dependent on each other in terms of predicting where the disease
is. Data on leprosy cases in different countries was collected for the
year 2011 and a summary is presented in table \#4.3.1 ("Leprosy:
Number of," 2013).
\end{quote}

\textbf{Table \#4.3.1: Number of Leprosy Cases}

\begin{longtable}[]{@{}llllll@{}}
\toprule
WHO Region World Bank I & ncome Group Row Total & & & &\tabularnewline
\midrule
\endhead
High Income Upper Midd & le Income Lower Middle I & ncome Low Income & & &\tabularnewline
Americas174 36028 615 & 036817 & & & &\tabularnewline
Eastern Mediterranean & 546 1883 604 2547 & & & &\tabularnewline
Europe 100 0 010 & & & & &\tabularnewline
Western Pacific 26216 & 3689 1155 5086 & & & &\tabularnewline
Africa 0 391986 159281 & 7953 & & & &\tabularnewline
South-East Asia 0 0 1498 & 9610236160132 & & & &\tabularnewline
Column Total264 36289 & 15806927923222545 & & & &\tabularnewline
\bottomrule
\end{longtable}

\begin{enumerate}
\def\labelenumi{\alph{enumi}.}
\tightlist
\item
  Find the probability that a person with leprosy is from the
  Americas.
\end{enumerate}

\textbf{Solution: }

There are 36817 cases of leprosy in the Americas out of 222,545
cases worldwide. So,

There is about a 16.5\% chance that a person with leprosy lives in a
country in the Americas.

\begin{enumerate}
\def\labelenumi{\alph{enumi}.}
\setcounter{enumi}{1}
\tightlist
\item
  Find the probability that a person with leprosy is from a
  high-income country.
\end{enumerate}

\textbf{Solution:}

There are 264 cases of leprosy in high-income countries out of
222,545 cases worldwide. So,

There is about a 0.1\% chance that a person with leprosy lives in a
high-income country.

\begin{enumerate}
\def\labelenumi{\alph{enumi}.}
\setcounter{enumi}{2}
\tightlist
\item
  Find the probability that a person with leprosy is from the Americas
  and a high-income country.
\end{enumerate}

\textbf{Solution:}

There are 174 cases of leprosy in countries in a high-income country
in the Americas out the 222,545 cases worldwide. So,

There is about a 0.08\% chance that a person with leprosy lives in a
high-income country in the Americas.

\begin{enumerate}
\def\labelenumi{\alph{enumi}.}
\setcounter{enumi}{3}
\tightlist
\item
  Find the probability that a person with leprosy is from a
  high-income country, given they are from the Americas.
\end{enumerate}

\textbf{Solution:}

In this case you know that the person is in the Americas. You don't
need to consider people from Easter Mediterranean, Europe, Western
Pacific, Africa, and South-east Asia. You only need to look at the
row with Americas at the start. In that row, look to see how many
leprosy cases there are from a high-income country. There are 174
countries out of the 36,817 leprosy cases in the Americas. So,

There is 0.47\% chance that a person with leprosy is from a
high-income country given that they are from the Americas.

\begin{enumerate}
\def\labelenumi{\alph{enumi}.}
\setcounter{enumi}{4}
\tightlist
\item
  Find the probability that a person with leprosy is from a low-income
  country.
\end{enumerate}

\textbf{Solution:}

There are 27,923 cases of leprosy in low-income countries out of the
222,545 leprosy cases worldwide. So,

There is a 12.5\% chance that a person with leprosy is from a
low-income country.

\begin{enumerate}
\def\labelenumi{\alph{enumi}.}
\setcounter{enumi}{5}
\tightlist
\item
  Find the probability that a person with leprosy is from Africa.
\end{enumerate}

\textbf{Solution:}

There are 17,953 cases of leprosy in Africa out of 222,545 leprosy
cases worldwide. So,

There is an 8.1\% chance that a person with leprosy is from Africa.

\begin{enumerate}
\def\labelenumi{\alph{enumi}.}
\setcounter{enumi}{6}
\tightlist
\item
  Find the probability that a person with leprosy is from Africa and a
  low-income country.
\end{enumerate}

\textbf{Solution:}

There are 15,928 cases of leprosy in low-income countries in Africa
out of all the 222,545 leprosy cases worldwide. So,

There is a 7.2\% chance that a person with leprosy is from a
low-income country in Africa.

\begin{enumerate}
\def\labelenumi{\alph{enumi}.}
\setcounter{enumi}{7}
\tightlist
\item
  Find the probability that a person with leprosy is from Africa,
  given they are from a low-income country.
\end{enumerate}

\textbf{Solution:}

In this case you know that the person with leprosy is from
low-income country. You don't need to include the high income,
upper-middle income, and lower-middle income country. You only need
to consider the column headed by low-income. In that column, there
are 15,928 cases of leprosy in Africa out of the 27,923 cases of
leprosy in low-income countries. So,

There is a 57.0\% chance that a person with leprosy is from Africa,
given that they are from a low-income country.

\begin{enumerate}
\def\labelenumi{\roman{enumi}.}
\tightlist
\item
  Are the events that a person with leprosy is from ``Africa'' and
  ``low-income country'' independent events? Why or why not?
\end{enumerate}

\textbf{Solution:}

In order for these events to be independent, either or have to be
true. Part (h) showed and part (f) showed . Since these are not
equal, then these two events are dependent.

\begin{enumerate}
\def\labelenumi{\alph{enumi}.}
\setcounter{enumi}{9}
\tightlist
\item
  Are the events that a person with leprosy is from ``Americas'' and
  ``high-income country'' independent events? Why or why not?
\end{enumerate}

\textbf{Solution:}

In order for these events to be independent, either or have to be
true. Part (d) showed and part (b) showed Since these are not equal,
then these two events are dependent.

A big deal has been made about the difference between dependent and
independent events while calculating the probability of \emph{and}~compound
events.~ You must multiply the probability of the first event with the
conditional probability of the second event.

~

Why do you care? ~You need to calculate probabilities when you are
performing sampling, as you will learn later.~~But here is a
simplification~that can make the calculations a lot easier: when
the~sample size is very small compared to the population size, you can
assume that the conditional probabilities~just don't change very much
over the sample.

~

For example, consider acceptance sampling. ~Suppose there is a big
population of parts delivered to you factory, say 12,000 parts. ~Suppose
there are 85 defective parts in the population. ~You decide to randomly
select ten parts, and reject the shipment. ~ ~What is the probability of
rejecting the shipment? ~

~

There are many different ways you could reject the shipment. ~For
example, maybe the first three parts are good, one is bad, and the rest
are good. ~Or all ten parts could be bad, or maybe the first five. ~So
many ways to reject! ~But there is only \textbf{\emph{one}}~way that you'd accept
the shipment: if \textbf{\emph{all} \emph{ten}}~parts are good. That would happen if
the first part is good, \textbf{and}~the second part is good, \textbf{and}~the
third part is good, and so on. ~Since the probability of the second part
being good is (slightly) dependent on whether the first part was good,
technically you should take~this~into consideration~when you calculate
the probability that all~ten are good.

~

The probability of getting the first sampled part good is . So the
probability that all ten being good is .

If instead you assume that the probability doesn't change much, you get
. So as you can see, there is not much difference. So here is the rule:
if the sample is very small compared to the size of the population, then
you can assume that the probabilities are independent, even though they
aren't technically. By the way, the probability of rejecting the
shipment is .

\hypertarget{homework-12}{%
\subsection{Homework}\label{homework-12}}

\begin{enumerate}
\def\labelenumi{\arabic{enumi}.}
\item
  Are owning a refrigerator and owning a car independent events? Why
  or why not?
\item
  Are owning a computer or tablet and paying for Internet service
  independent events? Why or why not?
\item
  Are passing your statistics class and passing your biology class
  independent events? Why or why not?
\item
  Are owning a bike and owning a car independent events? Why or why
  not?
\item
  An experiment is picking a card from a fair deck.
\end{enumerate}

\begin{enumerate}
\def\labelenumi{\alph{enumi}.}
\item
  What is the probability of picking a Jack given that the card is a
  face card?
\item
  What is the probability of picking a heart given that the card is a
  three?
\item
  What is the probability of picking a red card given that the card is
  an ace?
\item
  Are the events Jack and face card independent events? Why or why
  not?
\item
  Are the events red card and ace independent events? Why or why not?
\end{enumerate}

\begin{enumerate}
\def\labelenumi{\arabic{enumi}.}
\setcounter{enumi}{5}
\tightlist
\item
  An experiment is rolling two dice.
\end{enumerate}

\begin{enumerate}
\def\labelenumi{\alph{enumi}.}
\item
  What is the probability that the sum is 6 given that the first die
  is a 5?
\item
  What is the probability that the first die is a 3 given that the sum
  is 11?
\item
  What is the probability that the sum is 7 given that the fist die is
  a 2?
\item
  Are the two events sum of 6 and first die is a 5 independent events?
  Why or why not?
\item
  Are the two events sum of 7 and first die is a 2 independent events?
  Why or why not?
\end{enumerate}

\begin{enumerate}
\def\labelenumi{\arabic{enumi}.}
\setcounter{enumi}{6}
\item
  You flip a coin four times. What is the probability that all four of
  them are heads?
\item
  You flip a coin six times. What is the probability that all six of
  them are heads?
\item
  You pick three cards from a deck with replacing the card each time
  before picking the next card. What is the probability that all three
  cards are kings?
\item
  You pick three cards from a deck without replacing a card before
  picking the next card. What is the probability that all three cards
  are kings?
\item
  The number of people who survived the Titanic based on class and sex
  is in table \#4.3.2 ("Encyclopedia Titanica," 2013). Suppose a
  person is picked at random from the survivors.
\end{enumerate}

\textbf{\\
}

\textbf{Table \#4.3.2: Surviving the Titanic}

\begin{longtable}[]{@{}llll@{}}
\toprule
Class & Sex Tota & l &\tabularnewline
\midrule
\endhead
Female & Male & &\tabularnewline
1st 134 & 59 193 & &\tabularnewline
2nd 94 & 25 119 & &\tabularnewline
3rd 80 & 58 138 & &\tabularnewline
Total & 308 142 & 450 &\tabularnewline
\bottomrule
\end{longtable}

\begin{enumerate}
\def\labelenumi{\alph{enumi}.}
\item
  What is the probability that a survivor was female?
\item
  What is the probability that a survivor was in the 1\textsuperscript{st} class?
\item
  What is the probability that a survivor was a female given that the
  person was in 1\textsuperscript{st} class?
\item
  What is the probability that a survivor was a female and in the
  1\textsuperscript{st} class?
\item
  What is the probability that a survivor was a female or in the 1\textsuperscript{st}
  class?
\item
  Are the events survivor is a female and survivor is in 1\textsuperscript{st} class
  mutually exclusive? Why or why not?
\item
  Are the events survivor is a female and survivor is in 1\textsuperscript{st} class
  independent? Why or why not?
\end{enumerate}

\begin{enumerate}
\def\labelenumi{\arabic{enumi}.}
\setcounter{enumi}{11}
\tightlist
\item
  Researchers watched groups of dolphins off the coast of Ireland in
  1998 to determine what activities the dolphins partake in at certain
  times of the day ("Activities of dolphin," 2013). The numbers in
  table \#4.3.3 represent the number of groups of dolphins that were
  partaking in an activity at certain times of days.
\end{enumerate}

\textbf{Table \#4.3.3: Dolphin Activity}

\begin{longtable}[]{@{}llllll@{}}
\toprule
Activity & PeriodTota & l & & &\tabularnewline
\midrule
\endhead
orning No & onAfternoo & n Even & ing & &\tabularnewline
Travel 6 6 & 14 1339 & & & &\tabularnewline
Feed 284 & 0 5688 & & & &\tabularnewline
Social 385 & 9 1062 & & & &\tabularnewline
Total 7215 & 23 7918 & 9 & & &\tabularnewline
\bottomrule
\end{longtable}

\begin{enumerate}
\def\labelenumi{\alph{enumi}.}
\item
  What is the probability that a dolphin group is partaking in travel?
\item
  What is the probability that a dolphin group is around in the
  morning?
\item
  What is the probability that a dolphin group is partaking in travel
  given that it is morning?
\item
  What is the probability that a dolphin group is around in the
  morning given that it is partaking in socializing?
\item
  What is the probability that a dolphin group is around in the
  afternoon given that it is partaking in feeding?
\item
  What is the probability that a dolphin group is around in the
  afternoon and is partaking in feeding?
\item
  What is the probability that a dolphin group is around in the
  afternoon or is partaking in feeding?
\item
  Are the events dolphin group around in the afternoon and dolphin
  group feeding mutually exclusive events? Why or why not?
\item
  Are the events dolphin group around in the morning and dolphin group
  partaking in travel independent events? Why or why not?
\end{enumerate}

\hypertarget{counting-techniques}{%
\section{Counting Techniques}\label{counting-techniques}}

There are times when the sample space or event space are very large, that it isn't feasible to write it out. In that case, it helps to have mathematical tools for counting the size of the sample space and event space. These tools are known as counting techniques.

\textbf{Multiplication Rule in Counting Techniques}

If task 1 can be done ways, task 2 can be done ways, and so forth to task \emph{n} being done ways. Then the number of ways to do task 1, 2,\ldots{}, \emph{n} together would be {[}MISSING EQN?{]}.

\textbf{Example \#4.4.1: Multiplication Rule in Counting}

\begin{quote}
A menu offers a choice of 3 salads, 8 main dishes, and 5 desserts. How many different meals consisting of one salad, one main dish, and one dessert are possible?
\end{quote}

\begin{quote}
\textbf{Solution:}

There are three tasks, picking a salad, a main dish, and a dessert. The salad task can be done 3 ways, the main dish task can be done 8 ways, and the dessert task can be done 5 ways. The ways to pick a salad, main dish, and dessert are {[}MISSING EQ?{]}
\end{quote}

\textbf{Example \#4.4.2: Multiplication Rule in Counting}

\begin{quote}
How many three letter ``words'' can be made from the letters a, b, and c
with no letters repeating? A ``word'' is just an ordered group of
letters. It doesn't have to be a real word in a dictionary.

\textbf{Solution:}

There are three tasks that must be done in this case. The tasks are to
pick the first letter, then the second letter, and then the third
letter. The first task can be done 3 ways since there are 3 letters.
The second task can be done 2 ways, since the first task took one of
the letters. The third task can be done 1 ways, since the first and
second task took two of the letters. There are

Which is

You can also look at this in a tree diagram:

So, there are 6 different ``words.''
\end{quote}

In example \#4.4.2, the solution was found by find . Many counting
problems involve multiplying a list of decreasing numbers. This is
called a \textbf{factorial}. There is a special symbol for this and a special
button on your calculator.

\textbf{Factorial}

As an example:

0 factorial is defined to be and 1 factorial is defined to be .

Sometimes you are trying to select \emph{r} objects from \emph{n} total objects.
The number of ways to do this depends on if the order you choose the \emph{r}
objects matters or if it doesn't. As an example if you are trying to
call a person on the phone, you have to have their number in the right
order. Otherwise, you call someone you didn't mean to. In this case, the
order of the numbers matters. If however you were picking random numbers
for the lottery, it doesn't matter which number you pick first. As long
as you have the same numbers that the lottery people pick, you win. In
this case the order doesn't matter. A \textbf{permutation} is an arrangement
of items with a specific order. You use permutations to count items when
the order matters. When the order doesn't matter you use combinations. A
\textbf{combination} is an arrangement of items when order is not important.
When you do a counting problem, the first thing you should ask yourself
is ``does order matter?''

\textbf{Permutation Formula}

Picking \emph{r} objects from \emph{n} total objects when order matters

\textbf{Combination Formula}

Picking \emph{r} objects from \emph{n} total objects when order doesn't matter

\textbf{Example \#4.4.3: Calculating the Number of Ways}

\begin{quote}
In a club with 15 members, how many ways can a slate of 3 officers
consisting of a president, vice-president, and secretary/treasurer be
chosen?

\textbf{Solution:}

In this case the order matters. If you pick person 1 for president, person 2 for vice-president, and person 3 for secretary/treasurer you would have different officers than if you picked person 2 for president, person 1 for vice-president, and person 3 for secretary/treasurer. This is a permutation problem with and .
\end{quote}

\textbf{Example \#4.4.4: Calculating the Number of Ways}

\begin{quote}
Suppose you want to pick 7 people out of 20 people to take part in a survey. How many ways can you do this?

\textbf{Solution:}

In this case the order doesn't matter, since you just want 7 people. This is a combination with and {[}MISSING{]}.
\end{quote}

Most calculators have a factorial button on them, and many have the combination and permutation functions also. R has a combination command.
\#\#\# Homework

\begin{enumerate}
\def\labelenumi{\arabic{enumi}.}
\item
  You are going to a benefit dinner, and need to decide before the dinner what you want for salad, main dish, and dessert. You have 2 different salads to choose from, 3 main dishes, and 5 desserts. How many different meals are available?
\item
  How many different phone numbers are possible in the area code 928?
\item
  You are opening a T-shirt store. You can have long sleeves or short sleeves, three different colors, five different designs, and four different sizes. How many different shirts can you make?
\item
  The California license plate has one number followed by three
  letters followed by three numbers. How many different license plates
  are there?
\item
  Find {[}MISSING{]}
\item
  Find {[}MISSING{]}
\item
  Find {[}MISSING{]}
\item
  Find
\item
  You have a group of twelve people. You need to pick a president,
  treasurer, and secretary from the twelve. How many different ways
  can you do this?
\item
  A baseball team has a 25-man roster. A batting order has nine
  people. How many different batting orders are there?
\item
  An urn contains five red balls, seven yellow balls, and eight white
  balls. How many different ways can you pick two red balls?
\item
  How many ways can you choose seven people from a group of twenty?
\end{enumerate}

Data sources:

\emph{Aboriginal deaths in custody}. (2013, September 26). Retrieved from
\url{http://www.statsci.org/data/oz/custody.html}

\emph{Activities of dolphin groups}. (2013, September 26). Retrieved from
\url{http://www.statsci.org/data/general/dolpacti.html}

\emph{Car preferences}. (2013, September 26). Retrieved from
\url{http://www.statsci.org/data/oz/carprefs.html}

\emph{Encyclopedia Titanica}. (2013, November 09). Retrieved from
\url{http://www.encyclopedia-titanica.org/}

\emph{Leprosy: Number of reported cases by country}. (2013, September 04).
Retrieved from \url{http://apps.who.int/gho/data/node.main.A1639}

Madison, J. (2013, October 15). \emph{M\&M's color distribution analysis}.
Retrieved from
\url{http://joshmadison.com/2007/12/02/mms-color-distribution-analysis/}

\hypertarget{discrete-probability-distributions}{%
\chapter{Discrete Probability Distributions}\label{discrete-probability-distributions}}

\hypertarget{basics-of-probability-distributions}{%
\section{Basics of Probability Distributions}\label{basics-of-probability-distributions}}

As a reminder, a variable or what will be called the random variable from now on, is represented by the letter \emph{x} and it represents a quantitative (numerical) variable that is measured or observed in an experiment.

Also remember there are different types of quantitative variables, called discrete or continuous. What is the difference between discrete and continuous data? \textbf{Discrete} data can only take on particular values in a range. \textbf{Continuous} data can take on any value in a range. Discrete data usually arises from counting while continuous data usually arises from measuring.

\textbf{Examples of each:}

How tall is a plant given a new fertilizer? Continuous. This is something you measure.

How many fleas are on prairie dogs in a colony? Discrete. This is something you count.

If you have a variable, and can find a probability associated with that variable, it is called a \textbf{random variable}. In many cases the random variable is what you are measuring, but when it comes to discrete random variables, it is usually what you are counting. So for the example of how tall is a plant given a new fertilizer, the random variable is the height of the plant given a new fertilizer. For the example of how many fleas are on prairie dogs in a colony, the random variable is the number of fleas on a prairie dog in a colony.

Now suppose you put all the values of the random variable together with the probability that that random variable would occur. You could then have a distribution like before, but now it is called a probability distribution since it involves probabilities. A \textbf{probability distribution} is an assignment of probabilities to the values of the random variable. The abbreviation of pdf is used for a probability distribution function.

For probability distributions, and {[}MISSING{]}

\textbf{Example \#5.1.1: Probability Distribution}

\begin{quote}
The 2010 U.S. Census found the chance of a household being a certain
size. The data is in table \#5.1.1 ("Households by age," 2013).

\textbf{Table \#5.1.1: Household Size from U.S. Census of 2010}
\end{quote}

\begin{longtable}[]{@{}llllllll@{}}
\toprule
Size of household & 1 & 2 & 3 & 4 & 5 & 6 & 7 or more\tabularnewline
\midrule
\endhead
Probability & 26.7\% & 33.6\% & 15.8\% & 13.7\% & 6.3\% & 2.4\% & 1.5\%\tabularnewline
\bottomrule
\end{longtable}

\begin{quote}
\textbf{Solution:}

In this case, the random variable is \emph{x} = number of people in a household. This is a discrete random variable, since you are unting the number of people in a household.
\end{quote}

\begin{quote}
This is a probability distribution since you have the \emph{x} value and the probabilities that go with it, all of the probabilities are between zero and one, and the sum of all of the probabilities is one.
\end{quote}

You can give a probability distribution in table form (as in table \#5.1.1) or as a graph. The graph looks like a histogram. A probability distribution is basically a relative frequency distribution based on a very large sample.

\textbf{Example \#5.1.2: Graphing a Probability Distribution}

\begin{quote}
The 2010 U.S. Census found the chance of a household being a certain size. The data is in the table ("Households by age," 2013). Draw a histogram of the probability distribution.

\textbf{Table \#5.1.2: Household Size from U.S. Census of 2010}
\end{quote}

\begin{longtable}[]{@{}llllllll@{}}
\toprule
Size of household & 1 & 2 & 3 & 4 & 5 & 6 & 7 or more\tabularnewline
\midrule
\endhead
Probability & 26.7\% & 33.6\% & 15.8\% & 13.7\% & 6.3\% & 2.4\% & 1.5\%\tabularnewline
\bottomrule
\end{longtable}

\begin{quote}
\textbf{Solution:}

State random variable:

\emph{x} = number of people in a household

You draw a histogram, where the \emph{x} values are on the horizontal axis
and are the \emph{x} values of the classes (for the 7 or more category,
just call it 7). The probabilities are on the vertical axis.

\textbf{Graph \#5.1.1: Histogram of Household Size from U.S. Census of
2010}

\includegraphics[width=5.01389in,height=3.01389in]{media/image3.png}

Notice this graph is skewed right.
\end{quote}

Just as with any data set, you can calculate the mean and standard
deviation. In problems involving a probability distribution function
(pdf), you consider the probability distribution the population even
though the pdf in most cases come from repeating an experiment many
times. This is because you are using the data from repeated experiments
to estimate the true probability. Since a pdf is basically a population,
the mean and standard deviation that are calculated are actually the
population parameters and not the sample statistics. The notation used
is the same as the notation for population mean and population standard
deviation that was used in chapter 3. Note: the mean can be thought of
as the \textbf{expected value}. It is the value you expect to get if the
trials were repeated infinite number of times. The mean or expected
value does not need to be a whole number, even if the possible values of
\emph{x} are whole numbers.

For a discrete probability distribution function,

The mean or expected value is

The variance is

The standard deviation is

where \emph{x} = the value of the random variable and \emph{P(x)} = the
probability corresponding to a particular \emph{x} value.

\textbf{Example \#5.1.3: Calculating Mean, Variance, and Standard Deviation
for a Discrete Probability Distribution}

\begin{quote}
The 2010 U.S. Census found the chance of a household being a certain
size. The data is in the table ("Households by age," 2013).

\textbf{Table \#5.1.3: Household Size from U.S. Census of 2010}
\end{quote}

\begin{longtable}[]{@{}llllllll@{}}
\toprule
Size of household & 1 & 2 & 3 & 4 & 5 & 6 & 7 or more\tabularnewline
\midrule
\endhead
Probability & 26.7\% & 33.6\% & 15.8\% & 13.7\% & 6.3\% & 2.4\% & 1.5\%\tabularnewline
\bottomrule
\end{longtable}

\begin{quote}
\textbf{Solution:}

State random variable:

\emph{x} = number of people in a household
\end{quote}

\begin{enumerate}
\def\labelenumi{\alph{enumi}.}
\tightlist
\item
  Find the mean
\end{enumerate}

\begin{quote}
\textbf{Solution:}
\end{quote}

To find the mean it is easier to just use a table as shown below.
Consider the category 7 or more to just be 7. The formula for the mean
says to multiply the \emph{x} value by the \emph{P(x)} value, so add a row into
the table for this calculation. Also convert all \emph{P(x)} to decimal form.

\begin{quote}
\textbf{Table \#5.1.4: Calculating the Mean for a Discrete PDF}
\end{quote}

\begin{longtable}[]{@{}llllllll@{}}
\toprule
\emph{x} & 1 & 2 & 3 & 4 & 5 & 6 & 7\tabularnewline
\midrule
\endhead
\emph{P(x) } & 0.267 & 0.336 & 0.158 & 0.137 & 0.063 & 0.024 & 0.015\tabularnewline
& 0.267 & 0.672 & 0.474 & 0.548 & 0.315 & 0.144 & 0.098\tabularnewline
\bottomrule
\end{longtable}

\begin{quote}
Now add up the new row and you get the answer 2.525. This is the mean
or the expected value, . This means that you expect a household in the
U.S. to have 2.525 people in it. Now of course you can't have half a
person, but what this tells you is that you expect a household to have
either 2 or 3 people, with a little more 3-person households than
2-person households.
\end{quote}

\begin{enumerate}
\def\labelenumi{\alph{enumi}.}
\setcounter{enumi}{1}
\tightlist
\item
  Find the variance
\end{enumerate}

\begin{quote}
\textbf{Solution:}
\end{quote}

To find the variance, again it is easier to use a table version than try
to just the formula in a line. Looking at the formula, you will notice
that the first operation that you should do is to subtract the mean from
each \emph{x} value. Then you square each of these values. Then you multiply
each of these answers by the probability of each \emph{x} value. Finally you
add up all of these values.

\begin{quote}
\textbf{Table \#5.1.5: Calculating the Variance for a Discrete PDF}
\end{quote}

\begin{longtable}[]{@{}llllllll@{}}
\toprule
\emph{x} & 1 & 2 & 3 & 4 & 5 & 6 & 7\tabularnewline
\midrule
\endhead
\emph{P(x) } & 0.267 & 0.336 & 0.158 & 0.137 & 0.063 & 0.024 & 0.015\tabularnewline
& -1.525 & -0.525 & 0.475 & 1.475 & 2.475 & 3.475 & 4.475\tabularnewline
& 2.3256 & 0.2756 & 0.2256 & 2.1756 & 6.1256 & 12.0756 & 20.0256\tabularnewline
& 0.6209 & 0.0926 & 0.0356 & 0.2981 & 0.3859 & 0.2898 & 0.3004\tabularnewline
\bottomrule
\end{longtable}

Now add up the last row to find the variance, . (Note: try not to round
your numbers too much so you aren't creating rounding error in your
answer. The numbers in the table above were rounded off because of space
limitations, but the answer was calculated using many decimal places.)

\begin{enumerate}
\def\labelenumi{\alph{enumi}.}
\setcounter{enumi}{2}
\tightlist
\item
  Find the standard deviation
\end{enumerate}

\begin{quote}
\textbf{Solution:}
\end{quote}

To find the standard deviation, just take the square root of the
variance, . This means that you can expect a U.S. household to have
2.525 people in it, with a standard deviation of 1.42 people.

\textbf{\\
}

\begin{enumerate}
\def\labelenumi{\alph{enumi}.}
\setcounter{enumi}{3}
\tightlist
\item
  Use a TI-83/84 to calculate the mean and standard deviation.
\end{enumerate}

\begin{quote}
\textbf{Solution:}

Go into the STAT menu, then the Edit menu. Type the \emph{x} values into L1
and the \emph{P(x)} values into L2. Then go into the STAT menu, then the
CALC menu. Choose 1:1-Var Stats. This will put 1-Var Stats on the home
screen. Now type in L1,L2 (there is a comma between L1 and L2) and
then press ENTER. If you have the newer operating system on the TI-84,
then your input will be slightly different. You will see the output in
figure \#5.1.1.

\textbf{Figure \#5.1.1: TI-83/84 Output}

\includegraphics[width=3.09722in,height=2.32773in]{media/image14.png}

The mean is 2.525 people and the standard deviation is 1.422 people.
\end{quote}

\begin{enumerate}
\def\labelenumi{\alph{enumi}.}
\setcounter{enumi}{4}
\tightlist
\item
  Using R to calculate the mean.
\end{enumerate}

\begin{quote}
\textbf{Solution:}

The command would be weighted.mean(x, p). So for this example, the
process would look like:

x\textless{}-c(1, 2, 3, 4, 5, 6, 7)

p\textless{}-c(0.267, 0.336, 0.158, 0.137, 0.063, 0.024, 0.015)

weighted.mean(x, p)

Output:

\[1\] 2.525

So the mean is 2.525.

To find the standard deviation, you would need to program the process
into R. So it is easier to just do it using the formula.
\end{quote}

\textbf{Example \#5.1.4: Calculating the Expected Value}

\begin{quote}
In the Arizona lottery called Pick 3, a player pays \$1 and then picks
a three-digit number. If those three numbers are picked in that
specific order the person wins \$500. What is the expected value in
this game?

\textbf{Solution:}

To find the expected value, you need to first create the probability
distribution. In this case, the random variable \emph{x} = winnings. If you
pick the right numbers in the right order, then you win \$500, but you
paid \$1 to play, so you actually win \$499. If you didn't pick the
right numbers, you lose the \$1, the \emph{x} value is . You also need the
probability of winning and losing. Since you are picking a three-digit
number, and for each digit there are 10 numbers you can pick with each
independent of the others, you can use the multiplication rule. To
win, you have to pick the right numbers in the right order. The first
digit, you pick 1 number out of 10, the second digit you pick 1 number
out of 10, and the third digit you pick 1 number out of 10. The
probability of picking the right number in the right order is . The
probability of losing (not winning) would be . Putting this
information into a table will help to calculate the expected value.

\textbf{Table \#5.1.6: Finding Expected Value}
\end{quote}

\begin{longtable}[]{@{}llll@{}}
\toprule
Win or lose & \emph{x} & \emph{P(x)} &\tabularnewline
\midrule
\endhead
Win & \$499 & 0.001 & \$0.499\tabularnewline
Lose & & 0.999 &\tabularnewline
\bottomrule
\end{longtable}

\begin{quote}
Now add the two values together and you have the expected value. It is
. In the long run, you will expect to lose \$0.50. Since the expected
value is not 0, then this game is not fair. Since you lose money,
Arizona makes money, which is why they have the lottery.
\end{quote}

The reason probability is studied in statistics is to help in making
decisions in inferential statistics. To understand how that is done the
concept of a rare event is needed.

\textbf{Rare Event Rule for Inferential Statistics}

If, under a given assumption, the probability of a particular observed
event is extremely small, then you can conclude that the assumption is
probably not correct.

An example of this is suppose you roll an assumed fair die 1000 times
and get a six 600 times, when you should have only rolled a six around
160 times, then you should believe that your assumption about it being a
fair die is untrue.

\textbf{Determining if an event is unusual}

If you are looking at a value of x for a discrete variable, and the
\emph{P}(the variable has a value of x or more) \textless{} 0.05, then you can
consider the x an unusually high value.~ Another way to think of this is
if the probability of getting such a high value is less than 0.05, then
the event of getting the value x is unusual.

Similarly, if the \emph{P}(the variable has a value of x or less) \textless{} 0.05,
then you can consider this an unusually low value. Another way to think
of this is if the probability of getting a value as small as x is less
than 0.05, then the event x is considered unusual.

Why is it "x or more" or "x or less" instead of just "x" when you
are determining if an event is unusual? Consider this example: you and
your friend go out to lunch every day.~ Instead of Going Dutch (each
paying for their own lunch), you decide to flip a coin, and the loser
pays for both.~ Your friend seems to be winning more often than you'd
expect, so you want to determine if this is unusual before you decide to
change how you pay for lunch (or accuse your friend of cheating).~ The
process for how to calculate these probabilities will be presented in
the next section on the binomial distribution.~ If your friend won 6 out
of 10 lunches, the probability of that happening turns out to be about
20.5\%, not unusual.~ The probability of winning 6 or more is about
37.7\%.~ But what happens if your friend won 501 out of 1,000 lunches?~
That doesn't seem so unlikely!~ The probability of winning 501 or more
lunches is about 47.8\%, and that is consistent with your hunch that this
isn't so unusual.~ But the probability of winning exactly 501 lunches
is much less, only about 2.5\%.~ That is why the probability of getting
exactly that value is not the right question to ask: you should ask the
probability of getting that value or more (or that value or less on the
other side).

The value 0.05 will be explained later, and it is not the only value you
can use.

\textbf{Example \#5.1.5: Is the Event Unusual}

\begin{quote}
The 2010 U.S. Census found the chance of a household being a certain
size. The data is in the table ("Households by age," 2013).

\textbf{Table \#5.1.7: Household Size from U.S. Census of 2010}
\end{quote}

\begin{longtable}[]{@{}llllllll@{}}
\toprule
Size of household & 1 & 2 & 3 & 4 & 5 & 6 & 7 or more\tabularnewline
\midrule
\endhead
Probability & 26.7\% & 33.6\% & 15.8\% & 13.7\% & 6.3\% & 2.4\% & 1.5\%\tabularnewline
\bottomrule
\end{longtable}

\begin{quote}
\textbf{Solution:}

State random variable:

\emph{x} = number of people in a household
\end{quote}

\begin{enumerate}
\def\labelenumi{\alph{enumi}.}
\item
  Is it unusual for a household to have six people in the family?

  \textbf{Solution:}
\end{enumerate}

To determine this, you need to look at probabilities. However, you
cannot just look at the probability of six people. You need to look at
the probability of \emph{x} being six or more people or the probability of
\emph{x} being six or less people. The

\begin{quote}
Since this probability is more than 5\%, then six is not an unusually
low value.

The

Since this probability is less than 5\%, then six is an unusually high
value. It is unusual for a household to have six people in the family.
\end{quote}

\begin{enumerate}
\def\labelenumi{\alph{enumi}.}
\setcounter{enumi}{1}
\item
  If you did come upon many families that had six people in the
  family, what would you think?

  \textbf{Solution:}

  Since it is unusual for a family to have six people in it, then you
  may think that either the size of families is increasing from what
  it was or that you are in a location where families are larger than
  in other locations.
\item
  Is it unusual for a household to have four people in the family?

  \textbf{Solution:}

  To determine this, you need to look at probabilities. Again, look at
  the probability of \emph{x} being four or more or the probability of \emph{x}
  being four or less. The

  Since this probability is more than 5\%, four is not an unusually
  high value.

  The

  Since this probability is more than 5\%, four is not an unusually low
  value. Thus, four is not an unusual size of a family.
\item
  If you did come upon a family that has four people in it, what would
  you think?

  \textbf{Solution:}

  Since it is not unusual for a family to have four members, then you
  would not think anything is amiss.
\end{enumerate}

\hypertarget{homework-13}{%
\subsection{Homework}\label{homework-13}}

\begin{enumerate}
\def\labelenumi{\arabic{enumi}.}
\tightlist
\item
  Eyeglassomatic manufactures eyeglasses for different retailers. The number of days it takes to fix defects in an eyeglass and the probability that it will take that number of days are in the table.
\end{enumerate}

\begin{quote}
\textbf{Table \#5.1.8: Number of Days to Fix Defects}
\end{quote}

\begin{longtable}[]{@{}ll@{}}
\toprule
Number of days & Probabilities\tabularnewline
\midrule
\endhead
1 & 24.9\%\tabularnewline
2 & 10.8\%\tabularnewline
3 & 9.1\%\tabularnewline
4 & 12.3\%\tabularnewline
5 & 13.3\%\tabularnewline
6 & 11.4\%\tabularnewline
7 & 7.0\%\tabularnewline
8 & 4.6\%\tabularnewline
9 & 1.9\%\tabularnewline
10 & 1.3\%\tabularnewline
11 & 1.0\%\tabularnewline
12 & 0.8\%\tabularnewline
13 & 0.6\%\tabularnewline
14 & 0.4\%\tabularnewline
15 & 0.2\%\tabularnewline
16 & 0.2\%\tabularnewline
17 & 0.1\%\tabularnewline
18 & 0.1\%\tabularnewline
\bottomrule
\end{longtable}

\begin{enumerate}
\def\labelenumi{\alph{enumi}.}
\item
  State the random variable.
\item
  Draw a histogram of the number of days to fix defects
\item
  Find the mean number of days to fix defects.
\item
  Find the variance for the number of days to fix defects.
\item
  Find the standard deviation for the number of days to fix defects.
\item
  Find probability that a lens will take at least 16 days to make a
  fix the defect.
\item
  Is it unusual for a lens to take 16 days to fix a defect?
\item
  If it does take 16 days for eyeglasses to be repaired, what would
  you think?
\end{enumerate}

\begin{enumerate}
\def\labelenumi{\arabic{enumi}.}
\setcounter{enumi}{1}
\tightlist
\item
  Suppose you have an experiment where you flip a coin three times.
  You then count the number of heads.
\end{enumerate}

\begin{enumerate}
\def\labelenumi{\alph{enumi}.}
\item
  State the random variable.
\item
  Write the probability distribution for the number of heads.
\item
  Draw a histogram for the number of heads.
\item
  Find the mean number of heads.
\item
  Find the variance for the number of heads.
\item
  Find the standard deviation for the number of heads.
\item
  Find the probability of having two or more number of heads.
\item
  Is it unusual for to flip two heads?
\end{enumerate}

\begin{enumerate}
\def\labelenumi{\arabic{enumi}.}
\setcounter{enumi}{2}
\item
  The Ohio lottery has a game called Pick 4 where a player pays \$1
  and picks a four-digit number. If the four numbers come up in the
  order you picked, then you win \$2,500. What is your expected value?
\item
  An LG Dishwasher, which costs \$800, has a 20\% chance of needing to
  be replaced in the first 2 years of purchase. A two-year extended
  warrantee costs \$112.10 on a dishwasher. What is the expected value
  of the extended warranty assuming it is replaced in the first 2
  years?
\end{enumerate}

\textbf{\\
}

\hypertarget{binomial-probability-distribution}{%
\section{Binomial Probability Distribution}\label{binomial-probability-distribution}}

Section 5.1 introduced the concept of a probability distribution. The
focus of the section was on discrete probability distributions (pdf). To
find the pdf for a situation, you usually needed to actually conduct the
experiment and collect data. Then you can calculate the experimental
probabilities. Normally you cannot calculate the theoretical
probabilities instead. However, there are certain types of experiment
that allow you to calculate the theoretical probability. One of those
types is called a \textbf{Binomial Experiment}.

Properties of a \textbf{binomial experiment} (or Bernoulli trial):

1) Fixed number of trials, \emph{n}, which means that the experiment is
repeated a specific number of times.

2) The \emph{n} trials are independent, which means that what happens on one
trial does not influence the outcomes of other trials.

3) There are only two outcomes, which are called a success and a
failure.

4) The probability of a success doesn't change from trial to trial,
where \emph{p} = probability of success and \emph{q} = probability of failure, .

If you know you have a binomial experiment, then you can calculate
binomial probabilities. This is important because binomial probabilities
come up often in real life. Examples of binomial experiments are:

\begin{quote}
Toss a fair coin ten times, and find the probability of getting two
heads.

Question twenty people in class, and look for the probability of more
than half being women?

Shoot five arrows at a target, and find the probability of hitting it
five times?
\end{quote}

To develop the process for calculating the probabilities in a binomial
experiment, consider example \#5.2.1.

\textbf{Example \#5.2.1: Deriving the Binomial Probability Formula}

\begin{quote}
Suppose you are given a 3 question multiple-choice test. Each question
has 4 responses and only one is correct. Suppose you want to find the
probability that you can just guess at the answers and get 2 questions
right. (Teachers do this all the time when they make up a
multiple-choice test to see if students can still pass without
studying. In most cases the students can't.) To help with the idea
that you are going to guess, suppose the test is in Martian.
\end{quote}

\begin{enumerate}
\def\labelenumi{\alph{enumi}.}
\tightlist
\item
  What is the random variable?
\end{enumerate}

\textbf{Solution:}

\begin{quote}
\emph{x} = number of correct answers
\end{quote}

\begin{enumerate}
\def\labelenumi{\alph{enumi}.}
\setcounter{enumi}{1}
\tightlist
\item
  Is this a binomial experiment?
\end{enumerate}

\textbf{Solution:}

\begin{enumerate}
\def\labelenumi{\arabic{enumi}.}
\item
  There are 3 questions, and each question is a trial, so there are a
  fixed number of trials. In this case, \emph{n} = 3.
\item
  Getting the first question right has no affect on getting the second
  or third question right, thus the trials are independent.
\item
  Either you get the question right or you get it wrong, so there are
  only two outcomes. In this case, the success is getting the question
  right.
\item
  The probability of getting a question right is one out of four. This
  is the same for every trial since each question has 4 responses. In
  this case, and
\end{enumerate}

\begin{quote}
This is a binomial experiment, since all of the properties are met.
\end{quote}

\begin{enumerate}
\def\labelenumi{\alph{enumi}.}
\setcounter{enumi}{2}
\tightlist
\item
  What is the probability of getting 2 questions right?
\end{enumerate}

\begin{quote}
\textbf{Solution:}

To answer this question, start with the sample space.

SS = \{RRR, RRW, RWR, WRR, WWR, WRW, RWW, WWW\}, where RRW means you get
the first question right, the second question right, and the third
question wrong. The same is similar for the other outcomes.

Now the event space for getting 2 right is \{RRW, RWR, WRR\}. What you
did in chapter four was just to find three divided by eight. However,
this would not be right in this case. That is because the probability
of getting a question right is different from getting a question
wrong. What else can you do?

Look at just P(RRW) for the moment. Again, that means

P(RRW) = P(R on 1st, R on 2nd, and W on 3rd)

Since the trials are independent, then

P(RRW) = P(R on 1st, R on 2nd, and W on 3rd)

= P(R on 1st) * P(R on 2nd) * P(W on 3rd)

Just multiply p * p * q

The same is true for P(RWR) and P(WRR). To find the probability of 2
correct answers, just add these three probabilities together. You get
\end{quote}

\begin{enumerate}
\def\labelenumi{\alph{enumi}.}
\setcounter{enumi}{3}
\tightlist
\item
  What is the probability of getting zero right, one right, and all
  three right?
\end{enumerate}

\begin{quote}
\textbf{Solution:}

You could go through the same argument that you did above and come up
with the following:
\end{quote}

\begin{longtable}[]{@{}ll@{}}
\toprule
r right & P(r right)\tabularnewline
\midrule
\endhead
0 right &\tabularnewline
1 right &\tabularnewline
2 right &\tabularnewline
3 right &\tabularnewline
\bottomrule
\end{longtable}

\begin{quote}
Hopefully you see the pattern that results. You can now write the
general formula for the probabilities for a Binomial experiment
\end{quote}

First, the random variable in a binomial experiment is \emph{x} = number of
successes.

Be careful, a success is not always a good thing. Sometimes a success is
something that is bad, like finding a defect. A success just means you
observed the outcome you wanted to see happen.

Binomial Formula for the probability of \emph{r} successes in \emph{n} trials is

where

The is the number of combinations of \emph{n} things taking \emph{r} at a time. It
is read ``\emph{n} choose \emph{r}''. Some other common notations for \emph{n} choose \emph{r}
are , and . \emph{n!} means you are multiplying . As an example, .

When solving problems, make sure you define your random variable and
state what \emph{n}, \emph{p}, \emph{q}, and \emph{r} are. Without doing this, the problems
are a great deal harder.

\textbf{Example \#5.2.2: Calculating Binomial Probabilities}

\begin{quote}
When looking at a person's eye color, it turns out that 1\% of people
in the world has green eyes ("What percentage of," 2013). Consider a
group of 20 people.
\end{quote}

\begin{enumerate}
\def\labelenumi{\alph{enumi}.}
\tightlist
\item
  State the random variable.
\end{enumerate}

\textbf{Solution:}

\emph{x} = number of people with green eyes

\begin{enumerate}
\def\labelenumi{\alph{enumi}.}
\setcounter{enumi}{1}
\tightlist
\item
  Argue that this is a binomial experiment
\end{enumerate}

\textbf{Solution:}

\begin{enumerate}
\def\labelenumi{\arabic{enumi}.}
\item
  There are 20 people, and each person is a trial, so there are a
  fixed number of trials. In this case, \emph{n} = 20.
\item
  If you assume that each person in the group is chosen at random the
  eye color of one person doesn't affect the eye color of the next
  person, thus the trials are independent.
\item
  Either a person has green eyes or they do not have green eyes, so
  there are only two outcomes. In this case, the success is a person
  has green eyes.
\item
  The probability of a person having green eyes is 0.01. This is the
  same for every trial since each person has the same chance of having
  green eyes. and
\end{enumerate}

Find the probability that

\begin{enumerate}
\def\labelenumi{\alph{enumi}.}
\setcounter{enumi}{2}
\tightlist
\item
  None have green eyes.
\end{enumerate}

\textbf{Solution:}

\begin{enumerate}
\def\labelenumi{\alph{enumi}.}
\setcounter{enumi}{3}
\tightlist
\item
  Nine have green eyes.
\end{enumerate}

\textbf{Solution:}

\begin{enumerate}
\def\labelenumi{\alph{enumi}.}
\setcounter{enumi}{4}
\tightlist
\item
  At most three have green eyes.
\end{enumerate}

\textbf{Solution:}

At most three means that three is the highest value you will have. Find
the probability of \emph{x} is less than or equal to three.

The reason the answer is written as being greater than 0.999 is because
the answer is actually 0.9999573791, and when that is rounded to three
decimal places you get 1. But 1 means that the event will happen, when
in reality there is a slight chance that it won't happen. It is best to
write the answer as greater than 0.999 to represent that the number is
very close to 1, but isn't 1.

\begin{enumerate}
\def\labelenumi{\alph{enumi}.}
\setcounter{enumi}{5}
\tightlist
\item
  At most two have green eyes.
\end{enumerate}

\textbf{Solution:}

\begin{enumerate}
\def\labelenumi{\alph{enumi}.}
\setcounter{enumi}{6}
\tightlist
\item
  At least four have green eyes.
\end{enumerate}

\textbf{Solution:}

At least four means four or more. Find the probability of x being
greater than or equal to four. That would mean adding up all the
probabilities from four to twenty. This would take a long time, so it is
better to use the idea of complement. The complement of being greater
than or equal to four is being less than four. That would mean being
less than or equal to three. Part (e) has the answer for the probability
of being less than or equal to three. Just subtract that number from 1.

.

Actually the answer is less than 0.001, but it is fine to write it this
way.

\begin{enumerate}
\def\labelenumi{\alph{enumi}.}
\setcounter{enumi}{7}
\tightlist
\item
  In Europe, four people out of twenty have green eyes. Is this
  unusual? What does that tell you?
\end{enumerate}

\textbf{Solution:}

Since the probability of finding four or more people with green eyes is
much less than 0.05, it is unusual to find four people out of twenty
with green eyes. That should make you wonder if the proportion of people
in Europe with green eyes is more than the 1\% for the general
population. If this is true, then you may want to ask why Europeans have
a higher proportion of green-eyed people. That of course could lead to
more questions.

The binomial formula is cumbersome to use, so you can find the
probabilities by using technology. On the TI-83/84 calculator, the
commands on the TI-83/84 calculators when the number of trials is equal
to \emph{n} and the probability of a success is equal to \emph{p} are when you
want to find and when you want to find . If you want to find , then you
use the property that , since and are complementary events. Both
binompdf and binomcdf commands are found in the DISTR menu. Using R, the
commands are and .

\textbf{Example \#5.2.3: Using the Binomial Command on the TI-83/84}

\begin{quote}
When looking at a person's eye color, it turns out that 1\% of people
in the world has green eyes ("What percentage of," 2013). Consider a
group of 20 people.
\end{quote}

\begin{enumerate}
\def\labelenumi{\alph{enumi}.}
\tightlist
\item
  State the random variable.
\end{enumerate}

\textbf{Solution:}

\emph{x} = number of people with green eyes

\begin{quote}
Find the probability that
\end{quote}

\begin{enumerate}
\def\labelenumi{\alph{enumi}.}
\setcounter{enumi}{1}
\tightlist
\item
  None have green eyes.
\end{enumerate}

\textbf{Solution:}

You are looking for . Since this problem is , you use the binompdf
command on the TI-83/84 or dbinom command on R. On the TI-83/84, you go
to the DISTR menu, select the binompdf, and then type into the
parenthesis your \emph{n}, \emph{p}, and \emph{r} values into your calculator, making
sure you use the comma to separate the values. The command will look
like and when you press ENTER you will be given the answer. (If you have
the new software on the TI-84, the screen looks a bit different.)

\textbf{Figure \#5.2.1: Calculator Results for binompdf}

\includegraphics[width=2.75in,height=1.86111in]{media/image62.png}

On R, the command would look like dbinom(0, 20, 0.01)

. Thus there is an 81.8\% chance that in a group of 20 people none of
them will have green eyes.

\begin{enumerate}
\def\labelenumi{\alph{enumi}.}
\setcounter{enumi}{2}
\tightlist
\item
  Nine have green eyes.
\end{enumerate}

\textbf{Solution:}

In this case you want to find the . Again, you will use the binompdf
command or the dbinom command. Following the procedure above, you will
have on the TI-83/84 or dbinom(9,20,0.01) on R. Your answer is .
(Remember when the calculator gives you and R give you , this is how
they display scientific notation.) The probability that out of twenty
people, nine of them have green eyes is a very small chance.

\begin{enumerate}
\def\labelenumi{\alph{enumi}.}
\setcounter{enumi}{3}
\tightlist
\item
  At most three have green eyes.
\end{enumerate}

\textbf{Solution:}

At most three means that three is the highest value you will have. Find
the probability of \emph{x} being less than or equal to three, which is .
This uses the binomcdf command on the TI-83/84 and pbinom command in R.
You use the command on the TI-83/84 of and the command on R of
pbinom(3,20,0.01)

\textbf{Figure \#5.2.2: Calculator Results for binomcdf}

\includegraphics[width=2.75in,height=1.86111in]{media/image71.png}

Your answer is 0.99996. Thus there is a really good chance that in a
group of 20 people at most three will have green eyes. (Note: don't
round this to one, since one means that the event will happen, when in
reality there is a slight chance that it won't happen. It is best to
write the answer out to enough decimal points so it doesn't round off to
one.

\begin{enumerate}
\def\labelenumi{\alph{enumi}.}
\setcounter{enumi}{4}
\tightlist
\item
  At most two have green eyes.
\end{enumerate}

\textbf{Solution:}

You are looking for . Again use binomcdf or pbinom. Following the
procedure above you will have on the TI-83/84 and pbinom(2,20,0.01),
with . Again there is a really good chance that at most two people in
the room will have green eyes.

\begin{enumerate}
\def\labelenumi{\alph{enumi}.}
\setcounter{enumi}{5}
\tightlist
\item
  At least four have green eyes.
\end{enumerate}

\textbf{Solution:}

At least four means four or more. Find the probability of \emph{x} being
greater than or equal to four. That would mean adding up all the
probabilities from four to twenty. This would take a long time, so it is
better to use the idea of complement. The complement of being greater
than or equal to four is being less than four. That would mean being
less than or equal to three. Part (e) has the answer for the probability
of being less than or equal to three. Just subtract that number from 1.

. You can also find this answer by doing the following on TI-83/84:

on R:

Again, this is very unlikely to happen.

There are other technologies that will compute binomial probabilities.

\textbf{Example \#5.2.4: Calculating Binomial Probabilities}

\begin{quote}
According to the Center for Disease Control (CDC), about 1 in 88
children in the U.S. have been diagnosed with autism ("CDC-data and
statistics,," 2013). Suppose you consider a group of 10 children.
\end{quote}

\begin{enumerate}
\def\labelenumi{\alph{enumi}.}
\tightlist
\item
  State the random variable.
\end{enumerate}

\textbf{Solution:}

\emph{x} = number of children with autism.

\begin{enumerate}
\def\labelenumi{\alph{enumi}.}
\setcounter{enumi}{1}
\tightlist
\item
  Argue that this is a binomial experiment
\end{enumerate}

\textbf{Solution:}

\begin{enumerate}
\def\labelenumi{\arabic{enumi}.}
\item
  There are 10 children, and each child is a trial, so there are a
  fixed number of trials. In this case, \emph{n} = 10.
\item
  If you assume that each child in the group is chosen at random, then
  whether a child has autism does not affect the chance that the next
  child has autism. Thus the trials are independent.
\item
  Either a child has autism or they do not have autism, so there are
  two outcomes. In this case, the success is a child has autism.
\item
  The probability of a child having autism is 1/88. This is the same
  for every trial since each child has the same chance of having
  autism. and .
\end{enumerate}

\begin{quote}
Find the probability that
\end{quote}

\begin{enumerate}
\def\labelenumi{\alph{enumi}.}
\setcounter{enumi}{2}
\tightlist
\item
  None have autism.
\end{enumerate}

\textbf{Solution:}

Using the formula:

Using the TI-83/84 Calculator:

Using R:

\begin{enumerate}
\def\labelenumi{\alph{enumi}.}
\setcounter{enumi}{3}
\tightlist
\item
  Seven have autism.
\end{enumerate}

\textbf{Solution:}

Using the formula:

Using the TI-83/84 Calculator:

Using R:

\begin{enumerate}
\def\labelenumi{\alph{enumi}.}
\setcounter{enumi}{4}
\tightlist
\item
  At least five have autism.
\end{enumerate}

\textbf{Solution:}

Using the formula:

Using the TI-83/84 Calculator:

To use the calculator you need to use the complement.

Using R:

To use R you need to use the complement.

Notice, the answer is given as 0.000, since the answer is less than
0.000. Don't write 0, since 0 means that the event is impossible to
happen. The event of five or more is improbable, but not impossible.

\begin{enumerate}
\def\labelenumi{\alph{enumi}.}
\setcounter{enumi}{5}
\tightlist
\item
  At most two have autism.
\end{enumerate}

\textbf{Solution:}

Using the formula:

Using the TI-83/84 Calculator:

Using R:

\begin{enumerate}
\def\labelenumi{\alph{enumi}.}
\setcounter{enumi}{6}
\tightlist
\item
  Suppose five children out of ten have autism. Is this unusual? What
  does that tell you?
\end{enumerate}

\textbf{Solution:}

Since the probability of five or more children in a group of ten having
autism is much less than 5\%, it is unusual to happen. If this does
happen, then one may think that the proportion of children diagnosed
with autism is actually more than 1/88.

\hypertarget{homework-14}{%
\subsection{Homework}\label{homework-14}}

\begin{enumerate}
\def\labelenumi{\arabic{enumi}.}
\tightlist
\item
  Suppose a random variable, \emph{x}, arises from a binomial experiment. If \emph{n} = 14, and \emph{p} = 0.13, find the following probabilities using the binomial formula.
\end{enumerate}

\begin{quote}
a.)

b.)

c.)

d.)

e.)

f.)
\end{quote}

\begin{enumerate}
\def\labelenumi{\arabic{enumi}.}
\setcounter{enumi}{1}
\tightlist
\item
  Suppose a random variable, \emph{x}, arises from a binomial experiment. If \emph{n} = 22, and \emph{p} = 0.85, find the following probabilities using the binomial formula.
\end{enumerate}

\begin{quote}
a.)

b.)

c.)

d.)

e.)

f.)
\end{quote}

\begin{enumerate}
\def\labelenumi{\arabic{enumi}.}
\setcounter{enumi}{2}
\tightlist
\item
  Suppose a random variable, \emph{x}, arises from a binomial experiment. If \emph{n} = 10, and \emph{p} = 0.70, find the following probabilities using technology.
\end{enumerate}

\begin{quote}
a.)

b.)

c.)

d.)

e.)

f.)
\end{quote}

\begin{enumerate}
\def\labelenumi{\arabic{enumi}.}
\setcounter{enumi}{3}
\tightlist
\item
  Suppose a random variable, \emph{x}, arises from a binomial experiment.
  If \emph{n} = 6, and \emph{p} = 0.30, find the following probabilities using
  technology.
\end{enumerate}

\begin{quote}
a.)

b.)

c.)

d.)

e.)

f.)
\end{quote}

\begin{enumerate}
\def\labelenumi{\arabic{enumi}.}
\setcounter{enumi}{4}
\tightlist
\item
  Suppose a random variable, \emph{x}, arises from a binomial experiment.
  If \emph{n} = 17, and \emph{p} = 0.63, find the following probabilities using
  technology.
\end{enumerate}

\begin{quote}
a.)

b.)

c.)

d.)

e.)

f.)
\end{quote}

\begin{enumerate}
\def\labelenumi{\arabic{enumi}.}
\setcounter{enumi}{5}
\tightlist
\item
  Suppose a random variable, \emph{x}, arises from a binomial experiment.
  If \emph{n} = 23, and \emph{p} = 0.22, find the following probabilities using
  technology.
\end{enumerate}

\begin{quote}
a.)

b.)

c.)

d.)

e.)

f.)
\end{quote}

\begin{enumerate}
\def\labelenumi{\arabic{enumi}.}
\setcounter{enumi}{6}
\tightlist
\item
  Approximately 10\% of all people are left-handed ("11 little-known
  facts," 2013). Consider a grouping of fifteen people.
\end{enumerate}

\begin{enumerate}
\def\labelenumi{\alph{enumi}.}
\item
  State the random variable.
\item
  Argue that this is a binomial experiment
\end{enumerate}

\begin{quote}
Find the probability that
\end{quote}

\begin{enumerate}
\def\labelenumi{\alph{enumi}.}
\setcounter{enumi}{2}
\item
  None are left-handed.
\item
  Seven are left-handed.
\item
  At least two are left-handed.
\item
  At most three are left-handed.
\item
  At least seven are left-handed.
\item
  Seven of the last 15 U.S. Presidents were left-handed. Is this
  unusual? What does that tell you?
\end{enumerate}

\begin{enumerate}
\def\labelenumi{\arabic{enumi}.}
\setcounter{enumi}{7}
\tightlist
\item
  According to an article in the American Heart Association's
  publication \emph{Circulation}, 24\% of patients who had been hospitalized
  for an acute myocardial infarction did not fill their cardiac
  medication by the seventh day of being discharged (Ho, Bryson \&
  Rumsfeld, 2009). Suppose there are twelve people who have been
  hospitalized for an acute myocardial infarction.
\end{enumerate}

\begin{enumerate}
\def\labelenumi{\alph{enumi}.}
\item
  State the random variable.
\item
  Argue that this is a binomial experiment
\end{enumerate}

\begin{quote}
Find the probability that
\end{quote}

\begin{enumerate}
\def\labelenumi{\alph{enumi}.}
\setcounter{enumi}{2}
\item
  All filled their cardiac medication.
\item
  Seven did not fill their cardiac medication.
\item
  None filled their cardiac medication.
\item
  At most two did not fill their cardiac medication.
\item
  At least three did not fill their cardiac medication.
\item
  At least ten did not fill their cardiac medication.
\item
  Suppose of the next twelve patients discharged, ten did not fill
  their cardiac medication, would this be unusual? What does this tell
  you?
\end{enumerate}

\begin{enumerate}
\def\labelenumi{\arabic{enumi}.}
\setcounter{enumi}{8}
\tightlist
\item
  Eyeglassomatic manufactures eyeglasses for different retailers. In
  March 2010, they tested to see how many defective lenses they made,
  and there were 16.9\% defective lenses due to scratches. Suppose
  Eyeglassomatic examined twenty eyeglasses.
\end{enumerate}

\begin{enumerate}
\def\labelenumi{\alph{enumi}.}
\item
  State the random variable.
\item
  Argue that this is a binomial experiment
\end{enumerate}

\begin{quote}
Find the probability that
\end{quote}

\begin{enumerate}
\def\labelenumi{\alph{enumi}.}
\setcounter{enumi}{2}
\item
  None are scratched.
\item
  All are scratched.
\item
  At least three are scratched.
\item
  At most five are scratched.
\item
  At least ten are scratched.
\item
  Is it unusual for ten lenses to be scratched? If it turns out that
  ten lenses out of twenty are scratched, what might that tell you
  about the manufacturing process?
\end{enumerate}

\begin{enumerate}
\def\labelenumi{\arabic{enumi}.}
\setcounter{enumi}{9}
\tightlist
\item
  The proportion of brown M\&M's in a milk chocolate packet is
  approximately 14\% (Madison, 2013). Suppose a package of M\&M's
  typically contains 52 M\&M's.
\end{enumerate}

\begin{enumerate}
\def\labelenumi{\alph{enumi}.}
\item
  State the random variable.
\item
  Argue that this is a binomial experiment
\end{enumerate}

Find the probability that

\begin{enumerate}
\def\labelenumi{\alph{enumi}.}
\setcounter{enumi}{2}
\item
  Six M\&M's are brown.
\item
  Twenty-five M\&M's are brown.
\item
  All of the M\&M's are brown.
\item
  Would it be unusual for a package to have only brown M\&M's? If this
  were to happen, what would you think is the reason?
\end{enumerate}

\textbf{\\
}

\hypertarget{mean-and-standard-deviation-of-binomial-distribution}{%
\section{Mean and Standard Deviation of Binomial Distribution}\label{mean-and-standard-deviation-of-binomial-distribution}}

If you list all possible values of \emph{x} in a Binomial distribution, you get the \textbf{Binomial Probability Distribution} (pdf). You can draw a histogram of the pdf and find the mean, variance, and standard deviation of it.

For a general discrete probability distribution, you can find the mean, the variance, and the standard deviation for a pdf using the general formulas {[}MISSING EQ?{]}

\begin{quote}
, , and .
\end{quote}

These formulas are useful, but if you know the type of distribution, like Binomial, then you can find the mean and standard deviation using easier formulas. They are derived from the general formulas.

For a Binomial distribution, , the expected number of successes, , the
variance, and , the standard deviation for the number of success are
given by the formulas:

Where \emph{p} is the probability of success and .

\textbf{Example \#5.3.1: Finding the Probability Distribution, Mean, Variance,
and Standard Deviation of a Binomial Distribution}

\begin{quote}
When looking at a person's eye color, it turns out that 1\% of people in the world has green eyes ("What percentage of," 2013). Consider a group of 20 people.
\end{quote}

\begin{enumerate}
\def\labelenumi{\alph{enumi}.}
\item
  State the random variable.

  \textbf{Solution:}

  x = number of people who have green eyes
\item
  Write the probability distribution.

  \textbf{Solution:}

  In this case you need to write each value of \emph{x} and its
  corresponding probability. It is easiest to do this by using the
  binompdf command, but don't put in the \emph{r} value. You may want to
  set your calculator to only three decimal places, so it is easier to
  see the values and you don't need much more precision than that. The
  command would look like .

  This produces the information in table \#5.3.1

  \textbf{Table \#5.3.1: Probability Distribution for Number of People with
  Green Eyes}
\end{enumerate}

\begin{longtable}[]{@{}ll@{}}
\toprule
\emph{x} &\tabularnewline
\midrule
\endhead
0 & 0.818\tabularnewline
1 & 0.165\tabularnewline
2 & 0.016\tabularnewline
3 & 0.001\tabularnewline
4 & 0.000\tabularnewline
5 & 0.000\tabularnewline
6 & 0.000\tabularnewline
7 & 0.000\tabularnewline
8 & 0.000\tabularnewline
9 & 0.000\tabularnewline
10 & 0.000\tabularnewline
\bottomrule
\end{longtable}

20 0.000

Notice that after x = 4, the probability values are all 0.000. This just
means they are really small numbers.

\begin{enumerate}
\def\labelenumi{\alph{enumi}.}
\setcounter{enumi}{2}
\item
  Draw a histogram.

  \textbf{Solution:}

  You can draw the histogram on the TI-83/84 or other technology. The
  graph would look like in figure \#5.3.1.

  \textbf{Figure \#5.3.1: Histogram Created on TI-83/84}

  \includegraphics[width=2.75in,height=1.86111in]{media/image142.png}

  This graph is very skewed to the right.
\item
  Find the mean.

  \textbf{Solution:}
\end{enumerate}

Since this is a binomial, then you can use the formula . So

You expect on average that out of 20 people, less than 1 would have green eyes.

\begin{enumerate}
\def\labelenumi{\alph{enumi}.}
\setcounter{enumi}{4}
\item
  Find the variance.

  \textbf{Solution:}
\end{enumerate}

Since this is a binomial, then you can use the formula .

\begin{enumerate}
\def\labelenumi{\alph{enumi}.}
\setcounter{enumi}{5}
\item
  Find the standard deviation.

  \textbf{Solution:}
\end{enumerate}

Once you have the variance, you just take the square root of the variance to find the standard deviation.

\hypertarget{homework-15}{%
\subsection{Homework}\label{homework-15}}

\begin{enumerate}
\def\labelenumi{\arabic{enumi}.}
\tightlist
\item
  Suppose a random variable, \emph{x}, arises from a binomial experiment. Suppose \emph{n} = 6, and \emph{p} = 0.13.
\end{enumerate}

\begin{enumerate}
\def\labelenumi{\alph{enumi}.}
\item
  Write the probability distribution.
\item
  Draw a histogram.
\item
  Describe the shape of the histogram.
\item
  Find the mean.
\item
  Find the variance.
\item
  Find the standard deviation.
\end{enumerate}

\begin{enumerate}
\def\labelenumi{\arabic{enumi}.}
\setcounter{enumi}{1}
\tightlist
\item
  Suppose a random variable, \emph{x}, arises from a binomial experiment. Suppose \emph{n} = 10, and \emph{p} = 0.81.
\end{enumerate}

\begin{enumerate}
\def\labelenumi{\alph{enumi}.}
\item
  Write the probability distribution.
\item
  Draw a histogram.
\item
  Describe the shape of the histogram.
\item
  Find the mean.
\item
  Find the variance.
\item
  Find the standard deviation.
\end{enumerate}

\begin{enumerate}
\def\labelenumi{\arabic{enumi}.}
\setcounter{enumi}{2}
\tightlist
\item
  Suppose a random variable, \emph{x}, arises from a binomial experiment.
  Suppose \emph{n} = 7, and \emph{p} = 0.50.
\end{enumerate}

\begin{enumerate}
\def\labelenumi{\alph{enumi}.}
\item
  Write the probability distribution.
\item
  Draw a histogram.
\item
  Describe the shape of the histogram.
\item
  Find the mean.
\item
  Find the variance.
\item
  Find the standard deviation.
\end{enumerate}

\begin{enumerate}
\def\labelenumi{\arabic{enumi}.}
\setcounter{enumi}{3}
\tightlist
\item
  Approximately 10\% of all people are left-handed. Consider a grouping
  of fifteen people.
\end{enumerate}

\begin{enumerate}
\def\labelenumi{\alph{enumi}.}
\item
  State the random variable.
\item
  Write the probability distribution.
\item
  Draw a histogram.
\item
  Describe the shape of the histogram.
\item
  Find the mean.
\item
  Find the variance.
\item
  Find the standard deviation.
\end{enumerate}

\begin{enumerate}
\def\labelenumi{\arabic{enumi}.}
\setcounter{enumi}{4}
\tightlist
\item
  According to an article in the American Heart Association's
  publication \emph{Circulation}, 24\% of patients who had been hospitalized
  for an acute myocardial infarction did not fill their cardiac
  medication by the seventh day of being discharged (Ho, Bryson \&
  Rumsfeld, 2009). Suppose there are twelve people who have been
  hospitalized for an acute myocardial infarction.
\end{enumerate}

\begin{enumerate}
\def\labelenumi{\alph{enumi}.}
\item
  State the random variable.
\item
  Write the probability distribution.
\item
  Draw a histogram.
\item
  Describe the shape of the histogram.
\item
  Find the mean.
\item
  Find the variance.
\item
  Find the standard deviation.
\end{enumerate}

\begin{enumerate}
\def\labelenumi{\arabic{enumi}.}
\setcounter{enumi}{5}
\tightlist
\item
  Eyeglassomatic manufactures eyeglasses for different retailers. In
  March 2010, they tested to see how many defective lenses they made,
  and there were 16.9\% defective lenses due to scratches. Suppose
  Eyeglassomatic examined twenty eyeglasses.
\end{enumerate}

\begin{enumerate}
\def\labelenumi{\alph{enumi}.}
\item
  State the random variable.
\item
  Write the probability distribution.
\item
  Draw a histogram.
\item
  Describe the shape of the histogram.
\item
  Find the mean.
\item
  Find the variance.
\item
  Find the standard deviation.
\end{enumerate}

\begin{enumerate}
\def\labelenumi{\arabic{enumi}.}
\setcounter{enumi}{6}
\tightlist
\item
  The proportion of brown M\&M's in a milk chocolate packet is
  approximately 14\% (Madison, 2013). Suppose a package of M\&M's
  typically contains 52 M\&M's.
\end{enumerate}

\begin{enumerate}
\def\labelenumi{\alph{enumi}.}
\item
  State the random variable.
\item
  Find the mean.
\item
  Find the variance.
\item
  Find the standard deviation.
\end{enumerate}

Data Sources:

\emph{11 little-known facts about left-handers}. (2013, October 21).
Retrieved from
\url{http://www.huffingtonpost.com/2012/10/29/left-handed-facts-lefties_n_2005864.html}

\emph{CDC-data and statistics, autism spectrum disorders - ncbdd}. (2013,
October 21). Retrieved from \url{http://www.cdc.gov/ncbddd/autism/data.html}

Ho, P. M., Bryson, C. L., \& Rumsfeld, J. S. (2009). Medication
adherence. \emph{Circulation}, \emph{119}(23), 3028-3035. Retrieved from
\url{http://circ.ahajournals.org/content/119/23/3028}

\emph{Households by age of householder and size of household: 1990 to 2010}.
(2013, October 19). Retrieved from
\url{http://www.census.gov/compendia/statab/2012/tables/12s0062.pdf}

Madison, J. (2013, October 15). \emph{M\&M's color distribution analysis}.
Retrieved from
\url{http://joshmadison.com/2007/12/02/mms-color-distribution-analysis/}

\emph{What percentage of people have green eyes?}. (2013, October 21).
Retrieved from
\url{http://www.ask.com/question/what-percentage-of-people-have-green-eyes}

\hypertarget{continuous-probability-distributions}{%
\chapter{Continuous Probability Distributions}\label{continuous-probability-distributions}}

Chapter 5 dealt with probability distributions arising from discrete random variables. Mostly that chapter focused on the binomial experiment. There are many other experiments from discrete random variables that exist but are not covered in this book.

Chapter 6 deals with probability distributions that arise from continuous random variables. The focus of this chapter is a distribution known as the normal distribution, though realize that there are many other distributions that exist. A few others are examined in future chapters.

\hypertarget{uniform-distribution}{%
\section{Uniform Distribution}\label{uniform-distribution}}

If you have a situation where the probability is always the same, then this is known as a uniform distribution. An example would be waiting for a commuter train. The commuter trains on the Blue and Green Lines for the Regional Transit Authority (RTA) in Cleveland, OH, have a waiting time during peak hours of ten minutes ("2012 annual report," 2012). If you are waiting for a train, you have anywhere from zero minutes to ten minutes to wait. Your probability of having to wait any number of minutes in that interval is the same. This is a uniform distribution. The graph of this distribution is in figure \#6.1.1.

\textbf{Figure \#6.1.1: Uniform Distribution Graph}

\begin{quote}
\includegraphics[width=3.38889in,height=1.87386in]{media/image1.png}
\end{quote}

Suppose you want to know the probability that you will have to wait between five and ten minutes for the next train. You can look at the probability graphically such as in figure \#6.1.2.

\textbf{Figure \#6.1.2: Uniform Distribution with P(5 \textless{} x \textless{} 10)}

\begin{quote}
\includegraphics[width=3.36111in,height=1.8585in]{media/image2.png}
\end{quote}

How would you find this probability? Calculus says that the probability is the area under the curve. Notice that the shape of the shaded area is a rectangle, and the area of a rectangle is length times width. The length is and the width is 0.1. The probability is , where and \emph{x} is the waiting time during peak hours.

\textbf{Example \#6.1.1: Finding Probabilities in a Uniform Distribution}

\begin{quote}
The commuter trains on the Blue and Green Lines for the Regional Transit Authority (RTA) in Cleveland, OH, have a waiting time during peak rush hour periods of ten minutes ("2012 annual report," 2012).
\end{quote}

\begin{enumerate}
\def\labelenumi{\alph{enumi}.}
\tightlist
\item
  State the random variable.
\end{enumerate}

\begin{quote}
\textbf{Solution:}

\emph{x} = waiting time during peak hours
\end{quote}

\begin{enumerate}
\def\labelenumi{\alph{enumi}.}
\setcounter{enumi}{1}
\tightlist
\item
  Find the probability that you have to wait between four and six
  minutes for a train.
\end{enumerate}

\begin{quote}
\textbf{Solution:}
\end{quote}

\begin{enumerate}
\def\labelenumi{\alph{enumi}.}
\setcounter{enumi}{2}
\tightlist
\item
  Find the probability that you have to wait between three and seven
  minutes for a train.
\end{enumerate}

\begin{quote}
\textbf{Solution:}
\end{quote}

\begin{enumerate}
\def\labelenumi{\alph{enumi}.}
\setcounter{enumi}{3}
\tightlist
\item
  Find the probability that you have to wait between zero and ten
  minutes for a train.
\end{enumerate}

\begin{quote}
\textbf{Solution:}
\end{quote}

\begin{enumerate}
\def\labelenumi{\alph{enumi}.}
\setcounter{enumi}{4}
\tightlist
\item
  Find the probability of waiting exactly five minutes.
\end{enumerate}

\begin{quote}
\textbf{Solution:}

Since this would be just one line, and the width of the line is 0,
then the
\end{quote}

Notice that in example \#6.1.1d, the probability is equal to one. This
is because the probability that was computed is the area under the
entire curve. Just like in discrete probability distributions, where the
total probability was one, the probability of the entire curve is one.
This is the reason that the height of the curve is 0.1. In general, the
height of a uniform distribution that ranges between \emph{a} and \emph{b}, is .

\hypertarget{homework-16}{%
\subsection{Homework}\label{homework-16}}

\begin{enumerate}
\def\labelenumi{\arabic{enumi}.}
\tightlist
\item
  The commuter trains on the Blue and Green Lines for the Regional
  Transit Authority (RTA) in Cleveland, OH, have a waiting time during
  peak rush hour periods of ten minutes ("2012 annual report,"
  2012).
\end{enumerate}

\begin{enumerate}
\def\labelenumi{\alph{enumi}.}
\item
  State the random variable.
\item
  Find the probability of waiting between two and five minutes.
\item
  Find the probability of waiting between seven and ten minutes.
\item
  Find the probability of waiting eight minutes exactly.
\end{enumerate}

\begin{enumerate}
\def\labelenumi{\arabic{enumi}.}
\setcounter{enumi}{1}
\tightlist
\item
  The commuter trains on the Red Line for the Regional Transit
  Authority (RTA) in Cleveland, OH, have a waiting time during peak
  rush hour periods of eight minutes ("2012 annual report," 2012).
\end{enumerate}

\begin{enumerate}
\def\labelenumi{\alph{enumi}.}
\item
  State the random variable.
\item
  Find the height of this uniform distribution.
\item
  Find the probability of waiting between four and five minutes.
\item
  Find the probability of waiting between three and eight minutes.
\item
  Find the probability of waiting five minutes exactly.
\end{enumerate}

\textbf{\\
}

\hypertarget{graphs-of-the-normal-distribution}{%
\section{Graphs of the Normal Distribution}\label{graphs-of-the-normal-distribution}}

Many real life problems produce a histogram that is a symmetric, unimodal, and bell-shaped continuous probability distribution. For example: height, blood pressure, and cholesterol level. However, not every bell shaped curve is a normal curve. In a normal curve, there is a specific relationship between its ``height'' and its ``width.''

Normal curves can be tall and skinny or they can be short and fat. They are all symmetric, unimodal, and centered at , the population mean. Figure \#6.2.1 shows two different normal curves drawn on the same scale. Both have but the one on the left has a standard deviation of 10 and the one on the right has a standard deviation of 5. Notice that the larger standard deviation makes the graph wider (more spread out) and shorter.

\textbf{Figure \#6.2.1: Different Normal Distribution Graphs}

\includegraphics[width=2.5in,height=1.51389in]{media/image12.png}
\includegraphics[width=2.375in,height=1.55556in]{media/image13.png}

Every normal curve has common features. These are detailed in figure
\#6.2.2.

\textbf{Figure \#6.2.2: Typical Graph of a Normal Curve}

\includegraphics[width=6in,height=2.38889in]{media/image14.png}

\begin{itemize}
\item
  The center, or the highest point, is at the population mean, .
\item
  The transition points (inflection points) are the places where the
  curve changes from a ``hill'' to a ``valley''. The distance from the
  mean to the transition point is one standard deviation, .
\item
  The area under the whole curve is exactly 1. Therefore, the area
  under the half below or above the mean is 0.5.
\end{itemize}

The equation that creates this curve is .

Just as in a discrete probability distribution, the object is to find
the probability of an event occurring. However, unlike in a discrete
probability distribution where the event can be a single value, in a
continuous probability distribution the event must be a range. You are
interested in finding the probability of \emph{x} occurring in the range
between \emph{a} and \emph{b}, or . Calculus tells us that to find this you find
the area under the curve above the interval from \emph{a} to \emph{b}.

\begin{quote}
is the area under the curve above the interval from \emph{a} to \emph{b}.
\end{quote}

\textbf{Figure \#6.2.3: Probability of an Event}

\includegraphics[width=4.58333in,height=2.15012in]{media/image20.jpg}

Before looking at the process for finding the probabilities under the
normal curve, it is somewhat useful to look at the \textbf{Empirical Rule}
that gives approximate values for these areas. The Empirical Rule is
just an approximation and it will only be used in this section to give
you an idea of what the size of the probabilities is for different
shadings. A more precise method for finding probabilities for the normal
curve will be demonstrated in the next section. Please do not use the
empirical rule except for real rough estimates.

\textbf{The Empirical Rule} for any normal distribution:

Approximately 68\% of the data is within one standard deviation of the
mean.

Approximately 95\% of the data is within two standard deviations of the
mean.

Approximately 99.7\% of the data is within three standard deviations of
the mean.

\textbf{\\
}

\textbf{Figure \#6.2.4: Empirical Rule}

\includegraphics[width=5.02778in,height=2.41667in]{media/image21.png}

Be careful, there is still some area left over in each end. Remember,
the maximum a probability can be is 100\%, so if you calculate you will
see that for both ends together there is 0.3\% of the curve. Because of
symmetry, you can divide this equally between both ends and find that
there is 0.15\% in each tail beyond the .

\textbf{\\
}

\hypertarget{finding-probabilities-for-the-normal-distribution}{%
\section{Finding Probabilities for the Normal Distribution}\label{finding-probabilities-for-the-normal-distribution}}

The Empirical Rule is just an approximation and only works for certain values. What if you want to find the probability for \emph{x} values that are not integer multiples of the standard deviation? The probability is the area under the curve. To find areas under the curve, you need calculus. Before technology, you needed to convert every \emph{x} value to a standardized number, called the \emph{z}-score or \emph{z-}value or simply just \emph{z}. The \emph{z}-score is a measure of how many standard deviations an \emph{x} value is from the mean. To convert from a normally distributed \emph{x} value to a \emph{z}-score, you use the following formula.

\textbf{\emph{z}-score}

where = mean of the population of the \emph{x} value and = standard deviation for the population of the \emph{x} value

The \emph{z}-score is normally distributed, with a mean of 0 and a standard deviation of 1. It is known as the standard normal curve. Once you have the \emph{z-}score, you can look up the \emph{z-}score in the standard normal distribution table.

The \textbf{standard normal distribution, \emph{z},} has a mean of and a standard deviation of {[}MISSING EQ?{]}.

\textbf{Figure \#6.3.1: Standard Normal Curve}

\includegraphics[width=4.52778in,height=2.31944in]{media/image29.png}

Luckily, these days technology can find probabilities for you without converting to the \emph{z}-score and looking the probabilities up in a table. There are many programs available that will calculate the probability for a normal curve including Excel and the TI-83/84. There are also online sites available. The following examples show how to do the calculation on the TI-83/84 and with R. The command on the TI-83/84 is in the DISTR menu and is normalcdf(. You then type in the lower limit, upper limit, mean, standard deviation in that order and including the commas. The command on R to find the area to the left is pnorm(z-value or x-value, mean, standard deviation).

\textbf{Example \#6.3.1: General Normal Distribution}

\begin{quote}
The length of a human pregnancy is normally distributed with a mean of 272 days with a standard deviation of 9 days (Bhat \& Kushtagi, 2006).
\end{quote}

\begin{enumerate}
\def\labelenumi{\alph{enumi}.}
\tightlist
\item
  State the random variable.
\end{enumerate}

\begin{quote}
\textbf{Solution:}

\emph{x} = length of a human pregnancy
\end{quote}

\begin{enumerate}
\def\labelenumi{\alph{enumi}.}
\setcounter{enumi}{1}
\tightlist
\item
  Find the probability of a pregnancy lasting more than 280 days.
\end{enumerate}

\begin{quote}
\textbf{Solution:}

First translate the statement into a mathematical statement.

Now, draw a picture. Remember the center of this normal curve is 272.
\end{quote}

\textbf{Figure \#6.3.2: Normal Distribution Graph for Example \#6.3.1b}

\begin{quote}
{\[CHART\]}

To find the probability on the TI-83/84, looking at the picture you
realize the lower limit is 280. The upper limit is infinity. The
calculator doesn't have infinity on it, so you need to put in a really
big number. Some people like to put in 1000, but if you are working
with numbers that are bigger than 1000, then you would have to
remember to change the upper limit. The safest number to use is ,
which you put in the calculator as 1E99 (where E is the EE button on
the calculator). The command looks like:
\end{quote}

\textbf{Figure \#6.3.3: TI-83/84 Output for Example \#6.3.1b}

\begin{quote}
\includegraphics[width=2.75in,height=1.86111in]{media/image33.png}

To find the probability on R, R always gives the probability to the
left of the value. The total area under the curve is 1, so if you want
the area to the right, then you find the area to the left and subtract
from 1. The command looks like:

Thus,

Thus18.7\% of all pregnancies last more than 280 days. This is not
unusual since the probability is greater than 5\%.
\end{quote}

\begin{enumerate}
\def\labelenumi{\alph{enumi}.}
\setcounter{enumi}{2}
\tightlist
\item
  Find the probability of a pregnancy lasting less than 250 days.
\end{enumerate}

\begin{quote}
\textbf{Solution:}

First translate the statement into a mathematical statement.

Now, draw a picture. Remember the center of this normal curve is 272.
\end{quote}

\textbf{Figure \#6.3.4: Normal Distribution Graph for Example \#6.3.1c}

\begin{quote}
{\[CHART\]}

To find the probability on the TI-83/84, looking at the picture,
though it is hard to see in this case, the lower limit is negative
infinity. Again, the calculator doesn't have this on it, put in a
really small number, such as on the calculator.
\end{quote}

\textbf{Figure \#6.3.5: TI-83/84 Output for Example \#6.3.1c}

\begin{quote}
\includegraphics[width=2.75in,height=1.86111in]{media/image38.png}

.

To find the probability on R, R always gives the probability to the
left of the value. Looking at the figure, you can see the area you
want is to the left. The command looks like:

Thus 0.73\% of all pregnancies last less than 250 days. This is unusual
since the probability is less than 5\%.
\end{quote}

\begin{enumerate}
\def\labelenumi{\alph{enumi}.}
\setcounter{enumi}{3}
\tightlist
\item
  Find the probability that a pregnancy lasts between 265 and 280
  days.
\end{enumerate}

\begin{quote}
\textbf{Solution:}

First translate the statement into a mathematical statement.

Now, draw a picture. Remember the center of this normal curve is 272.

\textbf{Figure \#6.3.6: Normal Distribution Graph for Example \#6.3.1d}

{\[CHART\]}

In this case, the lower limit is 265 and the upper limit is 280.

Using the calculator
\end{quote}

\textbf{Figure \#6.3.7: TI-83/84 Output for Example \#6.3.1d}

\begin{quote}
\includegraphics[width=2.75in,height=1.86111in]{media/image42.png}

To use R, you have to remember that R gives you the area to the left.
So

is the area to the left of 280 and is the area to the left of 265. So
the area is between the two would be the bigger one minus the smaller
one. So,

Thus 59.5\% of all pregnancies last between 265 and 280 days.
\end{quote}

\begin{enumerate}
\def\labelenumi{\alph{enumi}.}
\setcounter{enumi}{4}
\tightlist
\item
  Find the length of pregnancy that 10\% of all pregnancies last less
  than.
\end{enumerate}

\begin{quote}
\textbf{Solution:}

This problem is asking you to find an \emph{x} value from a probability.
You want to find the \emph{x} value that has 10\% of the length of
pregnancies to the left of it. On the TI-83/84, the command is in the
DISTR menu and is called invNorm(. The invNorm( command needs the area
to the left. In this case, that is the area you are given. For the
command on the calculator, once you have invNorm( on the main screen
you type in the probability to the left, mean, standard deviation, in
that order with the commas.
\end{quote}

\textbf{Figure \#6.3.8: TI-83/84 Output for Example \#6.3.1e}

\begin{quote}
\includegraphics[width=2.75in,height=1.86111in]{media/image47.png}

On R, the command is qnorm(area to the left, mean, standard
deviation). For this example that would be qnorm(0.1, 272, 9)

Thus 10\% of all pregnancies last less than approximately 260 days.
\end{quote}

\begin{enumerate}
\def\labelenumi{\alph{enumi}.}
\setcounter{enumi}{5}
\tightlist
\item
  Suppose you meet a woman who says that she was pregnant for less
  than 250 days. Would this be unusual and what might you think?
\end{enumerate}

\begin{quote}
\textbf{Solution:}

From part (c) you found the probability that a pregnancy lasts less
than 250 days is 0.73\%. Since this is less than 5\%, it is very
unusual. You would think that either the woman had a premature baby,
or that she may be wrong about when she actually became pregnant.
\end{quote}

\textbf{Example \#6.3.2: General Normal Distribution}

\begin{quote}
The mean mathematics SAT score in 2012 was 514 with a standard
deviation of 117 ("Total group profile," 2012). Assume the
mathematics SAT score is normally distributed.
\end{quote}

\begin{enumerate}
\def\labelenumi{\alph{enumi}.}
\tightlist
\item
  State the random variable.
\end{enumerate}

\begin{quote}
\textbf{Solution:}

\emph{x} = mathematics SAT score
\end{quote}

\begin{enumerate}
\def\labelenumi{\alph{enumi}.}
\setcounter{enumi}{1}
\tightlist
\item
  Find the probability that a person has a mathematics SAT score
  over 700.
\end{enumerate}

\begin{quote}
\textbf{Solution:}

First translate the statement into a mathematical statement.

Now, draw a picture. Remember the center of this normal curve is 514.
\end{quote}

\textbf{Figure \#6.3.9: Normal Distribution Graph for Example \#6.3.2b}

{\[CHART\]}

\begin{quote}
On TI-83/84:

On R:

There is a 5.6\% chance that a person scored above a 700 on the
mathematics SAT test. This is not unusual.
\end{quote}

\begin{enumerate}
\def\labelenumi{\alph{enumi}.}
\setcounter{enumi}{2}
\tightlist
\item
  Find the probability that a person has a mathematics SAT score of
  less than 400.
\end{enumerate}

\begin{quote}
\textbf{Solution:}

First translate the statement into a mathematical statement.

Now, draw a picture. Remember the center of this normal curve is 514.
\end{quote}

\textbf{Figure \#6.3.10: Normal Distribution Graph for Example \#6.3.2c}

\begin{quote}
{\[CHART\]}

On TI-83/84:

On R:

So, there is a 16.5\% chance that a person scores less than a 400 on
the mathematics part of the SAT.
\end{quote}

\begin{enumerate}
\def\labelenumi{\alph{enumi}.}
\setcounter{enumi}{3}
\tightlist
\item
  Find the probability that a person has a mathematics SAT score
  between a 500 and a 650.
\end{enumerate}

\begin{quote}
\textbf{Solution:}

First translate the statement into a mathematical statement.

Now, draw a picture. Remember the center of this normal curve is 514.
\end{quote}

\textbf{Figure \#6.3.11: Normal Distribution Graph for Example \#6.3.2d}

{\[CHART\]}

\begin{quote}
On TI-83/84:

On R:

So, there is a 42.5\% chance that a person has a mathematical SAT score
between 500 and 650.
\end{quote}

\begin{enumerate}
\def\labelenumi{\alph{enumi}.}
\setcounter{enumi}{4}
\tightlist
\item
  Find the mathematics SAT score that represents the top 1\% of all
  scores.
\end{enumerate}

\begin{quote}
\textbf{Solution:}

This problem is asking you to find an \emph{x} value from a probability.
You want to find the \emph{x} value that has 1\% of the mathematics SAT
scores to the right of it. Remember, the calculator and R always need
the area to the left, you need to find the area to the left by .

On TI-83/84:

On R:

So, 1\% of all people who took the SAT scored over about 786 points on
the mathematics SAT.
\end{quote}

\hypertarget{homework-17}{%
\subsection{Homework}\label{homework-17}}

\begin{enumerate}
\def\labelenumi{\arabic{enumi}.}
\tightlist
\item
  Find each of the probabilities, where \emph{z} is a \emph{z}-score from the standard normal distribution with mean of and standard deviation . Make sure you draw a picture for each problem.
\end{enumerate}

\begin{quote}
a.)

b.)

c.)

d.)
\end{quote}

\begin{enumerate}
\def\labelenumi{\arabic{enumi}.}
\setcounter{enumi}{1}
\tightlist
\item
  Find the \emph{z}-score corresponding to the given area. Remember, \emph{z} is distributed as the standard normal distribution with mean of {[}MISSING EQ{]} and standard deviation {[}MISSING EQ{]}.
\end{enumerate}

\begin{enumerate}
\def\labelenumi{\alph{enumi}.}
\item
  The area to the left of \emph{z} is 15\%.
\item
  The area to the right of \emph{z} is 65\%.
\item
  The area to the left of \emph{z} is 10\%.
\item
  The area to the right of \emph{z} is 5\%.
\item
  The area between and \emph{z} is 95\%. (Hint draw a picture and figure out the area to the left of the .)
\item
  The area between and \emph{z} is 99\%.
\end{enumerate}

\begin{enumerate}
\def\labelenumi{\arabic{enumi}.}
\setcounter{enumi}{2}
\tightlist
\item
  If a random variable that is normally distributed has a mean of 25
  and a standard deviation of 3, convert the given value to a
  \emph{z}-score.
\end{enumerate}

\begin{enumerate}
\def\labelenumi{\alph{enumi}.}
\item
  \emph{x} = 23
\item
  \emph{x} = 33
\item
  \emph{x} = 19
\item
  \emph{x} = 45
\end{enumerate}

\begin{enumerate}
\def\labelenumi{\arabic{enumi}.}
\setcounter{enumi}{3}
\tightlist
\item
  According to the WHO MONICA Project the mean blood pressure for people in China is 128 mmHg with a standard deviation of 23 mmHg (Kuulasmaa, Hense \& Tolonen, 1998). Assume that blood pressure is normally distributed.
\end{enumerate}

\begin{enumerate}
\def\labelenumi{\alph{enumi}.}
\item
  State the random variable.
\item
  Find the probability that a person in China has blood pressure of
  135 mmHg or more.
\item
  Find the probability that a person in China has blood pressure of
  141 mmHg or less.
\item
  Find the probability that a person in China has blood pressure
  between 120 and 125 mmHg.
\item
  Is it unusual for a person in China to have a blood pressure of 135
  mmHg? Why or why not?
\item
  What blood pressure do 90\% of all people in China have less than?
\end{enumerate}

\begin{enumerate}
\def\labelenumi{\arabic{enumi}.}
\setcounter{enumi}{4}
\tightlist
\item
  The size of fish is very important to commercial fishing. A study
  conducted in 2012 found the length of Atlantic cod caught in nets in
  Karlskrona to have a mean of 49.9 cm and a standard deviation of
  3.74 cm (Ovegard, Berndt \& Lunneryd, 2012). Assume the length of
  fish is normally distributed.
\end{enumerate}

\begin{enumerate}
\def\labelenumi{\alph{enumi}.}
\item
  State the random variable.
\item
  Find the probability that an Atlantic cod has a length less than
  52 cm.
\item
  Find the probability that an Atlantic cod has a length of more than
  74 cm.
\item
  Find the probability that an Atlantic cod has a length between 40.5
  and 57.5 cm.
\item
  If you found an Atlantic cod to have a length of more than 74 cm,
  what could you conclude?
\item
  What length are 15\% of all Atlantic cod longer than?
\end{enumerate}

\begin{enumerate}
\def\labelenumi{\arabic{enumi}.}
\setcounter{enumi}{5}
\tightlist
\item
  The mean cholesterol levels of women age 45-59 in Ghana, Nigeria,
  and Seychelles is 5.1 mmol/l and the standard deviation is 1.0
  mmol/l (Lawes, Hoorn, Law \& Rodgers, 2004). Assume that cholesterol
  levels are normally distributed.
\end{enumerate}

\begin{enumerate}
\def\labelenumi{\alph{enumi}.}
\item
  State the random variable.
\item
  Find the probability that a woman age 45-59 in Ghana, Nigeria, or
  Seychelles has a cholesterol level above 6.2 mmol/l (considered a
  high level).
\item
  Find the probability that a woman age 45-59 in Ghana, Nigeria, or
  Seychelles has a cholesterol level below 5.2 mmol/l (considered a
  normal level).
\item
  Find the probability that a woman age 45-59 in Ghana, Nigeria, or
  Seychelles has a cholesterol level between 5.2 and 6.2 mmol/l
  (considered borderline high).
\item
  If you found a woman age 45-59 in Ghana, Nigeria, or Seychelles
  having a cholesterol level above 6.2 mmol/l, what could you
  conclude?
\item
  What value do 5\% of all woman ages 45-59 in Ghana, Nigeria, or
  Seychelles have a cholesterol level less than?
\end{enumerate}

\begin{enumerate}
\def\labelenumi{\arabic{enumi}.}
\setcounter{enumi}{6}
\tightlist
\item
  In the United States, males between the ages of 40 and 49 eat on
  average 103.1 g of fat every day with a standard deviation of 4.32 g
  ("What we eat," 2012). Assume that the amount of fat a person eats
  is normally distributed.
\end{enumerate}

\begin{enumerate}
\def\labelenumi{\alph{enumi}.}
\item
  State the random variable.
\item
  Find the probability that a man age 40-49 in the U.S. eats more than
  110 g of fat every day.
\item
  Find the probability that a man age 40-49 in the U.S. eats less than
  93 g of fat every day.
\item
  Find the probability that a man age 40-49 in the U.S. eats less than
  65 g of fat every day.
\item
  If you found a man age 40-49 in the U.S. who says he eats less than
  65 g of fat every day, would you believe him? Why or why not?
\item
  What daily fat level do 5\% of all men age 40-49 in the U.S. eat more
  than?
\end{enumerate}

\begin{enumerate}
\def\labelenumi{\arabic{enumi}.}
\setcounter{enumi}{7}
\tightlist
\item
  A dishwasher has a mean life of 12 years with an estimated standard
  deviation of 1.25 years ("Appliance life expectancy," 2013).
  Assume the life of a dishwasher is normally distributed.
\end{enumerate}

\begin{enumerate}
\def\labelenumi{\alph{enumi}.}
\item
  State the random variable.
\item
  Find the probability that a dishwasher will last more than 15 years.
\item
  Find the probability that a dishwasher will last less than 6 years.
\item
  Find the probability that a dishwasher will last between 8 and 10
  years.
\item
  If you found a dishwasher that lasted less than 6 years, would you
  think that you have a problem with the manufacturing process? Why or
  why not?
\item
  A manufacturer of dishwashers only wants to replace free of charge
  5\% of all dishwashers. How long should the manufacturer make the
  warranty period?
\end{enumerate}

\begin{enumerate}
\def\labelenumi{\arabic{enumi}.}
\setcounter{enumi}{8}
\tightlist
\item
  The mean starting salary for nurses is \$67,694 nationally ("Staff
  nurse -," 2013). The standard deviation is approximately \$10,333.
  Assume that the starting salary is normally distributed.
\end{enumerate}

\begin{enumerate}
\def\labelenumi{\alph{enumi}.}
\item
  State the random variable.
\item
  Find the probability that a starting nurse will make more than
  \$80,000.
\item
  Find the probability that a starting nurse will make less than
  \$60,000.
\item
  Find the probability that a starting nurse will make between
  \$55,000 and \$72,000.
\item
  If a nurse made less than \$50,000, would you think the nurse was
  under paid? Why or why not?
\item
  What salary do 30\% of all nurses make more than?
\end{enumerate}

\begin{enumerate}
\def\labelenumi{\arabic{enumi}.}
\setcounter{enumi}{9}
\tightlist
\item
  The mean yearly rainfall in Sydney, Australia, is about 137 mm and
  the standard deviation is about 69 mm ("Annual maximums of,"
  2013). Assume rainfall is normally distributed.
\end{enumerate}

\begin{enumerate}
\def\labelenumi{\alph{enumi}.}
\item
  State the random variable.
\item
  Find the probability that the yearly rainfall is less than 100 mm.
\item
  Find the probability that the yearly rainfall is more than 240 mm.
\item
  Find the probability that the yearly rainfall is between 140 and
  250 mm.
\item
  If a year has a rainfall less than 100mm, does that mean it is an
  unusually dry year? Why or why not?
\item
  What rainfall amount are 90\% of all yearly rainfalls more than?
\end{enumerate}

\textbf{\\
}

\hypertarget{assessing-normality}{%
\section{Assessing Normality}\label{assessing-normality}}

The distributions you have seen up to this point have been assumed to be normally distributed, but how do you determine if it is normally distributed. One way is to take a sample and look at the sample to determine if it appears normal. If the sample looks normal, then most likely the population is also. Here are some guidelines that are use to help make that determination.

\begin{enumerate}
\def\labelenumi{\arabic{enumi}.}
\item
  \textbf{Histogram:} Make a histogram. For a normal distribution, the histogram should be roughly bell-shaped. For small samples, this is not very accurate, and another method is needed. A distribution may not look normally distributed from the histogram, but it still may be normally distributed.
\item
  \textbf{Outliers:} For a normal distribution, there should not be more than one outlier. One way to check for outliers is to use a modified box plot. Outliers are values that are shown as dots outside of the rest of the values. If you don't have a modified box plot, outliers are those data values that are:
\end{enumerate}

\begin{quote}
Above \emph{Q3}, the third quartile, by an amount greater than 1.5 times the interquartile range (\emph{IQR})

Below \emph{Q1}, the first quartile, by an amount greater than 1.5 times the interquartile range (\emph{IQR})

Note: if there is one outlier, that outlier could have a dramatic effect on the results especially if it is an extreme outlier. However, there are times where a distribution has more than one outlier, but it is still normally distributed. The guideline of only one outlier is just a guideline.
\end{quote}

\begin{enumerate}
\def\labelenumi{\arabic{enumi}.}
\setcounter{enumi}{2}
\tightlist
\item
  \textbf{Normal quantile plot (or normal probability plot):} This plot is provided through statistical software on a computer or graphing calculator. If the points lie close to a line, the data comes from a distribution that is approximately normal. If the points do not lie close to a line or they show a pattern that is not a line, the data are likely to come from a distribution that is not normally distributed.
\end{enumerate}

\begin{quote}
\textbf{To create a histogram on the TI-83/84:}
\end{quote}

\begin{enumerate}
\def\labelenumi{\arabic{enumi}.}
\tightlist
\item
  Go into the STAT menu, and then Chose 1:Edit
\end{enumerate}

\begin{quote}
\textbf{Figure \#6.4.1: STAT Menu on TI-83/84}

\begin{figure}
\centering
\includegraphics[width=2.75in,height=1.86111in]{media/image71.gif}
\caption{stat\_menu.gif}
\end{figure}
\end{quote}

\begin{enumerate}
\def\labelenumi{\arabic{enumi}.}
\setcounter{enumi}{1}
\item
  Type your data values into L1.
\item
  Now click STAT PLOT (2\textsuperscript{nd} Y=).
\end{enumerate}

\begin{quote}
\textbf{Figure \#6.4.2: STAT PLOT Menu on TI-83/84}

\begin{figure}
\centering
\includegraphics[width=2.75in,height=1.86111in]{media/image72.gif}
\caption{stat\_plot\_menu.gif}
\end{figure}
\end{quote}

\begin{enumerate}
\def\labelenumi{\arabic{enumi}.}
\setcounter{enumi}{3}
\tightlist
\item
  Use 1:Plot1. Press ENTER.
\end{enumerate}

\begin{quote}
\textbf{Figure \#6.4.3: Plot1 Menu on TI-83/84}

\begin{figure}
\centering
\includegraphics[width=2.75in,height=1.86111in]{media/image73.gif}
\caption{stat\_plot1\_menu.gif}
\end{figure}
\end{quote}

\begin{enumerate}
\def\labelenumi{\arabic{enumi}.}
\setcounter{enumi}{4}
\item
  You will see a new window. The first thing you want to do is turn
  the plot on. At this point you should be on On, just press ENTER. It
  will make On dark.
\item
  Now arrow down to Type: and arrow right to the graph that looks like
  a histogram (3\textsuperscript{rd} one from the left in the top row).
\item
  Now arrow down to Xlist. Make sure this says L1. If it doesn't, then
  put L1 there (2\textsuperscript{nd} number 1). Freq: should be a 1.
\end{enumerate}

\textbf{\\
}

\begin{quote}
\textbf{Figure \#6.4.4: Plot1 Menu on TI-83/84 Setup for Histogram}

\begin{figure}
\centering
\includegraphics[width=2.75in,height=1.86111in]{media/image74.gif}
\caption{stat\_plot\_histogram.gif}
\end{figure}
\end{quote}

\begin{enumerate}
\def\labelenumi{\arabic{enumi}.}
\setcounter{enumi}{7}
\item
  Now you need to set up the correct window to graph on. Click on
  WINDOW. You need to set up the settings for the \emph{x} variable. Xmin
  should be your smallest data value. Xmax should just be a value
  sufficiently above your highest data value, but not too high. Xscl
  is your class width that you calculated. Ymin should be 0 and Ymax
  should be above what you think the highest frequency is going to be.
  You can always change this if you need to. Yscl is just how often
  you would like to see a tick mark on the \emph{y}-axis.
\item
  Now press GRAPH. You will see a histogram.
\end{enumerate}

\begin{quote}
\textbf{To find the IQR and create a box plot on the TI-83/84:}
\end{quote}

\begin{enumerate}
\def\labelenumi{\arabic{enumi}.}
\tightlist
\item
  Go into the STAT menu, and then Chose 1:Edit
\end{enumerate}

\begin{quote}
\textbf{Figure \#6.4.5: STAT Menu on TI-83/84}

\begin{figure}
\centering
\includegraphics[width=2.75in,height=1.86111in]{media/image71.gif}
\caption{stat\_menu.gif}
\end{figure}
\end{quote}

\begin{enumerate}
\def\labelenumi{\arabic{enumi}.}
\setcounter{enumi}{1}
\item
  Type your data values into L1. If L1 has data in it, arrow up to the
  name L1, click CLEAR and then press ENTER. The column will now be
  cleared and you can type the data in.
\item
  Go into the STAT menu, move over to CALC and choose 1-Var Stats.
  Press ENTER, then type L1 (2\textsuperscript{nd} 1) and then ENTER. This will give
  you the summary statistics. If you press the down arrow, you will
  see the five-number summary.
\item
  To draw the box plot press 2\textsuperscript{nd} STAT PLOT.
\end{enumerate}

\begin{quote}
\textbf{Figure \#6.4.6: STAT PLOT Menu on TI-83/84}

\begin{figure}
\centering
\includegraphics[width=2.75in,height=1.86111in]{media/image72.gif}
\caption{stat\_plot\_menu.gif}
\end{figure}
\end{quote}

\begin{enumerate}
\def\labelenumi{\arabic{enumi}.}
\setcounter{enumi}{4}
\tightlist
\item
  Use Plot1. Press ENTER
\end{enumerate}

\begin{quote}
\textbf{Figure \#6.4.7: Plot1 Menu on TI-83/84 Setup for Box Plot}

\begin{figure}
\centering
\includegraphics[width=2.75in,height=1.86111in]{media/image75.gif}
\caption{plot1\_setup.gif}
\end{figure}
\end{quote}

\begin{enumerate}
\def\labelenumi{\arabic{enumi}.}
\setcounter{enumi}{5}
\item
  Put the cursor on On and press Enter to turn the plot on. Use the
  down arrow and the right arrow to highlight the boxplot in the
  middle of the second row of types then press ENTER. Set Data List to
  L1 (it might already say that) and leave Freq as 1.
\item
  Now tell the calculator the set up for the units on the x-axis so
  you can see the whole plot. The calculator will do it automatically
  if you press ZOOM, which is in the middle of the top row.
\end{enumerate}

\textbf{\\
}

\begin{quote}
\textbf{Figure \#6.4.8: ZOOM Menu on TI-83/84}

\begin{figure}
\centering
\includegraphics[width=2.75in,height=1.86111in]{media/image76.gif}
\caption{zoom.gif}
\end{figure}

Then use the down arrow to get to 9:ZoomStat and press ENTER. The box
plot will be drawn.

\textbf{Figure \#6.4.9: ZOOM Menu on TI-83/84 with ZoomStat}

\begin{figure}
\centering
\includegraphics[width=2.75in,height=1.86111in]{media/image77.gif}
\caption{zoom\_stat.gif}
\end{figure}

\textbf{To create a normal quantile plot on the TI-83/84:}
\end{quote}

\begin{enumerate}
\def\labelenumi{\arabic{enumi}.}
\tightlist
\item
  Go into the STAT menu, and then Chose 1:Edit
\end{enumerate}

\begin{quote}
\textbf{Figure \#6.4.10: STAT Menu on TI-83/84}

\begin{figure}
\centering
\includegraphics[width=2.75in,height=1.86111in]{media/image71.gif}
\caption{stat\_menu.gif}
\end{figure}
\end{quote}

\begin{enumerate}
\def\labelenumi{\arabic{enumi}.}
\setcounter{enumi}{1}
\item
  Type your data values into L1. If L1 has data in it, arrow up to the
  name L1, click CLEAR and then press ENTER. The column will now be
  cleared and you can type the data in.
\item
  Now click STAT PLOT (2\textsuperscript{nd} Y=). You have three stat plots to choose
  from.
\end{enumerate}

\begin{quote}
\textbf{Figure \#6.4.11: STAT PLOT Menu on TI-83/84}

\begin{figure}
\centering
\includegraphics[width=2.75in,height=1.86111in]{media/image72.gif}
\caption{stat\_plot\_menu.gif}
\end{figure}
\end{quote}

\begin{enumerate}
\def\labelenumi{\arabic{enumi}.}
\setcounter{enumi}{3}
\item
  Use 1:Plot1. Press ENTER.
\item
  Put the curser on the word On and press ENTER. This turns on the
  plot. Arrow down to Type: and use the right arrow to move over to
  the last graph (it looks like an increasing linear graph). Set Data
  List to L1 (it might already say that) and set Data Axis to Y. The
  Mark is up to you.
\end{enumerate}

\begin{quote}
\textbf{Figure \#6.4.12: Plot1 Menu on TI-83/84 Setup for Normal Quantile
Plot}

\begin{figure}
\centering
\includegraphics[width=2.75in,height=1.86111in]{media/image78.gif}
\caption{normal\_prob\_input.gif}
\end{figure}
\end{quote}

\begin{enumerate}
\def\labelenumi{\arabic{enumi}.}
\setcounter{enumi}{5}
\item
  Now you need to set up the correct window on which to graph. Click
  on WINDOW. You need to set up the settings for the \emph{x} variable.
  Xmin should be . Xmax should be 4. Xscl should be 1. Ymin and Ymax
  are based on your data, the Ymin should be below your lowest data
  value and Ymax should be above your highest data value. Yscl is just
  how often you would like to see a tick mark on the \emph{y}-axis.
\item
  Now press GRAPH. You will see the normal quantile plot.
\end{enumerate}

\begin{quote}
\textbf{To create a histogram on R:}

Put the variable in using variable\textless{}-c(type in the data with commas
between values) using a name for the variable that makes sense for the
problem. The command for histogram is hist(variable). You can then
copy the histogram into a word processing program. There are options
that you can put in for title, and axis labels. See section 2.2 for
the commands for those.

\textbf{To create a modified boxplot on R:}

Put the variable in using variable\textless{}-c(type in the data with commas
between values) using a name for the variable that makes sense for the
problem. The command for box plot is boxplot(variable). You can then
copy the box plot into a word processing program. There are options
that you can put in for title, horizontal orientation, and axis
labels. See section 3.3 for the commands for those.

\textbf{To create a normal quantile plot on R:}

Put the variable in using variable\textless{}-c(type in the data with commas
between values) using a name for the variable that makes sense for the
problem. The command for normal quantile plot is qqnorm(variable). You
can then copy the normal quantile plot into a word processing program.
\end{quote}

Realize that your random variable may be normally distributed, even if
the sample fails the three tests. However, if the histogram definitely
doesn't look symmetric and bell shaped, there are outliers that are
very extreme, and the normal probability plot doesn't look linear, then
you can be fairly confident that the data set does not come from a
population that is normally distributed.

\textbf{Example \#6.4.1: Is It Normal?}

\begin{quote}
In Kiama, NSW, Australia, there is a blowhole. The data in table
\#6.4.1 are times in seconds between eruptions ("Kiama blowhole
eruptions," 2013). Do the data come from a population that is
normally distributed?

\textbf{Table \#6.4.1: Time (in Seconds) Between Kiama Blowhole Eruptions }
\end{quote}

\begin{longtable}[]{@{}llllllll@{}}
\toprule
83 & 51 & 87 & 60 & 28 & 95 & 8 & 27\tabularnewline
\midrule
\endhead
15 & 10 & 18 & 16 & 29 & 54 & 91 & 8\tabularnewline
17 & 55 & 10 & 35 & 47 & 77 & 36 & 17\tabularnewline
21 & 36 & 18 & 40 & 10 & 7 & 34 & 27\tabularnewline
28 & 56 & 8 & 25 & 68 & 146 & 89 & 18\tabularnewline
73 & 69 & 9 & 37 & 10 & 82 & 29 & 8\tabularnewline
60 & 61 & 61 & 18 & 169 & 25 & 8 & 26\tabularnewline
11 & 83 & 11 & 42 & 17 & 14 & 9 & 12\tabularnewline
\bottomrule
\end{longtable}

\begin{enumerate}
\def\labelenumi{\alph{enumi}.}
\tightlist
\item
  State the random variable
\end{enumerate}

\begin{quote}
\textbf{Solution:}

\emph{x} = time in seconds between eruptions of Kiama Blowhole
\end{quote}

\begin{enumerate}
\def\labelenumi{\alph{enumi}.}
\setcounter{enumi}{1}
\tightlist
\item
  Draw a histogram.
\end{enumerate}

\begin{quote}
\textbf{Solution:}

The histogram produced is in figure \#6.4.13.

\textbf{Figure \#6.4.13: Histogram for Kiama Blowhole}

\includegraphics[width=2.58333in,height=2.58333in]{media/image80.emf}

This looks skewed right and not symmetric.
\end{quote}

\begin{enumerate}
\def\labelenumi{\alph{enumi}.}
\setcounter{enumi}{2}
\tightlist
\item
  Find the number of outliers.
\end{enumerate}

\begin{quote}
\textbf{Solution:}

The box plot is in figure \#6.4.14.

\textbf{Figure \#6.4.14: Modified Box Plot from TI-83/84 for Kiama
Blowhole}

\includegraphics[width=2.19444in,height=2.19444in]{media/image81.emf}

There are two outliers.

Instead using:

Outliers are any numbers greater than 128.25 seconds and less than
seconds. Since all the numbers are measurements of time, then no data
values are less than 0 or seconds for that matter. There are two
numbers that are larger than 128.25 seconds, so there are two
outliers. Two outliers are not real indications that the sample does
not come from a normal distribution, but the fact that both are well
above 128.25 seconds is an indication of an issue.
\end{quote}

\begin{enumerate}
\def\labelenumi{\alph{enumi}.}
\setcounter{enumi}{3}
\tightlist
\item
  Draw the normal quantile plot.
\end{enumerate}

\begin{quote}
\textbf{Solution:}

The normal quantile plot is in figure \#6.4.15.

\textbf{Figure \#6.4.15: Normal Probability Plot}

\includegraphics[width=2.54167in,height=2.54167in]{media/image87.emf}

This graph looks more like an exponential growth than linear.
\end{quote}

\begin{enumerate}
\def\labelenumi{\alph{enumi}.}
\setcounter{enumi}{4}
\tightlist
\item
  Do the data come from a population that is normally distributed?
\end{enumerate}

\begin{quote}
\textbf{Solution:}

Considering the histogram is skewed right, there are two extreme
outliers, and the normal probability plot does not look linear, then
the conclusion is that this sample is not from a population that is
normally distributed.
\end{quote}

\textbf{Example \#6.4.2: Is It Normal?}

\begin{quote}
One way to measure intelligence is with an IQ score. Table \#6.4.2
contains 50 IQ scores. Determine if the sample comes from a population
that is normally distributed.
\end{quote}

\textbf{Table \#6.4.2: IQ Scores}

\begin{longtable}[]{@{}llllllllll@{}}
\toprule
78 & 92 & 96 & 100 & 67 & 105 & 109 & 75 & 127 & 111\tabularnewline
\midrule
\endhead
93 & 114 & 82 & 100 & 125 & 67 & 94 & 74 & 81 & 98\tabularnewline
102 & 108 & 81 & 96 & 103 & 91 & 90 & 96 & 86 & 92\tabularnewline
84 & 92 & 90 & 103 & 115 & 93 & 85 & 116 & 87 & 106\tabularnewline
85 & 88 & 106 & 104 & 102 & 98 & 116 & 107 & 102 & 89\tabularnewline
\bottomrule
\end{longtable}

\begin{enumerate}
\def\labelenumi{\alph{enumi}.}
\tightlist
\item
  State the random variable
\end{enumerate}

\begin{quote}
\textbf{Solution:}

\emph{x} = IQ score
\end{quote}

\begin{enumerate}
\def\labelenumi{\alph{enumi}.}
\setcounter{enumi}{1}
\tightlist
\item
  Draw a histogram.
\end{enumerate}

\begin{quote}
\textbf{Solution:}

The histogram is in figure \#6.4.16.

\textbf{Figure \#6.4.16: Histogram for IQ Score}

\includegraphics[width=3.25in,height=3.25in]{media/image88.emf}

This looks somewhat symmetric, though it could be thought of as
slightly skewed right.
\end{quote}

\begin{enumerate}
\def\labelenumi{\alph{enumi}.}
\setcounter{enumi}{2}
\tightlist
\item
  Find the number of outliers.
\end{enumerate}

\begin{quote}
\textbf{Solution:}

The modified box plot is in figure \#6.4.17.

\textbf{Figure \#6.4.17: Output from TI-83/84 for IQ Score}

\includegraphics[width=2.58333in,height=2.58333in]{media/image89.emf}

There are no outliers.

Or using

Outliers are any numbers greater than 132 and less than 60. Since the
maximum number is 127 and the minimum is 67, there are no outliers.
\end{quote}

\begin{enumerate}
\def\labelenumi{\alph{enumi}.}
\setcounter{enumi}{3}
\tightlist
\item
  Draw the normal quantile plot.
\end{enumerate}

\begin{quote}
\textbf{Solution:}

The normal quantile plot is in figure \#6.4.18.

\textbf{Figure \#6.4.18: Normal Quantile Plot}

\includegraphics[width=3.13889in,height=3.13889in]{media/image93.emf}

This graph looks fairly linear.
\end{quote}

\begin{enumerate}
\def\labelenumi{\alph{enumi}.}
\setcounter{enumi}{4}
\tightlist
\item
  Do the data come from a population that is normally distributed?
\end{enumerate}

\begin{quote}
\textbf{Solution:}

Considering the histogram is somewhat symmetric, there are no
outliers, and the normal probability plot looks linear, then the
conclusion is that this sample is from a population that is normally
distributed.
\end{quote}

\hypertarget{homework-18}{%
\subsection{Homework}\label{homework-18}}

\begin{enumerate}
\def\labelenumi{\arabic{enumi}.}
\tightlist
\item
  Cholesterol data was collected on patients four days after having a heart attack. The data is in table \#6.4.3. Determine if the data is from a population that is normally distributed.
\end{enumerate}

\begin{quote}
\textbf{Table \#6.4.3: Cholesterol Data Collected Four Days After a Heart Attack}
\end{quote}

\begin{longtable}[]{@{}lllllll@{}}
\toprule
218 & 234 & 214 & 116 & 200 & 276 & 146\tabularnewline
\midrule
\endhead
182 & 238 & 288 & 190 & 236 & 244 & 258\tabularnewline
240 & 294 & 220 & 200 & 220 & 186 & 352\tabularnewline
202 & 218 & 248 & 278 & 248 & 270 & 242\tabularnewline
\bottomrule
\end{longtable}

\begin{enumerate}
\def\labelenumi{\arabic{enumi}.}
\setcounter{enumi}{1}
\tightlist
\item
  The size of fish is very important to commercial fishing. A study conducted in 2012 collected the lengths of Atlantic cod caught in nets in Karlskrona (Ovegard, Berndt \& Lunneryd, 2012). Data based on information from the study is in table \#6.4.4. Determine if the data is from a population that is normally distributed.
\end{enumerate}

\begin{quote}
\textbf{Table \#6.4.4: Atlantic Cod Lengths}
\end{quote}

\begin{longtable}[]{@{}llllllll@{}}
\toprule
48 & 50 & 50 & 55 & 53 & 50 & 49 & 52\tabularnewline
\midrule
\endhead
61 & 48 & 45 & 47 & 53 & 46 & 50 & 48\tabularnewline
42 & 44 & 50 & 60 & 54 & 48 & 50 & 49\tabularnewline
53 & 48 & 52 & 56 & 46 & 46 & 47 & 48\tabularnewline
48 & 49 & 52 & 47 & 51 & 48 & 45 & 47\tabularnewline
\bottomrule
\end{longtable}

\begin{enumerate}
\def\labelenumi{\arabic{enumi}.}
\setcounter{enumi}{2}
\tightlist
\item
  The WHO MONICA Project collected blood pressure data for people in China (Kuulasmaa, Hense \& Tolonen, 1998). Data based on information from the study is in table \#6.4.5. Determine if the data is from a population that is normally distributed.
\end{enumerate}

\begin{quote}
\textbf{Table \#6.4.5: Blood Pressure Values for People in China}
\end{quote}

\begin{longtable}[]{@{}lllllllll@{}}
\toprule
114 & 141 & 154 & 137 & 131 & 132 & 133 & 156 & 119\tabularnewline
\midrule
\endhead
138 & 86 & 122 & 112 & 114 & 177 & 128 & 137 & 140\tabularnewline
171 & 129 & 127 & 104 & 97 & 135 & 107 & 136 & 118\tabularnewline
92 & 182 & 150 & 142 & 97 & 140 & 106 & 76 & 115\tabularnewline
119 & 125 & 162 & 80 & 138 & 124 & 132 & 143 & 119\tabularnewline
\bottomrule
\end{longtable}

\begin{enumerate}
\def\labelenumi{\arabic{enumi}.}
\setcounter{enumi}{3}
\item
  Annual rainfalls for Sydney, Australia are given in table \#6.4.6. ("Annual maximums of," 2013). Can you assume rainfall is normally distributed?

  \textbf{Table \#6.4.6: Annual Rainfall in Sydney, Australia}
\end{enumerate}

\begin{longtable}[]{@{}llllllll@{}}
\toprule
146.8 & 383 & 90.9 & 178.1 & 267.5 & 95.5 & 156.5 & 180\tabularnewline
\midrule
\endhead
90.9 & 139.7 & 200.2 & 171.7 & 187.2 & 184.9 & 70.1 & 58\tabularnewline
84.1 & 55.6 & 133.1 & 271.8 & 135.9 & 71.9 & 99.4 & 110.6\tabularnewline
47.5 & 97.8 & 122.7 & 58.4 & 154.4 & 173.7 & 118.8 & 88\tabularnewline
84.6 & 171.5 & 254.3 & 185.9 & 137.2 & 138.9 & 96.2 & 85\tabularnewline
45.2 & 74.7 & 264.9 & 113.8 & 133.4 & 68.1 & 156.4 &\tabularnewline
\bottomrule
\end{longtable}

\textbf{\\
}

\hypertarget{sampling-distribution-and-the-central-limit-theorem}{%
\section{Sampling Distribution and the Central Limit Theorem}\label{sampling-distribution-and-the-central-limit-theorem}}

You now have most of the skills to start statistical inference, but you need one more concept.

First, it would be helpful to state what statistical inference is in more accurate terms.

\textbf{Statistical Inference:} to make accurate decisions about parameters from statistics

When it says ``accurate decision,'' you want to be able to measure how accurate. You measure how accurate using probability. In both binomial and normal distributions, you needed to know that the random variable followed either distribution. You need to know how the statistic is distributed and then you can find probabilities. In other words, you need to know the shape of the sample mean or whatever statistic you want to make a decision about.

How is the statistic distributed? This is answered with a sampling distribution.

\textbf{Sampling Distribution:} how a sample statistic is distributed when repeated trials of size \emph{n} are taken.

\textbf{Example \#6.5.1: Sampling Distribution}

\begin{quote}
Suppose you throw a penny and count how often a head comes up. The random variable is \emph{x} = number of heads. The probability distribution (pdf) of this random variable is presented in figure \#6.5.1.

\textbf{Figure \#6.5.1: Distribution of Random Variable}

\includegraphics[width=2.15278in,height=2.91667in]{media/image94.png}

Repeat this experiment 10 times, which means \emph{n} = 10. Here is the data set:

\{1, 1, 1, 1, 0, 0, 0, 0, 0, 0\}. The mean of this sample is 0.4. Now take another sample. Here is that data set:

\{1, 1, 1, 0, 1, 0, 1, 1, 0, 0\}. The mean of this sample is 0.6.
Another sample looks like:

\{0, 1, 0, 1, 1, 1, 1, 1, 0, 1\}. The mean of this sample is 0.7. Repeat this 40 times. You could get these means:

\textbf{Table \#6.5.1: Sample Means When \emph{n} = 10}
\end{quote}

\begin{longtable}[]{@{}llllllllll@{}}
\toprule
0.4 & 0.6 & 0.7 & 0.3 & 0.3 & 0.2 & 0.5 & 0.5 & 0.5 & 0.5\tabularnewline
\midrule
\endhead
0.4 & 0.4 & 0.5 & 0.7 & 0.7 & 0.6 & 0.4 & 0.4 & 0.4 & 0.6\tabularnewline
0.7 & 0.7 & 0.3 & 0.5 & 0.6 & 0.3 & 0.3 & 0.8 & 0.3 & 0.6\tabularnewline
0.4 & 0.3 & 0.5 & 0.6 & 0.5 & 0.6 & 0.3 & 0.5 & 0.6 & 0.2\tabularnewline
\bottomrule
\end{longtable}

\begin{quote}
Table \#6.5.2 contains the distribution of these sample means (just count how many of each number there are and then divide by 40 to obtain the relative frequency).

\textbf{Table \#6.5.2: Distribution of Sample Means When \emph{n} = 10}
\end{quote}

\begin{longtable}[]{@{}ll@{}}
\toprule
Sample Mean & Probability\tabularnewline
\midrule
\endhead
0.1 & 0\tabularnewline
0.2 & 0.05\tabularnewline
0.3 & 0.2\tabularnewline
0.4 & 0.175\tabularnewline
0.5 & 0.225\tabularnewline
0.6 & 0.2\tabularnewline
0.7 & 0.125\tabularnewline
0.8 & 0.025\tabularnewline
0.9 & 0\tabularnewline
\bottomrule
\end{longtable}

\begin{quote}
Figure \#6.5.2 contains the histogram of these sample means.
\end{quote}

\begin{quote}
\textbf{Figure \#6.5.2: Histogram of Sample Means When \emph{n} = 10}

\includegraphics[width=5.01389in,height=3.01389in]{media/image95.png}

This distribution (represented graphically by the histogram) is a
sampling distribution. That is all a sampling distribution is. It is a
distribution created from statistics.

Notice the histogram does not look anything like the histogram of the original random variable. It also doesn't look anything like a normal distribution, which is the only one you really know how to find probabilities. Granted you have the binomial, but the normal is better.

What does this distribution look like if instead of repeating the experiment 10 times you repeat it 20 times instead?

Table \#6.5.3 contains 40 means when the experiment of flipping the coin is repeated 20 times.

\textbf{Table \#6.5.3: Sample Means When \emph{n} = 20}
\end{quote}

\begin{longtable}[]{@{}llllllllll@{}}
\toprule
0.5 & 0.45 & 0.7 & 0.55 & 0.65 & 0.6 & 0.4 & 0.35 & 0.45 & 0.6\tabularnewline
\midrule
\endhead
0.5 & 0.5 & 0.65 & 0.5 & 0.5 & 0.35 & 0.55 & 0.4 & 0.65 & 0.3\tabularnewline
0.4 & 0.5 & 0.45 & 0.45 & 0.65 & 0.7 & 0.6 & 0.5 & 0.7 & 0.7\tabularnewline
0.7 & 0.45 & 0.35 & 0.6 & 0.65 & 0.55 & 0.35 & 0.4 & 0.55 & 0.6\tabularnewline
\bottomrule
\end{longtable}

\begin{quote}
Table \#6.5.3 contains the sampling distribution of the sample means.
\end{quote}

\begin{quote}
\textbf{Table \#6.5.3: Distribution of Sample Means When \emph{n} = 20}
\end{quote}

\begin{longtable}[]{@{}ll@{}}
\toprule
Mean & Probability\tabularnewline
\midrule
\endhead
0.1 & 0\tabularnewline
0.2 & 0\tabularnewline
0.3 & 0.125\tabularnewline
0.4 & 0.2\tabularnewline
0.5 & 0.3\tabularnewline
0.6 & 0.25\tabularnewline
0.7 & 0.125\tabularnewline
0.8 & 0\tabularnewline
0.9 & 0\tabularnewline
\bottomrule
\end{longtable}

\begin{quote}
This histogram of the sampling distribution is displayed in figure
\#6.5.3.

\textbf{Figure \#6.5.3: Histogram of Sample Means When \emph{n} = 20}

\includegraphics[width=5.01389in,height=3.01389in]{media/image96.png}

Notice this histogram of the sample mean looks approximately
symmetrical and could almost be called normal. What if you keep
increasing \emph{n}? What will the sampling distribution of the sample mean
look like? In other words, what does the sampling distribution of look
like as \emph{n} gets even larger?
\end{quote}

This depends on how the original distribution is distributed. In Example
\#6.5.1, the random variable was uniform looking. But as \emph{n} increased
to 20, the distribution of the mean looked approximately normal. What if
the original distribution was normal? How big would \emph{n} have to be?
Before that question is answered, another concept is needed.

Suppose you have a random variable that has a population mean, , and a
population standard deviation, . If a sample of size \emph{n} is taken, then
the sample mean, has a mean and standard deviation of . The standard
deviation of is lower because by taking the mean you are averaging out
the extreme values, which makes the distribution of the original random
variable spread out.

You now know the center and the variability of . You also want to know
the shape of the distribution of . You hope it is normal, since you know
how to find probabilities using the normal curve. The following theorem
tells you the requirement to have normally distributed.

\textbf{Theorem \#6.5.1: Central Limit Theorem.}

Suppose a random variable is from any distribution. If a sample of size
\emph{n} is taken, then the sample mean, , becomes normally distributed as
\emph{n} increases.

What this says is that no matter what \emph{x} looks like, would look normal
if \emph{n} is large enough. Now, what size of \emph{n} is large enough? That
depends on how \emph{x} is distributed in the first place. If the original
random variable is normally distributed, then \emph{n} just needs to be 2 or
more data points. If the original random variable is somewhat mound
shaped and symmetrical, then \emph{n} needs to be greater than or equal to
30. Sometimes the sample size can be smaller, but this is a good rule of
thumb. The sample size may have to be much larger if the original random
variable is really skewed one way or another.

Now that you know when the sample mean will look like a normal
distribution, then you can find the probability related to the sample
mean. Remember that the mean of the sample mean is just the mean of the
original data (), but the standard deviation of the sample mean, , also
known as the standard error of the mean, is actually . Make sure you use
this in all calculations. If you are using the \emph{z}-score, the formula
when working with is . If you are using the TI-83/84 calculator, then
the input would be . If you are using R, then the input would be for
find the area to the left of . Remember to subtract from 1 if you want
the area to the right of .

\textbf{Example \#6.5.1: Finding Probabilities for Sample Means}

\begin{quote}
The birth weight of boy babies of European descent who were delivered
at 40 weeks is normally distributed with a mean of 3687.6 g with a
standard deviation of 410.5 g (Janssen, Thiessen, Klein, Whitfield,
MacNab \& Cullis-Kuhl, 2007). Suppose there were nine European descent
boy babies born on a given day and the mean birth weight is
calculated.
\end{quote}

\begin{enumerate}
\def\labelenumi{\alph{enumi}.}
\tightlist
\item
  State the random variable.
\end{enumerate}

\begin{quote}
\textbf{Solution:}

\emph{x} = birth weight of boy babies (Note: the random variable is
something you measure, and it is not the mean birth weight. Mean birth
weight is calculated.)
\end{quote}

\begin{enumerate}
\def\labelenumi{\alph{enumi}.}
\setcounter{enumi}{1}
\tightlist
\item
  What is the mean of the sample mean?
\end{enumerate}

\begin{quote}
\textbf{Solution:}
\end{quote}

\begin{enumerate}
\def\labelenumi{\alph{enumi}.}
\setcounter{enumi}{2}
\tightlist
\item
  What is the standard deviation of the sample mean?
\end{enumerate}

\begin{quote}
\textbf{Solution:}
\end{quote}

\begin{enumerate}
\def\labelenumi{\alph{enumi}.}
\setcounter{enumi}{3}
\tightlist
\item
  What distribution is the sample mean distributed as?
\end{enumerate}

\begin{quote}
\textbf{Solution:}

Since the original random variable is distributed normally, then the
sample mean is distributed normally.
\end{quote}

\begin{enumerate}
\def\labelenumi{\alph{enumi}.}
\setcounter{enumi}{4}
\tightlist
\item
  Find the probability that the mean weight of the nine boy babies
  born was less than 3500.4 g.
\end{enumerate}

\begin{quote}
\textbf{Solution:}

You are looking for the . You use the normalcdf command on the
calculator. Remember to use the standard deviation you found in part
c. However to reduce rounding error, type the division into the
command. On the TI-83/84 you would have

On R you would have

There is an 8.6\% chance that the mean birth weight of the nine boy
babies born would be less than 3500.4 g. Since this is more than 5\%,
this is not unusual.
\end{quote}

\begin{enumerate}
\def\labelenumi{\alph{enumi}.}
\setcounter{enumi}{5}
\tightlist
\item
  Find the probability that the mean weight of the nine babies born
  was less than 3452.5 g.
\end{enumerate}

\begin{quote}
\textbf{Solution:}

You are looking for the .

On TI-83/84:

On R:

There is a 4.3\% chance that the mean birth weight of the nine boy
babies born would be less than 3452.5 g. Since this is less than 5\%,
this would be an unusual event. If it actually happened, then you may
think there is something unusual about this sample. Maybe some of the
nine babies were born as multiples, which brings the mean weight down,
or some or all of the babies were not of European descent (in fact the
mean weight of South Asian boy babies is 3452.5 g), or some were born
before 40 weeks, or the babies were born at high altitudes.
\end{quote}

\textbf{Example \#6.5.2: Finding Probabilities for Sample Means}

\begin{quote}
The age that American females first have intercourse is on average
17.4 years, with a standard deviation of approximately 2 years ("The
Kinsey institute," 2013). This random variable is not normally
distributed, though it is somewhat mound shaped.
\end{quote}

\begin{enumerate}
\def\labelenumi{\alph{enumi}.}
\tightlist
\item
  State the random variable.
\end{enumerate}

\begin{quote}
\textbf{Solution:}

\emph{x} = age that American females first have intercourse
\end{quote}

\begin{enumerate}
\def\labelenumi{\alph{enumi}.}
\setcounter{enumi}{1}
\tightlist
\item
  Suppose a sample of 35 American females is taken. Find the
  probability that the mean age that these 35 females first had
  intercourse is more than 21 years.
\end{enumerate}

\begin{quote}
\textbf{Solution:}

Even though the original random variable is not normally distributed,
the sample size is over 30, by the central limit theorem the sample
mean will be normally distributed. The mean of the sample mean is .
The standard deviation of the sample mean is . You have all the
information you need to use the normal command on your technology.
Without the central limit theorem, you couldn't use the normal
command, and you would not be able to answer this question.

On the TI-83/84:

On R:

The probability of a sample mean of 35 women being more than 21 years
when they had their first intercourse is very small. This is extremely
unlikely to happen. If it does, it may make you wonder about the
sample. Could the population mean have increased from the 17.4 years
that was stated in the article? Could the sample not have been random,
and instead have been a group of women who had similar beliefs about
intercourse? These questions, and more, are ones that you would want
to ask as a researcher
\end{quote}

\hypertarget{homework-19}{%
\subsection{Homework}\label{homework-19}}

\begin{enumerate}
\def\labelenumi{\arabic{enumi}.}
\tightlist
\item
  A random variable is not normally distributed, but it is mound shaped. It has a mean of 14 and a standard deviation of 3.
\end{enumerate}

\begin{enumerate}
\def\labelenumi{\alph{enumi}.}
\item
  If you take a sample of size 10, can you say what the shape of the sampling distribution for the sample mean is? Why?
\item
  For a sample of size 10, state the mean of the sample mean and the standard deviation of the sample mean.
\item
  If you take a sample of size 35, can you say what the shape of the distribution of the sample mean is? Why?
\item
  For a sample of size 35, state the mean of the sample mean and the standard deviation of the sample mean.
\end{enumerate}

\begin{enumerate}
\def\labelenumi{\arabic{enumi}.}
\setcounter{enumi}{1}
\tightlist
\item
  A random variable is normally distributed. It has a mean of 245 and a standard deviation of 21.
\end{enumerate}

\begin{enumerate}
\def\labelenumi{\alph{enumi}.}
\item
  If you take a sample of size 10, can you say what the shape of the distribution for the sample mean is? Why?
\item
  For a sample of size 10, state the mean of the sample mean and the standard deviation of the sample mean.
\item
  For a sample of size 10, find the probability that the sample mean is more than 241.
\item
  If you take a sample of size 35, can you say what the shape of the distribution of the sample mean is? Why?
\item
  For a sample of size 35, state the mean of the sample mean and the standard deviation of the sample mean.
\item
  For a sample of size 35, find the probability that the sample mean is more than 241.
\item
  Compare your answers in part d and f. Why is one smaller than the other?
\end{enumerate}

\begin{enumerate}
\def\labelenumi{\arabic{enumi}.}
\setcounter{enumi}{2}
\tightlist
\item
  The mean starting salary for nurses is \$67,694 nationally ("Staff nurse -," 2013). The standard deviation is approximately \$10,333. The starting salary is not normally distributed but it is mound shaped. A sample of 42 starting salaries for nurses is taken.
\end{enumerate}

\begin{enumerate}
\def\labelenumi{\alph{enumi}.}
\item
  State the random variable.
\item
  What is the mean of the sample mean?
\item
  What is the standard deviation of the sample mean?
\item
  What is the shape of the sampling distribution of the sample mean? Why?
\item
  Find the probability that the sample mean is more than \$75,000.
\item
  Find the probability that the sample mean is less than \$60,000.
\item
  If you did find a sample mean of more than \$75,000 would you find that unusual? What could you conclude?
\end{enumerate}

\begin{enumerate}
\def\labelenumi{\arabic{enumi}.}
\setcounter{enumi}{3}
\tightlist
\item
  According to the WHO MONICA Project the mean blood pressure for people in China is 128 mmHg with a standard deviation of 23 mmHg (Kuulasmaa, Hense \& Tolonen, 1998). Blood pressure is normally distributed.
\end{enumerate}

\begin{enumerate}
\def\labelenumi{\alph{enumi}.}
\item
  State the random variable.
\item
  Suppose a sample of size 15 is taken. State the shape of the distribution of the sample mean.
\item
  Suppose a sample of size 15 is taken. State the mean of the sample mean.
\item
  Suppose a sample of size 15 is taken. State the standard deviation of the sample mean.
\item
  Suppose a sample of size 15 is taken. Find the probability that the sample mean blood pressure is more than 135 mmHg.
\item
  Would it be unusual to find a sample mean of 15 people in China of more than 135 mmHg? Why or why not?
\item
  If you did find a sample mean for 15 people in China to be more than 135 mmHg, what might you conclude?
\end{enumerate}

\begin{enumerate}
\def\labelenumi{\arabic{enumi}.}
\setcounter{enumi}{4}
\tightlist
\item
  The size of fish is very important to commercial fishing. A study conducted in 2012 found the length of Atlantic cod caught in nets in Karlskrona to have a mean of 49.9 cm and a standard deviation of 3.74 cm (Ovegard, Berndt \& Lunneryd, 2012). The length of fish is normally distributed. A sample of 15 fish is taken.
\end{enumerate}

\begin{enumerate}
\def\labelenumi{\alph{enumi}.}
\item
  State the random variable.
\item
  Find the mean of the sample mean.
\item
  Find the standard deviation of the sample mean
\item
  What is the shape of the distribution of the sample mean? Why?
\item
  Find the probability that the sample mean length of the Atlantic cod is less than 52 cm.
\item
  Find the probability that the sample mean length of the Atlantic cod is more than 74 cm.
\item
  If you found sample mean length for Atlantic cod to be more than 74 cm, what could you conclude?
\end{enumerate}

\begin{enumerate}
\def\labelenumi{\arabic{enumi}.}
\setcounter{enumi}{5}
\tightlist
\item
  The mean cholesterol levels of women age 45-59 in Ghana, Nigeria, and Seychelles is 5.1 mmol/l and the standard deviation is 1.0 mmol/l (Lawes, Hoorn, Law \& Rodgers, 2004). Assume that cholesterol levels are normally distributed.
\end{enumerate}

\begin{enumerate}
\def\labelenumi{\alph{enumi}.}
\item
  State the random variable.
\item
  Find the probability that a woman age 45-59 in Ghana has a cholesterol level above 6.2 mmol/l (considered a high level).
\item
  Suppose doctors decide to test the woman's cholesterol level again and average the two values. Find the probability that this woman's mean cholesterol level for the two tests is above 6.2 mmol/l.
\item
  Suppose doctors being very conservative decide to test the woman's cholesterol level a third time and average the three values. Find the probability that this woman's mean cholesterol level for the three tests is above 6.2 mmol/l.
\item
  If the sample mean cholesterol level for this woman after three tests is above 6.2 mmol/l, what could you conclude?
\end{enumerate}

\begin{enumerate}
\def\labelenumi{\arabic{enumi}.}
\setcounter{enumi}{6}
\tightlist
\item
  In the United States, males between the ages of 40 and 49 eat on average 103.1 g of fat every day with a standard deviation of 4.32 g ("What we eat," 2012). The amount of fat a person eats is not normally distributed but it is relatively mound shaped.
\end{enumerate}

\begin{enumerate}
\def\labelenumi{\alph{enumi}.}
\item
  State the random variable.
\item
  Find the probability that a sample mean amount of daily fat intake for 35 men age 40-59 in the U.S. is more than 100 g.
\item
  Find the probability that a sample mean amount of daily fat intake for 35 men age 40-59 in the U.S. is less than 93 g.
\item
  If you found a sample mean amount of daily fat intake for 35 men age 40-59 in the U.S. less than 93 g, what would you conclude?
\end{enumerate}

\begin{enumerate}
\def\labelenumi{\arabic{enumi}.}
\setcounter{enumi}{7}
\tightlist
\item
  A dishwasher has a mean life of 12 years with an estimated standard deviation of 1.25 years ("Appliance life expectancy," 2013). The life of a dishwasher is normally distributed. Suppose you are a manufacturer and you take a sample of 10 dishwashers that you made.
\end{enumerate}

\begin{enumerate}
\def\labelenumi{\alph{enumi}.}
\item
  State the random variable.
\item
  Find the mean of the sample mean.
\item
  Find the standard deviation of the sample mean.
\item
  What is the shape of the sampling distribution of the sample mean? Why?
\item
  Find the probability that the sample mean of the dishwashers is less than 6 years.
\item
  If you found the sample mean life of the 10 dishwashers to be less than 6 years, would you think that you have a problem with the manufacturing process? Why or why not?
\end{enumerate}

Data Sources:

\emph{Annual maximums of daily rainfall in Sydney}. (2013, September 25).
Retrieved from \url{http://www.statsci.org/data/oz/sydrain.html}

\emph{Appliance life expectancy}. (2013, November 8). Retrieved from
\url{http://www.mrappliance.com/expert/life-guide/}

Bhat, R., \& Kushtagi, P. (2006). A re-look at the duration of human
pregnancy. \emph{Singapore Med J.}, \emph{47}(12), 1044-8. Retrieved from
\url{http://www.ncbi.nlm.nih.gov/pubmed/17139400}

College Board, SAT. (2012). \emph{Total group profile report}. Retrieved from
website:
\url{http://media.collegeboard.com/digitalServices/pdf/research/TotalGroup-2012.pdf}

Greater Cleveland Regional Transit Authority, (2012). \emph{2012 annual
report}. Retrieved from website: \url{http://www.riderta.com/annual/2012}

Janssen, P. A., Thiessen, P., Klein, M. C., Whitfield, M. F., MacNab, Y.
C., \& Cullis-Kuhl, S. C. (2007). Standards for the measurement of birth
weight, length and head circumference at term in neonates of european,
chinese and south asian ancestry. \emph{Open Medicine}, \emph{1}(2), e74-e88.
Retrieved from \url{http://www.ncbi.nlm.nih.gov/pmc/articles/PMC2802014/}

\emph{Kiama blowhole eruptions}. (2013, September 25). Retrieved from
\url{http://www.statsci.org/data/oz/kiama.html}

Kuulasmaa, K., Hense, H., \& Tolonen, H. World Health Organization (WHO),
WHO Monica Project. (1998). \emph{Quality assessment of data on blood
pressure in the who monica project} (ISSN 2242-1246). Retrieved from WHO
MONICA Project e-publications website:
\url{http://www.thl.fi/publications/monica/bp/bpqa.htm}

Lawes, C., Hoorn, S., Law, M., \& Rodgers, A. (2004). High cholesterol.
In M. Ezzati, A. Lopez, A. Rodgers \& C. Murray (Eds.), \emph{Comparative
Quantification of Health Risks} (1 ed., Vol. 1, pp.~391-496). Retrieved
from
\url{http://www.who.int/publications/cra/chapters/volume1/0391-0496.pdf}

Ovegard, M., Berndt, K., \& Lunneryd, S. (2012). Condition indices of
atlantic cod (gadus morhua) biased by capturing method. \emph{ICES Journal of
Marine Science}, doi: 10.1093/icesjms/fss145

\emph{Staff nurse - RN salary}. (2013, November 08). Retrieved from
\url{http://www1.salary.com/Staff-Nurse-RN-salary.html}

\emph{The Kinsey institute - sexuality information links}. (2013, November
08). Retrieved from \url{http://www.iub.edu/~kinsey/resources/FAQ.html}

US Department of Argriculture, Agricultural Research Service. (2012).
\emph{What we eat in America}. Retrieved from website:
\url{http://www.ars.usda.gov/Services/docs.htm?docid=18349}

\hypertarget{one-sample-inference}{%
\chapter{One-Sample Inference}\label{one-sample-inference}}

Now that you have all this information about descriptive statistics and probabilities, it is time to start inferential statistics. There are two branches of inferential statistics: hypothesis testing and confidence intervals.

\textbf{Hypothesis Testing:} making a decision about a parameter(s) based on a statistic(s).

\textbf{Confidence Interval:} estimating a parameter(s) based on a statistic(s).

\hypertarget{basics-of-hypothesis-testing}{%
\section{Basics of Hypothesis Testing}\label{basics-of-hypothesis-testing}}

To understand the process of a hypothesis tests, you need to first have an understanding of what a hypothesis is, which is an educated guess about a parameter. Once you have the hypothesis, you collect data and use the data to make a determination to see if there is enough evidence to show that the hypothesis is true. However, in hypothesis testing you actually assume something else is true, and then you look at your data to see how likely it is to get an event that your data demonstrates with that assumption. If the event is very unusual, then you might think that your assumption is actually false. If you are able to say this assumption is false, then your hypothesis must be true. This is known as a proof by contradiction. You assume the opposite of your hypothesis is true and show that it can't be true. If this happens, then your hypothesis must be true. All hypothesis tests go through the same process. Once you have the process down, then the concept is much easier. It is easier to see the process by looking at an example. Concepts that are needed will be detailed in this example.

\textbf{Example \#7.1.1: Basics of Hypothesis Testing}

\begin{quote}
Suppose a manufacturer of the XJ35 battery claims the mean life of the battery is 500 days with a standard deviation of 25 days. You are the buyer of this battery and you think this claim is inflated. You would like to test your belief because without a good reason you can't get out of your contract.

What do you do?

Well first, you should know what you are trying to measure. Define the random variable.

Let \emph{x} = life of a XJ35 battery

Now you are not just trying to find different \emph{x} values. You are trying to find what the true mean is. Since you are trying to find it, it must be unknown. You don't think it is 500 days. If you did, you wouldn't be doing any testing. The true mean, , is unknown. That means you should define that too.

Let = mean life of a XJ35 battery

Now what?

You may want to collect a sample. What kind of sample?

You could ask the manufacturers to give you batteries, but there is a chance that there could be some bias in the batteries they pick. To reduce the chance of bias, it is best to take a random sample.

How big should the sample be?

A sample of size 30 or more means that you can use the central limit theorem. Pick a sample of size 30.

Table \#7.1.1 contains the data for the sample you collected:

\textbf{Table \#7.1.1: Data on Battery Life}
\end{quote}

\begin{longtable}[]{@{}llllll@{}}
\toprule
491 & 485 & 503 & 492 & 482 & 490\tabularnewline
\midrule
\endhead
489 & 495 & 497 & 487 & 493 & 480\tabularnewline
482 & 504 & 501 & 486 & 478 & 492\tabularnewline
482 & 502 & 485 & 503 & 497 & 500\tabularnewline
488 & 475 & 478 & 490 & 487 & 486\tabularnewline
\bottomrule
\end{longtable}

\begin{quote}
Now what should you do? Looking at the data set, you see some of the times are above 500 and some are below. But looking at all of the numbers is too difficult. It might be helpful to calculate the mean for this sample.

The sample mean is . Looking at the sample mean, one might think that you are right. However, the standard deviation and the sample size also plays a role, so maybe you are wrong.
\end{quote}

\begin{quote}
Before going any farther, it is time to formalize a few definitions.

You have a guess that the mean life of a battery is less than 500 days. This is opposed to what the manufacturer claims. There really are two hypotheses, which are just guesses here -- the one that the manufacturer claims and the one that you believe. It is helpful to have names for them.
\end{quote}

\textbf{Null Hypothesis}: historical value, claim, or product specification. The symbol used is {[}MISSING{]} .

\textbf{Alternate Hypothesis:} what you want to prove. This is what you want to accept as true when you reject the null hypothesis. There are two symbols that are commonly used for the alternative hypothesis: or {[}MISSING{]}. The symbol will be used in this book.

In general, the hypotheses look something like this:

{[}MISSING EQ{]}

where just represents the value that the claim says the population mean is actually equal to.

Also, can be less than, greater than, or not equal to.

For this problem:

\begin{quote}
, since the manufacturer says the mean life of a battery is 500 days.

, since you believe that the mean life of the battery is less than 500 days.
\end{quote}

Now back to the mean. You have a sample mean of 490 days. Is this small enough to believe that you are right and the manufacturer is wrong? How small does it have to be?

If you calculated a sample mean of 235, you would definitely believe the population mean is less than 500. But even if you had a sample mean of 435 you would probably believe that the true mean was less than 500. What about 475? Or 483? There is some point where you would stop being so sure that the population mean is less than 500. That point separates the values of where you are sure or pretty sure that the mean is less than 500 from the area where you are not so sure. How do you find that point?

Well it depends on how much error you want to make. Of course you don't want to make any errors, but unfortunately that is unavoidable in statistics. You need to figure out how much error you made with your sample. Take the sample mean, and find the probability of getting another sample mean less than it, assuming for the moment that the manufacturer is right. The idea behind this is that you want to know what is the chance that you could have come up with your sample mean even if the population mean really is 500 days.

\begin{quote}
You want to find
\end{quote}

To compute this probability, you need to know how the sample mean is distributed. Since the sample size is at least 30, then you know the sample mean is approximately normally distributed. Remember and

A picture is always useful.

\begin{quote}
{\[CHART\]}
\end{quote}

Before calculating the probability, it is useful to see how many standard deviations away from the mean the sample mean is. Using the formula for the \emph{z}-score from chapter 6, you find

This sample mean is more than two standard deviations away from the mean. That seems pretty far, but you should look at the probability too.

\begin{quote}
On TI-83/84:

On R:
\end{quote}

There is a 1.42\% chance that you could find a sample mean less than 490 when the population mean is 500 days. This is really small, so the chances are that the assumption that the population mean is 500 days is wrong, and you can reject the manufacturer's claim. But how do you quantify really small? Is 5\% or 10\% or 15\% really small? How do you decide?

Before you answer that question, a couple more definitions are needed.

\textbf{Test statistic:} since it is calculated as part of the testing of the hypothesis

\textbf{p -- value:} probability that the test statistic will take on more extreme values than the observed test statistic, given that the null hypothesis is true. It is the probability that was calculated above.

Now, how small is small enough? To answer that, you really want to know the types of errors you can make.

There are actually only two errors that can be made. The first error is if you say that is false, when in fact it is true. This means you reject when was true. The second error is if you say that is true, when in fact it is false. This means you fail to reject when is false. The following table organizes this for you:

Type of errors:

\begin{longtable}[]{@{}lll@{}}
\toprule
& true & false\tabularnewline
\midrule
\endhead
Reject & Type I error & No error\tabularnewline
Fail to reject & No error & Type II error\tabularnewline
\bottomrule
\end{longtable}

Thus

\textbf{Type I Error} is rejecting when is true, and

\textbf{Type II Error} is failing to reject when is false.

Since these are the errors, then one can define the probabilities
attached to each error.

= P(type I error) = P(rejecting / is true)

= P(type II error) = P(failing to reject / is false)

is also called the \textbf{level of significance}.

Another common concept that is used is Power = {[}MISSING{]}.

Now there is a relationship between and . They are not complements of each other. How are they related?

If increases that means the chances of making a type I error will increase. It is more likely that a type I error will occur. It makes sense that you are less likely to make type II errors, only because you will be rejecting more often. You will be failing to reject less, and therefore, the chance of making a type II error will decrease. Thus, as increases, will decrease, and vice versa. That makes them seem like complements, but they aren't complements. What gives? Consider one more factor -- sample size.

Consider if you have a larger sample that is representative of the population, then it makes sense that you have more accuracy then with a smaller sample. Think of it this way, which would you trust more, a sample mean of 490 if you had a sample size of 35 or sample size of 350 (assuming a representative sample)? Of course the 350 because there are more data points and so more accuracy. If you are more accurate, then there is less chance that you will make any error. By increasing the sample size of a representative sample, you decrease both and .

Summary of all of this:

\begin{enumerate}
\def\labelenumi{\arabic{enumi}.}
\item
  For a certain sample size, \emph{n}, if increases, decreases.
\item
  For a certain level of significance, , if \emph{n} increases, decreases.
\end{enumerate}

Now how do you find and ? Well is actually chosen. There are only three values that are usually picked for : 0.01, 0.05, and 0.10. is very difficult to find, so usually it isn't found. If you want to make sure it is small you take as large of a sample as you can afford provided it is a representative sample. This is one use of the Power. You want to be small and the Power of the test is large. The Power word sounds good.

Which pick of do you pick? Well that depends on what you are working on. Remember in this example you are the buyer who is trying to get out of a contract to buy these batteries. If you create a type I error, you said that the batteries are bad when they aren't, most likely the manufacturer will sue you. You want to avoid this. You might pick to be 0.01. This way you have a small chance of making a type I error. Of course this means you have more of a chance of making a type II error. No big deal right? What if the batteries are used in pacemakers and you tell the person that their pacemaker's batteries are good for 500 days when they actually last less, that might be bad. If you make a type II error, you say that the batteries do last 500 days when they last less, then you have the possibility of killing someone. You certainly do not want to do this. In this case you might want to pick as 0.10. If both errors are equally bad, then pick as 0.05.

The above discussion is why the choice of depends on what you are researching. As the researcher, you are the one that needs to decide what level to use based on your analysis of the consequences of making each error is.

If a type I error is really bad, then pick = 0.01.

If a type II error is really bad, then pick = 0.10

If neither error is bad, or both are equally bad, then pick = 0.05

The main thing is to always pick the before you collect the data and start the test.

The above discussion was long, but it is really important information. If you don't know what the errors of the test are about, then there really is no point in making conclusions with the tests. Make sure you understand what the two errors are and what the probabilities are for them.

Now it is time to go back to the example and put this all together. This is the basic structure of testing a hypothesis, usually called a hypothesis test. Since this one has a test statistic involving \emph{z}, it is also called a \emph{z}-test. And since there is only one sample, it is usually called a one-sample \emph{z}-test.

\textbf{Example \#7.1.2: Battery Example Revisited. }

\begin{enumerate}
\def\labelenumi{\arabic{enumi}.}
\tightlist
\item
  State the random variable and the parameter in words
\end{enumerate}

\begin{quote}
\emph{x} = life of battery

= mean life of a XJ35 battery
\end{quote}

\begin{enumerate}
\def\labelenumi{\arabic{enumi}.}
\setcounter{enumi}{1}
\tightlist
\item
  State the null and alternative hypothesis and the level of
  significance
\end{enumerate}

\begin{quote}
= 0.10 (from above discussion about consequences)
\end{quote}

\begin{enumerate}
\def\labelenumi{\arabic{enumi}.}
\setcounter{enumi}{2}
\item
  State and check the assumptions for a hypothesis test

  Every hypothesis has some assumptions that be met to make sure that
  the results of the hypothesis are valid. The assumptions are
  different for each test. This test has the following assumptions.
\end{enumerate}

\begin{enumerate}
\def\labelenumi{\alph{enumi}.}
\item
  A random sample of size \emph{n} is taken.

  This occurred in this example, since it was stated that a random
  sample of 30 battery lives were taken.
\item
  The population standard deviation is known.

  This is true, since it was given in the problem.
\item
  The sample size is at least 30 or the population of the random
  variable is normally distributed.

  The sample size was 30, so this condition is met.
\end{enumerate}

\begin{enumerate}
\def\labelenumi{\arabic{enumi}.}
\setcounter{enumi}{3}
\tightlist
\item
  Find the sample statistic, test statistic, and p-value
\end{enumerate}

\begin{quote}
The test statistic depends on how many samples there are, what
parameter you are testing, and assumptions that need to be checked. In
this case, there is one sample and you are testing the mean. The
assumptions were checked above.

Sample statistic:

Test statistic:

p-value:

{\[CHART\]}

Using TI-83/84:

Using R:
\end{quote}

\begin{enumerate}
\def\labelenumi{\arabic{enumi}.}
\setcounter{enumi}{4}
\tightlist
\item
  Conclusion:
\end{enumerate}

\begin{quote}
Now what? Well, this p-value is 0.0142. This is a lot smaller than the
amount of error you would accept in the problem - = 0.10. That means
that finding a sample mean less than 490 days is unusual to happen if
is true. This should make you think that is not true. You should
reject .
\end{quote}

In fact, in general:

Reject if the p-value \textless{} and

Fail to reject if the p-value .

\begin{enumerate}
\def\labelenumi{\arabic{enumi}.}
\setcounter{enumi}{5}
\tightlist
\item
  Interpretation:
\end{enumerate}

\begin{quote}
Since you rejected , what does this mean in the real world? That is
what goes in the interpretation. Since you rejected the claim by the
manufacturer that the mean life of the batteries is 500 days, then you
now can believe that your hypothesis was correct. In other words,
there is enough evidence to show that the mean life of the battery is
less than 500 days.

Now that you know that the batteries last less than 500 days, should
you cancel the contract? Statistically, there is evidence that the
batteries do not last as long as the manufacturer says they should.
However, based on this sample there are only ten days less on average
that the batteries last. There may not be practical significance in
this case. Ten days do not seem like a large difference. In reality,
if the batteries are used in pacemakers, then you would probably tell
the patient to have the batteries replaced every year. You have a
large buffer whether the batteries last 490 days or 500 days. It seems
that it might not be worth it to break the contract over ten days.
What if the 10 days was practically significant? Are there any other
things you should consider? You might look at the business
relationship with the manufacturer. You might also look at how much it
would cost to find a new manufacturer. These are also questions to
consider before making any changes. What this discussion should show
you is that just because a hypothesis has statistical significance
does not mean it has practical significance. The hypothesis test is
just one part of a research process. There are other pieces that you
need to consider.
\end{quote}

That's it. That is what a hypothesis test looks like. All hypothesis
tests are done with the same six steps. Those general six steps are
outlined below.

\begin{enumerate}
\def\labelenumi{\arabic{enumi}.}
\tightlist
\item
  State the random variable and the parameter in words. This is where
  you are defining what the unknowns are in this problem.
\end{enumerate}

\begin{quote}
x = random variable

= mean of random variable, if the parameter of interest is the mean.
There are other parameters you can test, and you would use the
appropriate symbol for that parameter.
\end{quote}

\begin{enumerate}
\def\labelenumi{\arabic{enumi}.}
\setcounter{enumi}{1}
\tightlist
\item
  State the null and alternative hypotheses and the level of
  significance
\end{enumerate}

\begin{quote}
, where is the known mean

, use the appropriate one for your problem

Also, state your level here.
\end{quote}

\begin{enumerate}
\def\labelenumi{\arabic{enumi}.}
\setcounter{enumi}{2}
\item
  State and check the assumptions for a hypothesis test

  Each hypothesis test has its own assumptions. They will be stated
  when the different hypothesis tests are discussed.
\item
  Find the sample statistic, test statistic, and p-value
\end{enumerate}

\begin{quote}
This depends on what parameter you are working with, how many samples,
and the assumptions of the test. The p-value depends on your . If you
are doing the with the less than, then it is a left-tailed test, and
you find the probability of being in that left tail. If you are doing
the with the greater than, then it is a right-tailed test, and you
find the probability of being in the right tail. If you are doing the
with the not equal to, then you are doing a two-tail test, and you
find the probability of being in both tails. Because of symmetry, you
could find the probability in one tail and double this value to find
the probability in both tails.
\end{quote}

\begin{enumerate}
\def\labelenumi{\arabic{enumi}.}
\setcounter{enumi}{4}
\tightlist
\item
  Conclusion
\end{enumerate}

\begin{quote}
This is where you write reject or fail to reject . The rule is: if the
p-value \textless{} , then reject . If the p-value , then fail to reject
\end{quote}

\begin{enumerate}
\def\labelenumi{\arabic{enumi}.}
\setcounter{enumi}{5}
\tightlist
\item
  Interpretation
\end{enumerate}

\begin{quote}
This is where you interpret in real world terms the conclusion to the
test. The conclusion for a hypothesis test is that you either have
enough evidence to show is true, or you do not have enough evidence to
show is true.
\end{quote}

Sorry, one more concept about the conclusion and interpretation. First,
the conclusion is that you reject or you fail to reject . Why was it
said like this? It is because you never \textbf{accept} the null hypothesis.
If you wanted to accept the null hypothesis, then why do the test in the
first place? In the interpretation, you either have enough evidence to
show is true, or you do not have enough evidence to show is true. You
wouldn't want to go to all this work and then find out you wanted to
accept the claim. Why go through the trouble? You always want to show
that the alternative hypothesis is true. Sometimes you can do that and
sometimes you can't. It doesn't mean you proved the null hypothesis; it
just means you can't prove the alternative hypothesis. Here is an
example to demonstrate this.

\textbf{Example \#7.1.3: Conclusions in Hypothesis Tests}

\begin{quote}
In the U.S. court system a jury trial could be set up as a hypothesis
test. To really help you see how this works, let's use OJ Simpson as
an example. In the court system, a person is presumed innocent until
he/she is proven guilty, and this is your null hypothesis. OJ Simpson
was a football player in the 1970s. In 1994 his ex-wife and her friend
were killed. OJ Simpson was accused of the crime, and in 1995 the case
was tried. The prosecutors wanted to prove OJ was guilty of killing
his wife and her friend, and that is the alternative hypothesis

In this case, a verdict of not guilty was given. That does not mean
that he is innocent of this crime. It means there was not enough
evidence to prove he was guilty. Many people believe that OJ was
guilty of this crime, but the jury did not feel that the evidence
presented was enough to show there was guilt. The verdict in a jury
trial is always guilty or not guilty!
\end{quote}

The same is true in a hypothesis test. There is either enough or not
enough evidence to show that alternative hypothesis. It is not that you
proved the null hypothesis true.

When identifying hypothesis, it is important to state your random
variable and the appropriate parameter you want to make a decision
about. If count something, then the random variable is the number of
whatever you counted. The parameter is the proportion of what you
counted. If the random variable is something you measured, then the
parameter is the mean of what you measured. (Note: there are other
parameters you can calculate, and some analysis of those will be
presented in later chapters.)

\textbf{Example \#7.1.4: Stating Hypotheses}

\begin{quote}
Identify the hypotheses necessary to test the following statements:
\end{quote}

\begin{enumerate}
\def\labelenumi{\alph{enumi}.}
\tightlist
\item
  The average salary of a teacher is more than \$30,000.
\end{enumerate}

\begin{quote}
\textbf{Solution:}

\emph{x} = salary of teacher

mean salary of teacher

The guess is that and that is the alternative hypothesis.

The null hypothesis has the same parameter and number with an equal
sign.
\end{quote}

\begin{enumerate}
\def\labelenumi{\alph{enumi}.}
\setcounter{enumi}{1}
\tightlist
\item
  The proportion of students who like math is less than 10\%.
\end{enumerate}

\begin{quote}
\textbf{Solution:}

\emph{x} = number of students who like math

\emph{p} = proportion of students who like math

The guess is that \emph{p} \textless{} 0.10 and that is the alternative hypothesis.
\end{quote}

\begin{enumerate}
\def\labelenumi{\alph{enumi}.}
\setcounter{enumi}{2}
\tightlist
\item
  The average age of students in this class differs from 21.
\end{enumerate}

\begin{quote}
\textbf{Solution:}

\emph{x} = age of students in this class

mean age of students in this class

The guess is that and that is the alternative hypothesis.
\end{quote}

\textbf{Example \#7.1.5: Stating Type I and II Errors and Picking Level of
Significance}

\begin{enumerate}
\def\labelenumi{\alph{enumi}.}
\tightlist
\item
  The plant-breeding department at a major university developed a new
  hybrid raspberry plant called YumYum Berry. Based on research data,
  the claim is made that from the time shoots are planted 90 days on
  average are required to obtain the first berry with a standard
  deviation of 9.2 days. A corporation that is interested in marketing
  the product tests 60 shoots by planting them and recording the
  number of days before each plant produces its first berry. The
  sample mean is 92.3 days. The corporation wants to know if the mean
  number of days is more than the 90 days claimed. State the type I
  and type II errors in terms of this problem, consequences of each
  error, and state which level of significance to use.
\end{enumerate}

\begin{quote}
\textbf{Solution:}

\emph{x} = time to first berry for YumYum Berry plant

= mean time to first berry for YumYum Berry plant

Type I Error: If the corporation does a type I error, then they will
say that the plants take longer to produce than 90 days when they
don't. They probably will not want to market the plants if they think
they will take longer. They will not market them even though in
reality the plants do produce in 90 days. They may have loss of future
earnings, but that is all.

Type II error: The corporation do not say that the plants take longer
then 90 days to produce when they do take longer. Most likely they
will market the plants. The plants will take longer, and so customers
might get upset and then the company would get a bad reputation. This
would be really bad for the company.

Level of significance: It appears that the corporation would not want
to make a type II error. Pick a 10\% level of significance, .
\end{quote}

\begin{enumerate}
\def\labelenumi{\alph{enumi}.}
\setcounter{enumi}{1}
\tightlist
\item
  A concern was raised in Australia that the percentage of deaths of
  Aboriginal prisoners was higher than the percent of deaths of
  non-indigenous prisoners, which is 0.27\%. State the type I and type
  II errors in terms of this problem, consequences of each error, and
  state which level of significance to use.
\end{enumerate}

\begin{quote}
\textbf{Solution:}

x = number of Aboriginal prisoners who have died

\emph{p} = proportion of Aboriginal prisoners who have died

Type I error: Rejecting that the proportion of Aboriginal prisoners
who died was 0.27\%, when in fact it was 0.27\%. This would mean you
would say there is a problem when there isn't one. You could anger the
Aboriginal community, and spend time and energy researching something
that isn't a problem.

Type II error: Failing to reject that the proportion of Aboriginal
prisoners who died was 0.27\%, when in fact it is higher than 0.27\%.
This would mean that you wouldn't think there was a problem with
Aboriginal prisoners dying when there really is a problem. You risk
causing deaths when there could be a way to avoid them.

Level of significance: It appears that both errors may be issues in
this case. You wouldn't want to anger the Aboriginal community when
there isn't an issue, and you wouldn't want people to die when there
may be a way to stop it. It may be best to pick a 5\% level of
significance, .
\end{quote}

Hint -- hypothesis testing is really easy if you follow the same recipe
every time. The only differences in the various problems are the
assumptions of the test and the test statistic you calculate so you can
find the p-value. Do the same steps, in the same order, with the same
words, every time and these problems become very easy.

\hypertarget{homework-20}{%
\subsection{Homework}\label{homework-20}}

\textbf{For the problems in this section, a question is being asked. This is
to help you understand what the hypotheses are. You are not to run any
hypothesis tests and come up with any conclusions in this section.}

\begin{enumerate}
\def\labelenumi{\arabic{enumi}.}
\item
  Eyeglassomatic manufactures eyeglasses for different retailers. They test to see how many defective lenses they made in a given time period and found that 11\% of all lenses had defects of some type. Looking at the type of defects, they found in a three-month time period that out of 34,641 defective lenses, 5865 were due to scratches. Are there more defects from scratches than from all other causes? State the random variable, population parameter, and hypotheses.
\item
  According to the February 2008 Federal Trade Commission report on consumer fraud and identity theft, 23\% of all complaints in 2007 were for identity theft. In that year, Alaska had 321 complaints of identity theft out of 1,432 consumer complaints ("Consumer fraud and," 2008). Does this data provide enough evidence to show that Alaska had a lower proportion of identity theft than 23\%? State the random variable, population parameter, and hypothese {[}MISSING{]}
\item
  The Kyoto Protocol was signed in 1997, and required countries to start reducing their carbon emissions. The protocol became enforceable in February 2005. In 2004, the mean CO\textsubscript{2} emission was 4.87 metric tons per capita. Is there enough evidence to show that the mean CO\textsubscript{2} emission is lower in 2010 than in 2004? State the random variable, population parameter, and hypotheses.
\item
  The FDA regulates that fish that is consumed is allowed to contain 1.0 mg/kg of mercury. In Florida, bass fish were collected in 53 different lakes to measure the amount of mercury in the fish. The data for the average amount of mercury in each lake is in table \#7.3.5 ("Multi-disciplinary niser activity," 2013). Do the data provide enough evidence to show that the fish in Florida lakes has more mercury than the allowable amount? State the random variable, population parameter, and hypotheses.
\item
  Eyeglassomatic manufactures eyeglasses for different retailers. They test to see how many defective lenses they made in a given time period and found that 11\% of all lenses had defects of some type. Looking at the type of defects, they found in a three-month time period that out of 34,641 defective lenses, 5865 were due to scratches. Are there more defects from scratches than from all other causes? State the type I and type II errors in this case, consequences of each error type for this situation from the perspective of the manufacturer, and the appropriate alpha level to use. State why you picked this alpha level.
\item
  According to the February 2008 Federal Trade Commission report on consumer fraud and identity theft, 23\% of all complaints in 2007 were for identity theft. In that year, Alaska had 321 complaints of identity theft out of 1,432 consumer complaints ("Consumer fraud and," 2008). Does this data provide enough evidence to show that Alaska had a lower proportion of identity theft than 23\%? State the type I and type II errors in this case, consequences of each error type for this situation from the perspective of the state of Arizona, and the appropriate alpha level to use. State why you picked this alpha level.
\item
  The Kyoto Protocol was signed in 1997, and required countries to start reducing their carbon emissions. The protocol became enforceable in February 2005. In 2004, the mean CO\textsubscript{2} emission was 4.87 metric tons per capita. Is there enough evidence to show that the mean CO\textsubscript{2} emission is lower in 2010 than in 2004? State the type I and type II errors in this case, consequences of each error type for this situation from the perspective of the agency overseeing the protocol, and the appropriate alpha level to use. State why you picked this alpha level.
\item
  The FDA regulates that fish that is consumed is allowed to contain 1.0 mg/kg of mercury. In Florida, bass fish were collected in 53 different lakes to measure the amount of mercury in the fish. The data for the average amount of mercury in each lake is in table \#7.3.5 ("Multi-disciplinary niser activity," 2013). Do the data provide enough evidence to show that the fish in Florida lakes has more mercury than the allowable amount? State the type I and type II errors in this case, consequences of each error type for this situation from the perspective of the FDA, and the appropriate alpha level to use. State why you picked this alpha level.
\end{enumerate}

\textbf{\\
}

\hypertarget{one-sample-proportion-test}{%
\section{One-Sample Proportion Test}\label{one-sample-proportion-test}}

There are many different parameters that you can test. There is a test for the mean, such as was introduced with the z-test. There is also a test for the population proportion, \emph{p}. This is where you might be curious if the proportion of students who smoke at your school is lower than the proportion in your area. Or you could question if the proportion of accidents caused by teenage drivers who do not have a drivers' education class is more than the national proportion.

To test a population proportion, there are a few things that need to be defined first. Usually, Greek letters are used for parameters and Latin letters for statistics. When talking about proportions, it makes sense to use \emph{p} for proportion. The Greek letter for \emph{p} is , but that is too confusing to use. Instead, it is best to use \emph{p} for the population proportion. That means that a different symbol is needed for the sample proportion. The convention is to use, , known as p-hat. This way you know that \emph{p} is the population proportion, and that is the sample proportion related to it.

Now proportion tests are about looking for the percentage of individuals who have a particular attribute. You are really looking for the number of successes that happen. Thus, a proportion test involves a binomial distribution.

\textbf{Hypothesis Test for One Population Proportion (1-Prop Test)}

\begin{enumerate}
\def\labelenumi{\arabic{enumi}.}
\tightlist
\item
  State the random variable and the parameter in words.
\end{enumerate}

\begin{quote}
x = number of successes

\emph{p} = proportion of successes
\end{quote}

\begin{enumerate}
\def\labelenumi{\arabic{enumi}.}
\setcounter{enumi}{1}
\tightlist
\item
  State the null and alternative hypotheses and the level of
  significance
\end{enumerate}

\begin{quote}
, where is the known proportion

, use the appropriate one for your problem

Also, state your level here.
\end{quote}

\begin{enumerate}
\def\labelenumi{\arabic{enumi}.}
\setcounter{enumi}{2}
\tightlist
\item
  State and check the assumptions for a hypothesis test
\end{enumerate}

\begin{enumerate}
\def\labelenumi{\alph{enumi}.}
\item
  A simple random sample of size \emph{n} is taken.
\item
  The conditions for the binomial distribution are satisfied
\item
  To determine the sampling distribution of , you need to show that
  and , where . If this requirement is true, then the sampling
  distribution of is well approximated by a normal curve.
\end{enumerate}

\begin{enumerate}
\def\labelenumi{\arabic{enumi}.}
\setcounter{enumi}{3}
\tightlist
\item
  Find the sample statistic, test statistic, and p-value
\end{enumerate}

\begin{quote}
Sample Proportion:

Test Statistic:

p-value:

TI-83/84: Use normalcdf(lower limit, upper limit, 0, 1)

(Note: if , then lower limit is and upper limit is your test
statistic. If , then lower limit is your test statistic and the upper
limit is . If , then find the p-value for , and multiply by 2.)

R: Use pnorm(z, 0, 1)

(Note: if , then you can use pnorm. If , then you have to find pnorm
and then subtract from 1. If , then find the p-value for , and
multiply by 2.)
\end{quote}

\begin{enumerate}
\def\labelenumi{\arabic{enumi}.}
\setcounter{enumi}{4}
\tightlist
\item
  Conclusion
\end{enumerate}

\begin{quote}
This is where you write reject or fail to reject . The rule is: if the
p-value \textless{} , then reject . If the p-value , then fail to reject
\end{quote}

\begin{enumerate}
\def\labelenumi{\arabic{enumi}.}
\setcounter{enumi}{5}
\tightlist
\item
  Interpretation
\end{enumerate}

\begin{quote}
This is where you interpret in real world terms the conclusion to the
test. The conclusion for a hypothesis test is that you either have
enough evidence to show is true, or you do not have enough evidence to
show is true.
\end{quote}

\textbf{Example \#7.2.1: Hypothesis Test for One Proportion Using Formula}

\begin{quote}
A concern was raised in Australia that the percentage of deaths of
Aboriginal prisoners was higher than the percent of deaths of
non-Aboriginal prisoners, which is 0.27\%. A sample of six years
(1990-1995) of data was collected, and it was found that out of 14,495
Aboriginal prisoners, 51 died ("Indigenous deaths in," 1996). Do the
data provide enough evidence to show that the proportion of deaths of
Aboriginal prisoners is more than 0.27\%?

\textbf{Solution:}
\end{quote}

\begin{enumerate}
\def\labelenumi{\arabic{enumi}.}
\tightlist
\item
  State the random variable and the parameter in words.
\end{enumerate}

\begin{quote}
\emph{x} = number of Aboriginal prisoners who die

\emph{p} = proportion of Aboriginal prisoners who die
\end{quote}

\begin{enumerate}
\def\labelenumi{\arabic{enumi}.}
\setcounter{enumi}{1}
\tightlist
\item
  State the null and alternative hypotheses and the level of
  significance
\end{enumerate}

\begin{quote}
Example \#7.1.5b argued that the .
\end{quote}

\begin{enumerate}
\def\labelenumi{\arabic{enumi}.}
\setcounter{enumi}{2}
\tightlist
\item
  State and check the assumptions for a hypothesis test
\end{enumerate}

\begin{enumerate}
\def\labelenumi{\alph{enumi}.}
\item
  A simple random sample of 14,495 Aboriginal prisoners was taken.
  However, the sample was not a random sample, since it was data from
  six years. It is the numbers for all prisoners in these six years,
  but the six years were not picked at random. Unless there was
  something special about the six years that were chosen, the sample
  is probably a representative sample. This assumption is probably
  met.
\item
  There are 14,495 prisoners in this case. The prisoners are all
  Aboriginals, so you are not mixing Aboriginal with non-Aboriginal
  prisoners. There are only two outcomes, either the prisoner dies or
  doesn't. The chance that one prisoner dies over another may not be
  constant, but if you consider all prisoners the same, then it may be
  close to the same probability. Thus the conditions for the binomial
  distribution are satisfied
\item
  In this case \emph{p} = 0.0027 and \emph{n} = 14,495. and . So, the sampling
  distribution for is a normal distribution.
\end{enumerate}

\begin{enumerate}
\def\labelenumi{\arabic{enumi}.}
\setcounter{enumi}{3}
\tightlist
\item
  Find the sample statistic, test statistic, and p-value
\end{enumerate}

\begin{quote}
Sample Proportion:

Test Statistic:

p-value:

TI-83/84: p-value =

R: p-value =
\end{quote}

\begin{enumerate}
\def\labelenumi{\arabic{enumi}.}
\setcounter{enumi}{4}
\tightlist
\item
  Conclusion
\end{enumerate}

\begin{quote}
Since the p-value \textless{} 0.05, then reject .
\end{quote}

\begin{enumerate}
\def\labelenumi{\arabic{enumi}.}
\setcounter{enumi}{5}
\tightlist
\item
  Interpretation
\end{enumerate}

\begin{quote}
There is enough evidence to show that the proportion of deaths of
Aboriginal prisoners is more than for non-Aboriginal prisoners.
\end{quote}

\textbf{Example \#7.2.2: Hypothesis Test for One Proportion Using Technology}

\begin{quote}
A researcher who is studying the effects of income levels on
breastfeeding of infants hypothesizes that countries where the income
level is lower have a higher rate of infant breastfeeding than higher
income countries. It is known that in Germany, considered a
high-income country by the World Bank, 22\% of all babies are
breastfeed. In Tajikistan, considered a low-income country by the
World Bank, researchers found that in a random sample of 500 new
mothers that 125 were breastfeeding their infant. At the 5\% level of
significance, does this show that low-income countries have a higher
incident of breastfeeding?

\textbf{Solution:}
\end{quote}

\begin{enumerate}
\def\labelenumi{\arabic{enumi}.}
\tightlist
\item
  State you random variable and the parameter in words.
\end{enumerate}

\begin{quote}
\emph{x} = number of woman who breastfeed in a low-income country

\emph{p} = proportion of woman who breastfeed in a low-income country
\end{quote}

\begin{enumerate}
\def\labelenumi{\arabic{enumi}.}
\setcounter{enumi}{1}
\item
  State the null and alternative hypotheses and the level of
  significance
\item
  State and check the assumptions for a hypothesis test
\end{enumerate}

\begin{enumerate}
\def\labelenumi{\alph{enumi}.}
\item
  A simple random sample of 500 breastfeeding habits of woman in a
  low-income country was taken as was stated in the problem.
\item
  There were 500 women in the study. The women are considered
  identical, though they probably have some differences. There are
  only two outcomes, either the woman breastfeeds or she doesn't. The
  probability of a woman breastfeeding is probably not the same for
  each woman, but it is probably not very different for each woman.
  The conditions for the binomial distribution are satisfied
\item
  In this case, \emph{n} = 500 and \emph{p} = 0.22. and , so the sampling
  distribution of is well approximated by a normal curve.
\end{enumerate}

\begin{enumerate}
\def\labelenumi{\arabic{enumi}.}
\setcounter{enumi}{3}
\tightlist
\item
  Find the sample statistic, test statistic, and p-value
\end{enumerate}

\begin{quote}
This time, all calculations will be done with technology.

On the TI-83/84 calculator. Go into the STAT menu, then arrow over to
TESTS. This test is a 1-propZTest. Then type in the information just
as shown in figure \#7.2.1.

\textbf{Figure \#7.2.1: Setup for 1-Proportion Test}

\includegraphics[width=2.75in,height=1.86111in]{media/image174.png}
\end{quote}

Once you press Calculate, you will see the results as in figure \#7.2.2.

\textbf{Figure \#7.2.2: Results for 1-Proportion Test}

\begin{quote}
\includegraphics[width=2.75in,height=1.86111in]{media/image175.png}

The \emph{z} in the results is the test statistic. The \emph{p} = 0.052683219 is
the p-value, and the is the sample proportion.

The p-value is approximately 0.053

On R, the command is prop.test(x, n, po, alternative = "less" or
"greater"), where po is what H\textsubscript{o} says p equals, and you use less if
your H\textsubscript{A} is less and greater if your H\textsubscript{A} is greater. If your H\textsubscript{A} is
not equal to, then leave off the alternative statement. So for this
example, the command would be prop.test(125, 500, .22, alternative =
"greater")

1-sample proportions test with continuity correction

data: 125 out of 500, null probability 0.22

X-squared = 2.4505, df = 1, p-value = 0.05874

alternative hypothesis: true p is greater than 0.22

95 percent confidence interval:

0.218598 1.000000

sample estimates:

p

0.25

Note: R does a continuity correction that the formula and the TI-83/84
calculator do not do. You can put in a command that says not to use
the continuity correction, but it is correct to use it. Also, R
doesn't give the z test statistic, so you don't need to worry about
this. It does give a p-value that is slightly off from the formula and
the calculator due to the continuity correction.

p-value = 0.05874
\end{quote}

\begin{enumerate}
\def\labelenumi{\arabic{enumi}.}
\setcounter{enumi}{4}
\tightlist
\item
  Conclusion
\end{enumerate}

\begin{quote}
Since the p-value is more than 0.05, you fail to reject .
\end{quote}

\begin{enumerate}
\def\labelenumi{\arabic{enumi}.}
\setcounter{enumi}{5}
\tightlist
\item
  Interpretation
\end{enumerate}

\begin{quote}
There is not enough evidence to show that the proportion of women who
breastfeed in low-income countries is more than in high-income
countries.
\end{quote}

Notice, the conclusion is that there wasn't enough evidence to show
what said. The conclusion was not that you proved true. There are many
reasons why you can't say that is true. It could be that the countries
you chose were not very representative of what truly happens. If you
instead looked at all high-income countries and compared them to
low-income countries, you might have different results. It could also be
that the sample you collected in the low-income country was not
representative. It could also be that income level is not an indication
of breastfeeding habits. There could be other factors involved. This is
why you can't say that you have proven is true. There are too many other
factors that could be the reason that you failed to reject .

\hypertarget{homework-21}{%
\subsection{Homework}\label{homework-21}}

\textbf{In each problem show all steps of the hypothesis test. If some of the
assumptions are not met, note that the results of the test may not be
correct and then continue the process of the hypothesis test.}

\begin{enumerate}
\def\labelenumi{\arabic{enumi}.}
\item
  Eyeglassomatic manufactures eyeglasses for different retailers. They
  test to see how many defective lenses they made in a given time
  period and found that 11\% of all lenses had defects of some type.
  Looking at the type of defects, they found in a three-month time
  period that out of 34,641 defective lenses, 5865 were due to
  scratches. Are there more defects from scratches than from all other
  causes? Use a 1\% level of significance.
\item
  In July of 1997, Australians were asked if they thought unemployment
  would increase, and 47\% thought that it would increase. In November
  of 1997, they were asked again. At that time 284 out of 631 said
  that they thought unemployment would increase ("Morgan gallup
  poll," 2013). At the 5\% level, is there enough evidence to show
  that the proportion of Australians in November 1997 who believe
  unemployment would increase is less than the proportion who felt it
  would increase in July 1997?
\item
  According to the February 2008 Federal Trade Commission report on
  consumer fraud and identity theft, 23\% of all complaints in 2007
  were for identity theft. In that year, Arkansas had 1,601 complaints
  of identity theft out of 3,482 consumer complaints ("Consumer fraud
  and," 2008). Does this data provide enough evidence to show that
  Arkansas had a higher proportion of identity theft than 23\%? Test at
  the 5\% level.
\item
  According to the February 2008 Federal Trade Commission report on
  consumer fraud and identity theft, 23\% of all complaints in 2007
  were for identity theft. In that year, Alaska had 321 complaints of
  identity theft out of 1,432 consumer complaints ("Consumer fraud
  and," 2008). Does this data provide enough evidence to show that
  Alaska had a lower proportion of identity theft than 23\%? Test at
  the 5\% level.
\item
  In 2001, the Gallup poll found that 81\% of American adults believed
  that there was a conspiracy in the death of President Kennedy. In
  2013, the Gallup poll asked 1,039 American adults if they believe
  there was a conspiracy in the assassination, and found that 634
  believe there was a conspiracy ("Gallup news service," 2013). Do
  the data show that the proportion of Americans who believe in this
  conspiracy has decreased? Test at the 1\% level.
\item
  In 2008, there were 507 children in Arizona out of 32,601 who were
  diagnosed with Autism Spectrum Disorder (ASD) ("Autism and
  developmental," 2008). Nationally 1 in 88 children are diagnosed
  with ASD ("CDC features -," 2013). Is there sufficient data to
  show that the incident of ASD is more in Arizona than nationally?
  Test at the 1\% level.
\end{enumerate}

\textbf{\\
}

\hypertarget{one-sample-test-for-the-mean}{%
\section{One-Sample Test for the Mean}\label{one-sample-test-for-the-mean}}

It is time to go back to look at the test for the mean that was
introduced in section 7.1 called the z-test. In the example, you knew
what the population standard deviation, , was. What if you don't know ?

You could just use the sample standard deviation, \emph{s}, as an
approximation of. That means the test statistic is now . Great, now you
can go and find the p-value using the normal curve. Or can you? Is this
new test statistic normally distributed? Actually, it is not. How is it
distributed? A man named W. S. Gossett figured out what this
distribution is and called it the Student's t-distribution. There are
some assumptions that must be made for this formula to be a Student's
t-distribution. These are outlined in the following theorem. Note: the
t-distribution is called the Student's t-distribution because that is
the name he published under because he couldn't publish under his own
name due to employer not wanting him to publish under his own name. His
employer by the way was Guinness and they didn't want competitors
knowing they had a chemist working for them. It is not called the
Student's t-distribution because it is only used by students.

Theorem: If the following assumptions are met

\begin{enumerate}
\def\labelenumi{\alph{enumi}.}
\item
  A random sample of size \emph{n} is taken.
\item
  The distribution of the random variable is normal or the sample size
  is 30 or more.
\end{enumerate}

\begin{quote}
Then the distribution of is a Student's t-distribution with degrees of
freedom.
\end{quote}

\textbf{Explanation of degrees of freedom:} Recall the formula for sample
standard deviation is . Notice the denominator is . This is the same as
the degrees of freedom. This is no accident. The reason the denominator
and the degrees of freedom are both comes from how the standard
deviation is calculated. Remember, first you take each data value and
subtract . If you add up all of these new values, you will get 0. This
must happen. Since it must happen, the first data values you have
``freedom of choice'', but the nth data value, you have no freedom to
choose. Hence, you have degrees of freedom. Another way to think about
it is that if you five people and five chairs, the first four people
have a choice of where they are sitting, but the last person does not.
They have no freedom of where to sit. Only people have freedom of
choice.

The Student's t-distribution is a bell-shape that is more spread out
than the normal distribution. There are many t-distributions, one for
each different degree of freedom.

Here is a graph of the normal distribution and the Student's
t-distribution for df = 1 and df = 2.

\textbf{Figure \#7.3.1: Typical Student t-Distributions}

\begin{quote}
\includegraphics[width=3.76389in,height=2.20833in]{media/image196.png}
\end{quote}

As the degrees of freedom increases, the student's t-distribution looks
more like the normal distribution.

To find probabilities for the t-distribution, again technology can do
this for you. There are many technologies out there that you can use.

On the TI-83/84, the command is in the DISTR menu and is tcdf(. The
syntax for this command is

On R: the command to find the area to the left of a t value is pt(t
value, df)

\textbf{Hypothesis Test for One Population Mean (t-Test)}

\begin{enumerate}
\def\labelenumi{\arabic{enumi}.}
\tightlist
\item
  State the random variable and the parameter in words.
\end{enumerate}

\begin{quote}
\emph{x} = random variable

= mean of random variable
\end{quote}

\begin{enumerate}
\def\labelenumi{\arabic{enumi}.}
\setcounter{enumi}{1}
\tightlist
\item
  State the null and alternative hypotheses and the level of
  significance
\end{enumerate}

\begin{quote}
, where is the known mean

, use the appropriate one for your problem

Also, state your level here.
\end{quote}

\begin{enumerate}
\def\labelenumi{\arabic{enumi}.}
\setcounter{enumi}{2}
\tightlist
\item
  State and check the assumptions for a hypothesis test
\end{enumerate}

\begin{enumerate}
\def\labelenumi{\alph{enumi}.}
\item
  A random sample of size \emph{n} is taken.
\item
  The population of the random variable is normally distributed,
  though the t-test is fairly robust to the condition if the sample
  size is large. This means that if this condition isn't met, but your
  sample size is quite large (over 30), then the results of the t-test
  are valid.
\item
  The population standard deviation, , is unknown.
\end{enumerate}

\begin{enumerate}
\def\labelenumi{\arabic{enumi}.}
\setcounter{enumi}{3}
\tightlist
\item
  Find the sample statistic, test statistic, and p-value
\end{enumerate}

\begin{quote}
Test Statistic:

with degrees of freedom =

p-value:

Using TI-83/84:

(Note: if , then lower limit is and upper limit is your test
statistic. If , then lower limit is your test statistic and the upper
limit is . If , then find the p-value for , and multiply by 2.)

Using R: pt(t value, df)

(Note: if , then the command is pt(t value, df). If , then the command
is . If , then find the p-value for , and multiply by 2.)
\end{quote}

\begin{enumerate}
\def\labelenumi{\arabic{enumi}.}
\setcounter{enumi}{4}
\tightlist
\item
  Conclusion
\end{enumerate}

\begin{quote}
This is where you write reject or fail to reject . The rule is: if the
p-value \textless{} , then reject . If the p-value , then fail to reject
\end{quote}

\begin{enumerate}
\def\labelenumi{\arabic{enumi}.}
\setcounter{enumi}{5}
\tightlist
\item
  Interpretation
\end{enumerate}

\begin{quote}
This is where you interpret in real world terms the conclusion to the
test. The conclusion for a hypothesis test is that you either have
enough evidence to show is true, or you do not have enough evidence to
show is true.
\end{quote}

\textbf{How to check the assumptions of t-test:}

In order for the t-test to be valid, the assumptions of the test must be
true. Whenever you run a t-test, you must make sure the assumptions are
true. You need to check them. Here is how you do this:

\begin{enumerate}
\def\labelenumi{\arabic{enumi}.}
\item
  For the condition that the sample is a random sample, describe how
  you took the sample. Make sure your sampling technique is random.
\item
  For the condition that population of the random variable is normal,
  remember the process of assessing normality from chapter 6.
\end{enumerate}

Note: if the assumptions behind this test are not valid, then the
conclusions you make from the test are not valid. If you do not have a
random sample, that is your fault. Make sure the sample you take is as
random as you can make it following sampling techniques from chapter 1.
If the population of the random variable is not normal, then take a
sample larger than 30. If you cannot afford to do that, or if it is not
logistically possible, then you do different tests called non-parametric
tests. There is an entire course on non-parametric tests, and they will
not be discussed in this book.

\textbf{\\
}

\textbf{Example \#7.3.1: Test of the Mean Using the Formula}

\begin{quote}
A random sample of 20 IQ scores of famous people was taken from the
website of IQ of Famous People ("IQ of famous," 2013) and a random
number generator was used to pick 20 of them. The data are in table
\#7.3.1. Do the data provide evidence at the 5\% level that the IQ of a
famous person is higher than the average IQ of 100?

\textbf{Table \#7.3.1: IQ Scores of Famous People}
\end{quote}

\begin{longtable}[]{@{}lllll@{}}
\toprule
158 & 180 & 150 & 137 & 109\tabularnewline
\midrule
\endhead
225 & 122 & 138 & 145 & 180\tabularnewline
118 & 118 & 126 & 140 & 165\tabularnewline
150 & 170 & 105 & 154 & 118\tabularnewline
\bottomrule
\end{longtable}

\begin{quote}
\textbf{Solution:}
\end{quote}

\begin{enumerate}
\def\labelenumi{\arabic{enumi}.}
\tightlist
\item
  State the random variable and the parameter in words.
\end{enumerate}

\begin{quote}
\emph{x} = IQ score of a famous person

= mean IQ score of a famous person
\end{quote}

\begin{enumerate}
\def\labelenumi{\arabic{enumi}.}
\setcounter{enumi}{1}
\item
  State the null and alternative hypotheses and the level of
  significance
\item
  State and check the assumptions for a hypothesis test
\end{enumerate}

\begin{enumerate}
\def\labelenumi{\alph{enumi}.}
\item
  A random sample of 20 IQ scores was taken. This was said in the
  problem.
\item
  The population of IQ score is normally distributed. This was shown
  in example \#6.4.2.
\end{enumerate}

\begin{enumerate}
\def\labelenumi{\arabic{enumi}.}
\setcounter{enumi}{3}
\tightlist
\item
  Find the sample statistic, test statistic, and p-value
\end{enumerate}

\begin{quote}
Sample Statistic:

Test Statistic:

p-value:

TI-83/84: p-value =

R: p-value =
\end{quote}

\begin{enumerate}
\def\labelenumi{\arabic{enumi}.}
\setcounter{enumi}{4}
\tightlist
\item
  Conclusion
\end{enumerate}

\begin{quote}
Since the p-value is less than 5\%, then reject .
\end{quote}

\begin{enumerate}
\def\labelenumi{\arabic{enumi}.}
\setcounter{enumi}{5}
\tightlist
\item
  Interpretation
\end{enumerate}

\begin{quote}
There is enough evidence to show that famous people have a higher IQ
than the average IQ of 100.
\end{quote}

\textbf{Example \#7.3.2:} \textbf{Test of the Mean Using Technology}

\begin{quote}
In 2011, the average life expectancy for a woman in Europe was 79.8
years. The data in table \#7.3.2 are the life expectancies for men in
European countries in 2011 ("WHO life expectancy," 2013). Do the
data indicate that men's life expectancy is less than women's? Test at
the 1\% level.

\textbf{Table \#7.3.2: Life Expectancies for Men in European Countries in
2011}
\end{quote}

\begin{longtable}[]{@{}llllllll@{}}
\toprule
73 & 79 & 67 & 78 & 69 & 66 & 78 & 74\tabularnewline
\midrule
\endhead
71 & 74 & 79 & 75 & 77 & 71 & 78 & 78\tabularnewline
68 & 78 & 78 & 71 & 81 & 79 & 80 & 80\tabularnewline
62 & 65 & 69 & 68 & 79 & 79 & 79 & 73\tabularnewline
79 & 79 & 72 & 77 & 67 & 70 & 63 & 82\tabularnewline
72 & 72 & 77 & 79 & 80 & 80 & 67 & 73\tabularnewline
73 & 60 & 65 & 79 & 66 & & &\tabularnewline
\bottomrule
\end{longtable}

\textbf{Solution:}

\begin{enumerate}
\def\labelenumi{\arabic{enumi}.}
\tightlist
\item
  State the random variable and the parameter in words.
\end{enumerate}

\begin{quote}
\emph{x} = life expectancy for a European man in 2011

= mean life expectancy for European men in 2011
\end{quote}

\begin{enumerate}
\def\labelenumi{\arabic{enumi}.}
\setcounter{enumi}{1}
\item
  State the null and alternative hypotheses and the level of
  significance
\item
  State and check the assumptions for a hypothesis test
\end{enumerate}

\begin{enumerate}
\def\labelenumi{\alph{enumi}.}
\item
  A random sample of 53 life expectancies of European men in 2011 was
  taken. The data is actually all of the life expectancies for every
  country that is considered part of Europe by the World Health
  Organization. However, the information is still sample information
  since it is only for one year that the data was collected. It may
  not be a random sample, but that is probably not an issue in this
  case.
\item
  The distribution of life expectancies of European men in 2011 is
  normally distributed. To see if this condition has been met, look at
  the histogram, number of outliers, and the normal probability plot.
  (If you wish, you can look at the normal probability plot first. If
  it doesn't look linear, then you may want to look at the histogram
  and number of outliers at this point.)
\end{enumerate}

\textbf{\\
}

\textbf{Figure \#7.3.2: Histogram for Life Expectancies of European Men in
2011}

\includegraphics[width=2.11111in,height=2.11111in]{media/image240.emf}

Not bell shaped

\begin{quote}
\textbf{Number of outliers:}

\textbf{Figure \#7.3.3: Modified Box Plot for Life Expectancies of European
Men in 2011}

\includegraphics[width=2.80556in,height=2.80556in]{media/image241.emf}

or:

Outliers are numbers below 54 and above 94. There are no outliers for
this data set.

\textbf{Figure \#7.3.4: Normal Quantile Plot for Life Expectancies of
European Men in 2011}

\includegraphics[width=3.48611in,height=3.48611in]{media/image243.emf}

Not linear
\end{quote}

This population does not appear to be normally distributed. This sample
is larger than 30, so it is good that the t-test is robust.

\begin{enumerate}
\def\labelenumi{\arabic{enumi}.}
\setcounter{enumi}{3}
\tightlist
\item
  Find the sample statistic, test statistic, and p-value
\end{enumerate}

\begin{quote}
The calculations will be conducted using technology.

On the TI-83/84 calculator. Go into STAT and type the data into L1.
Then go into STAT and move over to TESTS. Choose T-Test. The setup for
the calculator is in figure \#7.3.4.

\textbf{Figure \#7.3.5: Setup for T-Test on TI-83/84 Calculator}

\includegraphics[width=2.75in,height=1.86111in]{media/image244.png}

Once you press ENTER on Calculate you will see the result shown in
figure \#7.3.6.

\textbf{Figure \#7.3.6: Result of T-Test on TI-83/84 Calculator}

\includegraphics[width=2.75in,height=1.86111in]{media/image245.png}

On R, the command is t.test(variable, mu = number in H\textsubscript{o}, alternative
= "less" or "greater"), where mu = what Ho says the mean equals,
and you use less if your H\textsubscript{A} is less and greater if your H\textsubscript{A} is
greater. If your H\textsubscript{A} is not equal to, then leave off the alternative
statement. For this example, the command would be t.test(expectancy,
mu=79.8, alternative = "less")

One Sample t-test

data: expectancy

t = -7.7069, df = 52, p-value = 1.853e-10

alternative hypothesis: true mean is less than 79.8

95 percent confidence interval:

-Inf 75.05357

sample estimates:

mean of x

73.73585

Most of the output you don't need. You need the test statistic and the
p-value.

The is the test statistic. The p-value is .
\end{quote}

\begin{enumerate}
\def\labelenumi{\arabic{enumi}.}
\setcounter{enumi}{4}
\tightlist
\item
  Conclusion
\end{enumerate}

\begin{quote}
Since the p-value is less than 1\%, then reject .
\end{quote}

\begin{enumerate}
\def\labelenumi{\arabic{enumi}.}
\setcounter{enumi}{5}
\tightlist
\item
  Interpretation
\end{enumerate}

\begin{quote}
There is enough evidence to show that the mean life expectancy for
European men in 2011 was less than the mean life expectancy for
European women in 2011 of 79.8 years.
\end{quote}

\textbf{\\
}

\hypertarget{homework-22}{%
\subsection{Homework}\label{homework-22}}

\textbf{In each problem show all steps of the hypothesis test. If some of the
assumptions are not met, note that the results of the test may not be
correct and then continue the process of the hypothesis test.}

\begin{enumerate}
\def\labelenumi{\arabic{enumi}.}
\tightlist
\item
  The Kyoto Protocol was signed in 1997, and required countries to
  start reducing their carbon emissions. The protocol became
  enforceable in February 2005. In 2004, the mean CO\textsubscript{2} emission was
  4.87 metric tons per capita. Table 7.3.3 contains a random sample of
  CO\textsubscript{2} emissions in 2010 ("CO2 emissions," 2013). Is there enough
  evidence to show that the mean CO\textsubscript{2} emission is lower in 2010 than
  in 2004? Test at the 1\% level.
\end{enumerate}

\textbf{Table \#7.3.3: CO\textsubscript{2} Emissions (in metric tons per capita) in 2010}

\begin{longtable}[]{@{}lllllll@{}}
\toprule
1.36 & 1.42 & 5.93 & 5.36 & 0.06 & 9.11 & 7.32\tabularnewline
\midrule
\endhead
7.93 & 6.72 & 0.78 & 1.80 & 0.20 & 2.27 & 0.28\tabularnewline
5.86 & 3.46 & 1.46 & 0.14 & 2.62 & 0.79 & 7.48\tabularnewline
0.86 & 7.84 & 2.87 & 2.45 & & &\tabularnewline
\bottomrule
\end{longtable}

\begin{enumerate}
\def\labelenumi{\arabic{enumi}.}
\setcounter{enumi}{1}
\tightlist
\item
  The amount of sugar in a Krispy Kream glazed donut is 10 g. Many
  people feel that cereal is a healthier alternative for children over
  glazed donuts. Table \#7.3.4 contains the amount of sugar in a
  sample of cereal that is geared towards children ("Healthy
  breakfast story," 2013). Is there enough evidence to show that the
  mean amount of sugar in children's cereal is more than in a glazed
  donut? Test at the 5\% level.
\end{enumerate}

\begin{quote}
\textbf{Table \#7.3.4: Sugar Amounts in Children's Cereal}
\end{quote}

\begin{longtable}[]{@{}lllllll@{}}
\toprule
10 & 14 & 12 & 9 & 13 & 13 & 13\tabularnewline
\midrule
\endhead
11 & 12 & 15 & 9 & 10 & 11 & 3\tabularnewline
6 & 12 & 15 & 12 & 12 & &\tabularnewline
\bottomrule
\end{longtable}

\begin{enumerate}
\def\labelenumi{\arabic{enumi}.}
\setcounter{enumi}{2}
\tightlist
\item
  The FDA regulates that fish that is consumed is allowed to contain
  1.0 mg/kg of mercury. In Florida, bass fish were collected in 53
  different lakes to measure the amount of mercury in the fish. The
  data for the average amount of mercury in each lake is in table
  \#7.3.5 ("Multi-disciplinary niser activity," 2013). Do the data
  provide enough evidence to show that the fish in Florida lakes has
  more mercury than the allowable amount? Test at the 10\% level.
\end{enumerate}

\begin{quote}
\textbf{Table \#7.3.5: Average Mercury Levels (mg/kg) in Fish}
\end{quote}

\begin{longtable}[]{@{}llllll@{}}
\toprule
1.23 & 1.33 & 0.04 & 0.44 & 1.20 & 0.27\tabularnewline
\midrule
\endhead
0.48 & 0.19 & 0.83 & 0.81 & 0.71 & 0.5\tabularnewline
0.49 & 1.16 & 0.05 & 0.15 & 0.19 & 0.77\tabularnewline
1.08 & 0.98 & 0.63 & 0.56 & 0.41 & 0.73\tabularnewline
0.34 & 0.59 & 0.34 & 0.84 & 0.50 & 0.34\tabularnewline
0.28 & 0.34 & 0.87 & 0.56 & 0.17 & 0.18\tabularnewline
0.19 & 0.04 & 0.49 & 1.10 & 0.16 & 0.10\tabularnewline
0.48 & 0.21 & 0.86 & 0.52 & 0.65 & 0.27\tabularnewline
0.94 & 0.40 & 0.43 & 0.25 & 0.27 &\tabularnewline
\bottomrule
\end{longtable}

\begin{enumerate}
\def\labelenumi{\arabic{enumi}.}
\setcounter{enumi}{3}
\tightlist
\item
  Stephen Stigler determined in 1977 that the speed of light is
  299,710.5 km/sec.~In 1882, Albert Michelson had collected
  measurements on the speed of light ("Student t-distribution,"
  2013). His measurements are given in table \#7.3.6. Is there
  evidence to show that Michelson's data is different from Stigler's
  value of the speed of light? Test at the 5\% level.
\end{enumerate}

\begin{quote}
\textbf{Table \#7.3.6: Speed of Light Measurements in (km/sec)}
\end{quote}

\begin{longtable}[]{@{}lllll@{}}
\toprule
299883 & 299816 & 299778 & 299796 & 299682\tabularnewline
\midrule
\endhead
299711 & 299611 & 299599 & 300051 & 299781\tabularnewline
299578 & 299796 & 299774 & 299820 & 299772\tabularnewline
299696 & 299573 & 299748 & 299748 & 299797\tabularnewline
299851 & 299809 & 299723 & &\tabularnewline
\bottomrule
\end{longtable}

\begin{enumerate}
\def\labelenumi{\arabic{enumi}.}
\setcounter{enumi}{4}
\tightlist
\item
  Table \#7.3.7 contains pulse rates after running for 1 minute,
  collected from females who drink alcohol ("Pulse rates before,"
  2013). The mean pulse rate after running for 1 minute of females who
  do not drink is 97 beats per minute. Do the data show that the mean
  pulse rate of females who do drink alcohol is higher than the mean
  pulse rate of females who do not drink? Test at the 5\% level.
\end{enumerate}

\begin{quote}
\textbf{Table \#7.3.7: Pulse Rates of Woman Who Use Alcohol}
\end{quote}

\begin{longtable}[]{@{}llllll@{}}
\toprule
176 & 150 & 150 & 115 & 129 & 160\tabularnewline
\midrule
\endhead
120 & 125 & 89 & 132 & 120 & 120\tabularnewline
68 & 87 & 88 & 72 & 77 & 84\tabularnewline
92 & 80 & 60 & 67 & 59 & 64\tabularnewline
88 & 74 & 68 & & &\tabularnewline
\bottomrule
\end{longtable}

\begin{enumerate}
\def\labelenumi{\arabic{enumi}.}
\setcounter{enumi}{5}
\tightlist
\item
  The economic dynamism, which is the index of productive growth in
  dollars for countries that are designated by the World Bank as
  middle-income are in table \#7.3.8 ("SOCR data 2008," 2013).
  Countries that are considered high-income have a mean economic
  dynamism of 60.29. Do the data show that the mean economic dynamism
  of middle-income countries is less than the mean for high-income
  countries? Test at the 5\% level.
\end{enumerate}

\begin{quote}
\textbf{Table \#7.3.8: Economic Dynamism of Middle Income Countries}
\end{quote}

\begin{longtable}[]{@{}lllllll@{}}
\toprule
25.8057 & 37.4511 & 51.915 & 43.6952 & 47.8506 & 43.7178 & 58.0767\tabularnewline
\midrule
\endhead
41.1648 & 38.0793 & 37.7251 & 39.6553 & 42.0265 & 48.6159 & 43.8555\tabularnewline
49.1361 & 61.9281 & 41.9543 & 44.9346 & 46.0521 & 48.3652 & 43.6252\tabularnewline
50.9866 & 59.1724 & 39.6282 & 33.6074 & 21.6643 & &\tabularnewline
\bottomrule
\end{longtable}

\begin{enumerate}
\def\labelenumi{\arabic{enumi}.}
\setcounter{enumi}{6}
\tightlist
\item
  In 1999, the average percentage of women who received prenatal care
  per country is 80.1\%. Table \#7.3.9 contains the percentage of woman
  receiving prenatal care in 2009 for a sample of countries
  ("Pregnant woman receiving," 2013). Do the data show that the
  average percentage of women receiving prenatal care in 2009 is
  higher than in 1999? Test at the 5\% level.
\end{enumerate}

\begin{quote}
\textbf{Table \#7.3.9: Percentage of Woman Receiving Prenatal Care}
\end{quote}

\begin{longtable}[]{@{}llllll@{}}
\toprule
70.08 & 72.73 & 74.52 & 75.79 & 76.28 & 76.28\tabularnewline
\midrule
\endhead
76.65 & 80.34 & 80.60 & 81.90 & 86.30 & 87.70\tabularnewline
87.76 & 88.40 & 90.70 & 91.50 & 91.80 & 92.10\tabularnewline
92.20 & 92.41 & 92.47 & 93.00 & 93.20 & 93.40\tabularnewline
93.63 & 93.68 & 93.80 & 94.30 & 94.51 & 95.00\tabularnewline
95.80 & 95.80 & 96.23 & 96.24 & 97.30 & 97.90\tabularnewline
97.95 & 98.20 & 99.00 & 99.00 & 99.10 & 99.10\tabularnewline
100.00 & 100.00 & 100.00 & 100.00 & 100.00 &\tabularnewline
\bottomrule
\end{longtable}

\begin{enumerate}
\def\labelenumi{\arabic{enumi}.}
\setcounter{enumi}{7}
\tightlist
\item
  Maintaining your balance may get harder as you grow older. A study
  was conducted to see how steady the elderly is on their feet. They
  had the subjects stand on a force platform and have them react to a
  noise. The force platform then measured how much they swayed forward
  and backward, and the data is in table \#7.3.10 ("Maintaining
  balance while," 2013). Do the data show that the elderly sway more
  than the mean forward sway of younger people, which is 18.125 mm?
  Test at the 5\% level.
\end{enumerate}

\begin{quote}
\textbf{Table \#7.3.10: Forward/backward Sway (in mm) of Elderly Subjects}
\end{quote}

\begin{longtable}[]{@{}lllllllll@{}}
\toprule
\endhead
19 & 30 & 20 & 19 & 29 & 25 & 21 & 24 & 50\tabularnewline
\bottomrule
\end{longtable}

Data Sources:

Australian Human Rights Commission, (1996). \emph{Indigenous deaths in
custody 1989 - 1996}. Retrieved from website:
\url{http://www.humanrights.gov.au/publications/indigenous-deaths-custody}

\emph{CDC features - new data on autism spectrum disorders}. (2013, November
26). Retrieved from \url{http://www.cdc.gov/features/countingautism/}

Center for Disease Control and Prevention, Prevalence of Autism Spectrum
Disorders - Autism and Developmental Disabilities Monitoring Network.
(2008). \emph{Autism and developmental disabilities monitoring network-2012}.
Retrieved from website:
\url{http://www.cdc.gov/ncbddd/autism/documents/ADDM-2012-Community-Report.pdf}

\emph{CO2 emissions}. (2013, November 19). Retrieved from
\url{http://data.worldbank.org/indicator/EN.ATM.CO2E.PC}

Federal Trade Commission, (2008). \emph{Consumer fraud and identity theft
complaint data: January-December 2007}. Retrieved from website:
\url{http://www.ftc.gov/opa/2008/02/fraud.pdf}

\emph{Gallup news service}. (2013, November 7-10). Retrieved from
\url{http://www.gallup.com/file/poll/165896/JFK_Conspiracy_131115.pdf}

\emph{Healthy breakfast story}. (2013, November 16). Retrieved from
\url{http://lib.stat.cmu.edu/DASL/Stories/HealthyBreakfast.html}

\emph{IQ of famous people}. (2013, November 13). Retrieved from
\url{http://www.kidsiqtestcenter.com/IQ-famous-people.html}

\emph{Maintaining balance while concentrating}. (2013, September 25).
Retrieved from \url{http://www.statsci.org/data/general/balaconc.html}

\emph{Morgan Gallup poll on unemployment}. (2013, September 26). Retrieved
from \url{http://www.statsci.org/data/oz/gallup.html}

\emph{Multi-disciplinary niser activity - mercury in bass}. (2013, November
16). Retrieved from
\href{http://gozips.uakron.edu/~nmimoto/pages/datasets/MercuryInBass\%20-\%20description.txt}{http://gozips.uakron.edu/\textasciitilde{}nmimoto/pages/datasets/MercuryInBass -
description.txt}

\emph{Pregnant woman receiving prenatal care}. (2013, October 14). Retrieved
from \url{http://data.worldbank.org/indicator/SH.STA.ANVC.ZS}

\emph{Pulse rates before and after exercise}. (2013, September 25). Retrieved
from \url{http://www.statsci.org/data/oz/ms212.html}

\emph{SOCR data 2008 world countries rankings}. (2013, November 16).
Retrieved from
\url{http://wiki.stat.ucla.edu/socr/index.php/SOCR_Data_2008_World_CountriesRankings}

\emph{Student t-distribution}. (2013, November 25). Retrieved from
\url{http://lib.stat.cmu.edu/DASL/Stories/student.html}

\emph{WHO life expectancy}. (2013, September 19). Retrieved from
\url{http://www.who.int/gho/mortality_burden_disease/life_tables/situation_trends/en/index.html}

\hypertarget{estimation}{%
\chapter{Estimation}\label{estimation}}

In hypothesis tests, the purpose was to make a decision about a parameter, in terms of it being greater than, less than, or not equal to a value. But what if you want to actually know what the parameter is. You need to do estimation. There are two types of estimation -- point estimator and confidence interval.

\hypertarget{basics-of-confidence-intervals}{%
\section{Basics of Confidence Intervals}\label{basics-of-confidence-intervals}}

A point estimator is just the statistic that you have calculated previously. As an example, when you wanted to estimate the population mean, {[}MISSING{]} , the point estimator is the sample mean, {[}MISSING{]}. To estimate the population proportion, \emph{p}, you use the sample proportion, {[}MISSING{]}. In general, if you want to estimate any population parameter, we will call it {[}MISSING{]}, you use the sample statistic, {[}MISSING{]}.

Point estimators are really easy to find, but they have some drawbacks. First, if you have a large sample size, then the estimate is better. But with a point estimator, you don't know what the sample size is. Also, you don't know how accurate the estimate is. Both of these problems are solved with a confidence interval.

\textbf{Confidence interval:} This is where you have an interval surrounding your parameter, and the interval has a chance of being a true statement. In general, a confidence interval looks like: , where is the point estimator and \emph{E} is the margin of error term that is added and subtracted from the point estimator. Thus making an interval.

\textbf{Interpreting a confidence interval: }

The statistical interpretation is that the confidence interval has a probability (, where is the complement of the confidence level) of containing the population parameter. As an example, if you have a 95\% confidence interval of 0.65 \textless{} \emph{p} \textless{} 0.73, then you would say, ``there is a 95\% chance that the interval 0.65 to 0.73 contains the true population proportion.'' This means that if you have 100 intervals, 95 of them will contain the true proportion, and 5\% will not. The wrong interpretation is that there is a 95\% chance that the true value of \emph{p} will fall between 0.65 and 0.73. The reason that this interpretation is wrong is that the true value is fixed out there somewhere. You are trying to capture it with this interval. So this is the chance is that your interval captures it, and not that the true value falls in the interval.

There is also a real world interpretation that depends on the situation. It is where you are telling people what numbers you found the parameter to lie between. So your real world is where you tell what values your parameter is between. There is no probability attached to this statement. That probability is in the statistical interpretation.

The common probabilities used for confidence intervals are 90\%, 95\%, and 99\%. These are known as the confidence level. The confidence level and the alpha level are related. For a two-tailed test, the confidence level is . This is because the is both tails and the confidence level is area between the two tails. As an example, for a two-tailed test (H\textsubscript{A} is not equal to) with equal to 0.10, the confidence level would be 0.90 or 90\%. If you have a one-tailed test, then your is only one tail. Because of symmetry the other tail is also . So you have 2 with both tails. So the confidence level, which is the area between the two tails, is {[}MISSING{]}.

\textbf{Example \#8.1.1: Stating the Statistical and Real World
Interpretations for a Confidence Interval}

a. Suppose you have a 95\% confidence interval for the mean age a woman
gets married in 2013 is . State the statistical and real world
interpretations of this statement.

\textbf{Solution:}

\begin{quote}
Statistical Interpretation: There is a 95\% chance that the interval
contains the mean age a woman gets married in 2013.

Real World Interpretation: The mean age that a woman married in 2013
is between 26 and 28 years of age.
\end{quote}

b. Suppose a 99\% confidence interval for the proportion of Americans who
have tried marijuana as of 2013 is . State the statistical and real
world interpretations of this statement.

\textbf{Solution:}

\begin{quote}
Statistical Interpretation: There is a 99\% chance that the interval
contains the proportion of Americans who have tried marijuana as of
2013.

Real World Interpretation: The proportion of Americans who have tried
marijuana as of 2013 is between 0.35 and 0.41.
\end{quote}

One last thing to know about confidence is how the sample size and
confidence level affect how wide the interval is. The following
discussion demonstrates what happens to the width of the interval as you
get more confident.

Think about shooting an arrow into the target. Suppose you are really
good at that and that you have a 90\% chance of hitting the bull's eye.
Now the bull's eye is very small. Since you hit the bull's eye
approximately 90\% of the time, then you probably hit inside the next
ring out 95\% of the time. You have a better chance of doing this, but
the circle is bigger. You probably have a 99\% chance of hitting the
target, but that is a much bigger circle to hit. You can see, as your
confidence in hitting the target increases, the circle you hit gets
bigger. The same is true for confidence intervals. This is demonstrated
in figure \#8.1.1.

\textbf{Figure \#8.1.1: Affect of Confidence Level on Width}

\begin{quote}
\includegraphics[width=3.5in,height=1.30556in]{media/image21.png}
\end{quote}

The higher level of confidence makes a wider interval. There's a trade
off between width and confidence level. You can be really confident
about your answer but your answer will not be very precise. Or you can
have a precise answer (small margin of error) but not be very confident
about your answer.

Now look at how the sample size affects the size of the interval.
Suppose figure \#8.1.2 represents confidence intervals calculated on a
95\% interval. A larger sample size from a representative sample makes
the width of the interval narrower. This makes sense. Large samples are
closer to the true population so the point estimate is pretty close to
the true value.

\textbf{Figure \#8.1.2: Affect of Sample Size on Width}

\begin{quote}
\includegraphics[width=3.29167in,height=1.30556in]{media/image22.png}
\end{quote}

Now you know everything you need to know about confidence intervals
except for the actual formula. The formula depends on which parameter
you are trying to estimate. With different situations you will be given
the confidence interval for that parameter.

\hypertarget{homework-23}{%
\subsection{Homework}\label{homework-23}}

\begin{enumerate}
\def\labelenumi{\arabic{enumi}.}
\item
  Suppose you compute a confidence interval with a sample size of 25. What will happen to the confidence interval if the sample size increases to 50?
\item
  Suppose you compute a 95\% confidence interval. What will happen to the confidence interval if you increase the confidence level to 99\%?
\item
  Suppose you compute a 95\% confidence interval. What will happen to the confidence interval if you decrease the confidence level to 90\%?
\item
  Suppose you compute a confidence interval with a sample size of 100. What will happen to the confidence interval if the sample size decreases to 80?
\item
  A 95\% confidence interval is , where is the mean diameter of the Earth. State the statistical interpretation.
\item
  A 95\% confidence interval is , where is the mean diameter of the Earth. State the real world interpretation.
\item
  In 2013, Gallup conducted a poll and found a 95\% confidence interval of, where \emph{p} is the proportion of Americans who believe it is the government's responsibility for health care. Give the real world interpretation.
\item
  In 2013, Gallup conducted a poll and found a 95\% confidence interval of, where \emph{p} is the proportion of Americans who believe it is the government's responsibility for health care. Give the statistical interpretation.
\end{enumerate}

\textbf{~}

\hypertarget{one-sample-interval-for-the-proportion}{%
\section{One-Sample Interval for the Proportion}\label{one-sample-interval-for-the-proportion}}

Suppose you want to estimate the population proportion, \emph{p}. As an example you may be curious what proportion of students at your school smoke. Or you could wonder what is the proportion of accidents caused by teenage drivers who do not have a drivers' education class.
\textbf{Confidence Interval for One Population Proportion (1-Prop Interval)}

\begin{enumerate}
\def\labelenumi{\arabic{enumi}.}
\tightlist
\item
  State the random variable and the parameter in words.
\end{enumerate}

\begin{quote}
\emph{x} = number of successes

\emph{p} = proportion of successes
\end{quote}

\begin{enumerate}
\def\labelenumi{\arabic{enumi}.}
\setcounter{enumi}{1}
\tightlist
\item
  State and check the assumptions for confidence interval
\end{enumerate}

\begin{enumerate}
\def\labelenumi{\alph{enumi}.}
\item
  A simple random sample of size \emph{n} is taken.
\item
  The condition for the binomial distribution are satisfied
\item
  To determine the sampling distribution of , you need to show that
  and , where . If this requirement is true, then the sampling
  distribution of is well approximated by a normal curve. (In reality
  this is not really true, since the correct assumption deals with
  \emph{p}. However, in a confidence interval you do not know \emph{p}, so you
  must use . This means you just need to show that and .)
\end{enumerate}

\begin{enumerate}
\def\labelenumi{\arabic{enumi}.}
\setcounter{enumi}{2}
\tightlist
\item
  Find the sample statistic and the confidence interval
\end{enumerate}

\begin{quote}
Sample Proportion:

Confidence Interval:

Where

\emph{p} = population proportion

= sample proportion

\emph{n} = number of sample values

\emph{E} = margin of error

= critical value
\end{quote}

\begin{enumerate}
\def\labelenumi{\arabic{enumi}.}
\setcounter{enumi}{3}
\item
  Statistical Interpretation: In general this looks like, ``there is a
  C\% chance that contains the true proportion.''
\item
  Real World Interpretation: This is where you state what interval
  contains the true proportion.
\end{enumerate}

The critical value is a value from the normal distribution. Since a
confidence interval is found by adding and subtracting a margin of error
amount from the sample proportion, and the interval has a probability of
containing the true proportion, then you can think of this as the
statement . You can use the invNorm command on the TI-83/84 calculator
or qnorm command on R to find the critical value. The critical values
will always be the same value, so it is easier to just look at table A.1
in the appendix.

\textbf{Example \#8.2.1: Confidence Interval for the Population Proportion
Using the Formula}

\begin{quote}
A concern was raised in Australia that the percentage of deaths of
Aboriginal prisoners was higher than the percent of deaths of
non-Aboriginal prisoners, which is 0.27\%. A sample of six years
(1990-1995) of data was collected, and it was found that out of 14,495
Aboriginal prisoners, 51 died ("Indigenous deaths in," 1996). Find a
95\% confidence interval for the proportion of Aboriginal prisoners who
died.

\textbf{Solution:}
\end{quote}

\begin{enumerate}
\def\labelenumi{\arabic{enumi}.}
\tightlist
\item
  State the random variable and the parameter in words.
\end{enumerate}

\begin{quote}
\emph{x} = number of Aboriginal prisoners who die

\emph{p} = proportion of Aboriginal prisoners who die
\end{quote}

\begin{enumerate}
\def\labelenumi{\arabic{enumi}.}
\setcounter{enumi}{1}
\tightlist
\item
  State and check the assumptions for a confidence interval
\end{enumerate}

\begin{enumerate}
\def\labelenumi{\alph{enumi}.}
\item
  A simple random sample of 14,495 Aboriginal prisoners was taken.
  However, the sample was not a random sample, since it was data from
  six years. It is the numbers for all prisoners in these six years,
  but the six years were not picked at random. Unless there was
  something special about the six years that were chosen, the sample
  is probably a representative sample. This assumption is probably
  met.
\item
  There are 14,495 prisoners in this case. The prisoners are all
  Aboriginals, so you are not mixing Aboriginal with non-Aboriginal
  prisoners. There are only two outcomes, either the prisoner dies or
  doesn't. The chance that one prisoner dies over another may not be
  constant, but if you consider all prisoners the same, then it may be
  close to the same probability. Thus the assumptions for the binomial
  distribution are satisfied
\item
  In this case, and and both are greater than or equal to 5. The
  sampling distribution for is a normal distribution.
\end{enumerate}

\begin{enumerate}
\def\labelenumi{\arabic{enumi}.}
\setcounter{enumi}{2}
\tightlist
\item
  Find the sample statistic and the confidence interval
\end{enumerate}

\begin{quote}
Sample Proportion:

Confidence Interval:

, since 95\% confidence level
\end{quote}

\begin{enumerate}
\def\labelenumi{\arabic{enumi}.}
\setcounter{enumi}{3}
\item
  Statistical Interpretation: There is a 95\% chance that contains the
  proportion of Aboriginal prisoners who died.
\item
  Real World Interpretation: The proportion of Aboriginal prisoners
  who died is between 0.0026 and 0.0045.
\end{enumerate}

You can also do the calculations for the confidence interval with
technology. The following example shows the process on the TI-83/84.

\textbf{Example \#8.2.2: Confidence Interval for the Population Proportion
Using Technology}

\begin{quote}
A researcher studying the effects of income levels on breastfeeding of
infants hypothesizes that countries where the income level is lower
have a higher rate of infant breastfeeding than higher income
countries. It is known that in Germany, considered a high-income
country by the World Bank, 22\% of all babies are breastfeed. In
Tajikistan, considered a low-income country by the World Bank,
researchers found that in a random sample of 500 new mothers that 125
were breastfeeding their infants. Find a 90\% confidence interval of
the proportion of mothers in low-income countries who breastfeed their
infants?

\textbf{Solution:}
\end{quote}

\begin{enumerate}
\def\labelenumi{\arabic{enumi}.}
\tightlist
\item
  State you random variable and the parameter in words.
\end{enumerate}

\begin{quote}
\emph{x} = number of woman who breastfeed in a low-income country

\emph{p} = proportion of woman who breastfeed in a low-income country
\end{quote}

\begin{enumerate}
\def\labelenumi{\arabic{enumi}.}
\setcounter{enumi}{1}
\tightlist
\item
  State and check the assumptions for a confidence interval
\end{enumerate}

\begin{enumerate}
\def\labelenumi{\alph{enumi}.}
\item
  A simple random sample of 500 breastfeeding habits of woman in a low-income country was taken as was stated in the problem.
\item
  There were 500 women in the study. The women are considered identical, though they probably have some differences. There are only two outcomes, either the woman breastfeeds or she doesn't. The probability of a woman breastfeeding is probably not the same for each woman, but it is probably not very different for each woman. The assumptions for the binomial distribution are satisfied
\item
  and and both are greater than or equal to 5, so the sampling distribution of is well approximated by a normal curve.
\end{enumerate}

\begin{enumerate}
\def\labelenumi{\arabic{enumi}.}
\setcounter{enumi}{2}
\tightlist
\item
  Find the sample statistic and the confidence interval
\end{enumerate}

\begin{quote}
On the TI-83/84: Go into the STAT menu. Move over to TESTS and choose
1-PropZInt.

\textbf{Figure \#8.2.1: Setup for 1-Proportion Interval}

\includegraphics[width=2.75in,height=1.86111in]{media/image57.png}

Once you press Calculate, you will see the results as in figure
\#8.2.2.

\textbf{Figure \#8.2.2: Results for 1-Proportion Interval}

\includegraphics[width=2.75in,height=1.86111in]{media/image58.png}

On R: the command is prop.test(x, n, conf.level = C), where C is given
in decimal form. So for this example, the command is

prop.test(125, 500, conf.level = 0.90)

1-sample proportions test with continuity correction

data: 125 out of 500, null probability 0.5

X-squared = 124, df = 1, p-value \textless{} 2.2e-16

alternative hypothesis: true p is not equal to 0.5

90 percent confidence interval:

0.2185980 0.2841772

sample estimates:

p

0.25

Again, R does a continuity correction, so the answer is slightly off
from the formula and the TI-83/84 calculator.
\end{quote}

\begin{enumerate}
\def\labelenumi{\arabic{enumi}.}
\setcounter{enumi}{3}
\item
  Statistical Interpretation: There is a 90\% chance that contains the proportion of women in low-income countries who breastfeed their infants.
\item
  Real World Interpretation: The proportion of women in low-income countries who breastfeed their infants is between 0.219 and 0.284.
\end{enumerate}

\hypertarget{homework-24}{%
\subsection{Homework}\label{homework-24}}

In each problem show all steps of the confidence interval. If some of the assumptions are not met, note that the results of the interval may not be correct and then continue the process of the confidence interval.

\begin{enumerate}
\def\labelenumi{\arabic{enumi}.}
\item
  Eyeglassomatic manufactures eyeglasses for different retailers. They test to see how many defective lenses they make. Looking at the type of defects, they found in a three-month time period that out of 34,641 defective lenses, 5865 were due to scratches. Find a 99\% confidence interval for the proportion of defects that are from scratches.
\item
  In November of 1997, Australians were asked if they thought unemployment would increase. At that time 284 out of 631 said that they thought unemployment would increase ("Morgan gallup poll," 2013). Estimate the proportion of Australians in November 1997 who believed unemployment would increase using a 95\% confidence interval?
\item
  According to the February 2008 Federal Trade Commission report on consumer fraud and identity theft, Arkansas had 1,601 complaints of identity theft out of 3,482 consumer complaints ("Consumer fraud and," 2008). Calculate a 90\% confidence interval for the proportion of identity theft in Arkansas.
\item
  According to the February 2008 Federal Trade Commission report on consumer fraud and identity theft, Alaska had 321 complaints of identity theft out of 1,432 consumer complaints ("Consumer fraud and," 2008). Calculate a 90\% confidence interval for the proportion of identity theft in Alaska.
\item
  In 2013, the Gallup poll asked 1,039 American adults if they believe there was a conspiracy in the assassination of President Kennedy, and found that 634 believe there was a conspiracy ("Gallup news service," 2013). Estimate the proportion of American's who believe in this conspiracy using a 98\% confidence interval.
\item
  In 2008, there were 507 children in Arizona out of 32,601 who were diagnosed with Autism Spectrum Disorder (ASD) ("Autism and developmental," 2008). Find the proportion of ASD in Arizona with a confidence level of 99\%.
\end{enumerate}

\hypertarget{one-sample-interval-for-the-mean}{%
\section{One-Sample Interval for the Mean}\label{one-sample-interval-for-the-mean}}

Suppose you want to estimate the mean height of Americans, or you want to estimate the mean salary of college graduates. A confidence interval for the mean would be the way to estimate these means.

\textbf{Confidence Interval for One Population Mean (t-Interval)}

\begin{enumerate}
\def\labelenumi{\arabic{enumi}.}
\tightlist
\item
  State the random variable and the parameter in words.
\end{enumerate}

\begin{quote}
x = random variable

= mean of random variable
\end{quote}

\begin{enumerate}
\def\labelenumi{\arabic{enumi}.}
\setcounter{enumi}{1}
\tightlist
\item
  State and check the assumptions for a hypothesis test
\end{enumerate}

\begin{enumerate}
\def\labelenumi{\alph{enumi}.}
\item
  A random sample of size \emph{n} is taken.
\item
  The population of the random variable is normally distributed, though the t-test is fairly robust to the assumption if the sample size is large. This means that if this assumption isn't met, but your sample size is quite large (over 30), then the results of the t-test are valid.
\end{enumerate}

\begin{enumerate}
\def\labelenumi{\arabic{enumi}.}
\setcounter{enumi}{2}
\tightlist
\item
  Find the sample statistic and confidence interval
\end{enumerate}

\begin{quote}
where

is the point estimator for

is the critical value where degrees of freedom:

\emph{s} is the sample standard deviation

\emph{n} is the sample size
\end{quote}

\begin{enumerate}
\def\labelenumi{\arabic{enumi}.}
\setcounter{enumi}{3}
\item
  Statistical Interpretation: In general this looks like, ``there is a C\% chance that the statement contains the true mean.''
\item
  Real World Interpretation: This is where you state what interval contains the true mean.
  The critical value is a value from the Student's t-distribution. Since a
  confidence interval is found by adding and subtracting a margin of error
  amount from the sample mean, and the interval has a probability of
  containing the true mean, then you can think of this as the statement .
  The critical values are found in table A.2 in the appendix
\end{enumerate}

\textbf{How to check the assumptions of confidence interval:}

In order for the confidence interval to be valid, the assumptions of the
test must be true. Whenever you run a confidence interval, you must make
sure the assumptions are true. You need to check them. Here is how you
do this:

\begin{enumerate}
\def\labelenumi{\arabic{enumi}.}
\item
  For the assumption that the sample is a random sample, describe how
  you took the sample. Make sure your sampling technique is random.
\item
  For the assumption that population is normal, remember the process
  of assessing normality from chapter 6.
\end{enumerate}

\textbf{Example \#8.3.1: Confidence Interval for the Population Mean Using the
Formula}

\begin{quote}
A random sample of 20 IQ scores of famous people was taken information
from the website of IQ of Famous People ("IQ of famous," 2013) and
then using a random number generator to pick 20 of them. The data are
in table \#8.3.1 (this is the same data set that was used in example
\#6.4.2). Find a 98\% confidence interval for the IQ of a famous
person.

\textbf{Table \#8.3.1: IQ Scores of Famous People}
\end{quote}

\begin{longtable}[]{@{}lllll@{}}
\toprule
158 & 180 & 150 & 137 & 109\tabularnewline
\midrule
\endhead
225 & 122 & 138 & 145 & 180\tabularnewline
118 & 118 & 126 & 140 & 165\tabularnewline
150 & 170 & 105 & 154 & 118\tabularnewline
\bottomrule
\end{longtable}

\begin{quote}
\textbf{Solution:}
\end{quote}

\begin{enumerate}
\def\labelenumi{\arabic{enumi}.}
\tightlist
\item
  State the random variable and the parameter in words.
\end{enumerate}

\begin{quote}
\emph{x} = IQ score of a famous person

= mean IQ score of a famous person
\end{quote}

\begin{enumerate}
\def\labelenumi{\arabic{enumi}.}
\setcounter{enumi}{1}
\tightlist
\item
  State and check the assumptions for a confidence interval
\end{enumerate}

\begin{enumerate}
\def\labelenumi{\alph{enumi}.}
\item
  A random sample of 20 IQ scores was taken. This was stated in the
  problem.
\item
  The population of IQ score is normally distributed. This was shown
  in example \#6.4.2.
\end{enumerate}

\begin{enumerate}
\def\labelenumi{\arabic{enumi}.}
\setcounter{enumi}{2}
\tightlist
\item
  Find the sample statistic and confidence interval
\end{enumerate}

\begin{quote}
Sample Statistic:

Now you need the degrees of freedom, and the \emph{C,} which is 98\%. Now go
to table A.2, go down the first column to 19 degrees of freedom. Then
go over to the column headed with 98\%. Thus . (See table 8.3.2.)

\textbf{Table \#8.3.2: Excerpt From Table A.2}
\end{quote}

\begin{longtable}[]{@{}llllll@{}}
\toprule
\begin{minipage}[b]{0.32\columnwidth}\raggedright
Degrees of Freedom (\emph{df})\strut
\end{minipage} & \begin{minipage}[b]{0.09\columnwidth}\raggedright
80\%\strut
\end{minipage} & \begin{minipage}[b]{0.09\columnwidth}\raggedright
90\%\strut
\end{minipage} & \begin{minipage}[b]{0.10\columnwidth}\raggedright
95\%\strut
\end{minipage} & \begin{minipage}[b]{0.14\columnwidth}\raggedright
98\%\strut
\end{minipage} & \begin{minipage}[b]{0.10\columnwidth}\raggedright
99\%\strut
\end{minipage}\tabularnewline
\midrule
\endhead
\begin{minipage}[t]{0.32\columnwidth}\raggedright
1\strut
\end{minipage} & \begin{minipage}[t]{0.09\columnwidth}\raggedright
3.078\strut
\end{minipage} & \begin{minipage}[t]{0.09\columnwidth}\raggedright
6.314\strut
\end{minipage} & \begin{minipage}[t]{0.10\columnwidth}\raggedright
12.706\strut
\end{minipage} & \begin{minipage}[t]{0.14\columnwidth}\raggedright
31.821\strut
\end{minipage} & \begin{minipage}[t]{0.10\columnwidth}\raggedright
63.657\strut
\end{minipage}\tabularnewline
\begin{minipage}[t]{0.32\columnwidth}\raggedright
2\strut
\end{minipage} & \begin{minipage}[t]{0.09\columnwidth}\raggedright
1.886\strut
\end{minipage} & \begin{minipage}[t]{0.09\columnwidth}\raggedright
2.920\strut
\end{minipage} & \begin{minipage}[t]{0.10\columnwidth}\raggedright
4.303\strut
\end{minipage} & \begin{minipage}[t]{0.14\columnwidth}\raggedright
6.965\strut
\end{minipage} & \begin{minipage}[t]{0.10\columnwidth}\raggedright
9.925\strut
\end{minipage}\tabularnewline
\begin{minipage}[t]{0.32\columnwidth}\raggedright
3\strut
\end{minipage} & \begin{minipage}[t]{0.09\columnwidth}\raggedright
1.638\strut
\end{minipage} & \begin{minipage}[t]{0.09\columnwidth}\raggedright
2.353\strut
\end{minipage} & \begin{minipage}[t]{0.10\columnwidth}\raggedright
3.182\strut
\end{minipage} & \begin{minipage}[t]{0.14\columnwidth}\raggedright
4.541\strut
\end{minipage} & \begin{minipage}[t]{0.10\columnwidth}\raggedright
5.841\strut
\end{minipage}\tabularnewline
\begin{minipage}[t]{0.32\columnwidth}\raggedright
.

.

.\strut
\end{minipage} & \begin{minipage}[t]{0.09\columnwidth}\raggedright
.

.

.\strut
\end{minipage} & \begin{minipage}[t]{0.09\columnwidth}\raggedright
.

.

.\strut
\end{minipage} & \begin{minipage}[t]{0.10\columnwidth}\raggedright
.

.

.\strut
\end{minipage} & \begin{minipage}[t]{0.14\columnwidth}\raggedright
.

.

.\strut
\end{minipage} & \begin{minipage}[t]{0.10\columnwidth}\raggedright
.

.

.\strut
\end{minipage}\tabularnewline
\begin{minipage}[t]{0.32\columnwidth}\raggedright
19\strut
\end{minipage} & \begin{minipage}[t]{0.09\columnwidth}\raggedright
1.328\strut
\end{minipage} & \begin{minipage}[t]{0.09\columnwidth}\raggedright
1.729\strut
\end{minipage} & \begin{minipage}[t]{0.10\columnwidth}\raggedright
2.093\strut
\end{minipage} & \begin{minipage}[t]{0.14\columnwidth}\raggedright
\textbf{2.539}\strut
\end{minipage} & \begin{minipage}[t]{0.10\columnwidth}\raggedright
2.861\strut
\end{minipage}\tabularnewline
\bottomrule
\end{longtable}

\begin{enumerate}
\def\labelenumi{\arabic{enumi}.}
\setcounter{enumi}{3}
\item
  Statistical Interpretation: There is a 98\% chance that contains the
  mean IQ score of a famous person.
\item
  Real World Interpretation: The mean IQ score of a famous person is
  between 128.8 and 162.
\end{enumerate}

\textbf{Example \#8.3.2: Confidence Interval for the Population Mean Using
Technology}

\begin{quote}
The data in table \#8.3.3 are the life expectancies for men in
European countries in 2011 ("WHO life expectancy," 2013). Find the
99\% confident interval for the mean life expectancy of men in Europe

\textbf{Table \#8.3.3: Life Expectancies for Men in European Countries in
2011}
\end{quote}

\begin{longtable}[]{@{}llllllll@{}}
\toprule
73 & 79 & 67 & 78 & 69 & 66 & 78 & 74\tabularnewline
\midrule
\endhead
71 & 74 & 79 & 75 & 77 & 71 & 78 & 78\tabularnewline
68 & 78 & 78 & 71 & 81 & 79 & 80 & 80\tabularnewline
62 & 65 & 69 & 68 & 79 & 79 & 79 & 73\tabularnewline
79 & 79 & 72 & 77 & 67 & 70 & 63 & 82\tabularnewline
72 & 72 & 77 & 79 & 80 & 80 & 67 & 73\tabularnewline
73 & 60 & 65 & 79 & 66 & & &\tabularnewline
\bottomrule
\end{longtable}

\textbf{Solution:}

\begin{enumerate}
\def\labelenumi{\arabic{enumi}.}
\tightlist
\item
  State the random variable and the parameter in words.
\end{enumerate}

\begin{quote}
\emph{x} = life expectancy for a European man in 2011

= mean life expectancy for European men in 2011
\end{quote}

\begin{enumerate}
\def\labelenumi{\arabic{enumi}.}
\setcounter{enumi}{1}
\tightlist
\item
  State and check the assumptions for a confidence interval
\end{enumerate}

\begin{enumerate}
\def\labelenumi{\alph{enumi}.}
\item
  A random sample of 53 life expectancies of European men in 2011 was
  taken. The data is actually all of the life expectancies for every
  country that is considered part of Europe by the World Health
  Organization. However, the information is still sample information
  since it is only for one year that the data was collected. It may
  not be a random sample, but that is probably not an issue in this
  case.
\item
  The distribution of life expectancies of European men in 2011 is
  normally distributed. To see if this assumption has been met, look
  at the histogram, number of outliers, and the normal probability
  plot. (If you wish, you can look at the normal probability plot
  first. If it doesn't look linear, then you may want to look at the
  histogram and number of outliers at this point.)

  \textbf{Figure \#8.3.1: Histogram for Life Expectancies of European Men in
  2011}

  \includegraphics[width=2.30556in,height=2.30556in]{media/image78.emf}

  Not normally distributed
\end{enumerate}

\begin{quote}
\textbf{Number of outliers:}

\textbf{Figure \#8.3.2: Modified Box Plot for Life Expectancies of European
Men in 2011}

\includegraphics[width=2.44444in,height=2.44444in]{media/image79.emf}

Outliers are numbers below 54 and above 94. There are no outliers for
this data set.

\textbf{Figure \#8.3.3: Normal Quantile Plot for Life Expectancies of
European Men in 2011}

\includegraphics[width=2.27778in,height=2.27778in]{media/image81.emf}

Not linear
\end{quote}

This population does not appear to be normally distributed. The t-test
is robust for sample sizes larger than 30 so you can go ahead and
calculate the interval.

\begin{enumerate}
\def\labelenumi{\arabic{enumi}.}
\setcounter{enumi}{2}
\tightlist
\item
  Find the sample statistic and confidence interval
\end{enumerate}

\begin{quote}
On the TI-83/84: Go into the STAT menu, and type the data into L1.
Then go into STAT and over to TESTS. Choose TInterval.

\textbf{Figure \#8.3.4: Setup for TInterval}

\includegraphics[width=2.75in,height=1.86111in]{media/image82.png}

\textbf{Figure \#8.3.5: Results for TInterval}

\includegraphics[width=2.75in,height=1.86111in]{media/image83.png}

On R: t.test(variable, conf.level = C), where C is given in decimal
form. So for this example it would be t.test(expectancy, conf.level =
0.99)

One Sample t-test

data: expectancy

t = 93.711, df = 52, p-value \textless{} 2.2e-16

alternative hypothesis: true mean is not equal to 0

99 percent confidence interval:

71.63204 75.83966

sample estimates:

mean of x

73.73585
\end{quote}

\begin{enumerate}
\def\labelenumi{\arabic{enumi}.}
\setcounter{enumi}{3}
\item
  Statistical Interpretation: There is a 99\% chance that contains the
  mean life expectancy of European men.
\item
  Real World Interpretation: The mean life expectancy of European men
  is between 71.6 and 75.8 years.
\end{enumerate}

\hypertarget{homework-25}{%
\subsection{Homework}\label{homework-25}}

In each problem show all steps of the confidence interval. If some of the assumptions are not met, note that the results of the interval may not be correct and then continue the process of the confidence interval.

\begin{enumerate}
\def\labelenumi{\arabic{enumi}.}
\tightlist
\item
  The Kyoto Protocol was signed in 1997, and required countries to start reducing their carbon emissions. The protocol became enforceable in February 2005. Table 8.3.4 contains a random sample of CO\textsubscript{2} emissions in 2010 ("CO2 emissions," 2013). Compute a 99\% confidence interval to estimate the mean CO\textsubscript{2} emission in 2010.
\end{enumerate}

\textbf{Table \#8.3.4: CO\textsubscript{2} Emissions (metric tons per capita) in 2010}

\begin{longtable}[]{@{}lllllll@{}}
\toprule
1.36 & 1.42 & 5.93 & 5.36 & 0.06 & 9.11 & 7.32\tabularnewline
\midrule
\endhead
7.93 & 6.72 & 0.78 & 1.80 & 0.20 & 2.27 & 0.28\tabularnewline
5.86 & 3.46 & 1.46 & 0.14 & 2.62 & 0.79 & 7.48\tabularnewline
0.86 & 7.84 & 2.87 & 2.45 & & &\tabularnewline
\bottomrule
\end{longtable}

\begin{enumerate}
\def\labelenumi{\arabic{enumi}.}
\setcounter{enumi}{1}
\tightlist
\item
  Many people feel that cereal is healthier alternative for children over glazed donuts. Table \#8.3.5 contains the amount of sugar in a sample of cereal that is geared towards children ("Healthy breakfast story," 2013). Estimate the mean amount of sugar in children cereal using a 95\% confidence level.
\end{enumerate}

\begin{quote}
\textbf{Table \#8.3.5: Sugar Amounts (g) in Children's Cereal}
\end{quote}

\begin{longtable}[]{@{}lllllll@{}}
\toprule
10 & 14 & 12 & 9 & 13 & 13 & 13\tabularnewline
\midrule
\endhead
11 & 12 & 15 & 9 & 10 & 11 & 3\tabularnewline
6 & 12 & 15 & 12 & 12 & &\tabularnewline
\bottomrule
\end{longtable}

\begin{enumerate}
\def\labelenumi{\arabic{enumi}.}
\setcounter{enumi}{2}
\tightlist
\item
  In Florida, bass fish were collected in 53 different lakes to measure the amount of mercury in the fish. The data for the average amount of mercury in each lake is in table \#8.3.6 ("Multi-disciplinary niser activity," 2013). Compute a 90\% confidence interval for the mean amount of mercury in fish in Florida lakes.
\end{enumerate}

\begin{quote}
\textbf{Table \#8.3.6: Average Mercury Levels (mg/kg) in Fish}
\end{quote}

\begin{longtable}[]{@{}llllll@{}}
\toprule
1.23 & 1.33 & 0.04 & 0.44 & 1.20 & 0.27\tabularnewline
\midrule
\endhead
0.48 & 0.19 & 0.83 & 0.81 & 0.71 & 0.5\tabularnewline
0.49 & 1.16 & 0.05 & 0.15 & 0.19 & 0.77\tabularnewline
1.08 & 0.98 & 0.63 & 0.56 & 0.41 & 0.73\tabularnewline
0.34 & 0.59 & 0.34 & 0.84 & 0.50 & 0.34\tabularnewline
0.28 & 0.34 & 0.87 & 0.56 & 0.17 & 0.18\tabularnewline
0.19 & 0.04 & 0.49 & 1.10 & 0.16 & 0.10\tabularnewline
0.48 & 0.21 & 0.86 & 0.52 & 0.65 & 0.27\tabularnewline
0.94 & 0.40 & 0.43 & 0.25 & 0.27 &\tabularnewline
\bottomrule
\end{longtable}

\begin{enumerate}
\def\labelenumi{\arabic{enumi}.}
\setcounter{enumi}{3}
\tightlist
\item
  In 1882, Albert Michelson collected measurements on the speed of light ("Student t-distribution," 2013). His measurements are given in table \#8.3.7. Find the speed of light value that Michelson estimated from his data using a 95\% confidence interval.
\end{enumerate}

\begin{quote}
\textbf{Table \#8.3.7: Speed of Light Measurements (km/sec)}
\end{quote}

\begin{longtable}[]{@{}lllll@{}}
\toprule
299883 & 299816 & 299778 & 299796 & 299682\tabularnewline
\midrule
\endhead
299711 & 299611 & 299599 & 300051 & 299781\tabularnewline
299578 & 299796 & 299774 & 299820 & 299772\tabularnewline
299696 & 299573 & 299748 & 299748 & 299797\tabularnewline
299851 & 299809 & 299723 & &\tabularnewline
\bottomrule
\end{longtable}

\begin{enumerate}
\def\labelenumi{\arabic{enumi}.}
\setcounter{enumi}{4}
\tightlist
\item
  Table \#8.3.8 contains pulse rates after running for 1 minute, collected from females who drink alcohol ("Pulse rates before," 2013). Find a 95\% confidence interval for the mean pulse rate after exercise of women who do drink alcohol.
\end{enumerate}

\begin{quote}
\textbf{Table \#8.3.8: Pulse Rates (beats per minute) of Woman Who Use
Alcohol}
\end{quote}

\begin{longtable}[]{@{}llllll@{}}
\toprule
176 & 150 & 150 & 115 & 129 & 160\tabularnewline
\midrule
\endhead
120 & 125 & 89 & 132 & 120 & 120\tabularnewline
68 & 87 & 88 & 72 & 77 & 84\tabularnewline
92 & 80 & 60 & 67 & 59 & 64\tabularnewline
88 & 74 & 68 & & &\tabularnewline
\bottomrule
\end{longtable}

\begin{enumerate}
\def\labelenumi{\arabic{enumi}.}
\setcounter{enumi}{5}
\tightlist
\item
  The economic dynamism, which is the index of productive growth in dollars for countries that are designated by the World Bank as middle-income are in table \#8.3.9 ("SOCR data 2008," 2013). Compute a 95\% confidence interval for the mean economic dynamism of middle-income countries.
\end{enumerate}

\begin{quote}
\textbf{Table \#8.3.9: Economic Dynamism (\$) of Middle Income Countries}
\end{quote}

\begin{longtable}[]{@{}lllllll@{}}
\toprule
25.8057 & 37.4511 & 51.915 & 43.6952 & 47.8506 & 43.7178 & 58.0767\tabularnewline
\midrule
\endhead
41.1648 & 38.0793 & 37.7251 & 39.6553 & 42.0265 & 48.6159 & 43.8555\tabularnewline
49.1361 & 61.9281 & 41.9543 & 44.9346 & 46.0521 & 48.3652 & 43.6252\tabularnewline
50.9866 & 59.1724 & 39.6282 & 33.6074 & 21.6643 & &\tabularnewline
\bottomrule
\end{longtable}

\begin{enumerate}
\def\labelenumi{\arabic{enumi}.}
\setcounter{enumi}{6}
\tightlist
\item
  Table \#8.3.10 contains the percentage of woman receiving prenatal care in 2009 for a sample of countries ("Pregnant woman receiving," 2013). Estimate the average percentage of woman receiving prenatal care in 2009 using a 90\% confidence interval.
\end{enumerate}

\begin{quote}
\textbf{Table \#8.3.10: Percentage of Woman Receiving Prenatal Care}
\end{quote}

\begin{longtable}[]{@{}llllll@{}}
\toprule
70.08 & 72.73 & 74.52 & 75.79 & 76.28 & 76.28\tabularnewline
\midrule
\endhead
76.65 & 80.34 & 80.60 & 81.90 & 86.30 & 87.70\tabularnewline
87.76 & 88.40 & 90.70 & 91.50 & 91.80 & 92.10\tabularnewline
92.20 & 92.41 & 92.47 & 93.00 & 93.20 & 93.40\tabularnewline
93.63 & 93.68 & 93.80 & 94.30 & 94.51 & 95.00\tabularnewline
95.80 & 95.80 & 96.23 & 96.24 & 97.30 & 97.90\tabularnewline
97.95 & 98.20 & 99.00 & 99.00 & 99.10 & 99.10\tabularnewline
100.00 & 100.00 & 100.00 & 100.00 & 100.00 &\tabularnewline
\bottomrule
\end{longtable}

\begin{enumerate}
\def\labelenumi{\arabic{enumi}.}
\setcounter{enumi}{7}
\tightlist
\item
  Maintaining your balance may get harder as you grow older. A study was conducted to see how steady the elderly is on their feet. They had the subjects stand on a force platform and have them react to a noise. The force platform then measured how much they swayed forward and backward, and the data is in table \#8.3.11 ("Maintaining balance while," 2013). Find a 99\% confidence interval for the mean sway of elderly people.
\end{enumerate}

\begin{quote}
\textbf{Table \#8.3.11: Forward/backward Sway (mm) of Elderly Subjects}
\end{quote}

\begin{longtable}[]{@{}lllllllll@{}}
\toprule
\endhead
19 & 30 & 20 & 19 & 29 & 25 & 21 & 24 & 50\tabularnewline
\bottomrule
\end{longtable}

Data Sources:

Australian Human Rights Commission, (1996). \emph{Indigenous deaths in
custody 1989 - 1996}. Retrieved from website:
\url{http://www.humanrights.gov.au/publications/indigenous-deaths-custody}

\emph{CDC features - new data on autism spectrum disorders}. (2013, November
26). Retrieved from \url{http://www.cdc.gov/features/countingautism/}

\emph{CDC features - new data on autism spectrum disorders}. (2013, November
26). Retrieved from \url{http://www.cdc.gov/features/countingautism/}

Center for Disease Control and Prevention, Prevalence of Autism Spectrum
Disorders - Autism and Developmental Disabilities Monitoring Network.
(2008). \emph{Autism and developmental disabilities monitoring network-2012}.
Retrieved from website:
\url{http://www.cdc.gov/ncbddd/autism/documents/ADDM-2012-Community-Report.pdf}

Center for Disease Control and Prevention, Prevalence of Autism Spectrum
Disorders - Autism and Developmental Disabilities Monitoring Network.
(2008). \emph{Autism and developmental disabilities monitoring network-2012}.
Retrieved from website:
\url{http://www.cdc.gov/ncbddd/autism/documents/ADDM-2012-Community-Report.pdf}

\emph{CO2 emissions}. (2013, November 19). Retrieved from
\url{http://data.worldbank.org/indicator/EN.ATM.CO2E.PC}

Federal Trade Commission, (2008). \emph{Consumer fraud and identity theft
complaint data: January-december 2007}. Retrieved from website:
\url{http://www.ftc.gov/opa/2008/02/fraud.pdf}

Federal Trade Commission, (2008). \emph{Consumer fraud and identity theft
complaint data: January-december 2007}. Retrieved from website:
\url{http://www.ftc.gov/opa/2008/02/fraud.pdf}

\emph{Gallup news service}. (2013, November 7-10). Retrieved from
\url{http://www.gallup.com/file/poll/165896/JFK_Conspiracy_131115.pdf}

\emph{Gallup news service}. (2013, November 7-10). Retrieved from
\url{http://www.gallup.com/file/poll/165896/JFK_Conspiracy_131115.pdf}

\emph{Healthy breakfast story}. (2013, November 16). Retrieved from
\url{http://lib.stat.cmu.edu/DASL/Stories/HealthyBreakfast.html}

\emph{IQ of famous people}. (2013, November 13). Retrieved from
\url{http://www.kidsiqtestcenter.com/IQ-famous-people.html}

\emph{Maintaining balance while concentrating}. (2013, September 25).
Retrieved from \url{http://www.statsci.org/data/general/balaconc.html}

\emph{Morgan Gallup poll on unemployment}. (2013, September 26). Retrieved
from \url{http://www.statsci.org/data/oz/gallup.html}

\emph{Morgan Gallup poll on unemployment}. (2013, September 26). Retrieved
from \url{http://www.statsci.org/data/oz/gallup.html}

\emph{Multi-disciplinary niser activity - mercury in bass}. (2013, November
16). Retrieved from
\href{http://gozips.uakron.edu/~nmimoto/pages/datasets/MercuryInBass\%20-\%20description.txt}{http://gozips.uakron.edu/\textasciitilde{}nmimoto/pages/datasets/MercuryInBass -
description.txt}

\emph{Pregnant woman receiving prenatal care}. (2013, October 14). Retrieved
from \url{http://data.worldbank.org/indicator/SH.STA.ANVC.ZS}

\emph{Pulse rates before and after exercise}. (2013, September 25). Retrieved
from \url{http://www.statsci.org/data/oz/ms212.html}

\emph{SOCR data 2008 world countries rankings}. (2013, November 16).
Retrieved from
\url{http://wiki.stat.ucla.edu/socr/index.php/SOCR_Data_2008_World_CountriesRankings}

\emph{Student t-distribution}. (2013, November 25). Retrieved from
\url{http://lib.stat.cmu.edu/DASL/Stories/student.html}

\emph{WHO life expectancy}. (2013, September 19). Retrieved from
\url{http://www.who.int/gho/mortality_burden_disease/life_tables/situation_trends/en/index.html}

\hypertarget{two-sample-inference}{%
\chapter{Two-Sample Inference}\label{two-sample-inference}}

Chapter 7 discussed methods of hypothesis testing about one-population parameters. Chapter 8 discussed methods of estimating population parameters from one sample using confidence intervals. This chapter will look at methods of confidence intervals and hypothesis testing for two populations. Since there are two populations, there are two random variables, two means or proportions, and two samples (though with paired samples you usually consider there to be one sample with pairs collected). Examples of where you would do this are:

\begin{quote}
Testing and estimating the difference in testosterone levels of men
before and after they had children (Gettler, McDade, Feranil \& Kuzawa,
2011).

Testing the claim that a diet works by looking at the weight before
and after subjects are on the diet.

Estimating the difference in proportion of those who approve of
President Obama in the age group 18 to 26 year olds and the 55 and
over age group.
\end{quote}

All of these are examples of hypothesis tests or confidence intervals for two populations. The methods to conduct these hypothesis tests and confidence intervals will be explored in this method. As a reminder, all hypothesis tests are the same process. The only thing that changes is the formula that you use. Confidence intervals are also the same process, except that the formula is different.

\hypertarget{two-proportions}{%
\section{Two Proportions}\label{two-proportions}}

There are times you want to test a claim about two population
proportions or construct a confidence interval estimate of the
difference between two population proportions. As with all other
hypothesis tests and confidence intervals, the process is the same
though the formulas and assumptions are different.

\textbf{Hypothesis Test for Two Population Proportion (2-Prop Test)}

\begin{enumerate}
\def\labelenumi{\arabic{enumi}.}
\tightlist
\item
  State the random variables and the parameters in words.
\end{enumerate}

\begin{quote}
= number of successes from group 1

= number of successes from group 2

= proportion of successes in group 1

= proportion of successes in group 2
\end{quote}

\begin{enumerate}
\def\labelenumi{\arabic{enumi}.}
\setcounter{enumi}{1}
\tightlist
\item
  State the null and alternative hypotheses and the level of
  significance
\end{enumerate}

\begin{quote}
or

Also, state your level here.
\end{quote}

\begin{enumerate}
\def\labelenumi{\arabic{enumi}.}
\setcounter{enumi}{2}
\tightlist
\item
  State and check the assumptions for a hypothesis test
\end{enumerate}

\begin{enumerate}
\def\labelenumi{\alph{enumi}.}
\item
  A simple random sample of size is taken from population 1, and a
  simple random sample of size is taken from population 2.
\item
  The samples are independent.
\item
  The assumptions for the binomial distribution are satisfied for both
  populations.
\item
  To determine the sampling distribution of , you need to show that
  and , where . If this requirement is true, then the sampling
  distribution of is well approximated by a normal curve. To determine
  the sampling distribution of , you need to show that and , where .
  If this requirement is true, then the sampling distribution of is
  well approximated by a normal curve. However, you do not know and ,
  so you need to use and instead. This is not perfect, but it is the
  best you can do. Since (and similar for the other calculations) you
  just need to make sure that , , ,and are all more than 5.
\end{enumerate}

\begin{enumerate}
\def\labelenumi{\arabic{enumi}.}
\setcounter{enumi}{3}
\tightlist
\item
  Find the sample statistics, test statistic, and p-value
\end{enumerate}

\begin{quote}
Sample Proportion:

Pooled Sample Proportion, :

Test Statistic:

Usually , since

p-value:

On TI-83/84: use normalcdf(lower limit, upper limit, 0, 1)

(Note: if , then lower limit is and upper limit is your test
statistic. If , then lower limit is your test statistic and the upper
limit is . If , then find the p-value for , and multiply by 2.)

On R: use pnorm(z, 0, 1)

(Note: if , then use pnorm(z, 0, 1). If , then use . If , then find
the p-value for , and multiply by 2.)
\end{quote}

\begin{enumerate}
\def\labelenumi{\arabic{enumi}.}
\setcounter{enumi}{4}
\tightlist
\item
  Conclusion
\end{enumerate}

\begin{quote}
This is where you write reject or fail to reject . The rule is: if the
p-value \textless{} , then reject . If the p-value , then fail to reject
\end{quote}

\begin{enumerate}
\def\labelenumi{\arabic{enumi}.}
\setcounter{enumi}{5}
\tightlist
\item
  Interpretation
\end{enumerate}

\begin{quote}
This is where you interpret in real world terms the conclusion to the
test. The conclusion for a hypothesis test is that you either have
enough evidence to show is true, or you do not have enough evidence to
show is true.
\end{quote}

\textbf{Confidence Interval for the Difference Between Two Population
Proportion (2-Prop Interval)}

The confidence interval for the difference in proportions has the same
random variables and proportions and the same assumptions as the
hypothesis test for two proportions. If you have already completed the
hypothesis test, then you do not need to state them again. If you
haven't completed the hypothesis test, then state the random variables
and proportions and state and check the assumptions before completing
the confidence interval step.

\begin{enumerate}
\def\labelenumi{\arabic{enumi}.}
\tightlist
\item
  Find the sample statistics and the confidence interval
\end{enumerate}

\begin{quote}
Sample Proportion:

Confidence Interval:

The confidence interval estimate of the difference is

where the margin of error E is given by

= critical value
\end{quote}

\begin{enumerate}
\def\labelenumi{\arabic{enumi}.}
\setcounter{enumi}{1}
\item
  Statistical Interpretation: In general this looks like, ``there is a
  C\% chance that contains the true difference in proportions.''
\item
  Real World Interpretation: This is where you state how much more (or
  less) the first proportion is from the second proportion.
\end{enumerate}

The critical value is a value from the normal distribution. Since a
confidence interval is found by adding and subtracting a margin of error
amount from the sample proportion, and the interval has a probability of
being true, then you can think of this as the statement . So you can use
the invNorm command on the TI-83/84 calculator or qnorm on R to find the
critical value. These are always the same value, so it is easier to just
look at the table A.1 in the Appendix.

\textbf{Example \#9.1.1: Hypothesis Test for Two Population Proportions}

\begin{quote}
Do husbands cheat on their wives more than wives cheat on their
husbands ("Statistics brain," 2013)? Suppose you take a group of
1000 randomly selected husbands and find that 231 had cheated on their
wives. Suppose in a group of 1200 randomly selected wives, 176 cheated
on their husbands. Do the data show that the proportion of husbands
who cheat on their wives are more than the proportion of wives who
cheat on their husbands. Test at the 5\% level.

\textbf{Solution:}
\end{quote}

\begin{enumerate}
\def\labelenumi{\arabic{enumi}.}
\tightlist
\item
  State the random variables and the parameters in words.
\end{enumerate}

\begin{quote}
= number of husbands who cheat on his wife

= number of wives who cheat on her husband

= proportion of husbands who cheat on his wife

= proportion of wives who cheat on her husband
\end{quote}

\begin{enumerate}
\def\labelenumi{\arabic{enumi}.}
\setcounter{enumi}{1}
\tightlist
\item
  State the null and alternative hypotheses and the level of
  significance
\end{enumerate}

\begin{quote}
or
\end{quote}

\begin{enumerate}
\def\labelenumi{\arabic{enumi}.}
\setcounter{enumi}{2}
\tightlist
\item
  State and check the assumptions for a hypothesis test
\end{enumerate}

\begin{enumerate}
\def\labelenumi{\alph{enumi}.}
\item
  A simple random sample of 1000 responses about cheating from
  husbands is taken. This was stated in the problem. A simple random
  sample of 1200 responses about cheating from wives is taken. This
  was stated in the problem.
\item
  The samples are independent. This is true since the samples involved
  different genders.
\item
  The properties of the binomial distribution are satisfied in both
  populations. This is true since there are only two responses, there
  are a fixed number of trials, the probability of a success is the
  same, and the trials are independent.
\item
  The sampling distributions of and can be approximated with a normal
  distribution.

  , , , and are all greater than or equal to 5. So both sampling
  distributions of and can be approximated with a normal distribution.
\end{enumerate}

\begin{enumerate}
\def\labelenumi{\arabic{enumi}.}
\setcounter{enumi}{3}
\tightlist
\item
  Find the sample statistics, test statistic, and p-value
\end{enumerate}

\begin{quote}
Sample Proportion:

Pooled Sample Proportion, :

Test Statistic:

p-value:

On TI-83/84:

On R:
\end{quote}

\textbf{Figure \#9.1.1: Setup for 2-PropZTest on TI-83/84 Calculator}

\begin{quote}
\includegraphics[width=2.75in,height=1.86111in]{media/image89.png}
\end{quote}

\textbf{Figure \#9.1.2: Results for 2-PropZTest on TI-83/84 Calculator}

\includegraphics[width=2.75in,height=1.86111in]{media/image90.png}

\includegraphics[width=2.75in,height=1.86111in]{media/image91.png}

On R: . For this example, prop.test(c(231,176), c(1000, 1200),
alternative="greater")

2-sample test for equality of proportions with continuity correction

data: c(231, 176) out of c(1000, 1200)

X-squared = 25.173, df = 1, p-value = 2.621e-07

alternative hypothesis: greater

95 percent confidence interval:

0.05579805 1.00000000

sample estimates:

prop 1 prop 2

0.2310000 0.1466667

Note: the answer from R is the p-value. It is different from the formula
or the TI-83/84 calculator due to a continuity correction that R does.

\begin{enumerate}
\def\labelenumi{\arabic{enumi}.}
\setcounter{enumi}{4}
\tightlist
\item
  Conclusion
\end{enumerate}

\begin{quote}
Reject , since the p-value is less than 5\%.
\end{quote}

\begin{enumerate}
\def\labelenumi{\arabic{enumi}.}
\setcounter{enumi}{5}
\tightlist
\item
  Interpretation
\end{enumerate}

\begin{quote}
This is enough evidence to show that the proportion of husbands having
affairs is more than the proportion of wives having affairs.
\end{quote}

\textbf{Example \#9.1.2: Confidence Interval for Two Population Proportions}

\begin{quote}
Do more husbands cheat on their wives more than wives cheat on the
husbands ("Statistics brain," 2013)? Suppose you take a group of
1000 randomly selected husbands and find that 231 had cheated on their
wives. Suppose in a group of 1200 randomly selected wives, 176 cheated
on their husbands. Estimate the difference in the proportion of
husbands and wives who cheat on their spouses using a 95\% confidence
level.

\textbf{Solution:}
\end{quote}

\begin{enumerate}
\def\labelenumi{\arabic{enumi}.}
\item
  State the random variables and the parameters in words.

  These were stated in example \#9.3.1, but are reproduced here for
  reference.
\end{enumerate}

\begin{quote}
= number of husbands who cheat on his wife

= number of wives who cheat on her husband

= proportion of husbands who cheat on his wife

= proportion of wives who cheat on her husband
\end{quote}

\begin{enumerate}
\def\labelenumi{\arabic{enumi}.}
\setcounter{enumi}{1}
\tightlist
\item
  State and check the assumptions for the confidence interval
\end{enumerate}

\begin{quote}
The assumptions were stated and checked in example \#9.1.1.
\end{quote}

\begin{enumerate}
\def\labelenumi{\arabic{enumi}.}
\setcounter{enumi}{2}
\tightlist
\item
  Find the sample statistics and the confidence interval
\end{enumerate}

\begin{quote}
Sample Proportion:

Confidence Interval:

The confidence interval estimate of the difference is
\end{quote}

\textbf{Figure \#9.1.3: Setup for 2-PropZInt on TI-83/84 Calculator}

\begin{quote}
\includegraphics[width=2.75in,height=1.86111in]{media/image104.png}
\end{quote}

\textbf{Figure \#9.1.4: Results for 2-PropZInt on TI-83/84 Calculator}

\includegraphics[width=2.75in,height=1.86111in]{media/image105.png}

On R: , where C is in decimal form. For this example,
prop.test(c(231,176), c(1000, 1200), conf.level=0.95)

2-sample test for equality of proportions with continuity correction

data: c(231, 176) out of c(1000, 1200)

X-squared = 25.173, df = 1, p-value = 5.241e-07

alternative hypothesis: two.sided

95 percent confidence interval:

0.05050705 0.11815962

sample estimates:

prop 1 prop 2

0.2310000 0.1466667

Note: the answer from R is the confidence interval. It is different from the formula or the TI-83/84 calculator due to a continuity correction that R does.

\begin{enumerate}
\def\labelenumi{\arabic{enumi}.}
\setcounter{enumi}{3}
\item
  Statistical Interpretation: There is a 95\% chance that contains the true difference in proportions.
\item
  Real World Interpretation: The proportion of husbands who cheat is anywhere from 5.05\% to 11.82\% higher than the proportion of wives who cheat.
\end{enumerate}

\hypertarget{homework-26}{%
\subsection{Homework}\label{homework-26}}

In each problem show all steps of the hypothesis test or confidence interval. If some of the assumptions are not met, note that the results of the test or interval may not be correct and then continue the process of the hypothesis test or confidence interval.

\begin{enumerate}
\def\labelenumi{\arabic{enumi}.}
\item
  Many high school students take the AP tests in different subject areas. In 2007, of the 144,796 students who took the biology exam 84,199 of them were female. In that same year, of the 211,693 students who took the calculus AB exam 102,598 of them were female ("AP exam scores," 2013). Is there enough evidence to show that the proportion of female students taking the biology exam is higher than the proportion of female students taking the calculus AB exam? Test at the 5\% level.
\item
  Many high school students take the AP tests in different subject areas. In 2007, of the 144,796 students who took the biology exam 84,199 of them were female. In that same year, of the 211,693 students who took the calculus AB exam 102,598 of them were female ("AP exam scores," 2013). Estimate the difference in the proportion of female students taking the biology exam and female students taking the calculus AB exam using a 90\% confidence level.
\item
  Many high school students take the AP tests in different subject areas. In 2007, of the 211,693 students who took the calculus AB exam 102,598 of them were female and 109,095 of them were male ("AP exam scores," 2013). Is there enough evidence to show that the proportion of female students taking the calculus AB exam is different from the proportion of male students taking the calculus AB exam? Test at the 5\% level.
\item
  Many high school students take the AP tests in different subject areas. In 2007, of the 211,693 students who took the calculus AB exam 102,598 of them were female and 109,095 of them were male ("AP exam scores," 2013). Estimate using a 90\% level the difference in proportion of female students taking the calculus AB exam versus male students taking the calculus AB exam.
\item
  Are there more children diagnosed with Autism Spectrum Disorder (ASD) in states that have larger urban areas over states that are mostly rural? In the state of Pennsylvania, a fairly urban state, there are 245 eight year olds diagnosed with ASD out of 18,440 eight year olds evaluated. In the state of Utah, a fairly rural state, there are 45 eight year olds diagnosed with ASD out of 2,123 eight year olds evaluated ("Autism and developmental," 2008). Is there enough evidence to show that the proportion of children diagnosed with ASD in Pennsylvania is more than the proportion in Utah? Test at the 1\% level.
\item
  Are there more children diagnosed with Autism Spectrum Disorder (ASD) in states that have larger urban areas over states that are mostly rural? In the state of Pennsylvania, a fairly urban state, there are 245 eight year olds diagnosed with ASD out of 18,440 eight year olds evaluated. In the state of Utah, a fairly rural state, there are 45 eight year olds diagnosed with ASD out of 2,123 eight year olds evaluated ("Autism and developmental," 2008). Estimate the difference in proportion of children diagnosed with ASD between Pennsylvania and Utah. Use a 98\% confidence level.
\item
  A child dying from an accidental poisoning is a terrible incident. Is it more likely that a male child will get into poison than a female child? To find this out, data was collected that showed that out of 1830 children between the ages one and four who pass away from poisoning, 1031 were males and 799 were females (Flanagan, Rooney \& Griffiths, 2005). Do the data show that there are more male children dying of poisoning than female children? Test at the 1\% level.
\item
  A child dying from an accidental poisoning is a terrible incident. Is it more likely that a male child will get into poison than a female child? To find this out, data was collected that showed that out of 1830 children between the ages one and four who pass away from poisoning, 1031 were males and 799 were females (Flanagan, Rooney \& Griffiths, 2005). Compute a 99\% confidence interval for the difference in proportions of poisoning deaths of male and female children ages one to four.
\end{enumerate}

\textbf{\\
}

\hypertarget{paired-samples-for-two-means}{%
\section{Paired Samples for Two Means}\label{paired-samples-for-two-means}}

Are two populations the same? Is the average height of men taller than the average height of women? Is the mean weight less after a diet than before?

You can compare populations by comparing their means. You take a sample from each population and compare the statistics.

Anytime you compare two populations you need to know if the samples are independent or dependent. The formulas you use are different for different types of samples.

If how you choose one sample has no effect on the way you choose the other sample, the two samples are \textbf{independent.} The way to think about it is that in independent samples, the individuals from one sample are overall different from the individuals from the other sample. This will mean that sample one has no affect on sample two. The sample values from one sample are not related or paired with values from the other sample.

If you choose the samples so that a measurement in one sample is paired with a measurement from the other sample, the samples are \textbf{dependent} or \textbf{matched} or \textbf{paired}. (Often a before and after situation.) You want to make sure the there is a meaning for pairing data values from one sample with a specific data value from the other sample. One way to think about it is that in dependent samples, the individuals from one sample are the same individuals from the other sample, though there can be other reasons to pair values. This makes the sample values from each sample paired.

\textbf{Example \#9.2.1: Independent or Dependent Samples}

\begin{quote}
Determine if the following are dependent or independent samples.
\end{quote}

\begin{enumerate}
\def\labelenumi{\alph{enumi}.}
\tightlist
\item
  Randomly choose 5 men and 6 women and compare their heights
\end{enumerate}

\begin{quote}
\textbf{Solution:}

Independent, since there is no reason that one value belongs to another. The individuals are not the same for both samples. The individuals are definitely different. A way to think about this is that the knowledge that a man is chosen in one sample does not give any information about any of the woman chosen in the other sample.
\end{quote}

\begin{enumerate}
\def\labelenumi{\alph{enumi}.}
\setcounter{enumi}{1}
\tightlist
\item
  Choose 10 men and weigh them. Give them a new wonder diet drug and later weigh them again.
\end{enumerate}

\begin{quote}
\textbf{Solution:}

Dependent, since each person's before weight can be matched with their after weight. The individuals are the same for both samples. A way to think about this is that the knowledge that a person weighs 400 pounds at the beginning will tell you something about their weight after the diet drug.
\end{quote}

\begin{enumerate}
\def\labelenumi{\alph{enumi}.}
\setcounter{enumi}{2}
\tightlist
\item
  Take 10 people and measure the strength of their dominant arm and their non-dominant arm.
\end{enumerate}

\begin{quote}
\textbf{Solution:}
Dependent, since you can match the two arm strengths. The individuals are the same for both samples. So the knowledge of one person's dominant arm strength will tell you something about the strength of their non-dominant arm.
\end{quote}

To analyze data when there are matched or paired samples, called dependent samples, you conduct a paired t-test. Since the samples are matched, you can find the difference between the values of the two random variables.

\textbf{Hypothesis Test for Two Sample Paired t-Test}

\begin{enumerate}
\def\labelenumi{\arabic{enumi}.}
\tightlist
\item
  State the random variables and the parameters in words.
\end{enumerate}

\begin{quote}
= random variable 1

= random variable 2

= mean of random variable 1

= mean of random variable 2
\end{quote}

\begin{enumerate}
\def\labelenumi{\arabic{enumi}.}
\setcounter{enumi}{1}
\tightlist
\item
  State the null and alternative hypotheses and the level of significance
\end{enumerate}

\begin{quote}
The usual hypotheses would be

or

However, since you are finding the differences, then you can actually
think of ,

So the hypotheses become

Also, state your level here.
\end{quote}

\begin{enumerate}
\def\labelenumi{\arabic{enumi}.}
\setcounter{enumi}{2}
\tightlist
\item
  State and check the assumptions for the hypothesis test
\end{enumerate}

\begin{enumerate}
\def\labelenumi{\alph{enumi}.}
\item
  A random sample of \emph{n} pairs is taken.
\item
  The population of the difference between random variables is normally distributed. In this case the population you are interested in has to do with the differences that you find. It does not matter if each random variable is normally distributed. It is only important if the differences you find are normally distributed. Just as before, the t-test is fairly robust to the assumption if the sample size is large. This means that if this assumption isn't met, but your sample size is quite large (over 30), then the results of the t-test are valid.
\end{enumerate}

\begin{enumerate}
\def\labelenumi{\arabic{enumi}.}
\setcounter{enumi}{3}
\tightlist
\item
  Find the sample statistic, test statistic, and p-value
\end{enumerate}

\begin{quote}
Sample Statistic:

Difference: for each pair

Sample mean of the differences:

Standard deviation of the differences:

Number of pairs:

Test Statistic:

with degrees of freedom =

Note: in most cases.

p-value:

On TI-83/84: Use

(Note: if , then lower limit is and upper limit is your test
statistic. If , then lower limit is your test statistic and the upper
limit is . If , then find the p-value for , and multiply by 2.)

On R: Use

(Note: if , use . If , use . If , then find the p-value for , and
multiply by 2.)
\end{quote}

\begin{enumerate}
\def\labelenumi{\arabic{enumi}.}
\setcounter{enumi}{4}
\tightlist
\item
  Conclusion
\end{enumerate}

\begin{quote}
This is where you write reject or fail to reject . The rule is: if the p-value \textless{} , then reject . If the p-value , then fail to reject
\end{quote}

\begin{enumerate}
\def\labelenumi{\arabic{enumi}.}
\setcounter{enumi}{5}
\tightlist
\item
  Interpretation
\end{enumerate}

\begin{quote}
This is where you interpret in real world terms the conclusion to the test. The conclusion for a hypothesis test is that you either have enough evidence to show is true, or you do not have enough evidence to show is true.
\end{quote}

\textbf{Confidence Interval for Difference in Means from Paired Samples
(t-Interval)}

The confidence interval for the difference in means has the same random variables and means and the same assumptions as the hypothesis test for two paired samples. If you have already completed the hypothesis test, then you do not need to state them again. If you haven't completed the hypothesis test, then state the random variables and means, and state and check the assumptions before completing the confidence interval step.

\begin{enumerate}
\def\labelenumi{\arabic{enumi}.}
\tightlist
\item
  Find the sample statistic and confidence interval
\end{enumerate}

\begin{quote}
Sample Statistic:

Difference:

Sample mean of the differences:

Standard deviation of the differences:

Number of pairs:

Confidence Interval:

The confidence interval estimate of the difference is

is the critical value where degrees of freedom
\end{quote}

\begin{enumerate}
\def\labelenumi{\arabic{enumi}.}
\setcounter{enumi}{1}
\item
  Statistical Interpretation: In general this looks like, ``there is a C\% chance that the statement contains the true mean difference.''
\item
  Real World Interpretation: This is where you state what interval contains the true mean difference.
\end{enumerate}

The critical value is a value from the Student's t-distribution. Since a confidence interval is found by adding and subtracting a margin of error amount from the sample mean, and the interval has a probability of containing the true mean difference, then you can think of this as the statement . To find the critical value, you use table A.2 in the Appendix.

\textbf{How to check the assumptions of t-test and confidence interval:}

In order for the t-test or confidence interval to be valid, the assumptions of the test must be met. So whenever you run a t-test or confidence interval, you must make sure the assumptions are met. So you need to check them. Here is how you do this:

\begin{enumerate}
\def\labelenumi{\arabic{enumi}.}
\item
  For the assumption that the sample is a random sample, describe how you took the samples. Make sure your sampling technique is random and that the samples were dependent.
\item
  For the assumption that the population of the differences is normal, remember the process of assessing normality from chapter 6.
\end{enumerate}

\textbf{Example \#9.2.2: Hypothesis Test for Paired Samples Using the Formula}

\begin{quote}
A researcher wants to see if a weight loss program is effective. She measures the weight of 6 randomly selected women before and after the weight loss program (see table \#9.2.1). Is there evidence that the weight loss program is effective? Test at the 5\% level.

\textbf{Table \#9.2.1: Data of Before and After Weights}
\end{quote}

\begin{longtable}[]{@{}lllllll@{}}
\toprule
Person & 1 & 2 & 3 & 4 & 5 & 6\tabularnewline
\midrule
\endhead
Weight before & 165 & 172 & 181 & 185 & 168 & 175\tabularnewline
Weight after & 143 & 151 & 156 & 161 & 152 & 154\tabularnewline
\bottomrule
\end{longtable}

\textbf{Solution:}

\begin{enumerate}
\def\labelenumi{\arabic{enumi}.}
\tightlist
\item
  State the random variables and the parameters in words.
\end{enumerate}

\begin{quote}
= weight of a woman after the weight loss program

= weight of a woman before the weight loss program

= mean weight of a woman after the weight loss program

= mean weight of a woman before the weight loss program
\end{quote}

\begin{enumerate}
\def\labelenumi{\arabic{enumi}.}
\setcounter{enumi}{1}
\item
  State the null and alternative hypotheses and the level of significance
\item
  State and check the assumptions for the hypothesis test
\end{enumerate}

\begin{enumerate}
\def\labelenumi{\alph{enumi}.}
\item
  A random sample of 6 pairs of weights before and after was taken. This was stated in the problem, since the women were chosen randomly.
\item
  The population of the difference in after and before weights is normally distributed. To see if this is true, look at the histogram, number of outliers, and the normal probability plot. (If you wish, you can look at the normal probability plot first. If it doesn't look linear, then you may want to look at the histogram and number of outliers at this point.)
\end{enumerate}

\begin{quote}
\textbf{Figure \#9.2.1: Histogram of Differences in Weights}

\includegraphics[width=2.23611in,height=2.23611in]{media/image165.emf}

This histogram looks somewhat bell shaped.

\textbf{Figure \#9.2.2: Modified Box Plot of Differences in Weights}

\includegraphics[width=2.5in,height=2.5in]{media/image166.emf}

There is only one outlier in the difference data set.

\textbf{Figure \#9.2.3: Normal Quantile Plot of Differences in Weights}

\includegraphics[width=2.26389in,height=2.26389in]{media/image167.emf}

The probability plot on the differences looks somewhat linear.

So you can assume that the distribution of the difference in weights is normal.
\end{quote}

\begin{enumerate}
\def\labelenumi{\arabic{enumi}.}
\setcounter{enumi}{3}
\tightlist
\item
  Find the sample statistic, test statistic, and p-value
\end{enumerate}

\begin{quote}
Sample Statistics:

\textbf{Table \#9.2.2: Differences Between Before and After Weights}
\end{quote}

\begin{longtable}[]{@{}lllllll@{}}
\toprule
Person & 1 & 2 & 3 & 4 & 5 & 6\tabularnewline
\midrule
\endhead
Weight after, & 143 & 151 & 156 & 161 & 152 & 154\tabularnewline
Weight before, & 165 & 172 & 181 & 185 & 168 & 175\tabularnewline
\bottomrule
\end{longtable}

\begin{quote}
The mean and standard deviation are

Test Statistic:

p-value:

There are six pairs so the degrees of freedom are

Since , then p-value

Using TI-83/84:

Using R:
\end{quote}

\begin{enumerate}
\def\labelenumi{\arabic{enumi}.}
\setcounter{enumi}{4}
\tightlist
\item
  Conclusion
\end{enumerate}

\begin{quote}
Since the p-value \textless{} 0.05, reject .
\end{quote}

\begin{enumerate}
\def\labelenumi{\arabic{enumi}.}
\setcounter{enumi}{5}
\tightlist
\item
  Interpretation
\end{enumerate}

\begin{quote}
There is enough evidence to show that the weight loss program is effective.
\end{quote}

Note: Just because the hypothesis test says the program is effective doesn't mean you should go out and use it right away. The program has statistical significance, but that doesn't mean it has practical significance. You need to see how much weight a person loses, and you need to look at how safe it is, how expensive, does it work in the long term, and other type questions. Remember to look at the practical significance in all situations. In this case, the average weight loss was 21.5 pounds, which is very practically significant. Do remember to look at the safety and expense of the drug also.

\textbf{Example \#9.2.3: Hypothesis Test for Paired Samples Using Technology}

\begin{quote}
The New Zealand Air Force purchased a batch of flight helmets. They then found out that the helmets didn't fit. In order to make sure that they order the correct size helmets, they measured the head size of recruits. To save money, they wanted to use cardboard calipers, but were not sure if they will be accurate enough. So they took 18 recruits and measured their heads with the cardboard calipers and also with metal calipers. The data in centimeters (cm) is in table \#9.2.3 ("NZ helmet size," 2013). Do the data provide enough evidence to show that there is a difference in measurements between the cardboard and metal calipers? Use a 5\% level of significance.

\textbf{Table \#9.2.3: Data for Head Measurements}
\end{quote}

\begin{longtable}[]{@{}ll@{}}
\toprule
Cardboard & Metal\tabularnewline
\midrule
\endhead
146 & 145\tabularnewline
151 & 153\tabularnewline
163 & 161\tabularnewline
152 & 151\tabularnewline
151 & 145\tabularnewline
151 & 150\tabularnewline
149 & 150\tabularnewline
166 & 163\tabularnewline
149 & 147\tabularnewline
155 & 154\tabularnewline
155 & 150\tabularnewline
156 & 156\tabularnewline
162 & 161\tabularnewline
150 & 152\tabularnewline
156 & 154\tabularnewline
158 & 154\tabularnewline
149 & 147\tabularnewline
163 & 160\tabularnewline
\bottomrule
\end{longtable}

\textbf{Solution:}

\begin{enumerate}
\def\labelenumi{\arabic{enumi}.}
\tightlist
\item
  State the random variables and the parameters in words.
\end{enumerate}

\begin{quote}
= head measurement of recruit using cardboard caliper

= head measurement of recruit using metal caliper

= mean head measurement of recruit using cardboard caliper

= mean head measurement of recruit using metal caliper
\end{quote}

\begin{enumerate}
\def\labelenumi{\arabic{enumi}.}
\setcounter{enumi}{1}
\item
  State the null and alternative hypotheses and the level of significance
\item
  State and check the assumptions for the hypothesis test
\end{enumerate}

\begin{enumerate}
\def\labelenumi{\alph{enumi}.}
\item
  A random sample of 18 pairs of head measures of recruits with cardboard and metal caliper was taken. This was not stated, but probably could be safely assumed.
\item
  The population of the difference in head measurements between cardboard and metal calipers is normally distributed. To see if this is true, look at the histogram, number of outliers, and the normal probability plot. (If you wish, you can look at the normal probability plot first. If it doesn't look linear, then you may want to look at the histogram and number of outliers at this point.)

  \textbf{Figure \#9.2.4: Histogram of Differences in Head Measurements}
\end{enumerate}

\begin{quote}
\includegraphics[width=2.84722in,height=2.84722in]{media/image189.emf}

This histogram looks bell shaped.

\textbf{Figure \#9.2.5: Modified Box Plot of Differences in Head
Measurements}

\includegraphics[width=2.76389in,height=2.76389in]{media/image190.emf}

There are no outliers in the difference data set.

\textbf{Figure \#9.2.6: Normal Quantile Plot of Differences in Head
Measurements}

\includegraphics[width=2.66667in,height=2.66667in]{media/image191.emf}

The probability plot on the differences looks somewhat linear.

So you can assume that the distribution of the difference in weights is normal.
\end{quote}

\begin{enumerate}
\def\labelenumi{\arabic{enumi}.}
\setcounter{enumi}{3}
\tightlist
\item
  Find the sample statistic, test statistic, and p-value
\end{enumerate}

\begin{quote}
Using the TI-83/84, put into L1 and into L2. Then go onto the name L3, and type . The calculator will calculate the differences for you and put them in L3. Now go into STAT and move over to TESTS. Choose T-Test. The setup for the calculator is in figure \#9.2.7.

\textbf{Figure \#9.2.7: Setup for T-Test on Ti-83/84 Calculator}

\includegraphics[width=2.75in,height=1.86111in]{media/image195.png}

Once you press ENTER on Calculate you will see the result shown in figure \#9.2.8.

\textbf{Figure \#9.2.8: Results of T-Test on TI-83/84 Calculator}

\includegraphics[width=2.75in,height=1.86111in]{media/image196.png}

Using R: command is t.test(variable1, variable2, paired = TRUE, alternative = "less" or "greater"). For this example, the command would be

t.test(cardboard, metal, paired = TRUE)

Paired t-test

data: cardboard and metal

t = 3.1854, df = 17, p-value = 0.005415

alternative hypothesis: true difference in means is not equal to 0

95 percent confidence interval:

0.5440163 2.6782060

sample estimates:

mean of the differences

1.611111

The is the test statistic. The p-value is .
\end{quote}

\begin{enumerate}
\def\labelenumi{\arabic{enumi}.}
\setcounter{enumi}{4}
\tightlist
\item
  Conclusion
\end{enumerate}

\begin{quote}
Since the p-value \textless{} 0.05, reject .
\end{quote}

\begin{enumerate}
\def\labelenumi{\arabic{enumi}.}
\setcounter{enumi}{5}
\tightlist
\item
  Interpretation
\end{enumerate}

\begin{quote}
There is enough evidence to show that the mean head measurements using the cardboard calipers are not the same as when using the metal calipers. So it looks like the New Zealand Air Force shouldn't use the cardboard calipers.
\end{quote}

\textbf{Example \#9.2.4: Confidence Interval for Paired Samples Using the
Formula}

\begin{quote}
A researcher wants to estimate the mean weight loss that people experience using a new program. She measures the weight of 6 randomly selected women before and after the weight loss program (see table \#9.2.1). Find a 90\% confidence interval for the mean the weight loss using the new program.

\textbf{Solution:}
\end{quote}

\begin{enumerate}
\def\labelenumi{\arabic{enumi}.}
\tightlist
\item
  State the random variables and the parameters in words.
\end{enumerate}

These were stated in example \#9.2.2, but are reproduced here for reference.

\begin{quote}
= weight of a woman after the weight loss program

= weight of a woman before the weight loss program

= mean weight of a woman after the weight loss program

= mean weight of a woman before the weight loss program
\end{quote}

\begin{enumerate}
\def\labelenumi{\arabic{enumi}.}
\setcounter{enumi}{1}
\tightlist
\item
  State and check the assumptions for the confidence interval
\end{enumerate}

\begin{quote}
The assumptions were stated and checked in example \#9.2.2.
\end{quote}

\begin{enumerate}
\def\labelenumi{\arabic{enumi}.}
\setcounter{enumi}{2}
\tightlist
\item
  Find the sample statistic and confidence interval
\end{enumerate}

\begin{quote}
Sample Statistics:

From example \#9.2.2

The confidence level is 90\%, so

There are six pairs, so the degrees of freedom are

Now look in table A.2. Go down the first column to 5, then over to the
column headed with 90\%.
\end{quote}

\begin{enumerate}
\def\labelenumi{\arabic{enumi}.}
\setcounter{enumi}{3}
\item
  Statistical Interpretation: There is a 90\% chance that contains the true mean difference in weight loss.
\item
  Real World Interpretation: The mean weight loss is between 18.9 and 24.1 pounds. (Note, the negative signs tell you that the first mean is less than the second mean, and thus a weight loss in this case.)
\end{enumerate}

\textbf{Example \#9.2.5: Confidence Interval for Paired Samples Using Technology}

\begin{quote}
The New Zealand Air Force purchased a batch of flight helmets. They then found out that the helmets didn't fit. In order to make sure that they order the correct size helmets, they measured the head size of recruits. To save money, they wanted to use cardboard calipers, but were not sure if they will be accurate enough. So they took 18 recruits and measured their heads with the cardboard calipers and also with metal calipers. The data in centimeters (cm) is in table \#9.2.3 ("NZ helmet size," 2013). Estimate the mean difference in measurements between the cardboard and metal calipers using a 95\% confidence interval.
\end{quote}

\textbf{Solution:}

\begin{enumerate}
\def\labelenumi{\arabic{enumi}.}
\tightlist
\item
  State the random variables and the parameters in words.
\end{enumerate}

These were stated in example \#9.2.3, but are reproduced here for reference.

\begin{quote}
= head measurement of recruit using cardboard caliper

= head measurement of recruit using metal caliper

= mean head measurement of recruit using cardboard caliper

= mean head measurement of recruit using metal caliper
\end{quote}

\begin{enumerate}
\def\labelenumi{\arabic{enumi}.}
\setcounter{enumi}{1}
\tightlist
\item
  State and check the assumptions for the hypothesis test
\end{enumerate}

\begin{quote}
The assumptions were stated and checked in example \#9.2.3.
\end{quote}

\begin{enumerate}
\def\labelenumi{\arabic{enumi}.}
\setcounter{enumi}{2}
\tightlist
\item
  Find the sample statistic and confidence interval
\end{enumerate}

\begin{quote}
Using the TI-83/84, put into L1 and into L2. Then go onto the name L3, and type . The calculator will now calculate the differences for you and put them in L3. Now go into STAT and move over to TESTS. Then chose TInterval. The setup for the calculator is in figure \#9.2.9.
\end{quote}

\begin{quote}
\textbf{Figure \#9.2.9: Setup for TInterval on Ti-83/84 Calculator}

\includegraphics[width=2.75in,height=1.86111in]{media/image218.png}

Once you press ENTER on Calculate you will see the result shown in
figure \#9.2.10.

\textbf{Figure \#9.2.10: Results of TInterval on TI-83/84 Calculator}

\includegraphics[width=2.75in,height=1.86111in]{media/image219.png}

Using R: the command is t.test(variable1, variable2, paired = TRUE, conf.level = C), where C is in decimal form. For this example the command would be

t.test(cardboard, metal, paired = TRUE, conf.level=0.95)

Paired t-test

data: cardboard and metal

t = 3.1854, df = 17, p-value = 0.005415

alternative hypothesis: true difference in means is not equal to 0

95 percent confidence interval:

0.5440163 2.6782060

sample estimates:

mean of the differences

1.611111

So
\end{quote}

\begin{enumerate}
\def\labelenumi{\arabic{enumi}.}
\setcounter{enumi}{3}
\item
  Statistical Interpretation: There is a 95\% chance that contains the true mean difference in head measurements between cardboard and metal calibers.
\item
  Real World Interpretation: The mean difference in head measurements between the cardboard and metal calibers is between 0.54 and 2.68 cm. This means that the cardboard calibers measure on average the head of a recruit to be between 0.54 and 2.68 cm more in diameter than the metal calibers. That makes it seem that the cardboard calibers are not measuring the same as the metal calibers. (The positive values on the confidence interval imply that the first mean is higher than the second mean.)
\end{enumerate}

Examples \#9.2.2 and \#9.2.4 use the same data set, but one is conducting a hypothesis test and the other is conducting a confidence interval. Notice that the hypothesis test's conclusion was to reject and say that there was a difference in the means, and the confidence interval does not contain the number 0. If the confidence interval did contain the number 0, then that would mean that the two means could be the same. Since the interval did not contain 0, then you could say that the means are different just as in the hypothesis test. This means that the hypothesis test and the confidence interval can produce the same interpretation. Do be careful though, you can run a hypothesis test with a particular significance level and a confidence interval with a confidence level that is not compatible with your significance level. This will mean that the conclusion from the confidence interval would not be the same as with a hypothesis test. So if you want to estimate the mean difference, then conduct a confidence interval. If you want to show that the means are different, then conduct a hypothesis test.

\textbf{\\
}

\hypertarget{homework-27}{%
\subsection{Homework}\label{homework-27}}

In each problem show all steps of the hypothesis test or confidence interval. If some of the assumptions are not met, note that the results of the test or interval may not be correct and then continue the process of the hypothesis test or confidence interval.

\begin{enumerate}
\def\labelenumi{\arabic{enumi}.}
\tightlist
\item
  The cholesterol level of patients who had heart attacks was measured two days after the heart attack and then again four days after the heart attack. The researchers want to see if the cholesterol level of patients who have heart attacks reduces as the time since their heart attack increases. The data is in table \#9.2.4 ("Cholesterol levels after," 2013). Do the data show that the mean cholesterol level of patients that have had a heart attack reduces as the time increases since their heart attack? Test at the 1\% level.
\end{enumerate}

\begin{quote}
\textbf{Table \#9.2.4: Cholesterol Levels (in mg/dL) of Heart Attack
Patients}
\end{quote}

\begin{longtable}[]{@{}lll@{}}
\toprule
Patient & Cholesterol Level Day 2 & Cholesterol Level Day 4\tabularnewline
\midrule
\endhead
1 & 270 & 218\tabularnewline
2 & 236 & 234\tabularnewline
3 & 210 & 214\tabularnewline
4 & 142 & 116\tabularnewline
5 & 280 & 200\tabularnewline
6 & 272 & 276\tabularnewline
7 & 160 & 146\tabularnewline
8 & 220 & 182\tabularnewline
9 & 226 & 238\tabularnewline
10 & 242 & 288\tabularnewline
11 & 186 & 190\tabularnewline
12 & 266 & 236\tabularnewline
13 & 206 & 244\tabularnewline
14 & 318 & 258\tabularnewline
15 & 294 & 240\tabularnewline
16 & 282 & 294\tabularnewline
17 & 234 & 220\tabularnewline
18 & 224 & 200\tabularnewline
19 & 276 & 220\tabularnewline
20 & 282 & 186\tabularnewline
21 & 360 & 352\tabularnewline
22 & 310 & 202\tabularnewline
23 & 280 & 218\tabularnewline
24 & 278 & 248\tabularnewline
25 & 288 & 278\tabularnewline
26 & 288 & 248\tabularnewline
27 & 244 & 270\tabularnewline
28 & 236 & 242\tabularnewline
\bottomrule
\end{longtable}

\begin{enumerate}
\def\labelenumi{\arabic{enumi}.}
\setcounter{enumi}{1}
\item
  The cholesterol level of patients who had heart attacks was measured two days after the heart attack and then again four days after the heart attack. The researchers want to see if the cholesterol level of patients who have heart attacks reduces as the time since their heart attack increases. The data is in table \#9.2.4 ("Cholesterol levels after," 2013). Calculate a 98\% confidence interval for the mean difference in cholesterol levels from day two to day four.
\item
  All Fresh Seafood is a wholesale fish company based on the east coast of the U.S. Catalina Offshore Products is a wholesale fish company based on the west coast of the U.S. Table \#9.2.5 contains prices from both companies for specific fish types ("Seafood online," 2013) ("Buy sushi grade," 2013). Do the data provide enough evidence to show that a west coast fish wholesaler is more expensive than an east coast wholesaler? Test at the 5\% level.
\end{enumerate}

\begin{quote}
\textbf{Table \#9.2.5: Wholesale Prices of Fish in Dollars}
\end{quote}

\begin{longtable}[]{@{}lll@{}}
\toprule
Fish & All Fresh Seafood Prices & Catalina Offshore Products Prices\tabularnewline
\midrule
\endhead
Cod & 19.99 & 17.99\tabularnewline
Tilapi & 6.00 & 13.99\tabularnewline
Farmed Salmon & 19.99 & 22.99\tabularnewline
Organic Salmon & 24.99 & 24.99\tabularnewline
Grouper Fillet & 29.99 & 19.99\tabularnewline
Tuna & 28.99 & 31.99\tabularnewline
Swordfish & 23.99 & 23.99\tabularnewline
Sea Bass & 32.99 & 23.99\tabularnewline
Striped Bass & 29.99 & 14.99\tabularnewline
\bottomrule
\end{longtable}

\begin{enumerate}
\def\labelenumi{\arabic{enumi}.}
\setcounter{enumi}{3}
\item
  All Fresh Seafood is a wholesale fish company based on the east coast of the U.S. Catalina Offshore Products is a wholesale fish company based on the west coast of the U.S. Table \#9.2.5 contains prices from both companies for specific fish types ("Seafood online," 2013) ("Buy sushi grade," 2013). Find a 95\% confidence interval for the mean difference in wholesale price between the east coast and west coast suppliers.
\item
  The British Department of Transportation studied to see if people avoid driving on Friday the 13\textsuperscript{th}. They did a traffic count on a Friday and then again on a Friday the 13\textsuperscript{th} at the same two locations ("Friday the 13th," 2013). The data for each location on the two different dates is in table \#9.2.6. Do the data show that on average fewer people drive on Friday the 13\textsuperscript{th}? Test at the 5\% level.
\end{enumerate}

\begin{quote}
\textbf{Table \#9.2.6: Traffic Count}
\end{quote}

\begin{longtable}[]{@{}lll@{}}
\toprule
Dates & 6th & 13th\tabularnewline
\midrule
\endhead
1990, July & 139246 & 138548\tabularnewline
1990, July & 134012 & 132908\tabularnewline
1991, September & 137055 & 136018\tabularnewline
1991, September & 133732 & 131843\tabularnewline
1991, December & 123552 & 121641\tabularnewline
1991, December & 121139 & 118723\tabularnewline
1992, March & 128293 & 125532\tabularnewline
1992, March & 124631 & 120249\tabularnewline
1992, November & 124609 & 122770\tabularnewline
1992, November & 117584 & 117263\tabularnewline
\bottomrule
\end{longtable}

\begin{enumerate}
\def\labelenumi{\arabic{enumi}.}
\setcounter{enumi}{5}
\item
  The British Department of Transportation studied to see if people avoid driving on Friday the 13\textsuperscript{th}. They did a traffic count on a Friday and then again on a Friday the 13\textsuperscript{th} at the same two locations ("Friday the 13th," 2013). The data for each location on the two different dates is in table \#9.2.6. Estimate the mean difference in traffic count between the 6\textsuperscript{th} and the 13\textsuperscript{th} using a 90\% level.
\item
  To determine if Reiki is an effective method for treating pain, a pilot study was carried out where a certified second-degree Reiki therapist provided treatment on volunteers. Pain was measured using a visual analogue scale (VAS) immediately before and after the Reiki treatment (Olson \& Hanson, 1997). The data is in table \#9.2.7. Do the data show that Reiki treatment reduces pain? Test at the 5\% level.
\end{enumerate}

\begin{quote}
\textbf{Table \#9.2.7: Pain Measures Before and After Reiki Treatment}
\end{quote}

\begin{longtable}[]{@{}ll@{}}
\toprule
VAS before & VAS after\tabularnewline
\midrule
\endhead
6 & 3\tabularnewline
2 & 1\tabularnewline
2 & 0\tabularnewline
9 & 1\tabularnewline
3 & 0\tabularnewline
3 & 2\tabularnewline
4 & 1\tabularnewline
5 & 2\tabularnewline
2 & 2\tabularnewline
3 & 0\tabularnewline
5 & 1\tabularnewline
1 & 0\tabularnewline
6 & 4\tabularnewline
6 & 1\tabularnewline
4 & 4\tabularnewline
4 & 1\tabularnewline
7 & 6\tabularnewline
2 & 1\tabularnewline
4 & 3\tabularnewline
8 & 8\tabularnewline
\bottomrule
\end{longtable}

\begin{enumerate}
\def\labelenumi{\arabic{enumi}.}
\setcounter{enumi}{7}
\item
  To determine if Reiki is an effective method for treating pain, a pilot study was carried out where a certified second-degree Reiki therapist provided treatment on volunteers. Pain was measured using a visual analogue scale (VAS) immediately before and after the Reiki treatment (Olson \& Hanson, 1997). The data is in table \#9.2.7. Compute a 90\% confidence level for the mean difference in VAS score from before and after Reiki treatment.
\item
  The female labor force participation rates (FLFPR) of women in randomly selected countries in 1990 and latest years of the 1990s are in table \#9.2.8 (Lim, 2002). Do the data show that the mean female labor force participation rate in 1990 is different from that in the latest years of the 1990s using a 5\% level of significance?
\end{enumerate}

\begin{quote}
\textbf{Table \#9.2.8: Female Labor Force Participation Rates}
\end{quote}

\begin{longtable}[]{@{}lll@{}}
\toprule
\begin{minipage}[b]{0.29\columnwidth}\raggedright
Region and country\strut
\end{minipage} & \begin{minipage}[b]{0.10\columnwidth}\raggedright
FLFPR

25-54

1990\strut
\end{minipage} & \begin{minipage}[b]{0.29\columnwidth}\raggedright
FLFPR

25-54

Latest year of 1990s\strut
\end{minipage}\tabularnewline
\midrule
\endhead
\begin{minipage}[t]{0.29\columnwidth}\raggedright
\strut
\end{minipage} & \begin{minipage}[t]{0.10\columnwidth}\raggedright
\strut
\end{minipage} & \begin{minipage}[t]{0.29\columnwidth}\raggedright
\strut
\end{minipage}\tabularnewline
\begin{minipage}[t]{0.29\columnwidth}\raggedright
\strut
\end{minipage} & \begin{minipage}[t]{0.10\columnwidth}\raggedright
\strut
\end{minipage} & \begin{minipage}[t]{0.29\columnwidth}\raggedright
\strut
\end{minipage}\tabularnewline
\begin{minipage}[t]{0.29\columnwidth}\raggedright
\strut
\end{minipage} & \begin{minipage}[t]{0.10\columnwidth}\raggedright
\strut
\end{minipage} & \begin{minipage}[t]{0.29\columnwidth}\raggedright
\strut
\end{minipage}\tabularnewline
\begin{minipage}[t]{0.29\columnwidth}\raggedright
Iran\strut
\end{minipage} & \begin{minipage}[t]{0.10\columnwidth}\raggedright
22.6\strut
\end{minipage} & \begin{minipage}[t]{0.29\columnwidth}\raggedright
12.5\strut
\end{minipage}\tabularnewline
\begin{minipage}[t]{0.29\columnwidth}\raggedright
Morocco\strut
\end{minipage} & \begin{minipage}[t]{0.10\columnwidth}\raggedright
41.4\strut
\end{minipage} & \begin{minipage}[t]{0.29\columnwidth}\raggedright
34.5\strut
\end{minipage}\tabularnewline
\begin{minipage}[t]{0.29\columnwidth}\raggedright
Qatar\strut
\end{minipage} & \begin{minipage}[t]{0.10\columnwidth}\raggedright
42.3\strut
\end{minipage} & \begin{minipage}[t]{0.29\columnwidth}\raggedright
46.5\strut
\end{minipage}\tabularnewline
\begin{minipage}[t]{0.29\columnwidth}\raggedright
Syrian Arab Republic\strut
\end{minipage} & \begin{minipage}[t]{0.10\columnwidth}\raggedright
25.6\strut
\end{minipage} & \begin{minipage}[t]{0.29\columnwidth}\raggedright
19.5\strut
\end{minipage}\tabularnewline
\begin{minipage}[t]{0.29\columnwidth}\raggedright
United Arab Emirates\strut
\end{minipage} & \begin{minipage}[t]{0.10\columnwidth}\raggedright
36.4\strut
\end{minipage} & \begin{minipage}[t]{0.29\columnwidth}\raggedright
39.7\strut
\end{minipage}\tabularnewline
\begin{minipage}[t]{0.29\columnwidth}\raggedright
Cape Verde\strut
\end{minipage} & \begin{minipage}[t]{0.10\columnwidth}\raggedright
46.7\strut
\end{minipage} & \begin{minipage}[t]{0.29\columnwidth}\raggedright
50.9\strut
\end{minipage}\tabularnewline
\begin{minipage}[t]{0.29\columnwidth}\raggedright
Ghana\strut
\end{minipage} & \begin{minipage}[t]{0.10\columnwidth}\raggedright
89.8\strut
\end{minipage} & \begin{minipage}[t]{0.29\columnwidth}\raggedright
90.0\strut
\end{minipage}\tabularnewline
\begin{minipage}[t]{0.29\columnwidth}\raggedright
Kenya\strut
\end{minipage} & \begin{minipage}[t]{0.10\columnwidth}\raggedright
82.1\strut
\end{minipage} & \begin{minipage}[t]{0.29\columnwidth}\raggedright
82.6\strut
\end{minipage}\tabularnewline
\begin{minipage}[t]{0.29\columnwidth}\raggedright
Lesotho\strut
\end{minipage} & \begin{minipage}[t]{0.10\columnwidth}\raggedright
51.9\strut
\end{minipage} & \begin{minipage}[t]{0.29\columnwidth}\raggedright
68.0\strut
\end{minipage}\tabularnewline
\begin{minipage}[t]{0.29\columnwidth}\raggedright
South Africa\strut
\end{minipage} & \begin{minipage}[t]{0.10\columnwidth}\raggedright
54.7\strut
\end{minipage} & \begin{minipage}[t]{0.29\columnwidth}\raggedright
61.7\strut
\end{minipage}\tabularnewline
\begin{minipage}[t]{0.29\columnwidth}\raggedright
Bangladesh\strut
\end{minipage} & \begin{minipage}[t]{0.10\columnwidth}\raggedright
73.5\strut
\end{minipage} & \begin{minipage}[t]{0.29\columnwidth}\raggedright
60.6\strut
\end{minipage}\tabularnewline
\begin{minipage}[t]{0.29\columnwidth}\raggedright
Malaysia\strut
\end{minipage} & \begin{minipage}[t]{0.10\columnwidth}\raggedright
49.0\strut
\end{minipage} & \begin{minipage}[t]{0.29\columnwidth}\raggedright
50.2\strut
\end{minipage}\tabularnewline
\begin{minipage}[t]{0.29\columnwidth}\raggedright
Mongolia\strut
\end{minipage} & \begin{minipage}[t]{0.10\columnwidth}\raggedright
84.7\strut
\end{minipage} & \begin{minipage}[t]{0.29\columnwidth}\raggedright
71.3\strut
\end{minipage}\tabularnewline
\begin{minipage}[t]{0.29\columnwidth}\raggedright
Myanmar\strut
\end{minipage} & \begin{minipage}[t]{0.10\columnwidth}\raggedright
72.1\strut
\end{minipage} & \begin{minipage}[t]{0.29\columnwidth}\raggedright
72.3\strut
\end{minipage}\tabularnewline
\begin{minipage}[t]{0.29\columnwidth}\raggedright
Argentina\strut
\end{minipage} & \begin{minipage}[t]{0.10\columnwidth}\raggedright
36.8\strut
\end{minipage} & \begin{minipage}[t]{0.29\columnwidth}\raggedright
54\strut
\end{minipage}\tabularnewline
\begin{minipage}[t]{0.29\columnwidth}\raggedright
Belize\strut
\end{minipage} & \begin{minipage}[t]{0.10\columnwidth}\raggedright
28.8\strut
\end{minipage} & \begin{minipage}[t]{0.29\columnwidth}\raggedright
42.5\strut
\end{minipage}\tabularnewline
\begin{minipage}[t]{0.29\columnwidth}\raggedright
Bolivia\strut
\end{minipage} & \begin{minipage}[t]{0.10\columnwidth}\raggedright
27.3\strut
\end{minipage} & \begin{minipage}[t]{0.29\columnwidth}\raggedright
69.8\strut
\end{minipage}\tabularnewline
\begin{minipage}[t]{0.29\columnwidth}\raggedright
Brazil\strut
\end{minipage} & \begin{minipage}[t]{0.10\columnwidth}\raggedright
51.1\strut
\end{minipage} & \begin{minipage}[t]{0.29\columnwidth}\raggedright
63.2\strut
\end{minipage}\tabularnewline
\begin{minipage}[t]{0.29\columnwidth}\raggedright
Colombia\strut
\end{minipage} & \begin{minipage}[t]{0.10\columnwidth}\raggedright
57.4\strut
\end{minipage} & \begin{minipage}[t]{0.29\columnwidth}\raggedright
72.7\strut
\end{minipage}\tabularnewline
\begin{minipage}[t]{0.29\columnwidth}\raggedright
Ecuador\strut
\end{minipage} & \begin{minipage}[t]{0.10\columnwidth}\raggedright
33.5\strut
\end{minipage} & \begin{minipage}[t]{0.29\columnwidth}\raggedright
64\strut
\end{minipage}\tabularnewline
\begin{minipage}[t]{0.29\columnwidth}\raggedright
Nicaragua\strut
\end{minipage} & \begin{minipage}[t]{0.10\columnwidth}\raggedright
50.1\strut
\end{minipage} & \begin{minipage}[t]{0.29\columnwidth}\raggedright
42.5\strut
\end{minipage}\tabularnewline
\begin{minipage}[t]{0.29\columnwidth}\raggedright
Uruguay\strut
\end{minipage} & \begin{minipage}[t]{0.10\columnwidth}\raggedright
59.5\strut
\end{minipage} & \begin{minipage}[t]{0.29\columnwidth}\raggedright
71.5\strut
\end{minipage}\tabularnewline
\begin{minipage}[t]{0.29\columnwidth}\raggedright
Albania\strut
\end{minipage} & \begin{minipage}[t]{0.10\columnwidth}\raggedright
77.4\strut
\end{minipage} & \begin{minipage}[t]{0.29\columnwidth}\raggedright
78.8\strut
\end{minipage}\tabularnewline
\begin{minipage}[t]{0.29\columnwidth}\raggedright
Uzbekistan\strut
\end{minipage} & \begin{minipage}[t]{0.10\columnwidth}\raggedright
79.6\strut
\end{minipage} & \begin{minipage}[t]{0.29\columnwidth}\raggedright
82.8\strut
\end{minipage}\tabularnewline
\bottomrule
\end{longtable}

\begin{enumerate}
\def\labelenumi{\arabic{enumi}.}
\setcounter{enumi}{9}
\item
  The female labor force participation rates of women in randomly selected countries in 1990 and latest years of the 1990s are in table \#9.2.8 (Lim, 2002). Estimate the mean difference in the female labor force participation rate in 1990 to latest years of the 1990s using a 95\% confidence level?
\item
  Table \#9.2.9 contains pulse rates collected from males, who are non-smokers but do drink alcohol ("Pulse rates before," 2013). The before pulse rate is before they exercised, and the after pulse rate was taken after the subject ran in place for one minute. Do the data indicate that the pulse rate before exercise is less than after exercise? Test at the 1\% level.
\end{enumerate}

\begin{quote}
\textbf{Table\#9.2.9: Pulse Rate of Males Before and After Exercise}
\end{quote}

\begin{longtable}[]{@{}ll@{}}
\toprule
Pulse before & Pulse after\tabularnewline
\midrule
\endhead
76 & 88\tabularnewline
56 & 110\tabularnewline
64 & 126\tabularnewline
50 & 90\tabularnewline
49 & 83\tabularnewline
68 & 136\tabularnewline
68 & 125\tabularnewline
88 & 150\tabularnewline
80 & 146\tabularnewline
78 & 168\tabularnewline
59 & 92\tabularnewline
60 & 104\tabularnewline
65 & 82\tabularnewline
76 & 150\tabularnewline
145 & 155\tabularnewline
84 & 140\tabularnewline
78 & 141\tabularnewline
85 & 131\tabularnewline
78 & 132\tabularnewline
\bottomrule
\end{longtable}

\begin{enumerate}
\def\labelenumi{\arabic{enumi}.}
\setcounter{enumi}{11}
\tightlist
\item
  Table \#9.2.9 contains pulse rates collected from males, who are non-smokers but do drink alcohol ("Pulse rates before," 2013). The before pulse rate is before they exercised, and the after pulse rate was taken after the subject ran in place for one minute. Compute a 98\% confidence interval for the mean difference in pulse rates from before and after exercise.
\end{enumerate}

\textbf{\\
}

\hypertarget{independent-samples-for-two-means}{%
\section{Independent Samples for Two Means}\label{independent-samples-for-two-means}}

This section will look at how to analyze when two samples are collected that are independent. As with all other hypothesis tests and confidence intervals, the process is the same though the formulas and assumptions are different. The only difference with the independent t-test, as opposed to the other tests that have been done, is that there are actually two different formulas to use depending on if a particular assumption is met or not.

\textbf{Hypothesis Test for Independent t-Test (2-Sample t-Test)}

\begin{enumerate}
\def\labelenumi{\arabic{enumi}.}
\tightlist
\item
  State the random variables and the parameters in words.
\end{enumerate}

\begin{quote}
= random variable 1

= random variable 2

= mean of random variable 1

= mean of random variable 2
\end{quote}

\begin{enumerate}
\def\labelenumi{\arabic{enumi}.}
\setcounter{enumi}{1}
\tightlist
\item
  State the null and alternative hypotheses and the level of significance
\end{enumerate}

\begin{quote}
The normal hypotheses would be

or

Also, state your level here.
\end{quote}

\begin{enumerate}
\def\labelenumi{\arabic{enumi}.}
\setcounter{enumi}{2}
\tightlist
\item
  State and check the assumptions for the hypothesis test
\end{enumerate}

\begin{enumerate}
\def\labelenumi{\alph{enumi}.}
\item
  A random sample of size is taken from population 1. A random sample of size is taken from population 2. Note: the samples do not need to be the same size, but the test is more robust if they are.
\item
  The two samples are independent.
\item
  Population 1 is normally distributed. Population 2 is normally distributed. Just as before, the t-test is fairly robust to the assumption if the sample size is large. This means that if this assumption isn't met, but your sample sizes are quite large (over 30), then the results of the t-test are valid.
\item
  The population variances are unknown and not assumed to be equal. The old assumption is that the variances are equal. However, this assumption is no longer an assumption that most statisticians use. This is because it isn't really realistic to assume that the variances are equal. So we will just assume the assumption of the variances being unknown and not assumed to be equal is true, and it will not be checked.
\end{enumerate}

\begin{enumerate}
\def\labelenumi{\arabic{enumi}.}
\setcounter{enumi}{3}
\tightlist
\item
  Find the sample statistic, test statistic, and p-value
\end{enumerate}

\begin{quote}
Sample Statistic:

Calculate

Test Statistic:

Since the assumption that isn't being satisfied, then

Usually , since

Degrees of freedom: (the Welch--Satterthwaite equation)

p-value:

Using the TI-83/84:

(Note: if , then lower limit is and upper limit is your test
statistic. If , then lower limit is your test statistic and the upper
limit is . If , then find the p-value for , and multiply by 2.)

Using R:

(Note: if , then use . If , then use . If , then find the p-value for , and multiply by 2.)
\end{quote}

\begin{enumerate}
\def\labelenumi{\arabic{enumi}.}
\setcounter{enumi}{4}
\tightlist
\item
  Conclusion
\end{enumerate}

\begin{quote}
This is where you write reject or fail to reject . The rule is: if the p-value \textless{} , then reject . If the p-value , then fail to reject
\end{quote}

\begin{enumerate}
\def\labelenumi{\arabic{enumi}.}
\setcounter{enumi}{5}
\tightlist
\item
  Interpretation
\end{enumerate}

\begin{quote}
This is where you interpret in real world terms the conclusion to the test. The conclusion for a hypothesis test is that you either have enough evidence to show is true, or you do not have enough evidence to show is true.
\end{quote}

\textbf{Confidence Interval for the Difference in Means from Two Independent Samples (2 Samp T-Int)}

The confidence interval for the difference in means has the same random variables and means and the same assumptions as the hypothesis test for independent samples. If you have already completed the hypothesis test, then you do not need to state them again. If you haven't completed the hypothesis test, then state the random variables and means and state and check the assumptions before completing the confidence interval step.

\begin{enumerate}
\def\labelenumi{\arabic{enumi}.}
\tightlist
\item
  Find the sample statistic and confidence interval
\end{enumerate}

\begin{quote}
Sample Statistic:

Calculate

Confidence Interval:

The confidence interval estimate of the difference is

Since the assumption that isn't being satisfied, then

where is the critical value with degrees of freedom:

Degrees of freedom: (the Welch--Satterthwaite equation)
\end{quote}

\begin{enumerate}
\def\labelenumi{\arabic{enumi}.}
\setcounter{enumi}{1}
\item
  Statistical Interpretation: In general this looks like, ``there is a C\% chance that contains the true mean difference.''
\item
  Real World Interpretation: This is where you state what interval contains the true difference in means, though often you state how much more (or less) the first mean is from the second mean.
\end{enumerate}

The critical value is a value from the Student's t-distribution. Since a confidence interval is found by adding and subtracting a margin of error amount from the difference in sample means, and the interval has a probability of containing the true difference in means, then you can think of this as the statement . To find the critical value you use table A.2 in the Appendix.

\textbf{How to check the assumptions of two sample t-test and confidence interval:}

In order for the t-test or confidence interval to be valid, the assumptions of the test must be true. So whenever you run a t-test or confidence interval, you must make sure the assumptions are true. So you need to check them. Here is how you do this:

\begin{enumerate}
\def\labelenumi{\arabic{enumi}.}
\item
  For the random sample assumption, describe how you took the two samples. Make sure your sampling technique is random for both samples.
\item
  For the independent assumption, describe how they are independent samples.
\item
  For the assumption about each population being normally distributed, remember the process of assessing normality from chapter 6. Make sure you assess each sample separately.
\item
  You do not need to check the equal variance assumption since it is not being assumed.
\end{enumerate}

\textbf{Example \#9.3.1: Hypothesis Test for Two Means}

\begin{quote}
The cholesterol level of patients who had heart attacks was measured two days after the heart attack. The researchers want to see if patients who have heart attacks have higher cholesterol levels over healthy people, so they also measured the cholesterol level of healthy adults who show no signs of heart disease. The data is in table \#9.3.1 ("Cholesterol levels after," 2013). Do the data show that people who have had heart attacks have higher cholesterol levels over patients that have not had heart attacks? Test at the 1\% level.

\textbf{Table \#9.3.1: Cholesterol Levels in mg/dL}
\end{quote}

\begin{longtable}[]{@{}ll@{}}
\toprule
\begin{minipage}[b]{0.47\columnwidth}\raggedright
Cholesterol Level of Heart Attack
Patients\strut
\end{minipage} & \begin{minipage}[b]{0.47\columnwidth}\raggedright
\begin{quote}
Cholesterol Level of Healthy
Individual
\end{quote}\strut
\end{minipage}\tabularnewline
\midrule
\endhead
\begin{minipage}[t]{0.47\columnwidth}\raggedright
270\strut
\end{minipage} & \begin{minipage}[t]{0.47\columnwidth}\raggedright
196\strut
\end{minipage}\tabularnewline
\begin{minipage}[t]{0.47\columnwidth}\raggedright
236\strut
\end{minipage} & \begin{minipage}[t]{0.47\columnwidth}\raggedright
232\strut
\end{minipage}\tabularnewline
\begin{minipage}[t]{0.47\columnwidth}\raggedright
210\strut
\end{minipage} & \begin{minipage}[t]{0.47\columnwidth}\raggedright
200\strut
\end{minipage}\tabularnewline
\begin{minipage}[t]{0.47\columnwidth}\raggedright
142\strut
\end{minipage} & \begin{minipage}[t]{0.47\columnwidth}\raggedright
242\strut
\end{minipage}\tabularnewline
\begin{minipage}[t]{0.47\columnwidth}\raggedright
280\strut
\end{minipage} & \begin{minipage}[t]{0.47\columnwidth}\raggedright
206\strut
\end{minipage}\tabularnewline
\begin{minipage}[t]{0.47\columnwidth}\raggedright
272\strut
\end{minipage} & \begin{minipage}[t]{0.47\columnwidth}\raggedright
178\strut
\end{minipage}\tabularnewline
\begin{minipage}[t]{0.47\columnwidth}\raggedright
160\strut
\end{minipage} & \begin{minipage}[t]{0.47\columnwidth}\raggedright
184\strut
\end{minipage}\tabularnewline
\begin{minipage}[t]{0.47\columnwidth}\raggedright
220\strut
\end{minipage} & \begin{minipage}[t]{0.47\columnwidth}\raggedright
198\strut
\end{minipage}\tabularnewline
\begin{minipage}[t]{0.47\columnwidth}\raggedright
226\strut
\end{minipage} & \begin{minipage}[t]{0.47\columnwidth}\raggedright
160\strut
\end{minipage}\tabularnewline
\begin{minipage}[t]{0.47\columnwidth}\raggedright
242\strut
\end{minipage} & \begin{minipage}[t]{0.47\columnwidth}\raggedright
182\strut
\end{minipage}\tabularnewline
\begin{minipage}[t]{0.47\columnwidth}\raggedright
186\strut
\end{minipage} & \begin{minipage}[t]{0.47\columnwidth}\raggedright
182\strut
\end{minipage}\tabularnewline
\begin{minipage}[t]{0.47\columnwidth}\raggedright
266\strut
\end{minipage} & \begin{minipage}[t]{0.47\columnwidth}\raggedright
198\strut
\end{minipage}\tabularnewline
\begin{minipage}[t]{0.47\columnwidth}\raggedright
206\strut
\end{minipage} & \begin{minipage}[t]{0.47\columnwidth}\raggedright
182\strut
\end{minipage}\tabularnewline
\begin{minipage}[t]{0.47\columnwidth}\raggedright
318\strut
\end{minipage} & \begin{minipage}[t]{0.47\columnwidth}\raggedright
238\strut
\end{minipage}\tabularnewline
\begin{minipage}[t]{0.47\columnwidth}\raggedright
294\strut
\end{minipage} & \begin{minipage}[t]{0.47\columnwidth}\raggedright
198\strut
\end{minipage}\tabularnewline
\begin{minipage}[t]{0.47\columnwidth}\raggedright
282\strut
\end{minipage} & \begin{minipage}[t]{0.47\columnwidth}\raggedright
188\strut
\end{minipage}\tabularnewline
\begin{minipage}[t]{0.47\columnwidth}\raggedright
234\strut
\end{minipage} & \begin{minipage}[t]{0.47\columnwidth}\raggedright
166\strut
\end{minipage}\tabularnewline
\begin{minipage}[t]{0.47\columnwidth}\raggedright
224\strut
\end{minipage} & \begin{minipage}[t]{0.47\columnwidth}\raggedright
204\strut
\end{minipage}\tabularnewline
\begin{minipage}[t]{0.47\columnwidth}\raggedright
276\strut
\end{minipage} & \begin{minipage}[t]{0.47\columnwidth}\raggedright
182\strut
\end{minipage}\tabularnewline
\begin{minipage}[t]{0.47\columnwidth}\raggedright
282\strut
\end{minipage} & \begin{minipage}[t]{0.47\columnwidth}\raggedright
178\strut
\end{minipage}\tabularnewline
\begin{minipage}[t]{0.47\columnwidth}\raggedright
360\strut
\end{minipage} & \begin{minipage}[t]{0.47\columnwidth}\raggedright
212\strut
\end{minipage}\tabularnewline
\begin{minipage}[t]{0.47\columnwidth}\raggedright
310\strut
\end{minipage} & \begin{minipage}[t]{0.47\columnwidth}\raggedright
164\strut
\end{minipage}\tabularnewline
\begin{minipage}[t]{0.47\columnwidth}\raggedright
280\strut
\end{minipage} & \begin{minipage}[t]{0.47\columnwidth}\raggedright
230\strut
\end{minipage}\tabularnewline
\begin{minipage}[t]{0.47\columnwidth}\raggedright
278\strut
\end{minipage} & \begin{minipage}[t]{0.47\columnwidth}\raggedright
186\strut
\end{minipage}\tabularnewline
\begin{minipage}[t]{0.47\columnwidth}\raggedright
288\strut
\end{minipage} & \begin{minipage}[t]{0.47\columnwidth}\raggedright
162\strut
\end{minipage}\tabularnewline
\begin{minipage}[t]{0.47\columnwidth}\raggedright
288\strut
\end{minipage} & \begin{minipage}[t]{0.47\columnwidth}\raggedright
182\strut
\end{minipage}\tabularnewline
\begin{minipage}[t]{0.47\columnwidth}\raggedright
244\strut
\end{minipage} & \begin{minipage}[t]{0.47\columnwidth}\raggedright
218\strut
\end{minipage}\tabularnewline
\begin{minipage}[t]{0.47\columnwidth}\raggedright
236\strut
\end{minipage} & \begin{minipage}[t]{0.47\columnwidth}\raggedright
170\strut
\end{minipage}\tabularnewline
\begin{minipage}[t]{0.47\columnwidth}\raggedright
\strut
\end{minipage} & \begin{minipage}[t]{0.47\columnwidth}\raggedright
200\strut
\end{minipage}\tabularnewline
\begin{minipage}[t]{0.47\columnwidth}\raggedright
\strut
\end{minipage} & \begin{minipage}[t]{0.47\columnwidth}\raggedright
176\strut
\end{minipage}\tabularnewline
\bottomrule
\end{longtable}

\begin{quote}
\textbf{Solution:}
\end{quote}

\begin{enumerate}
\def\labelenumi{\arabic{enumi}.}
\tightlist
\item
  State the random variables and the parameters in words.
\end{enumerate}

\begin{quote}
= Cholesterol level of patients who had a heart attack

= Cholesterol level of healthy individuals

= mean cholesterol level of patients who had a heart attack

= mean cholesterol level of healthy individuals
\end{quote}

\begin{enumerate}
\def\labelenumi{\arabic{enumi}.}
\setcounter{enumi}{1}
\tightlist
\item
  State the null and alternative hypotheses and the level of significance
\end{enumerate}

\begin{quote}
The normal hypotheses would be

or
\end{quote}

\begin{enumerate}
\def\labelenumi{\arabic{enumi}.}
\setcounter{enumi}{2}
\tightlist
\item
  State and check the assumptions for the hypothesis test
\end{enumerate}

\begin{enumerate}
\def\labelenumi{\alph{enumi}.}
\item
  A random sample of 28 cholesterol levels of patients who had a heart attack is taken. A random sample of 30 cholesterol levels of healthy individuals is taken. The problem does not state if either sample was randomly selected. So this assumption may not be valid.
\item
  The two samples are independent. This is because either they were dealing with patients who had heart attacks or healthy individuals.
\item
  Population of all cholesterol levels of patients who had a heart attack is normally distributed. Population of all cholesterol levels of healthy individuals is normally distributed.

  Patients who had heart attacks:

  \textbf{Figure \#9.3.1: Histogram of Cholesterol Levels of Patients who
  had Heart Attacks}

  \includegraphics[width=2.56944in,height=2.56944in]{media/image279.emf}

  This looks somewhat bell shaped.

  \textbf{Figure \#9.3.2: Modified Box Plot of Cholesterol Levels of
  Patients who had Heart Attacks}

  \includegraphics[width=2.56944in,height=2.56944in]{media/image280.emf}

  There are no outliers.

  \textbf{Figure \#9.3.3: Normal Quantile Plot of Cholesterol Levels of
  Patients who had Heart Attacks}

  \includegraphics[width=2.43056in,height=2.43056in]{media/image281.emf}

  This looks somewhat linear.

  So, the population of all cholesterol levels of patients who had
  heart attacks is probably somewhat normally distributed.

  Healthy individuals:

  \textbf{Figure \#9.3.4: Histogram of Cholesterol Levels of Healthy
  Individuals}

  \includegraphics[width=2.23611in,height=2.23611in]{media/image282.emf}

  This does not look bell shaped.

  \textbf{Figure \#9.3.5: Modified Box Plot of Cholesterol Levels of Healthy
  Individuals}

  \includegraphics[width=2.125in,height=2.125in]{media/image283.emf}

  There are no outliers.

  \textbf{Figure \#9.3.6: Normal Quantile Plot of Cholesterol Levels of
  Healthy Individuals}

  \includegraphics[width=2.31944in,height=2.31944in]{media/image284.emf}

  This doesn't look linear.

  So, the population of all cholesterol levels of healthy individuals
  is probably not normally distributed.

  This assumption is not valid for the second sample. Since the sample
  is fairly large, and the t-test is robust, it may not be an issue.
  However, just realize that the conclusions of the test may not be
  valid.
\end{enumerate}

\begin{enumerate}
\def\labelenumi{\arabic{enumi}.}
\setcounter{enumi}{3}
\tightlist
\item
  Find the sample statistic, test statistic, and p-value
\end{enumerate}

\begin{quote}
Sample Statistic:

Test Statistic:

Degrees of freedom: (the Welch--Satterthwaite equation)

p-value:

Using TI-83/84:

Using R:

Using Technology: Using the TI-83/84:
\end{quote}

\textbf{Figure \#9.3.7: Setup for 2-SampTTest on TI-83/84 Calculator}

\includegraphics[width=2.75in,height=1.86111in]{media/image292.png}

Note: the Pooled question on the calculator is for whether you are assuming the variances are equal. Since this assumption is not being made, then the answer to this question is no. Pooled means that you assume the variances are equal and can pool the sample variances together.

\textbf{Figure \#9.3.8 Results for 2-SampTTest on TI-83/84 Calculator}

\includegraphics[width=2.40278in,height=1.80581in]{media/image293.png}\includegraphics[width=2.43056in,height=1.82669in]{media/image294.png}

Using R: command in general: t.test(variable1, variable2, alternative =
"less" or "greater")

For this example, the R command is:

t.test(heartattack, healthy, alternative="greater")

Welch Two Sample t-test

data: heartattack and healthy

t = 6.1452, df = 37.675, p-value = 1.86e-07

alternative hypothesis: true difference in means is greater than 0

95 percent confidence interval:

44.1124 Inf

sample estimates:

mean of x mean of y

253.9286 193.1333

The test statistic is t = 6.1452. The p-value is

\begin{enumerate}
\def\labelenumi{\arabic{enumi}.}
\setcounter{enumi}{4}
\tightlist
\item
  Conclusion
\end{enumerate}

\begin{quote}
Reject since the p-value \textless{} .
\end{quote}

\begin{enumerate}
\def\labelenumi{\arabic{enumi}.}
\setcounter{enumi}{5}
\tightlist
\item
  Interpretation
\end{enumerate}

\begin{quote}
This is enough evidence to show that patients who have had heart attacks have higher cholesterol level on average from healthy individuals. (Though do realize that some of assumptions are not valid, so this interpretation may be invalid.)
\end{quote}

\textbf{Example \#9.3.2: Confidence Interval for }

\begin{quote}
The cholesterol level of patients who had heart attacks was measured two days after the heart attack. The researchers want to see if patients who have heart attacks have higher cholesterol levels over healthy people, so they also measured the cholesterol level of healthy adults who show no signs of heart disease. The data is in table \#9.3.1 ("Cholesterol levels after," 2013). Find a 99\% confidence interval for the mean difference in cholesterol levels between heart attack patients and healthy individuals.

\textbf{Solution:}
\end{quote}

\begin{enumerate}
\def\labelenumi{\arabic{enumi}.}
\tightlist
\item
  State the random variables and the parameters in words.
\end{enumerate}

These were stated in example \#9.3.1, but are reproduced here for reference.

\begin{quote}
= Cholesterol level of patients who had a heart attack

= Cholesterol level of healthy individuals

= mean cholesterol level of patients who had a heart attack

= mean cholesterol level of healthy individuals
\end{quote}

\begin{enumerate}
\def\labelenumi{\arabic{enumi}.}
\setcounter{enumi}{1}
\tightlist
\item
  State and check the assumptions for the hypothesis test
\end{enumerate}

\begin{quote}
The assumptions were stated and checked in example \#9.3.1.
\end{quote}

\begin{enumerate}
\def\labelenumi{\arabic{enumi}.}
\setcounter{enumi}{2}
\tightlist
\item
  Find the sample statistic and confidence interval
\end{enumerate}

\begin{quote}
Sample Statistic:

Test Statistic:

Degrees of freedom: (the Welch--Satterthwaite equation)
\end{quote}

Since this df is not in the table, round to the nearest whole number.

\begin{quote}
Using Technology:

Using TI-83/84:

\textbf{Figure \#9.3.9: Setup for 2-SampTInt on TI-83/84 Calculator}
\end{quote}

\includegraphics[width=2.75in,height=1.86111in]{media/image310.png}

Note: the Pooled question on the calculator is for whether you are assuming the variances are equal. Since this assumption is not being made, then the answer to this question is no. Pooled means that you assume the variances are equal and can pool the sample variances together.

\textbf{Figure \#9.3.10: Results for 2-SampTInt on TI-83/84 Calculator}

\begin{quote}
\includegraphics[width=2.75in,height=1.86111in]{media/image311.png}

\includegraphics[width=2.75in,height=1.86111in]{media/image312.png}
\end{quote}

Using R: the commands is t.test(variable1, variable2, conf.level=C), where C is in decimal form.

For this example, the command is

t.test(heartattack, healthy, conf.level=.99)

Output:

Welch Two Sample t-test

data: heartattack and healthy

t = 6.1452, df = 37.675, p-value = 3.721e-07

alternative hypothesis: true difference in means is not equal to 0

99 percent confidence interval:

33.95750 87.63298

sample estimates:

mean of x mean of y

253.9286 193.1333

The confidence interval is .

\begin{enumerate}
\def\labelenumi{\arabic{enumi}.}
\setcounter{enumi}{3}
\item
  Statistical Interpretation: There is a 99\% chance that contains the true difference in means.
\item
  Real World Interpretation: The mean cholesterol level for patients who had heart attacks is anywhere from 32.66 mg/dL to 85,72 mg/dL more than the mean cholesterol level for healthy patients. (Though do realize that many of assumptions are not valid, so this interpretation may be invalid.)
\end{enumerate}

If you do assume that the variances are equal, that is , then the test statistic is:

\begin{quote}
The Degrees of Freedom is:
\end{quote}

The confidence interval if you do assume that has been met, is

\begin{quote}
Degrees of Freedom:

is the critical value where
\end{quote}

To show that the variances are equal, just show that the ratio of your sample variances is not unusual (probability is greater than 0.05). In other words, make sure the following is true.

\begin{quote}
(or so that the larger variance is in the numerator). This probability is from an F-distribution. To find the probability on the TI-83/84 calculator use . To find the probability on R, use .

Note: the F-distribution is very sensitive to the normal distribution. A better test for equal variances is Levene's test, though it is more complicated. It is best to do Levene's test when using statistical software (such as SPSS or Minitab) to perform the two-sample independent t-test.
\end{quote}

\textbf{Example \#9.3.3: Hypothesis Test for Two Means}

\begin{quote}
The amount of sodium in beef hotdogs was measured. In addition, the amount of sodium in poultry hotdogs was also measured ("SOCR 012708 id," 2013). The data is in table \#9.3.2. Is there enough evidence to show that beef has less sodium on average than poultry hotdogs? Use a 5\% level of significance.

\textbf{Table \#9.3.2: Hotdog Data}
\end{quote}

\begin{longtable}[]{@{}ll@{}}
\toprule
Sodium in Beef Hotdogs & Sodium in Poultry Hotdogs\tabularnewline
\midrule
\endhead
495 & 430\tabularnewline
477 & 375\tabularnewline
425 & 396\tabularnewline
322 & 383\tabularnewline
482 & 387\tabularnewline
587 & 542\tabularnewline
370 & 359\tabularnewline
322 & 357\tabularnewline
479 & 528\tabularnewline
375 & 513\tabularnewline
330 & 426\tabularnewline
300 & 513\tabularnewline
386 & 358\tabularnewline
401 & 581\tabularnewline
645 & 588\tabularnewline
440 & 522\tabularnewline
317 & 545\tabularnewline
319 & 430\tabularnewline
298 & 375\tabularnewline
253 & 396\tabularnewline
\bottomrule
\end{longtable}

\textbf{\\
}

\begin{quote}
\textbf{Solution:}
\end{quote}

\begin{enumerate}
\def\labelenumi{\arabic{enumi}.}
\tightlist
\item
  State the random variables and the parameters in words.
\end{enumerate}

\begin{quote}
= sodium level in beef hotdogs

= sodium level in poultry hotdogs

= mean sodium level in beef hotdogs

= mean sodium level in poultry hotdogs
\end{quote}

\begin{enumerate}
\def\labelenumi{\arabic{enumi}.}
\setcounter{enumi}{1}
\tightlist
\item
  State the null and alternative hypotheses and the level of
  significance
\end{enumerate}

\begin{quote}
The normal hypotheses would be

or
\end{quote}

\begin{enumerate}
\def\labelenumi{\arabic{enumi}.}
\setcounter{enumi}{2}
\tightlist
\item
  State and check the assumptions for the hypothesis test
\end{enumerate}

\begin{enumerate}
\def\labelenumi{\alph{enumi}.}
\item
  A random sample of 20 sodium levels in beef hotdogs is taken. A
  random sample of 20 sodium levels in poultry hotdogs. The problem
  does not state if either sample was randomly selected. So this
  assumption may not be valid.
\item
  The two samples are independent since these are different types of
  hotdogs.
\item
  Population of all sodium levels in beef hotdogs is normally
  distributed. Population of all sodium levels in poultry hotdogs is
  normally distributed.

  Beef Hotdogs:

  \textbf{Figure \#9.3.11: Histogram of Sodium Levels in Beef Hotdogs}

  \includegraphics[width=3.04167in,height=3.04167in]{media/image338.emf}

  This looks somewhat bell shaped.

  \textbf{Figure \#9.3.12: Modified Box Plot of Sodium Levels in Beef
  Hotdogs}

  \includegraphics[width=2.13889in,height=2.13889in]{media/image339.emf}

  There are no outliers.

  \textbf{Figure \#9.3.13: Normal Quantile Plot of Sodium Levels in Beef
  Hotdogs}

  \includegraphics[width=2.27778in,height=2.27778in]{media/image340.emf}

  This looks somewhat linear.

  So, the population of all sodium levels in beef hotdogs may be
  normally distributed.

  Poultry Hotdogs:

  \textbf{Figure \#9.3.14: Histogram of Sodium Levels in Poultry Hotdogs}

  \includegraphics[width=2.38889in,height=2.38889in]{media/image341.emf}

  This does not look bell shaped.

  \textbf{Figure \#9.3.15: Modified Box Plot of Sodium Levels in Poultry
  Hotdogs}

  \includegraphics[width=2.58333in,height=2.58333in]{media/image342.emf}

  There are no outliers.

  \textbf{Figure \#9.3.16: Normal Quantile Plot of Sodium Levels in Poultry
  Hotdogs}

  \includegraphics[width=3.04167in,height=3.04167in]{media/image343.emf}

  This does not look linear.

  So, the population of all sodium levels in poultry hotdogs is
  probably not normally distributed.

  This assumption is not valid. Since the samples are fairly large,
  and the t-test is robust, it may not be a large issue. However, just
  realize that the conclusions of the test may not be valid.
\item
  The population variances are equal, i.e. .

  Using TI-83/84:

  Using R:

  So you can say that these variances are equal.
\end{enumerate}

\begin{enumerate}
\def\labelenumi{\arabic{enumi}.}
\setcounter{enumi}{3}
\tightlist
\item
  Find the sample statistic, test statistic, and p-value
\end{enumerate}

\begin{quote}
Sample Statistic:

Test Statistic:

The assumption has been met, so

Though you should try to do the calculations in the problem so you
don't create round off error.

p-value:

Using TI-83/84:

Using R:
\end{quote}

Using technology to find the t and p-value:

Using TI-83/84:

\textbf{Figure \#9.3.17: Setup for 2-SampTTest on TI-83/84 Calculator}

\includegraphics[width=2.75in,height=1.86111in]{media/image355.png}

Note: the Pooled question on the calculator is for whether you are using
the pooled standard deviation or not. In this example, the pooled
standard deviation was used since you are assuming the variances are
equal. That is why the answer to the question is Yes.

\textbf{Figure \#9.3.18: Results for 2-SampTTest on TI-83/84 Calculator}

\includegraphics[width=2.75in,height=1.86111in]{media/image356.png}\includegraphics[width=2.75in,height=1.86111in]{media/image357.png}

Using R: the command is t.test(variable1, variable2,
alternative="less" or "greater")

For this example, the command is

t.test(beef, poultry, alternative="less", equalvar=TRUE)

Welch Two Sample t-test

data: beef and poultry

t = -1.6783, df = 36.115, p-value = 0.05096

alternative hypothesis: true difference in means is less than 0

95 percent confidence interval:

-Inf 0.2875363

sample estimates:

mean of x mean of y

401.15 450.20

The and the p-value = 0.05096.

\begin{enumerate}
\def\labelenumi{\arabic{enumi}.}
\setcounter{enumi}{4}
\tightlist
\item
  Conclusion
\end{enumerate}

\begin{quote}
Fail to reject since the p-value \textgreater{} .
\end{quote}

\begin{enumerate}
\def\labelenumi{\arabic{enumi}.}
\setcounter{enumi}{5}
\tightlist
\item
  Interpretation
\end{enumerate}

\begin{quote}
This is not enough evidence to show that beef hotdogs have less sodium
than poultry hotdogs. (Though do realize that many of assumptions are
not valid, so this interpretation may be invalid.)
\end{quote}

\textbf{Example \#9.3.4: Confidence Interval for }

\begin{quote}
The amount of sodium in beef hotdogs was measured. In addition, the
amount of sodium in poultry hotdogs was also measured ("SOCR 012708
id," 2013). The data is in table \#9.3.2. Find a 95\% confidence
interval for the mean difference in sodium levels between beef and
poultry hotdogs.

\textbf{Solution:}
\end{quote}

\begin{enumerate}
\def\labelenumi{\arabic{enumi}.}
\item
  State the random variables and the parameters in words.

  These were stated in example \#9.3.1, but are reproduced here for
  reference.
\end{enumerate}

\begin{quote}
= sodium level in beef hotdogs

= sodium level in poultry hotdogs

= mean sodium level in beef hotdogs

= mean sodium level in poultry hotdogs
\end{quote}

\begin{enumerate}
\def\labelenumi{\arabic{enumi}.}
\setcounter{enumi}{1}
\tightlist
\item
  State and check the assumptions for the hypothesis test
\end{enumerate}

\begin{quote}
The assumptions were stated and checked in example \#9.3.3.
\end{quote}

\begin{enumerate}
\def\labelenumi{\arabic{enumi}.}
\setcounter{enumi}{2}
\tightlist
\item
  Find the sample statistic and confidence interval
\end{enumerate}

\begin{quote}
Sample Statistic:

Confidence Interval:

The confidence interval estimate of the difference is

The assumption has been met, so

Though you should try to do the calculations in the formula for E so
you don't create round off error.

Using technology:

Using the TI-83/84:
\end{quote}

\textbf{Figure \#9.3.19: Setup for 2-SampTInt on TI-83/84 Calculator}

\begin{quote}
\includegraphics[width=2.75in,height=1.86111in]{media/image373.png}
\end{quote}

Note: the Pooled question on the calculator is for whether you are using
the pooled standard deviation or not. In this example, the pooled
standard deviation was used since you are assuming the variances are
equal. That is why the answer to the question is Yes.

\textbf{\\
}

\textbf{Figure \#9.3.20: Results for 2-SampTInt on TI-83/84 Calculator}

\begin{quote}
\includegraphics[width=2.75in,height=1.86111in]{media/image374.png}

\includegraphics[width=2.75in,height=1.86111in]{media/image375.png}

Using R: the command is t.test(variable1, variable2, equalvar=TRUE,
conf.level=C), where C is in decimal form.

For this example, the command is

t.test(beef, poultry, conf.level=.95, equalvar=TRUE)

Welch Two Sample t-test

data: beef and poultry

t = -1.6783, df = 36.115, p-value = 0.1019

alternative hypothesis: true difference in means is not equal to 0

95 percent confidence interval:

-108.31592 10.21592

sample estimates:

mean of x mean of y

401.15 450.20

The confidence interval is .
\end{quote}

\begin{enumerate}
\def\labelenumi{\arabic{enumi}.}
\setcounter{enumi}{3}
\item
  Statistical Interpretation: There is a 95\% chance that contains the
  true difference in means.
\item
  Real World Interpretation: The mean sodium level of beef hotdogs is
  anywhere from 108.20 g less than the mean sodium level of poultry
  hotdogs to 10.10 g more. (The negative sign on the lower limit
  implies that the first mean is less than the second mean. The
  positive sign on the upper limit implies that the first mean is
  greater than the second mean.)

  Realize that many of assumptions are not valid in this example, so
  the interpretation may be invalid.
\end{enumerate}

\hypertarget{homework-28}{%
\subsection{Homework}\label{homework-28}}

In each problem show all steps of the hypothesis test or confidence interval. If some of the assumptions are not met, note that the results of the test or interval may not be correct and then continue the process of the hypothesis test or confidence interval. Unless directed by your instructor, do not assume the variances are equal (except in problems 11 through 16).

\begin{enumerate}
\def\labelenumi{\arabic{enumi}.}
\tightlist
\item
  The income of males in each state of the United States, including the District of Columbia and Puerto Rico, are given in table \#9.3.3, and the income of females is given in table \#9.3.4 ("Median income of," 2013). Is there enough evidence to show that the mean income of males is more than of females? Test at the 1\% level.
\end{enumerate}

\begin{quote}
\textbf{Table \#9.3.3: Data of Income for Males}
\end{quote}

\begin{longtable}[]{@{}lllllll@{}}
\toprule
\$42,951 & \$52,379 & \$42,544 & \$37,488 & \$49,281 & \$50,987 & \$60,705\tabularnewline
\midrule
\endhead
\$50,411 & \$66,760 & \$40,951 & \$43,902 & \$45,494 & \$41,528 & \$50,746\tabularnewline
\$45,183 & \$43,624 & \$43,993 & \$41,612 & \$46,313 & \$43,944 & \$56,708\tabularnewline
\$60,264 & \$50,053 & \$50,580 & \$40,202 & \$43,146 & \$41,635 & \$42,182\tabularnewline
\$41,803 & \$53,033 & \$60,568 & \$41,037 & \$50,388 & \$41,950 & \$44,660\tabularnewline
\$46,176 & \$41,420 & \$45,976 & \$47,956 & \$22,529 & \$48,842 & \$41,464\tabularnewline
\$40,285 & \$41,309 & \$43,160 & \$47,573 & \$44,057 & \$52,805 & \$53,046\tabularnewline
\$42,125 & \$46,214 & \$51,630 & & & &\tabularnewline
\bottomrule
\end{longtable}

\begin{quote}
\textbf{Table \#9.3.4: Data of Income for Females}
\end{quote}

\begin{longtable}[]{@{}llllllll@{}}
\toprule
\$31,862 & \$40,550 & \$36,048 & \$30,752 & \$41,817 & \$40,236 & \$47,476 & \$40,500\tabularnewline
\midrule
\endhead
\$60,332 & \$33,823 & \$35,438 & \$37,242 & \$31,238 & \$39,150 & \$34,023 & \$33,745\tabularnewline
\$33,269 & \$32,684 & \$31,844 & \$34,599 & \$48,748 & \$46,185 & \$36,931 & \$40,416\tabularnewline
\$29,548 & \$33,865 & \$31,067 & \$33,424 & \$35,484 & \$41,021 & \$47,155 & \$32,316\tabularnewline
\$42,113 & \$33,459 & \$32,462 & \$35,746 & \$31,274 & \$36,027 & \$37,089 & \$22,117\tabularnewline
\$41,412 & \$31,330 & \$31,329 & \$33,184 & \$35,301 & \$32,843 & \$38,177 & \$40,969\tabularnewline
\$40,993 & \$29,688 & \$35,890 & \$34,381 & & & &\tabularnewline
\bottomrule
\end{longtable}

\begin{enumerate}
\def\labelenumi{\arabic{enumi}.}
\setcounter{enumi}{1}
\item
  The income of males in each state of the United States, including the District of Columbia and Puerto Rico, are given in table \#9.3.3, and the income of females is given in table \#9.3.4 ("Median income of," 2013). Compute a 99\% confidence interval for the difference in incomes between males and females in the U.S.
\item
  A study was conducted that measured the total brain volume (TBV) (in ) of patients that had schizophrenia and patients that are considered normal. Table \#9.3.5 contains the TBV of the normal patients and table \#9.3.6 contains the TBV of schizophrenia patients ("SOCR data oct2009," 2013). Is there enough evidence to show that the patients with schizophrenia have less TBV on average than a patient that is considered normal? Test at the 10\% level.
\end{enumerate}

\begin{quote}
\textbf{Table \#9.3.5: Total Brain Volume (in ) of Normal Patients}
\end{quote}

\begin{longtable}[]{@{}llllll@{}}
\toprule
1663407 & 1583940 & 1299470 & 1535137 & 1431890 & 1578698\tabularnewline
\midrule
\endhead
1453510 & 1650348 & 1288971 & 1366346 & 1326402 & 1503005\tabularnewline
1474790 & 1317156 & 1441045 & 1463498 & 1650207 & 1523045\tabularnewline
1441636 & 1432033 & 1420416 & 1480171 & 1360810 & 1410213\tabularnewline
1574808 & 1502702 & 1203344 & 1319737 & 1688990 & 1292641\tabularnewline
1512571 & 1635918 & & & &\tabularnewline
\bottomrule
\end{longtable}

\begin{quote}
\textbf{Table \#9.3.6: Total Brain Volume (in ) of Schizophrenia Patients}
\end{quote}

\begin{longtable}[]{@{}llllll@{}}
\toprule
1331777 & 1487886 & 1066075 & 1297327 & 1499983 & 1861991\tabularnewline
\midrule
\endhead
1368378 & 1476891 & 1443775 & 1337827 & 1658258 & 1588132\tabularnewline
1690182 & 1569413 & 1177002 & 1387893 & 1483763 & 1688950\tabularnewline
1563593 & 1317885 & 1420249 & 1363859 & 1238979 & 1286638\tabularnewline
1325525 & 1588573 & 1476254 & 1648209 & 1354054 & 1354649\tabularnewline
1636119 & & & & &\tabularnewline
\bottomrule
\end{longtable}

\begin{enumerate}
\def\labelenumi{\arabic{enumi}.}
\setcounter{enumi}{3}
\item
  A study was conducted that measured the total brain volume (TBV) (in ) of patients that had schizophrenia and patients that are considered normal. Table \#9.3.5 contains the TBV of the normal patients and table \#9.3.6 contains the TBV of schizophrenia patients ("SOCR data oct2009," 2013). Compute a 90\% confidence interval for the difference in TBV of normal patients and patients with Schizophrenia.
\item
  The length of New Zealand (NZ) rivers that travel to the Pacific Ocean are given in table \#9.3.7 and the lengths of NZ rivers that travel to the Tasman Sea are given in table \#9.3.8 ("Length of NZ," 2013). Do the data provide enough evidence to show on average that the rivers that travel to the Pacific Ocean are longer than the rivers that travel to the Tasman Sea? Use a 5\% level of significance.
\end{enumerate}

\begin{quote}
\textbf{Table \#9.3.7: Lengths (in km) of NZ Rivers that Flow into the
Pacific Ocean}
\end{quote}

\begin{longtable}[]{@{}lllll@{}}
\toprule
209 & 48 & 169 & 138 & 64\tabularnewline
\midrule
\endhead
97 & 161 & 95 & 145 & 90\tabularnewline
121 & 80 & 56 & 64 & 209\tabularnewline
64 & 72 & 288 & 322 &\tabularnewline
\bottomrule
\end{longtable}

\begin{quote}
\textbf{Table \#9.3.8: Lengths (in km) of NZ Rivers that Flow into the
Tasman Sea}
\end{quote}

\begin{longtable}[]{@{}llllll@{}}
\toprule
76 & 64 & 68 & 64 & 37 & 32\tabularnewline
\midrule
\endhead
32 & 51 & 56 & 40 & 64 & 56\tabularnewline
80 & 121 & 177 & 56 & 80 & 35\tabularnewline
72 & 72 & 108 & 48 & &\tabularnewline
\bottomrule
\end{longtable}

\begin{enumerate}
\def\labelenumi{\arabic{enumi}.}
\setcounter{enumi}{5}
\item
  The length of New Zealand (NZ) rivers that travel to the Pacific Ocean are given in table \#9.3.7 and the lengths of NZ rivers that travel to the Tasman Sea are given in table \#9.3.8 ("Length of NZ," 2013). Estimate the difference in mean lengths of rivers between rivers in NZ that travel to the Pacific Ocean and ones that travel to the Tasman Sea. Use a 95\% confidence level.
\item
  The number of cell phones per 100 residents in countries in Europe is given in table \#9.3.9 for the year 2010. The number of cell phones per 100 residents in countries of the Americas is given in table \#9.3.10 also for the year 2010 ("Population reference bureau," 2013). Is there enough evidence to show that the mean number of cell phones in countries of Europe is more than in countries of the Americas? Test at the 1\% level.
\end{enumerate}

\begin{quote}
\textbf{Table \#9.3.9: Number of Cell Phones per 100 Residents in Europe}
\end{quote}

\begin{longtable}[]{@{}llllll@{}}
\toprule
100 & 76 & 100 & 130 & 75 & 84\tabularnewline
\midrule
\endhead
112 & 84 & 138 & 133 & 118 & 134\tabularnewline
126 & 188 & 129 & 93 & 64 & 128\tabularnewline
124 & 122 & 109 & 121 & 127 & 152\tabularnewline
96 & 63 & 99 & 95 & 151 & 147\tabularnewline
123 & 95 & 67 & 67 & 118 & 125\tabularnewline
110 & 115 & 140 & 115 & 141 & 77\tabularnewline
98 & 102 & 102 & 112 & 118 & 118\tabularnewline
54 & 23 & 121 & 126 & 47 &\tabularnewline
\bottomrule
\end{longtable}

\begin{quote}
\textbf{Table \#9.3.10: Number of Cell Phones per 100 Residents in the
Americas}
\end{quote}

\begin{longtable}[]{@{}llllll@{}}
\toprule
158 & 117 & 106 & 159 & 53 & 50\tabularnewline
\midrule
\endhead
78 & 66 & 88 & 92 & 42 & 3\tabularnewline
150 & 72 & 86 & 113 & 50 & 58\tabularnewline
70 & 109 & 37 & 32 & 85 & 101\tabularnewline
75 & 69 & 55 & 115 & 95 & 73\tabularnewline
86 & 157 & 100 & 119 & 81 & 113\tabularnewline
87 & 105 & 96 & & &\tabularnewline
\bottomrule
\end{longtable}

\begin{enumerate}
\def\labelenumi{\arabic{enumi}.}
\setcounter{enumi}{7}
\item
  The number of cell phones per 100 residents in countries in Europe is given in table \#9.3.9 for the year 2010. The number of cell phones per 100 residents in countries of the Americas is given in table \#9.3.10 also for the year 2010 ("Population reference bureau," 2013). Find the 98\% confidence interval for the difference in mean number of cell phones per 100 residents in Europe and the Americas.
\item
  A vitamin K shot is given to infants soon after birth. Nurses at Northbay Healthcare were involved in a study to see if how they handle the infants could reduce the pain the infants feel ("SOCR data nips," 2013). One of the measurements taken was how long, in seconds, the infant cried after being given the shot. A random sample was taken from the group that was given the shot using conventional methods (table \#9.3.11), and a random sample was taken from the group that was given the shot where the mother held the infant prior to and during the shot (table \#9.3.12). Is there enough evidence to show that infants cried less on average when they are held by their mothers than if held using conventional methods? Test at the 5\% level.
\end{enumerate}

\begin{quote}
\textbf{Table \#9.3.11: Crying Time of Infants Given Shots Using
Conventional Methods}
\end{quote}

\begin{longtable}[]{@{}llllll@{}}
\toprule
63 & 0 & 2 & 46 & 33 & 33\tabularnewline
\midrule
\endhead
29 & 23 & 11 & 12 & 48 & 15\tabularnewline
33 & 14 & 51 & 37 & 24 & 70\tabularnewline
63 & 0 & 73 & 39 & 54 & 52\tabularnewline
39 & 34 & 30 & 55 & 58 & 18\tabularnewline
\bottomrule
\end{longtable}

\begin{quote}
\textbf{Table \#9.3.12: Crying Time of Infants Given Shots Using New
Methods}
\end{quote}

\begin{longtable}[]{@{}llllll@{}}
\toprule
0 & 32 & 20 & 23 & 14 & 19\tabularnewline
\midrule
\endhead
60 & 59 & 64 & 64 & 72 & 50\tabularnewline
44 & 14 & 10 & 58 & 19 & 41\tabularnewline
17 & 5 & 36 & 73 & 19 & 46\tabularnewline
9 & 43 & 73 & 27 & 25 & 18\tabularnewline
\bottomrule
\end{longtable}

\begin{enumerate}
\def\labelenumi{\arabic{enumi}.}
\setcounter{enumi}{9}
\item
  A vitamin K shot is given to infants soon after birth. Nurses at Northbay Healthcare were involved in a study to see if how they handle the infants could reduce the pain the infants feel ("SOCR data nips," 2013). One of the measurements taken was how long, in seconds, the infant cried after being given the shot. A random sample was taken from the group that was given the shot using conventional methods (table \#9.3.11), and a random sample was taken from the group that was given the shot where the mother held the infant prior to and during the shot (table \#9.3.12). Calculate a 95\% confidence interval for the mean difference in mean crying time after being given a vitamin K shot between infants held using conventional methods and infants held by their mothers.
\item
  Redo problem 1 testing for the assumption of equal variances and then use the formula that utilizes the assumption of equal variances (follow the procedure in example 9.3.3).
\item
  Redo problem 2 testing for the assumption of equal variances and then use the formula that utilizes the assumption of equal variances (follow the procedure in example 9.3.3).
\item
  Redo problem 7 testing for the assumption of equal variances and then use the formula that utilizes the assumption of equal variances (follow the procedure in example 9.3.3).
\item
  Redo problem 8 testing for the assumption of equal variances and then use the formula that utilizes the assumption of equal variances (follow the procedure in example 9.3.3).
\item
  Redo problem 9 testing for the assumption of equal variances and then use the formula that utilizes the assumption of equal variances (follow the procedure in example 9.3.3).
\item
  Redo problem 10 testing for the assumption of equal variances and then use the formula that utilizes the assumption of equal variances (follow the procedure in example 9.3.3).
\end{enumerate}

\textbf{\\
}

\hypertarget{which-analysis-should-you-conduct}{%
\section{Which Analysis Should You Conduct?}\label{which-analysis-should-you-conduct}}

One of the most important concept that you need to understand is deciding which analysis you should conduct for a particular situation. To help you to figure out the analysis to conduct, there are a series of questions you should ask yourself.

\begin{enumerate}
\def\labelenumi{\arabic{enumi}.}
\tightlist
\item
  Does the problem deal with mean or proportion?
\end{enumerate}

\begin{quote}
Sometimes the problem states explicitly the words mean or proportion, but other times you have to figure it out based on the information you are given. If you counted number of individuals that responded in the affirmative to a question, then you are dealing with proportion. If you measured something, then you are dealing with mean.
\end{quote}

\begin{enumerate}
\def\labelenumi{\arabic{enumi}.}
\setcounter{enumi}{1}
\tightlist
\item
  Does the problem have one or two samples?
\end{enumerate}

\begin{quote}
So look to see if one group was measured or if two groups were measured. If you have the data sets, then it is usually easy to figure out if there is one or two samples, then there is either one data set or two data sets. If you don't have the data, then you need to decide if the problem describes collecting data from one group or from two groups.
\end{quote}

\begin{enumerate}
\def\labelenumi{\arabic{enumi}.}
\setcounter{enumi}{2}
\tightlist
\item
  If you have two samples, then you need to determine if the samples are independent or dependent.
\end{enumerate}

If the individuals are different for both samples, then most likely the samples are independent. If you can't tell, then determine if a data value from the first sample influences the data value in the second sample. In other words, can you pair data values together so you can find the difference, and that difference has meaning. If the answer is yes, then the samples are paired. Otherwise, the samples are independent.

\begin{enumerate}
\def\labelenumi{\arabic{enumi}.}
\setcounter{enumi}{3}
\item
  Does the situation involve a hypothesis test or a confidence interval?

  If the problem talks about "do the data show", "is there evidence of", "test to see", then you are doing a hypothesis test. If the problem talks about "find the value", "estimate the" or "find the interval", then you are doing a confidence interval.
\end{enumerate}

So if you have a situation that has two samples, independent samples, involving the mean, and is a hypothesis test, then you have a two-sample independent t-test. Now you look up the assumptions and the formula or technology process for doing this test. Every hypothesis test involves the same six steps, and you just have to use the correct assumptions and calculations. Every confidence interval has the same five steps, and again you just need to use the correct assumptions and calculations. So this is why it is so important to figure out what analysis you should conduct.

\textbf{\\
}

Data Sources:

\emph{AP exam scores}. (2013, November 20). Retrieved from
\url{http://wiki.stat.ucla.edu/socr/index.php/SOCR_Data_Dinov_030708_APExamScores}

\emph{Buy sushi grade fish online}. (2013, November 20). Retrieved from
\url{http://www.catalinaop.com/}

Center for Disease Control and Prevention, Prevalence of Autism Spectrum
Disorders - Autism and Developmental Disabilities Monitoring Network.
(2008). \emph{Autism and developmental disabilities monitoring network-2012}.
Retrieved from website:
\url{http://www.cdc.gov/ncbddd/autism/documents/ADDM-2012-Community-Report.pdf}

\emph{Cholesterol levels after heart attack}. (2013, September 25). Retrieved
from \url{http://www.statsci.org/data/general/cholest.html}

Flanagan, R., Rooney, C., \& Griffiths, C. (2005). Fatal poisoning in
childhood, england \& wales 1968-2000. \emph{Forensic Science International},
\emph{148:121-129}, Retrieved from
\url{http://www.cdc.gov/nchs/data/ice/fatal_poisoning_child.pdf}

\emph{Friday the 13th datafile}. (2013, November 25). Retrieved from
\url{http://lib.stat.cmu.edu/DASL/Datafiles/Fridaythe13th.html}

Gettler, L. T., McDade, T. W., Feranil, A. B., \& Kuzawa, C. W. (2011).
Longitudinal evidence that fatherhood decreases testosterone in human
males. \emph{The Proceedings of the National Academy of Sciences, PNAS 2011},
doi: 10.1073/pnas.1105403108

\emph{Length of NZ rivers}. (2013, September 25). Retrieved from
\url{http://www.statsci.org/data/oz/nzrivers.html}

Lim, L. L. United Nations, International Labour Office. (2002). \emph{Female
labour-force participation}. Retrieved from website:
\url{http://www.un.org/esa/population/publications/completingfertility/RevisedLIMpaper.PDF}

\emph{Median income of males}. (2013, October 9). Retrieved from
\url{http://www.prb.org/DataFinder/Topic/Rankings.aspx?ind=137}

\emph{Median income of males}. (2013, October 9). Retrieved from
\url{http://www.prb.org/DataFinder/Topic/Rankings.aspx?ind=136}

\emph{NZ helmet size}. (2013, September 25). Retrieved from
\url{http://www.statsci.org/data/oz/nzhelmet.html}

Olson, K., \& Hanson, J. (1997). Using reiki to manage pain: a
preliminary report. \emph{Cancer Prev Control}, \emph{1}(2), 108-13. Retrieved
from \url{http://www.ncbi.nlm.nih.gov/pubmed/9765732}

\emph{Population reference bureau}. (2013, October 8). Retrieved from
\url{http://www.prb.org/DataFinder/Topic/Rankings.aspx?ind=25}

\emph{Seafood online}. (2013, November 20). Retrieved from
\url{http://www.allfreshseafood.com/}

\emph{SOCR 012708 id data hotdogs}. (2013, November 13). Retrieved from
\url{http://wiki.stat.ucla.edu/socr/index.php/SOCR_012708_ID_Data_HotDogs}

\emph{SOCR data nips infantvitK shotdata}. (2013, November 16). Retrieved
from
\url{http://wiki.stat.ucla.edu/socr/index.php/SOCR_Data_NIPS_InfantVitK_ShotData}

\emph{SOCR data Oct2009 id ni}. (2013, November 16). Retrieved from
\url{http://wiki.stat.ucla.edu/socr/index.php/SOCR_Data_Oct2009_ID_NI}

\emph{Statistics brain}. (2013, November 30). Retrieved from
\url{http://www.statisticbrain.com/infidelity-statistics/}

\emph{Student t-distribution}. (2013, November 25). Retrieved from
\url{http://lib.stat.cmu.edu/DASL/Stories/student.html}

\hypertarget{regression-and-correlation}{%
\chapter{Regression and Correlation}\label{regression-and-correlation}}

The previous chapter looked at comparing populations to see if there is a difference between the two. That involved two random variables that are similar measures. This chapter will look at two random variables that are not similar measures, and see if there is a relationship between the two variables. To do this, you look at regression, which finds the linear relationship, and correlation, which measures the strength of a linear relationship.

Please note: there are many other types of relationships besides linear that can be found for the data. This book will only explore linear, but realize that there are other relationships that can be used to describe data.

\hypertarget{regression}{%
\section{Regression}\label{regression}}

When comparing two different variables, two questions come to mind: ``Is there a relationship between two variables?'' and ``How strong is that relationship?'' These questions can be answered using \textbf{regression} and \textbf{correlation}. Regression answers whether there is a relationship (again this book will explore linear only) and correlation answers how strong the linear relationship is. To introduce both of these concepts, it is easier to look at a set of data.

\textbf{Example \#10.1.1: Determining If There Is a Relationship}

\begin{quote}
Is there a relationship between the alcohol content and the number of calories in 12-ounce beer? To determine if there is one a random sample was taken of beer's alcohol content and calories ("Calories in beer,," 2011), and the data is in table \#10.1.1.

\textbf{Table \#10.1.1: Alcohol and Calorie Content in Beer}
\end{quote}

\begin{longtable}[]{@{}llll@{}}
\toprule
Brand & Brewery & Alcohol Content & Calories in 12 oz\tabularnewline
\midrule
\endhead
Big Sky Scape Goat Pale Ale & Big Sky Brewing & 4.70\% & 163\tabularnewline
Sierra Nevada Harvest Ale & Sierra Nevada & 6.70\% & 215\tabularnewline
Steel Reserve & MillerCoors & 8.10\% & 222\tabularnewline
O'Doul's & Anheuser Busch & 0.40\% & 70\tabularnewline
Coors Light & MillerCoors & 4.15\% & 104\tabularnewline
Genesee Cream Ale & High Falls Brewing & 5.10\% & 162\tabularnewline
Sierra Nevada Summerfest Beer & Sierra Nevada & 5.00\% & 158\tabularnewline
Michelob Beer & Anheuser Busch & 5.00\% & 155\tabularnewline
Flying Dog Doggie Style & Flying Dog Brewery & 4.70\% & 158\tabularnewline
Big Sky I.P.A. & Big Sky Brewing & 6.20\% & 195\tabularnewline
\bottomrule
\end{longtable}

\begin{quote}
\textbf{Solution:}

To aid in figuring out if there is a relationship, it helps to draw a scatter plot of the data. It is helpful to state the random variables, and since in an algebra class the variables are represented as \emph{x} and \emph{y}, those labels will be used here. It helps to state which variable is \emph{x} and which is \emph{y}.

State random variables

\emph{x} = alcohol content in the beer

\emph{y} = calories in 12 ounce beer

\textbf{Figure \#10.1.1: Scatter Plot of Beer Data}

\includegraphics[width=4.13889in,height=4.13889in]{media/image1.emf}

This scatter plot looks fairly linear. However, notice that there is one beer in the list that is actually considered a non-alcoholic beer. That value is probably an outlier since it is a non-alcoholic beer. The rest of the analysis will not include O'Doul's. You cannot just remove data points, but in this case it makes more sense to, since all the other beers have a fairly large alcohol content.
\end{quote}

To find the equation for the linear relationship, the process of regression is used to find the line that best fits the data (sometimes called the best fitting line). The process is to draw the line through the data and then find the distances from a point to the line, which are called the residuals. The regression line is the line that makes the square of the residuals as small as possible, so the regression line is also sometimes called the least squares line. The regression line and the residuals are displayed in figure \#10.1.2.

\textbf{\\
}

\textbf{Figure \#10.1.2: Scatter Plot of Beer Data with Regression Line and
Residuals}

\includegraphics[width=5.08333in,height=5.08333in]{media/image2.jpg}

\textbf{The find the regression equation (also known as best fitting line or
least squares line)}

Given a collection of paired sample data, the regression equation is

where the slope = and \emph{y}-intercept =

The \textbf{residuals} are the difference between the actual values and the
estimated values.

SS stands for sum of squares. So you are summing up squares. With the
subscript \emph{xy}, you aren't really summing squares, but you can think of
it that way in a weird sense.

Note: the easiest way to find the regression equation is to use the technology.

The \textbf{independent variable}, also called the \textbf{explanatory variable} or \textbf{predictor variable}, is the \emph{x}-value in the equation. The independent variable is the one that you use to predict what the other variable is. The \textbf{dependent variable} depends on what independent value you pick. It also responds to the explanatory variable and is sometimes called the \textbf{response variable}. In the alcohol content and calorie example, it makes slightly more sense to say that you would use the alcohol content on a beer to predict the number of calories in the beer.

The \textbf{population equation} looks like:

is used to predict \emph{y}.

Assumptions of the regression line:

\begin{enumerate}
\def\labelenumi{\alph{enumi}.}
\item
  The set of ordered pairs is a random sample from the population of all such possible pairs.
\item
  For each fixed value of \emph{x}, the \emph{y}-values have a normal distribution. All of the \emph{y} distributions have the same variance, and for a given \emph{x}-value, the distribution of \emph{y}-values has a mean that lies on the least squares line. You also assume that for a fixed \emph{y}, each \emph{x} has its own normal distribution. This is difficult to figure out, so you can use the following to determine if you have a normal distribution.
\end{enumerate}

\begin{enumerate}
\def\labelenumi{\roman{enumi}.}
\item
  Look to see if the scatter plot has a linear pattern.
\item
  Examine the residuals to see if there is randomness in the residuals. If there is a pattern to the residuals, then there is an issue in the data.
\end{enumerate}

\textbf{Example \#10.1.2: Find the Equation of the Regression Line}

\begin{enumerate}
\def\labelenumi{\alph{enumi}.}
\tightlist
\item
  Is there a positive relationship between the alcohol content and the number of calories in 12-ounce beer? To determine if there is a positive linear relationship, a random sample was taken of beer's alcohol content and calories for several different beers ("Calories in beer,," 2011), and the data are in table \#10.1.2.
\end{enumerate}

\textbf{\\
}

\begin{quote}
\textbf{Table \#10.1.2: Alcohol and Calorie Content in Beer without
Outlier}
\end{quote}

\begin{longtable}[]{@{}llll@{}}
\toprule
Brand & Brewery & Alcohol Content & Calories in 12 oz\tabularnewline
\midrule
\endhead
Big Sky Scape Goat Pale Ale & Big Sky Brewing & 4.70\% & 163\tabularnewline
Sierra Nevada Harvest Ale & Sierra Nevada & 6.70\% & 215\tabularnewline
Steel Reserve & MillerCoors & 8.10\% & 222\tabularnewline
Coors Light & MillerCoors & 4.15\% & 104\tabularnewline
Genesee Cream Ale & High Falls Brewing & 5.10\% & 162\tabularnewline
Sierra Nevada Summerfest Beer & Sierra Nevada & 5.00\% & 158\tabularnewline
Michelob Beer & Anheuser Busch & 5.00\% & 155\tabularnewline
Flying Dog Doggie Style & Flying Dog Brewery & 4.70\% & 158\tabularnewline
Big Sky I.P.A. & Big Sky Brewing & 6.20\% & 195\tabularnewline
\bottomrule
\end{longtable}

\begin{quote}
\textbf{Solution:}

State random variables

\emph{x} = alcohol content in the beer

\emph{y} = calories in 12 ounce beer

Assumptions check:
\end{quote}

\begin{enumerate}
\def\labelenumi{\alph{enumi}.}
\item
  A random sample was taken as stated in the problem.
\item
  The distribution for each calorie value is normally distributed for every value of alcohol content in the beer.
\end{enumerate}

\begin{enumerate}
\def\labelenumi{\roman{enumi}.}
\item
  From Example \#10.2.1, the scatter plot looks fairly linear.
\item
  The residual versus the \emph{x-}values plot looks fairly random. (See figure \#10.1.5.)
\end{enumerate}

\begin{quote}
It appears that the distribution for calories is a normal distribution.

To find the regression equation on the TI-83/84 calculator, put the \emph{x}'s in L1 and the \emph{y}'s in L2. Then go to STAT, over to TESTS, and choose LinRegTTest. The setup is in figure \#10.1.3. The reason that \textgreater{}0 was chosen is because the question was asked if there was a positive relationship. If you are asked if there is a negative relationship, then pick \textless{}0. If you are just asked if there is a relationship, then pick . Right now the choice will not make a different, but it will be important later.

\textbf{Figure \#10.1.3: Setup for Linear Regression Test on TI-83/84}

\includegraphics[width=2.75in,height=1.86111in]{media/image15.png}

\textbf{Figure \#10.1.4: Results for Linear Regression Test on TI-83/84}

\includegraphics[width=2.75in,height=1.86111in]{media/image16.png}\includegraphics[width=2.75in,height=1.86111in]{media/image17.png}

From this you can see that

To find the regression equation using R, the command is lm(dependent variable \textasciitilde{} independent variable), where \textasciitilde{} is the tilde symbol located on the upper left of most keyboards. So for this example, the command would be lm(calories \textasciitilde{} alcohol), and the output would be

Call:

lm(formula = calories \textasciitilde{} alcohol)

Coefficients:

(Intercept) alcohol

25.03 26.32

From this you can see that the y-intercept is 25.03 and the slope is 26.32. So the regression equation is .

Remember, this is an estimate for the true regression. A different random sample would produce a different estimate.
\end{quote}

\begin{enumerate}
\def\labelenumi{\alph{enumi}.}
\setcounter{enumi}{1}
\tightlist
\item
  Use the regression equation to find the number of calories when the alcohol content is 6.50\%.
\end{enumerate}

\begin{quote}
\textbf{Solution:}

If you are drinking a beer that is 6.50\% alcohol content, then it is probably close to 196 calories. Notice, the mean number of calories is 170 calories. This value of 196 seems like a better estimate than the mean when looking at the original data. The regression equation is a better estimate than just the mean.
\end{quote}

\begin{enumerate}
\def\labelenumi{\alph{enumi}.}
\setcounter{enumi}{2}
\tightlist
\item
  Use the regression equation to find the number of calories when the alcohol content is 2.00\%.
\end{enumerate}

\begin{quote}
\textbf{Solution:}

If you are drinking a beer that is 2.00\% alcohol content, then it has probably close to 78 calories. This doesn't seem like a very good estimate. This estimate is what is called extrapolation. It is not a good idea to predict values that are far outside the range of the original data. This is because you can never be sure that the regression equation is valid for data outside the original data.
\end{quote}

\begin{enumerate}
\def\labelenumi{\alph{enumi}.}
\setcounter{enumi}{3}
\tightlist
\item
  Find the residuals and then plot the residuals versus the \emph{x}-values.
\end{enumerate}

\begin{quote}
\textbf{Solution:}
\end{quote}

To find the residuals, find for each \emph{x}-value. Then subtract each from the given \emph{y} value to find the residuals. Realize that these are sample residuals since they are calculated from sample values. It is best to do this in a spreadsheet.

\textbf{Table \#10.1.3: Residuals for Beer Calories}

\begin{longtable}[]{@{}llll@{}}
\toprule
\emph{x} & \emph{y} & &\tabularnewline
\midrule
\endhead
4.70 & 163 & 148.61 & 14.390\tabularnewline
6.70 & 215 & 201.21 & 13.790\tabularnewline
8.10 & 222 & 238.03 & -16.030\tabularnewline
4.15 & 104 & 134.145 & -30.145\tabularnewline
5.10 & 162 & 159.13 & 2.870\tabularnewline
5.00 & 158 & 156.5 & 1.500\tabularnewline
5.00 & 155 & 156.5 & -1.500\tabularnewline
4.70 & 158 & 148.61 & 9.390\tabularnewline
6.20 & 195 & 188.06 & 6.940\tabularnewline
\bottomrule
\end{longtable}

Notice the residuals add up to close to 0. They don't add up to exactly
0 in this example because of rounding error. Normally the residuals add
up to 0.

You can use R to get the residuals. The command is

lm.out = lm(dependent variable \textasciitilde{} independent variable) -- this defines
the linear model with a name so you can use it later. Then

residual(lm.out) -- produces the residuals.

For this example, the command would be

lm(calories\textasciitilde{}alcohol)

Call:

lm(formula = calories \textasciitilde{} alcohol)

Coefficients:

(Intercept) alcohol

25.03 26.32

\textgreater{} residuals(lm.out)

1 2 3 4 5 6

14.271307 13.634092 -16.211959 -30.253458 2.743864 1.375725 -

7 8 9

1.624275 9.271307 6.793396

So the first residual is 14.271307 and it belongs to the first x value.
The residual 13.634092 belongs to the second x value, and so forth.

You can then graph the residuals versus the independent variable using
the plot command. For this example, the command would be plot(alcohol,
residuals(lm.out), main="Residuals for Beer Calories versus Alcohol
Content", xlab="Alcohol Content", ylab="Residuals"). Sometimes it
is useful to see the x-axis on the graph, so after creating the plot,
type the command abline(0,0).

The graph of the residuals versus the \emph{x}-values is in figure \#10.1.5.
They appear to be somewhat random.

\textbf{Figure \#10.1.5: Residuals of Beer Calories versus Content}

\begin{quote}
\includegraphics[width=3.59722in,height=3.59722in]{media/image26.emf}
\end{quote}

Notice, that the 6.50\% value falls into the range of the original \emph{x}-values. The processes of predicting values using an \emph{x} within the range of original \emph{x}-values is called \textbf{interpolating}. The 2.00\% value is outside the range of original \emph{x}-values. Using an \emph{x}-value that is outside the range of the original \emph{x}-values is called \textbf{extrapolating}. When predicting values using interpolation, you can usually feel pretty confident that that value will be close to the true value. When you extrapolate, you are not really sure that the predicted value is close to the true value. This is because when you interpolate, you know the equation that predicts, but when you extrapolate, you are not really sure that your relationship is still valid. The relationship could in fact change for different \emph{x}-values.

An example of this is when you use regression to come up with an equation to predict the growth of a city, like Flagstaff, AZ. Based on analysis it was determined that the population of Flagstaff would be well over 50,000 by 1995. However, when a census was undertaken in 1995, the population was less than 50,000. This is because they extrapolated and the growth factor they were using had obviously changed from the early 1990's. Growth factors can change for many reasons, such as employment growth, employment stagnation, disease, articles saying great place to live, etc. Realize that when you extrapolate, your predicted value may not be anywhere close to the actual value that you observe.

What does the slope mean in the context of this problem?

The calories increase 26.3 calories for every 1\% increase in alcohol
content.

The \emph{y}-intercept in many cases is meaningless. In this case, it means that if a drink has 0 alcohol content, then it would have 25.0 calories. This may be reasonable, but remember this value is an extrapolation so it may be wrong.

Consider the residuals again. According to the data, a beer with 6.7\% alcohol has 215 calories. The predicted value is 201 calories.

This deviation means that the actual value was 14 above the predicted value. That isn't that far off. Some of the actual values differ by a large amount from the predicted value. This is due to variability in the dependent variable. The larger the residuals the less the model explains the variability in the dependent variable. There needs to be a way to calculate how well the model explains the variability in the dependent variable. This will be explored in the next section.

The following example demonstrates the process to go through when using the formulas for finding the regression equation, though it is better to use technology. This is because if the linear model doesn't fit the data well, then you could try some of the other models that are available through technology.

\textbf{Example \#10.1.3: Calculating the Regression Equation with the
Formula}

\begin{quote}
Is there a relationship between the alcohol content and the number of calories in 12-ounce beer? To determine if there is one a random sample was taken of beer's alcohol content and calories ("Calories in beer,," 2011), and the data are in table \#10.1.2. Find the regression equation from the formula.

\textbf{Solution:}

State random variables

\emph{x} = alcohol content in the beer

\emph{y} = calories in 12 ounce beer
\end{quote}

\textbf{\\
}

\textbf{Table \#10.1.4: Calculations for Regression Equation}

\begin{longtable}[]{@{}lllllll@{}}
\toprule
Alcohol Content & Calories & & & & &\tabularnewline
\midrule
\endhead
4.70 & 163 & -0.8167 & -7.2222 & 0.6669 & 52.1605 & 5.8981\tabularnewline
6.70 & 215 & 1.1833 & 44.7778 & 1.4003 & 2005.0494 & 52.9870\tabularnewline
8.10 & 222 & 2.5833 & 51.7778 & 6.6736 & 2680.9383 & 133.7593\tabularnewline
4.15 & 104 & -1.3667 & -66.2222 & 1.8678 & 4385.3827 & 90.5037\tabularnewline
5.10 & 162 & -0.4167 & -8.2222 & 0.1736 & 67.6049 & 3.4259\tabularnewline
5.00 & 158 & -0.5167 & -12.2222 & 0.2669 & 149.3827 & 6.3148\tabularnewline
5.00 & 155 & -0.5167 & -15.2222 & 0.2669 & 231.7160 & 7.8648\tabularnewline
4.70 & 158 & -0.8167 & -12.2222 & 0.6669 & 149.3827 & 9.9815\tabularnewline
6.20 & 195 & 0.6833 & 24.7778 & 0.4669 & 613.9383 & 16.9315\tabularnewline
5.516667 & 170.2222 & & & 12.45 & 10335.5556 & 327.6667\tabularnewline
\bottomrule
\end{longtable}

\begin{quote}
slope:

\emph{y}-intercept:

Regression equation:
\end{quote}

\hypertarget{homework-29}{%
\subsection{Homework}\label{homework-29}}

For each problem, state the random variables. Also, look to see if there
are any outliers that need to be removed. Do the regression analysis
with and without the suspected outlier points to determine if their
removal affects the regression. The data sets in this section are used
in the homework for sections 10.2 and 10.3 also.

\begin{enumerate}
\def\labelenumi{\arabic{enumi}.}
\tightlist
\item
  When an anthropologist finds skeletal remains, they need to figure
  out the height of the person. The height of a person (in cm) and the
  length of their metacarpal bone 1 (in cm) were collected and are in
  table \#10.1.5 ("Prediction of height," 2013). Create a scatter
  plot and find a regression equation between the height of a person
  and the length of their metacarpal. Then use the regression equation
  to find the height of a person for a metacarpal length of 44 cm and
  for a metacarpal length of 55 cm. Which height that you calculated
  do you think is closer to the true height of the person? Why?
\end{enumerate}

\begin{quote}
\textbf{Table \#10.1.5: Data of Metacarpal versus Height}
\end{quote}

\begin{longtable}[]{@{}ll@{}}
\toprule
Length of Metacarpal (cm) & Height of Person (cm)\tabularnewline
\midrule
\endhead
45 & 171\tabularnewline
51 & 178\tabularnewline
39 & 157\tabularnewline
41 & 163\tabularnewline
48 & 172\tabularnewline
49 & 183\tabularnewline
46 & 173\tabularnewline
43 & 175\tabularnewline
47 & 173\tabularnewline
\bottomrule
\end{longtable}

\begin{enumerate}
\def\labelenumi{\arabic{enumi}.}
\setcounter{enumi}{1}
\tightlist
\item
  Table \#10.1.6 contains the value of the house and the amount of
  rental income in a year that the house brings in ("Capital and
  rental," 2013). Create a scatter plot and find a regression
  equation between house value and rental income. Then use the
  regression equation to find the rental income a house worth
  \$230,000 and for a house worth \$400,000. Which rental income that
  you calculated do you think is closer to the true rental income?
  Why?
\end{enumerate}

\begin{quote}
\textbf{Table \#10.1.6: Data of House Value versus Rental}
\end{quote}

\begin{longtable}[]{@{}llllllll@{}}
\toprule
Value & Rental & Value & Rental & Value & Rental & Value & Rental\tabularnewline
\midrule
\endhead
81000 & 6656 & 77000 & 4576 & 75000 & 7280 & 67500 & 6864\tabularnewline
95000 & 7904 & 94000 & 8736 & 90000 & 6240 & 85000 & 7072\tabularnewline
121000 & 12064 & 115000 & 7904 & 110000 & 7072 & 104000 & 7904\tabularnewline
135000 & 8320 & 130000 & 9776 & 126000 & 6240 & 125000 & 7904\tabularnewline
145000 & 8320 & 140000 & 9568 & 140000 & 9152 & 135000 & 7488\tabularnewline
165000 & 13312 & 165000 & 8528 & 155000 & 7488 & 148000 & 8320\tabularnewline
178000 & 11856 & 174000 & 10400 & 170000 & 9568 & 170000 & 12688\tabularnewline
200000 & 12272 & 200000 & 10608 & 194000 & 11232 & 190000 & 8320\tabularnewline
214000 & 8528 & 208000 & 10400 & 200000 & 10400 & 200000 & 8320\tabularnewline
240000 & 10192 & 240000 & 12064 & 240000 & 11648 & 225000 & 12480\tabularnewline
289000 & 11648 & 270000 & 12896 & 262000 & 10192 & 244500 & 11232\tabularnewline
325000 & 12480 & 310000 & 12480 & 303000 & 12272 & 300000 & 12480\tabularnewline
\bottomrule
\end{longtable}

\begin{enumerate}
\def\labelenumi{\arabic{enumi}.}
\setcounter{enumi}{2}
\tightlist
\item
  The World Bank collects information on the life expectancy of a
  person in each country ("Life expectancy at," 2013) and the
  fertility rate per woman in the country ("Fertility rate," 2013).
  The data for 24 randomly selected countries for the year 2011 are in
  table \#10.1.7. Create a scatter plot of the data and find a linear
  regression equation between fertility rate and life expectancy. Then
  use the regression equation to find the life expectancy for a
  country that has a fertility rate of 2.7 and for a country with
  fertility rate of 8.1. Which life expectancy that you calculated do
  you think is closer to the true life expectancy? Why?
\end{enumerate}

\begin{quote}
\textbf{Table \#10.1.7: Data of Fertility Rates versus Life Expectancy}
\end{quote}

\begin{longtable}[]{@{}ll@{}}
\toprule
Fertility Rate & Life Expectancy\tabularnewline
\midrule
\endhead
1.7 & 77.2\tabularnewline
5.8 & 55.4\tabularnewline
2.2 & 69.9\tabularnewline
2.1 & 76.4\tabularnewline
1.8 & 75.0\tabularnewline
2.0 & 78.2\tabularnewline
2.6 & 73.0\tabularnewline
2.8 & 70.8\tabularnewline
1.4 & 82.6\tabularnewline
2.6 & 68.9\tabularnewline
1.5 & 81.0\tabularnewline
6.9 & 54.2\tabularnewline
2.4 & 67.1\tabularnewline
1.5 & 73.3\tabularnewline
2.5 & 74.2\tabularnewline
1.4 & 80.7\tabularnewline
2.9 & 72.1\tabularnewline
2.1 & 78.3\tabularnewline
4.7 & 62.9\tabularnewline
6.8 & 54.4\tabularnewline
5.2 & 55.9\tabularnewline
4.2 & 66.0\tabularnewline
1.5 & 76.0\tabularnewline
3.9 & 72.3\tabularnewline
\bottomrule
\end{longtable}

\begin{enumerate}
\def\labelenumi{\arabic{enumi}.}
\setcounter{enumi}{3}
\tightlist
\item
  The World Bank collected data on the percentage of GDP that a
  country spends on health expenditures ("Health expenditure," 2013)
  and also the percentage of women receiving prenatal care ("Pregnant
  woman receiving," 2013). The data for the countries where this
  information are available for the year 2011 is in table \#10.1.8.
  Create a scatter plot of the data and find a regression equation
  between percentage spent on health expenditure and the percentage of
  women receiving prenatal care. Then use the regression equation to
  find the percent of women receiving prenatal care for a country that
  spends 5.0\% of GDP on health expenditure and for a country that
  spends 12.0\% of GDP. Which prenatal care percentage that you
  calculated do you think is closer to the true percentage? Why?
\end{enumerate}

\begin{quote}
\textbf{Table \#10.1.8: Data of Health Expenditure versus Prenatal Care}
\end{quote}

\begin{longtable}[]{@{}ll@{}}
\toprule
Health Expenditure (\% of GDP) & Prenatal Care (\%)\tabularnewline
\midrule
\endhead
9.6 & 47.9\tabularnewline
3.7 & 54.6\tabularnewline
5.2 & 93.7\tabularnewline
5.2 & 84.7\tabularnewline
10.0 & 100.0\tabularnewline
4.7 & 42.5\tabularnewline
4.8 & 96.4\tabularnewline
6.0 & 77.1\tabularnewline
5.4 & 58.3\tabularnewline
4.8 & 95.4\tabularnewline
4.1 & 78.0\tabularnewline
6.0 & 93.3\tabularnewline
9.5 & 93.3\tabularnewline
6.8 & 93.7\tabularnewline
6.1 & 89.8\tabularnewline
\bottomrule
\end{longtable}

\begin{enumerate}
\def\labelenumi{\arabic{enumi}.}
\setcounter{enumi}{4}
\item
  The height and weight of baseball players are in table \#10.1.9
  ("MLB heightsweights," 2013). Create a scatter plot and find a
  regression equation between height and weight of baseball players.
  Then use the regression equation to find the weight of a baseball
  player that is 75 inches tall and for a baseball player that is 68
  inches tall. Which weight that you calculated do you think is closer
  to the true weight? Why?

  \textbf{Table \#10.1.9: Heights and Weights of Baseball Players}
\end{enumerate}

\begin{longtable}[]{@{}ll@{}}
\toprule
Height (inches) & Weight (pounds)\tabularnewline
\midrule
\endhead
76 & 212\tabularnewline
76 & 224\tabularnewline
72 & 180\tabularnewline
74 & 210\tabularnewline
75 & 215\tabularnewline
71 & 200\tabularnewline
77 & 235\tabularnewline
78 & 235\tabularnewline
77 & 194\tabularnewline
76 & 185\tabularnewline
72 & 180\tabularnewline
72 & 170\tabularnewline
75 & 220\tabularnewline
74 & 228\tabularnewline
73 & 210\tabularnewline
72 & 180\tabularnewline
70 & 185\tabularnewline
73 & 190\tabularnewline
71 & 186\tabularnewline
74 & 200\tabularnewline
74 & 200\tabularnewline
75 & 210\tabularnewline
78 & 240\tabularnewline
72 & 208\tabularnewline
75 & 180\tabularnewline
\bottomrule
\end{longtable}

\begin{enumerate}
\def\labelenumi{\arabic{enumi}.}
\setcounter{enumi}{5}
\item
  Different species have different body weights and brain weights are
  in table \#10.1.10. ("Brain2bodyweight," 2013). Create a scatter
  plot and find a regression equation between body weights and brain
  weights. Then use the regression equation to find the brain weight
  for a species that has a body weight of 62 kg and for a species that
  has a body weight of 180,000 kg. Which brain weight that you
  calculated do you think is closer to the true brain weight? Why?

  \textbf{Table \#10.1.10: Body Weights and Brain Weights of Species}
\end{enumerate}

\begin{longtable}[]{@{}lll@{}}
\toprule
Species & Body Weight (kg) & Brain Weight (kg)\tabularnewline
\midrule
\endhead
Newborn Human & 3.20 & 0.37\tabularnewline
Adult Human & 73.00 & 1.35\tabularnewline
Pithecanthropus Man & 70.00 & 0.93\tabularnewline
Squirrel & 0.80 & 0.01\tabularnewline
Hamster & 0.15 & 0.00\tabularnewline
Chimpanzee & 50.00 & 0.42\tabularnewline
Rabbit & 1.40 & 0.01\tabularnewline
Dog\_(Beagle) & 10.00 & 0.07\tabularnewline
Cat & 4.50 & 0.03\tabularnewline
Rat & 0.40 & 0.00\tabularnewline
Bottle-Nosed Dolphin & 400.00 & 1.50\tabularnewline
Beaver & 24.00 & 0.04\tabularnewline
Gorilla & 320.00 & 0.50\tabularnewline
Tiger & 170.00 & 0.26\tabularnewline
Owl & 1.50 & 0.00\tabularnewline
Camel & 550.00 & 0.76\tabularnewline
Elephant & 4600.00 & 6.00\tabularnewline
Lion & 187.00 & 0.24\tabularnewline
Sheep & 120.00 & 0.14\tabularnewline
Walrus & 800.00 & 0.93\tabularnewline
Horse & 450.00 & 0.50\tabularnewline
Cow & 700.00 & 0.44\tabularnewline
Giraffe & 950.00 & 0.53\tabularnewline
Green Lizard & 0.20 & 0.00\tabularnewline
Sperm Whale & 35000.00 & 7.80\tabularnewline
Turtle & 3.00 & 0.00\tabularnewline
Alligator & 270.00 & 0.01\tabularnewline
\bottomrule
\end{longtable}

\begin{enumerate}
\def\labelenumi{\arabic{enumi}.}
\setcounter{enumi}{6}
\tightlist
\item
  A random sample of beef hotdogs was taken and the amount of sodium
  (in mg) and calories were measured. ("Data hotdogs," 2013) The
  data are in table \#10.1.11. Create a scatter plot and find a
  regression equation between amount of calories and amount of sodium.
  Then use the regression equation to find the amount of sodium a beef
  hotdog has if it is 170 calories and if it is 120 calories. Which
  sodium level that you calculated do you think is closer to the true
  sodium level? Why?
\end{enumerate}

\begin{quote}
\textbf{Table \#10.1.11: Calories and Sodium Levels in Beef Hotdogs}
\end{quote}

\begin{longtable}[]{@{}ll@{}}
\toprule
Calories & Sodium\tabularnewline
\midrule
\endhead
186 & 495\tabularnewline
181 & 477\tabularnewline
176 & 425\tabularnewline
149 & 322\tabularnewline
184 & 482\tabularnewline
190 & 587\tabularnewline
158 & 370\tabularnewline
139 & 322\tabularnewline
175 & 479\tabularnewline
148 & 375\tabularnewline
152 & 330\tabularnewline
111 & 300\tabularnewline
141 & 386\tabularnewline
153 & 401\tabularnewline
190 & 645\tabularnewline
157 & 440\tabularnewline
131 & 317\tabularnewline
149 & 319\tabularnewline
135 & 298\tabularnewline
132 & 253\tabularnewline
\bottomrule
\end{longtable}

\begin{enumerate}
\def\labelenumi{\arabic{enumi}.}
\setcounter{enumi}{7}
\tightlist
\item
  Per capita income in 1960 dollars for European countries and the
  percent of the labor force that works in agriculture in 1960 are in
  table \#10.1.12 ("OECD economic development," 2013). Create a
  scatter plot and find a regression equation between percent of labor
  force in agriculture and per capita income. Then use the regression
  equation to find the per capita income in a country that has 21
  percent of labor in agriculture and in a country that has 2 percent
  of labor in agriculture. Which per capita income that you calculated
  do you think is closer to the true income? Why?
\end{enumerate}

\begin{quote}
\textbf{Table \#10.1.12: Percent of Labor in Agriculture and Per Capita
Income for European Countries}
\end{quote}

\begin{longtable}[]{@{}lll@{}}
\toprule
Country & Percent in Agriculture & Per capita income\tabularnewline
\midrule
\endhead
Sweden & 14 & 1644\tabularnewline
Switzerland & 11 & 1361\tabularnewline
Luxembourg & 15 & 1242\tabularnewline
U. Kingdom & 4 & 1105\tabularnewline
Denmark & 18 & 1049\tabularnewline
W. Germany & 15 & 1035\tabularnewline
France & 20 & 1013\tabularnewline
Belgium & 6 & 1005\tabularnewline
Norway & 20 & 977\tabularnewline
Iceland & 25 & 839\tabularnewline
Netherlands & 11 & 810\tabularnewline
Austria & 23 & 681\tabularnewline
Ireland & 36 & 529\tabularnewline
Italy & 27 & 504\tabularnewline
Greece & 56 & 324\tabularnewline
Spain & 42 & 290\tabularnewline
Portugal & 44 & 238\tabularnewline
Turkey & 79 & 177\tabularnewline
\bottomrule
\end{longtable}

\begin{enumerate}
\def\labelenumi{\arabic{enumi}.}
\setcounter{enumi}{8}
\tightlist
\item
  Cigarette smoking and cancer have been linked. The number of deaths
  per one hundred thousand from bladder cancer and the number of
  cigarettes sold per capita in 1960 are in table \#10.1.13 ("Smoking
  and cancer," 2013). Create a scatter plot and find a regression
  equation between cigarette smoking and deaths of bladder cancer.
  Then use the regression equation to find the number of deaths from
  bladder cancer when the cigarette sales were 20 per capita and when
  the cigarette sales were 6 per capita. Which number of deaths that
  you calculated do you think is closer to the true number? Why?
\end{enumerate}

\begin{quote}
\textbf{Table \#10.1.13: Number of Cigarettes and Number of Bladder Cancer
Deaths in 1960}
\end{quote}

\begin{longtable}[]{@{}llll@{}}
\toprule
Cigarette Sales (per Capita) & Bladder Cancer Deaths (per 100 Thousand) & Cigarette Sales (per Capita) & Bladder Cancer Deaths (per 100 Thousand)\tabularnewline
\midrule
\endhead
18.20 & 2.90 & 42.40 & 6.54\tabularnewline
25.82 & 3.52 & 28.64 & 5.98\tabularnewline
18.24 & 2.99 & 21.16 & 2.90\tabularnewline
28.60 & 4.46 & 29.14 & 5.30\tabularnewline
31.10 & 5.11 & 19.96 & 2.89\tabularnewline
33.60 & 4.78 & 26.38 & 4.47\tabularnewline
40.46 & 5.60 & 23.44 & 2.93\tabularnewline
28.27 & 4.46 & 23.78 & 4.89\tabularnewline
20.10 & 3.08 & 29.18 & 4.99\tabularnewline
27.91 & 4.75 & 18.06 & 3.25\tabularnewline
26.18 & 4.09 & 20.94 & 3.64\tabularnewline
22.12 & 4.23 & 20.08 & 2.94\tabularnewline
21.84 & 2.91 & 22.57 & 3.21\tabularnewline
23.44 & 2.86 & 14.00 & 3.31\tabularnewline
21.58 & 4.65 & 25.89 & 4.63\tabularnewline
28.92 & 4.79 & 21.17 & 4.04\tabularnewline
25.91 & 5.21 & 21.25 & 5.14\tabularnewline
26.92 & 4.69 & 22.86 & 4.78\tabularnewline
24.96 & 5.27 & 28.04 & 3.20\tabularnewline
22.06 & 3.72 & 30.34 & 3.46\tabularnewline
16.08 & 3.06 & 23.75 & 3.95\tabularnewline
27.56 & 4.04 & 23.32 & 3.72\tabularnewline
\bottomrule
\end{longtable}

\begin{enumerate}
\def\labelenumi{\arabic{enumi}.}
\setcounter{enumi}{9}
\tightlist
\item
  The weight of a car can influence the mileage that the car can
  obtain. A random sample of cars' weights and mileage was collected
  and are in table \#10.1.14 ("Passenger car mileage," 2013). Create
  a scatter plot and find a regression equation between weight of cars
  and mileage. Then use the regression equation to find the mileage on
  a car that weighs 3800 pounds and on a car that weighs 2000 pounds.
  Which mileage that you calculated do you think is closer to the true
  mileage? Why?
\end{enumerate}

\begin{quote}
\textbf{Table \#10.1.14: Weights and Mileages of Cars}
\end{quote}

\begin{longtable}[]{@{}ll@{}}
\toprule
Weight (100 pounds) & Mileage (mpg)\tabularnewline
\midrule
\endhead
22.5 & 53.3\tabularnewline
22.5 & 41.1\tabularnewline
22.5 & 38.9\tabularnewline
25.0 & 40.9\tabularnewline
27.5 & 46.9\tabularnewline
27.5 & 36.3\tabularnewline
30.0 & 32.2\tabularnewline
30.0 & 32.2\tabularnewline
30.0 & 31.5\tabularnewline
30.0 & 31.4\tabularnewline
30.0 & 31.4\tabularnewline
35.0 & 32.6\tabularnewline
35.0 & 31.3\tabularnewline
35.0 & 31.3\tabularnewline
35.0 & 28.0\tabularnewline
35.0 & 28.0\tabularnewline
35.0 & 28.0\tabularnewline
40.0 & 23.6\tabularnewline
40.0 & 23.6\tabularnewline
40.0 & 23.4\tabularnewline
40.0 & 23.1\tabularnewline
45.0 & 19.5\tabularnewline
45.0 & 17.2\tabularnewline
45.0 & 17.0\tabularnewline
55.0 & 13.2\tabularnewline
\bottomrule
\end{longtable}

\textbf{\\
}

\hypertarget{correlation}{%
\section{Correlation}\label{correlation}}

A \textbf{correlation} exists between two variables when the values of one variable are somehow associated with the values of the other variable.

When you see a pattern in the data you say there is a correlation in the data. Though this book is only dealing with linear patterns, patterns can be exponential, logarithmic, or periodic. To see this pattern, you can draw a scatter plot of the data.

Remember to read graphs from left to right, the same as you read words. If the graph goes up the correlation is positive and if the graph goes down the correlation is negative.

The words " weak``,''moderate``, and''strong" are used to describe the strength of the relationship between the two variables.

\textbf{Figure 10.2.1: Correlation Graphs}

\includegraphics[width=6in,height=3.76243in]{media/image42.png}

The \textbf{linear} \textbf{correlation coefficient} is a number that describes the strength of the linear relationship between the two variables. It is also called the Pearson correlation coefficient after Karl Pearson who developed it. The symbol for the sample linear correlation coefficient is \emph{r}. The symbol for the population correlation coefficient is \(\rho\) (Greek letter rho).

The formula for \emph{r} is

\begin{quote}
Where
\end{quote}

Assumptions of linear correlation are the same as the assumptions for the regression line:

\begin{enumerate}
\def\labelenumi{\alph{enumi}.}
\item
  The set of ordered pairs is a random sample from the population of all such possible pairs.
\item
  For each fixed value of \emph{x}, the \emph{y}-values have a normal distribution. All of the \emph{y} distributions have the same variance, and for a given \emph{x}-value, the distribution of \emph{y}-values has a mean that lies on the least squares line. You also assume that for a fixed \emph{y}, each \emph{x} has its own normal distribution. This is difficult to figure out, so you can use the following to determine if you have a normal distribution.
\end{enumerate}

\begin{enumerate}
\def\labelenumi{\roman{enumi}.}
\item
  Look to see if the scatter plot has a linear pattern.
\item
  Examine the residuals to see if there is randomness in the residuals. If there is a pattern to the residuals, then there is an issue in the data.
\end{enumerate}

\textbf{Interpretation of the correlation coefficient}

\emph{r} is always between and 1. \emph{r} = means there is a perfect negative linear correlation and \emph{r} = 1 means there is a perfect positive correlation. The closer \emph{r} is to 1 or , the stronger the correlation. The closer \emph{r} is to 0, the weaker the correlation. CAREFUL: \emph{r} = 0 does not mean there is no correlation. It just means there is \textbf{no linear correlation.} There might be a very strong curved pattern.

\textbf{Example \#10.2.1: Calculating the Linear Correlation Coefficient, \emph{r}}

\begin{quote}
How strong is the positive relationship between the alcohol content and the number of calories in 12-ounce beer? To determine if there is a positive linear correlation, a random sample was taken of beer's alcohol content and calories for several different beers ("Calories in beer,," 2011), and the data are in table \#10.2.1. Find the correlation coefficient and interpret that value.
\end{quote}

\textbf{\\
}

\begin{quote}
\textbf{Table \#10.2.1: Alcohol and Calorie Content in Beer without Outlier}
\end{quote}

\begin{longtable}[]{@{}llll@{}}
\toprule
Brand & Brewery & Alcohol Content & Calories in 12 oz\tabularnewline
\midrule
\endhead
Big Sky Scape Goat Pale Ale & Big Sky Brewing & 4.70\% & 163\tabularnewline
Sierra Nevada Harvest Ale & Sierra Nevada & 6.70\% & 215\tabularnewline
Steel Reserve & MillerCoors & 8.10\% & 222\tabularnewline
Coors Light & MillerCoors & 4.15\% & 104\tabularnewline
Genesee Cream Ale & High Falls Brewing & 5.10\% & 162\tabularnewline
Sierra Nevada Summerfest Beer & Sierra Nevada & 5.00\% & 158\tabularnewline
Michelob Beer & Anheuser Busch & 5.00\% & 155\tabularnewline
Flying Dog Doggie Style & Flying Dog Brewery & 4.70\% & 158\tabularnewline
Big Sky I.P.A. & Big Sky Brewing & 6.20\% & 195\tabularnewline
\bottomrule
\end{longtable}

\begin{quote}
\textbf{Solution:}

State random variables

\emph{x} = alcohol content in the beer

\emph{y} = calories in 12 ounce beer

Assumptions check:

From example \#10.1.2, the assumptions have been met.

To compute the correlation coefficient using the TI-83/84 calculator, use the LinRegTTest in the STAT menu. The setup is in figure 10.2.2. The reason that \textgreater{}0 was chosen is because the question was asked if there was a positive correlation. If you are asked if there is a negative correlation, then pick \textless{}0. If you are just asked if there is a correlation, then pick . Right now the choice will not make a different, but it will be important later.

\textbf{Figure \#10.2.2: Setup for Linear Regression Test on TI-83/84}

\includegraphics[width=2.75in,height=1.86111in]{media/image15.png}
\end{quote}

\textbf{\\
}

\begin{quote}
\textbf{Figure \#10.2.3: Results for Linear Regression Test on TI-83/84}

\includegraphics[width=2.75in,height=1.86111in]{media/image52.png}
\includegraphics[width=2.75in,height=1.86111in]{media/image53.png}

To compute the correlation coefficient in R, the command is cor(independent variable, dependent variable). So for this example the command would be cor(alcohol, calories). The output is

\[1\] 0.9134414

The correlation coefficient is . This is close to 1, so it looks like there is a strong, positive correlation.
\end{quote}

\textbf{Causation}

One common mistake people make is to assume that because there is a correlation, then one variable causes the other. This is usually not the case. That would be like saying the amount of alcohol in the beer causes it to have a certain number of calories. However, fermentation of sugars is what causes the alcohol content. The more sugars you have, the more alcohol can be made, and the more sugar, the higher the calories. It is actually the amount of sugar that causes both. Do not confuse the idea of correlation with the concept of causation. Just because two variables are correlated does not mean one causes the other to happen.

\textbf{Example \#10.2.2: Correlation Versus Causation}

\begin{enumerate}
\def\labelenumi{\alph{enumi}.}
\tightlist
\item
  A study showed a strong linear correlation between per capita beer consumption and teacher's salaries. Does giving a teacher a raise cause people to buy more beer? Does buying more beer cause teachers to get a raise?
\end{enumerate}

\begin{quote}
\textbf{Solution:}

There is probably some other factor causing both of them to increase at the same time. Think about this: In a town where people have little extra money, they won't have money for beer and they won't give teachers raises. In another town where people have more extra money to spend it will be easier for them to buy more beer and they would be more willing to give teachers raises.
\end{quote}

\begin{enumerate}
\def\labelenumi{\alph{enumi}.}
\setcounter{enumi}{1}
\tightlist
\item
  A study shows that there is a correlation between people who have had a root canal and those that have cancer. Does that mean having a root canal causes cancer?
\end{enumerate}

\begin{quote}
\textbf{Solution:}

Just because there is positive correlation doesn't mean that one caused the other. It turns out that there is a positive correlation between eating carrots and cancer, but that doesn't mean that eating carrots causes cancer. In other words, there are lots of relationships you can find between two variables, but that doesn't mean that one caused the other.
\end{quote}

Remember a correlation only means a pattern exists. It does not mean that one variable causes the other variable to change.

\textbf{Explained Variation}

As stated before, there is some variability in the dependent variable values, such as calories. Some of the variation in calories is due to alcohol content and some is due to other factors. How much of the variation in the calories is due to alcohol content?

When considering this question, you want to look at how much of the variation in calories is explained by alcohol content and how much is explained by other variables. Realize that some of the changes in calories have to do with other ingredients. You can have two beers at the same alcohol content, but beer one has higher calories because of the other ingredients. Some variability is explained by the model and some variability is not explained. Together, both of these give the total variability. This is

The proportion of the variation that is explained by the model is

This is known as the \textbf{coefficient of determination}.

To find the coefficient of determination, you square the correlation coefficient. In addition, is part of the calculator results.

\textbf{Example \#10.2.3: Finding the Coefficient of Determination}

\begin{quote}
Find the coefficient of variation in calories that is explained by the linear relationship between alcohol content and calories and interpret the value.
\end{quote}

\textbf{Solution:}

From the calculator results,

\begin{quote}
Using R, you can do (cor(independent variable, dependent variable))\^{}2. So that would be (cor(alcohol, calories))\^{}2, and the output would be

\[1\] 0.8343751

Or you can just use a calculator and square the correlation value.

Thus, 83.44\% of the variation in calories is explained to the linear relationship between alcohol content and calories. The other 16.56\% of the variation is due to other factors. A really good coefficient of determination has a very small, unexplained part.
\end{quote}

\textbf{Example \#10.2.4: Using the Formula to Calculate \emph{r} and }

\begin{quote}
How strong is the relationship between the alcohol content and the number of calories in 12-ounce beer? To determine if there is a positive linear correlation, a random sample was taken of beer's alcohol content and calories for several different beers ("Calories in beer,," 2011), and the data are in table \#10.2.1. Find the correlation coefficient and the coefficient of determination using the formula.

\textbf{Solution:}

From example \#10.1.2,

Correlation coefficient:

Coefficient of determination:
\end{quote}

Now that you have a correlation coefficient, how can you tell if it is significant or not? This will be answered in the next section.

\hypertarget{homework-30}{%
\subsection{Homework}\label{homework-30}}

For each problem, state the random variables. Also, look to see if there are any outliers that need to be removed. Do the correlation analysis with and without the suspected outlier points to determine if their removal affects the correlation. The data sets in this section are in section 10.1 and will be used in section 10.3.

\begin{enumerate}
\def\labelenumi{\arabic{enumi}.}
\item
  When an anthropologist finds skeletal remains, they need to figure out the height of the person. The height of a person (in cm) and the length of their metacarpal bone 1 (in cm) were collected and are in table \#10.1.5 ("Prediction of height," 2013). Find the correlation coefficient and coefficient of determination and then interpret both.
\item
  Table \#10.1.6 contains the value of the house and the amount of rental income in a year that the house brings in ("Capital and rental," 2013). Find the correlation coefficient and coefficient of determination and then interpret both.
\item
  The World Bank collects information on the life expectancy of a person in each country ("Life expectancy at," 2013) and the fertility rate per woman in the country ("Fertility rate," 2013). The data for 24 randomly selected countries for the year 2011 are in table \#10.1.7. Find the correlation coefficient and coefficient of determination and then interpret both.
\item
  The World Bank collected data on the percentage of GDP that a country spends on health expenditures ("Health expenditure," 2013) and also the percentage of women receiving prenatal care ("Pregnant woman receiving," 2013). The data for the countries where this information is available for the year 2011 are in table \#10.1.8. Find the correlation coefficient and coefficient of determination and then interpret both.
\item
  The height and weight of baseball players are in table \#10.1.9 ("MLB heightsweights," 2013). Find the correlation coefficient and coefficient of determination and then interpret both.
\item
  Different species have different body weights and brain weights are in table \#10.1.10. ("Brain2bodyweight," 2013). Find the correlation coefficient and coefficient of determination and then interpret both.
\item
  A random sample of beef hotdogs was taken and the amount of sodium (in mg) and calories were measured. ("Data hotdogs," 2013) The data are in table \#10.1.11. Find the correlation coefficient and coefficient of determination and then interpret both.
\item
  Per capita income in 1960 dollars for European countries and the percent of the labor force that works in agriculture in 1960 are in table \#10.1.12 ("OECD economic development," 2013). Find the correlation coefficient and coefficient of determination and then interpret both.
\item
  Cigarette smoking and cancer have been linked. The number of deaths per one hundred thousand from bladder cancer and the number of cigarettes sold per capita in 1960 are in table \#10.1.13 ("Smoking and cancer," 2013). Find the correlation coefficient and coefficient of determination and then interpret both.
\item
  The weight of a car can influence the mileage that the car can obtain. A random sample of cars weights and mileage was collected and are in table \#10.1.14 ("Passenger car mileage," 2013). Find the correlation coefficient and coefficient of determination and then interpret both.
\item
  There is a negative correlation between police expenditure and crime rate. Does this mean that spending more money on police causes the crime rate to decrease? Explain your answer.
\item
  There is a positive correlation between tobacco sales and alcohol sales. Does that mean that using tobacco causes a person to also drink alcohol? Explain your answer.
\item
  There is a positive correlation between the average temperature in a location and the morality rate from breast cancer. Does that mean that higher temperatures cause more women to die of breast cancer? Explain your answer.
\item
  There is a positive correlation between the length of time a tableware company polishes a dish and the price of the dish. Does that mean that the time a plate is polished determines the price of the dish? Explain your answer.
\end{enumerate}

\textbf{\\
}

\hypertarget{inference-for-regression-and-correlation}{%
\section{Inference for Regression and Correlation}\label{inference-for-regression-and-correlation}}

How do you really say you have a correlation? Can you test to see if there really is a correlation? Of course, the answer is yes. The hypothesis test for correlation is as follows:

\textbf{Hypothesis Test for Correlation:}

\begin{enumerate}
\def\labelenumi{\arabic{enumi}.}
\tightlist
\item
  State the random variables in words.
\end{enumerate}

\begin{quote}
\emph{x} = independent variable

\emph{y} = dependent variable
\end{quote}

\begin{enumerate}
\def\labelenumi{\arabic{enumi}.}
\setcounter{enumi}{1}
\tightlist
\item
  State the null and alternative hypotheses and the level of significance
\end{enumerate}

\begin{quote}
Also, state your level here.
\end{quote}

\begin{enumerate}
\def\labelenumi{\arabic{enumi}.}
\setcounter{enumi}{2}
\tightlist
\item
  State and check the assumptions for the hypothesis test
\end{enumerate}

The assumptions for the hypothesis test are the same assumptions for regression and correlation.

\begin{enumerate}
\def\labelenumi{\arabic{enumi}.}
\setcounter{enumi}{3}
\tightlist
\item
  Find the test statistic and p-value
\end{enumerate}

\begin{quote}
with degrees of freedom =

p-value:

Using the TI-83/84:

(Note: if , then lower limit is and upper limit is your test
statistic. If , then lower limit is your test statistic and the upper
limit is . If , then find the p-value for , and multiply by 2.)

Using R:

(Note: if , then use , If , then use . If , then find the p-value for
, and multiply by 2.)
\end{quote}

\begin{enumerate}
\def\labelenumi{\arabic{enumi}.}
\setcounter{enumi}{4}
\tightlist
\item
  Conclusion
\end{enumerate}

\begin{quote}
This is where you write reject or fail to reject . The rule is: if the p-value \textless{} , then reject . If the p-value , then fail to reject
\end{quote}

\begin{enumerate}
\def\labelenumi{\arabic{enumi}.}
\setcounter{enumi}{5}
\tightlist
\item
  Interpretation
\end{enumerate}

\begin{quote}
This is where you interpret in real world terms the conclusion to the test. The conclusion for a hypothesis test is that you either have enough evidence to show is true, or you do not have enough evidence to show is true.
\end{quote}

Note: the TI-83/84 calculator results give you the test statistic and the p-value. In R, the command for getting the test statistic and p-value is cor.test(independent variable, dependent variable, alternative = "less" or "greater"). Use less for {[}MISSING{]}, use greater for {[}MISSING{]}, and leave off this command for {[}MISSING{]} .

\textbf{Example \#10.3.1: Testing the Claim of a Linear Correlation}

\begin{quote}
Is there a positive correlation between beer's alcohol content and calories? To determine if there is a positive linear correlation, a random sample was taken of beer's alcohol content and calories for several different beers ("Calories in beer,," 2011), and the data is in table \#10.2.1. Test at the 5\% level.
\end{quote}

\textbf{Solution:}

\begin{enumerate}
\def\labelenumi{\arabic{enumi}.}
\tightlist
\item
  State the random variables in words.
\end{enumerate}

\begin{quote}
\emph{x} = alcohol content in the beer

\emph{y} = calories in 12 ounce beer
\end{quote}

\begin{enumerate}
\def\labelenumi{\arabic{enumi}.}
\setcounter{enumi}{1}
\tightlist
\item
  State the null and alternative hypotheses and the level of significance
\end{enumerate}

\begin{quote}
Since you are asked if there is a positive correlation, .
\end{quote}

\begin{enumerate}
\def\labelenumi{\arabic{enumi}.}
\setcounter{enumi}{2}
\tightlist
\item
  State and check the assumptions for the hypothesis test
\end{enumerate}

The assumptions for the hypothesis test were already checked in example \#10.1.2.

\begin{enumerate}
\def\labelenumi{\arabic{enumi}.}
\setcounter{enumi}{3}
\tightlist
\item
  Find the test statistic and p-value
\end{enumerate}

\begin{quote}
The results from the TI-83/84 calculator are in figure \#10.3.1.
\end{quote}

\textbf{\\
}

\begin{quote}
\textbf{Figure \#10.3.1: Results for Linear Regression Test on TI-83/84}

\includegraphics[width=2.75in,height=1.86111in]{media/image52.png}

Test statistic: and p-value:

The results from R are

cor.test(alcohol, calories, alternative = "greater")

Pearson's product-moment correlation

data: alcohol and calories

t = 5.9384, df = 7, p-value = 0.0002884

alternative hypothesis: true correlation is greater than 0

95 percent confidence interval:

0.7046161 1.0000000

sample estimates:

cor

0.9134414

Test statistic: and p-value:
\end{quote}

\begin{enumerate}
\def\labelenumi{\arabic{enumi}.}
\setcounter{enumi}{4}
\tightlist
\item
  Conclusion
\end{enumerate}

\begin{quote}
Reject since the p-value is less than 0.05.
\end{quote}

\begin{enumerate}
\def\labelenumi{\arabic{enumi}.}
\setcounter{enumi}{5}
\tightlist
\item
  Interpretation
\end{enumerate}

\begin{quote}
There is enough evidence to show that there is a positive correlation between alcohol content and number of calories in a 12-ounce bottle of beer.
\end{quote}

\textbf{Prediction Interval}

Using the regression equation you can predict the number of calories from the alcohol content. However, you only find one value. The problem is that beers vary a bit in calories even if they have the same alcohol content. It would be nice to have a range instead of a single value. The range is called a prediction interval. To find this, you need to figure out how much error is in the estimate from the regression equation. This is known as the \textbf{standard error of the estimate}.

\textbf{Standard Error of the Estimate}

This is the sum of squares of the residuals

This formula is hard to work with, so there is an easier formula. You can also find the value from technology, such as the calculator.

\textbf{Example \#10.3.2: Finding the Standard Error of the Estimate}

\begin{quote}
Find the standard error of the estimate for the beer data. To determine if there is a positive linear correlation, a random sample was taken of beer's alcohol content and calories for several different beers ("Calories in beer,," 2011), and the data are in table \#10.2.1.

\textbf{Solution:}

\emph{x} = alcohol content in the beer

\emph{y} = calories in 12 ounce beer

Using the TI-83/84, the results are in figure \#10.3.2.

\textbf{Figure \#10.3.2: Results for Linear Regression Test on TI-83/84}

\includegraphics[width=2.75in,height=1.86111in]{media/image102.png}

The \emph{s} in the results is the standard error of the estimate. So .
\end{quote}

To find the standard error of the estimate in R, the commands are

lm.out = lm(dependent variable \textasciitilde{} independent variable) -- this defines the linear model with a name so you can use it later. Then

summary(lm.out) -- this will produce most of the information you need for a regression and correlation analysis. In fact, the only thing R doesn't produce with this command is the correlation coefficient. Otherwise, you can use the command to find the regression equation, coefficient of determination, test statistic, p-value for a two-tailed test, and standard error of the estimate.

\begin{quote}
The results from R are

lm.out=lm(calories\textasciitilde{}alcohol)

summary(lm.out)

Call:

lm(formula = calories \textasciitilde{} alcohol)

Residuals:

Min 1Q Median 3Q Max

-30.253 -1.624 2.744 9.271 14.271

Coefficients:

Estimate Std. Error t value Pr(\textgreater{}\textbar{}t\textbar{})

(Intercept) 25.031 24.999 1.001 0.350038

alcohol 26.319 4.432 5.938 0.000577 ***

-\/-\/-

Signif. codes: 0 `***' 0.001 `**' 0.01 `*' 0.05 `.' 0.1 ' ' 1

Residual standard error: 15.64 on 7 degrees of freedom

Multiple R-squared: 0.8344, Adjusted R-squared: 0.8107

F-statistic: 35.26 on 1 and 7 DF, p-value: 0.0005768

From this output, you can find the y-intercept is 25.031, the slope is
26.319, the test statistic is t = 5.938, the p-value for the
two-tailed test is 0.000577. If you want the p-value for a one-tailed
test, divide this number by 2. The standard error of the estimate is
the residual standard error and is 15.64. There is some information in
this output that you do not need.
\end{quote}

If you want to know how to calculate the standard error of the estimate from the formula, refer to example\# 10.3.3.

\textbf{Example \#10.3.3: Finding the Standard Error of the Estimate from the Formula}

\begin{quote}
Find the standard error of the estimate for the beer data using the formula. To determine if there is a positive linear correlation, a random sample was taken of beer's alcohol content and calories for several different beers ("Calories in beer,," 2011), and the data are in table \#10.2.1.

\textbf{Solution:}

\emph{x} = alcohol content in the beer

\emph{y} = calories in 12 ounce beer

From Example \#10.1.3:

The standard error of the estimate is
\end{quote}

\textbf{Prediction Interval for an Individual \emph{y}}

Given the fixed value , the prediction interval for an individual \emph{y} is

where

Note: to find

You can get the standard deviation from technology.

R will produce the prediction interval for you. The commands are (Note you probably already did the lm.out command. You do not need to do it again.)

lm.out = lm(dependent variable \textasciitilde{} independent variable) -- calculates
the linear model

predict(lm.out, newdata=list(independent variable = value),
interval="prediction", level=C) -- will compute a prediction interval
for the independent variable set to a particular value (put that value
in place of the word value), at a particular C level (given as a
decimal)

\textbf{Example \#10.3.4: Find the Prediction Interval}

\begin{quote}
Find a 95\% prediction interval for the number of calories when the
alcohol content is 6.5\% using a random sample taken of beer's alcohol
content and calories ("Calories in beer,," 2011). The data are in
table \#10.2.1.

\textbf{Solution:}

\emph{x} = alcohol content in the beer

\emph{y} = calories in 12 ounce beer

Computing the prediction interval using the TI-83/84 calculator:

From Example \#10.1.2

From Example \#10.3.2

\textbf{Figure \#10.3.3: Results of 1-Var Stats on TI-83/84}

\includegraphics[width=2.75in,height=1.86111in]{media/image114.png}

Now you can find

Now look in table A.2. Go down the first column to 7, then over to the
column headed with 95\%.

Prediction interval is

Computing the prediction interval using R:

predict(lm.out, newdata=list(alcohol=6.5), interval = "prediction",
level=0.95)

fit lwr upr

1 196.1022 155.7847 236.4196

fit = when x = 6.5\%. lwr = lower limit of prediction interval. upr =
upper limit of prediction interval. So the prediction interval is .

Statistical interpretation: There is a 95\% chance that the interval
contains the true value for the calories when the alcohol content is
6.5\%.

Real world interpretation: If a beer has an alcohol content of 6.50\%
then it has between 156 and 236 calories.
\end{quote}

\textbf{Example \#10.3.5:~ Doing a Correlation and Regression Analysis Using
the TI-83/84}

Table \#10.3.1 contains randomly selected high temperatures at various
cities on a single day and the elevation of the city.

\begin{quote}
\textbf{Table \#10.3.1: Temperatures and Elevation of Cities on a Given
Day}
\end{quote}

\begin{longtable}[]{@{}llllllll@{}}
\toprule
Elevation (in feet) & 7000 & 4000 & 6000 & 3000 & 7000 & 4500 & 5000\tabularnewline
\midrule
\endhead
Temperature (°F) & 50 & 60 & 48 & 70 & 55 & 55 & 60\tabularnewline
\bottomrule
\end{longtable}

\begin{enumerate}
\def\labelenumi{\alph{enumi}.}
\item
  State the random variables.

  \textbf{Solution:}
\end{enumerate}

\begin{quote}
\emph{x} = elevation

\emph{y} = high temperature
\end{quote}

\begin{enumerate}
\def\labelenumi{\alph{enumi}.}
\setcounter{enumi}{1}
\item
  Find a regression equation for elevation and high temperature on a
  given day.

  \textbf{Solution:}
\end{enumerate}

\begin{enumerate}
\def\labelenumi{\alph{enumi}.}
\item
  A random sample was taken as stated in the problem.
\item
  The distribution for each high temperature value is normally
  distributed for every value of elevation.
\end{enumerate}

\begin{enumerate}
\def\labelenumi{\roman{enumi}.}
\tightlist
\item
  Look at the scatter plot of high temperature versus elevation.
\end{enumerate}

\begin{quote}
\textbf{Figure \#10.3.4: Scatter Plot of Temperature Versus Elevation}

{\[CHART\]}

The scatter plot looks fairly linear.
\end{quote}

\begin{enumerate}
\def\labelenumi{\roman{enumi}.}
\setcounter{enumi}{1}
\item
  There are no points that appear to be outliers.
\item
  The residual plot for temperature versus elevation appears to be
  fairly random. (See figure \#10.3.7.)
\end{enumerate}

\begin{quote}
It appears that the high temperature is normally distributed.

All calculations computed using the TI-83/84 calculator.

\textbf{Figure \#10.3.5: Setup for Linear Regression on TI-83/84
Calculator}

\includegraphics[width=2.75in,height=1.86111in]{media/image124.png}
\end{quote}

\textbf{\\
}

\begin{quote}
\textbf{Figure \#10.3.6: Results for Linear Regression on TI-83/84
Calculator}

\includegraphics[width=2.75in,height=1.86111in]{media/image125.png}

\includegraphics[width=2.75in,height=1.86111in]{media/image126.png}
\end{quote}

\begin{enumerate}
\def\labelenumi{\alph{enumi}.}
\setcounter{enumi}{2}
\tightlist
\item
  Find the residuals and create a residual plot.
\end{enumerate}

\begin{quote}
\textbf{Solution:}
\end{quote}

\textbf{Table \#10.3.2: Residuals for Elevation vs.~Temperature Data}

\begin{longtable}[]{@{}llll@{}}
\toprule
\emph{x} & \emph{y} & &\tabularnewline
\midrule
\endhead
7000 & 50 & 50.1 & -0.1\tabularnewline
4000 & 60 & 61.8 & -1.8\tabularnewline
6000 & 48 & 54.0 & -6.0\tabularnewline
3000 & 70 & 65.7 & 4.3\tabularnewline
7000 & 55 & 50.1 & 4.9\tabularnewline
4500 & 55 & 59.85 & -4.85\tabularnewline
5000 & 60 & 57.9 & 2.1\tabularnewline
\bottomrule
\end{longtable}

\textbf{Figure \#10.3.7: Residual Plot for Temperature vs.~Elevation}

{\[CHART\]}

The residuals appear to be fairly random.

\begin{enumerate}
\def\labelenumi{\alph{enumi}.}
\setcounter{enumi}{3}
\tightlist
\item
  Use the regression equation to estimate the high temperature on that
  day at an elevation of 5500 ft.
\end{enumerate}

\begin{quote}
\textbf{Solution:}
\end{quote}

\begin{enumerate}
\def\labelenumi{\alph{enumi}.}
\setcounter{enumi}{4}
\tightlist
\item
  Use the regression equation to estimate the high temperature on that
  day at an elevation of 8000 ft.
\end{enumerate}

\begin{quote}
\textbf{Solution:}
\end{quote}

\begin{enumerate}
\def\labelenumi{\alph{enumi}.}
\setcounter{enumi}{5}
\tightlist
\item
  Between the answers to parts d and e, which estimate is probably
  more accurate and why?
\end{enumerate}

\begin{quote}
\textbf{Solution:}

Part d is more accurate, since it is interpolation and part e is
extrapolation.
\end{quote}

\begin{enumerate}
\def\labelenumi{\alph{enumi}.}
\setcounter{enumi}{6}
\tightlist
\item
  Find the correlation coefficient and coefficient of determination
  and interpret both.
\end{enumerate}

\begin{quote}
\textbf{Solution:}

From figure \#10.3.6, the correlation coefficient is

, which is moderate to strong negative correlation.

From figure \#10.3.6, the coefficient of determination is

, which means that 66.3\% of the variability in high temperature is
explained by the linear model. The other 33.7\% is explained by other
variables such as local weather conditions.
\end{quote}

\begin{enumerate}
\def\labelenumi{\alph{enumi}.}
\setcounter{enumi}{7}
\tightlist
\item
  Is there enough evidence to show a negative correlation between
  elevation and high temperature? Test at the 5\% level.
\end{enumerate}

\begin{quote}
\textbf{Solution:}
\end{quote}

\begin{enumerate}
\def\labelenumi{\arabic{enumi}.}
\tightlist
\item
  State the random variables in words.
\end{enumerate}

\begin{quote}
\emph{x} = elevation

\emph{y} = high temperature
\end{quote}

\begin{enumerate}
\def\labelenumi{\arabic{enumi}.}
\setcounter{enumi}{1}
\item
  State the null and alternative hypotheses and the level of
  significance
\item
  State and check the assumptions for the hypothesis test

  The assumptions for the hypothesis test were already checked part b.
\item
  Find the test statistic and p-value
\end{enumerate}

\begin{quote}
From figure \#10.3.6,

Test statistic:
\end{quote}

p-value:

\begin{enumerate}
\def\labelenumi{\arabic{enumi}.}
\setcounter{enumi}{4}
\tightlist
\item
  Conclusion
\end{enumerate}

\begin{quote}
Reject since the p-value is less than 0.05.
\end{quote}

\begin{enumerate}
\def\labelenumi{\arabic{enumi}.}
\setcounter{enumi}{5}
\tightlist
\item
  Interpretation
\end{enumerate}

\begin{quote}
There is enough evidence to show that there is a negative correlation
between elevation and high temperatures.
\end{quote}

\begin{enumerate}
\def\labelenumi{\roman{enumi}.}
\tightlist
\item
  Find the standard error of the estimate.
\end{enumerate}

\begin{quote}
\textbf{Solution:}

From figure \#10.3.6,
\end{quote}

\begin{enumerate}
\def\labelenumi{\alph{enumi}.}
\setcounter{enumi}{9}
\item
  Using a 95\% prediction interval, find a range for high temperature
  for an elevation of 6500 feet.

  \textbf{Solution:}
\end{enumerate}

\begin{quote}
\textbf{Figure \#10.3.8: Results of 1-Var Stats on TI-83/84}

\includegraphics[width=2.75in,height=1.86111in]{media/image143.png}

Now you can find

Now look in table A.2. Go down the first column to 5, then over to the
column headed with 95\%.

So

Prediction interval is

Statistical interpretation: There is a 95\% chance that the interval
contains the true value for the temperature at an elevation of 6500
feet.

Real world interpretation: A city of 6500 feet will have a high
temperature between 38.6°F and 65.6°F. Though this interval is fairly
wide, at least the interval tells you that the temperature isn't that
warm.
\end{quote}

\textbf{\\
}

\textbf{Example \#10.3.6:~ Doing a Correlation and Regression Analysis Using
R}

Table \#10.3.1 contains randomly selected high temperatures at various
cities on a single day and the elevation of the city.

\begin{enumerate}
\def\labelenumi{\alph{enumi}.}
\item
  State the random variables.

  \textbf{Solution:}
\end{enumerate}

\begin{quote}
\emph{x} = elevation

\emph{y} = high temperature
\end{quote}

\begin{enumerate}
\def\labelenumi{\alph{enumi}.}
\setcounter{enumi}{1}
\item
  Find a regression equation for elevation and high temperature on a
  given day.

  \textbf{Solution:}
\end{enumerate}

\begin{enumerate}
\def\labelenumi{\alph{enumi}.}
\item
  A random sample was taken as stated in the problem.
\item
  The distribution for each high temperature value is normally
  distributed for every value of elevation.
\end{enumerate}

\begin{enumerate}
\def\labelenumi{\roman{enumi}.}
\tightlist
\item
  Look at the scatter plot of high temperature versus elevation.
\end{enumerate}

\begin{quote}
R command: plot(elevation, temperature, main="Scatter Plot for
Temperature vs Elevation", xlab="Elevation (feet)",
ylab="Temperature (degrees F)", ylim=c(0,80))

\textbf{Figure \#10.3.9: Scatter Plot of Temperature Versus Elevation}

\includegraphics[width=4.15278in,height=4.15278in]{media/image151.emf}

The scatter plot looks fairly linear.
\end{quote}

\begin{enumerate}
\def\labelenumi{\roman{enumi}.}
\setcounter{enumi}{1}
\tightlist
\item
  The residual plot for temperature versus elevation appears to be
  fairly random. (See figure \#10.3.10.)
\end{enumerate}

\begin{quote}
It appears that the high temperature is normally distributed.

Using R:

Commands:

lm.out=lm(temperature \textasciitilde{} elevation)

summary(lm.out)

Output:

Call:

lm(formula = temperature \textasciitilde{} elevation)

Residuals:

1 2 3 4 5 6 7

0.1667 -1.6333 -5.7667 4.4333 5.1667 -4.6667 2.3000

Coefficients:

Estimate Std. Error t value Pr(\textgreater{}\textbar{}t\textbar{})

(Intercept) 77.366667 6.769182 11.429 8.98e-05 ***

elevation -0.003933 0.001253 -3.139 0.0257 *

-\/-\/-

Signif. codes: 0 `***' 0.001 `**' 0.01 `*' 0.05 `.' 0.1 ' ' 1

Residual standard error: 4.677 on 5 degrees of freedom

Multiple R-squared: 0.6633, Adjusted R-squared: 0.596

F-statistic: 9.852 on 1 and 5 DF, p-value: 0.0257

From the output you can see the slope = -0.0039 and the y-intercept =
77.4. So the regression equation is:
\end{quote}

\begin{enumerate}
\def\labelenumi{\alph{enumi}.}
\setcounter{enumi}{2}
\tightlist
\item
  Find the residuals and create a residual plot.
\end{enumerate}

\begin{quote}
\textbf{Solution:}
\end{quote}

R command: (notice these are also in the summary(lm.out) output, but if
you have too many data points, then R only gives a numerical summary of
the residuals.)

residuals(lm.out)

1 2 3 4 5 6

0.1666667 -1.6333333 -5.7666667 4.4333333 5.1666667 -4.6666667

7

2.3000000

So for the first x of 7000, the residual is approximately 0.1667. This
means if you find the for when x is 7000 and then subtract this answer
from the y value of 50 that was measured, you would obtain 0.1667.
Similar process is computed for the other residual values.

To plot the residuals, the R command is

plot(elevation, residuals(lm.out), main="Residual Plot for Temperautre
vs Elevation", xlab="Elevation (feet)", ylab="Residuals")

abline(0,0)

\textbf{Figure \#10.3.10: Residual Plot for Temperature vs.~Elevation}

\includegraphics[width=2.52778in,height=2.52778in]{media/image154.emf}

The residuals appear to be fairly random.

\begin{enumerate}
\def\labelenumi{\alph{enumi}.}
\setcounter{enumi}{3}
\tightlist
\item
  Use the regression equation to estimate the high temperature on that
  day at an elevation of 5500 ft.
\end{enumerate}

\begin{quote}
\textbf{Solution:}
\end{quote}

\begin{enumerate}
\def\labelenumi{\alph{enumi}.}
\setcounter{enumi}{4}
\tightlist
\item
  Use the regression equation to estimate the high temperature on that
  day at an elevation of 8000 ft.
\end{enumerate}

\begin{quote}
\textbf{Solution:}
\end{quote}

\begin{enumerate}
\def\labelenumi{\alph{enumi}.}
\setcounter{enumi}{5}
\tightlist
\item
  Between the answers to parts d and e, which estimate is probably
  more accurate and why?
\end{enumerate}

\begin{quote}
\textbf{Solution:}

Part d is more accurate, since it is interpolation and part e is
extrapolation.
\end{quote}

\begin{enumerate}
\def\labelenumi{\alph{enumi}.}
\setcounter{enumi}{6}
\tightlist
\item
  Find the correlation coefficient and coefficient of determination
  and interpret both.
\end{enumerate}

\begin{quote}
\textbf{Solution:}

The R command for the correlation coefficient is

cor(elevation, temperature)

\[1\] -0.8144564

So, , which is moderate to strong negative correlation.

From summary(lm.out), the coefficient of determination is the Multiple
R-squared.

So, which means that 66.3\% of the variability in high temperature is
explained by the linear model. The other 33.7\% is explained by other
variables such as local weather conditions.
\end{quote}

\begin{enumerate}
\def\labelenumi{\alph{enumi}.}
\setcounter{enumi}{7}
\tightlist
\item
  Is there enough evidence to show a negative correlation between
  elevation and high temperature? Test at the 5\% level.
\end{enumerate}

\begin{quote}
\textbf{Solution:}
\end{quote}

\begin{enumerate}
\def\labelenumi{\arabic{enumi}.}
\tightlist
\item
  State the random variables in words.
\end{enumerate}

\begin{quote}
\emph{x} = elevation

\emph{y} = high temperature
\end{quote}

\begin{enumerate}
\def\labelenumi{\arabic{enumi}.}
\setcounter{enumi}{1}
\item
  State the null and alternative hypotheses and the level of
  significance
\item
  State and check the assumptions for the hypothesis test

  The assumptions for the hypothesis test were already checked part b.
\item
  Find the test statistic and p-value
\end{enumerate}

\begin{quote}
The R command is cor.test(elevation, temperature, alternative =
"less")

Pearson's product-moment correlation

data: elevation and temperature

t = -3.1387, df = 5, p-value = 0.01285

alternative hypothesis: true correlation is less than 0

95 percent confidence interval:

-1.0000000 -0.3074247

sample estimates:

cor

-0.8144564

Test statistic: and p-value:
\end{quote}

\begin{enumerate}
\def\labelenumi{\arabic{enumi}.}
\setcounter{enumi}{4}
\tightlist
\item
  Conclusion
\end{enumerate}

\begin{quote}
Reject since the p-value is less than 0.05.
\end{quote}

\begin{enumerate}
\def\labelenumi{\arabic{enumi}.}
\setcounter{enumi}{5}
\tightlist
\item
  Interpretation
\end{enumerate}

\begin{quote}
There is enough evidence to show that there is a negative correlation
between elevation and high temperatures.
\end{quote}

\begin{enumerate}
\def\labelenumi{\roman{enumi}.}
\tightlist
\item
  Find the standard error of the estimate.
\end{enumerate}

\begin{quote}
\textbf{Solution:}

From summary(lm.out), Residual standard error: 4.677.

So,
\end{quote}

\begin{enumerate}
\def\labelenumi{\alph{enumi}.}
\setcounter{enumi}{9}
\item
  Using a 95\% prediction interval, find a range for high temperature
  for an elevation of 6500 feet.

  \textbf{Solution:}
\end{enumerate}

\begin{quote}
R command is predict(lm.out, newdata=list(elevation = 6500), interval
= "prediction", level=0.95)

fit lwr upr

1 51.8 38.29672 65.30328

So when x = 6500 feet, and .

Statistical interpretation: There is a 95\% chance that the interval
contains the true value for the temperature at an elevation of 6500
feet.

Real world interpretation: A city of 6500 feet will have a high
temperature between 38.3°F and 65.3°F. Though this interval is fairly
wide, at least the interval tells you that the temperature isn't that
warm.
\end{quote}

\hypertarget{homework-31}{%
\subsection{Homework}\label{homework-31}}

For each problem, state the random variables. The data sets in this section are in the homework for section 10.1 and were also used in section 10.2. If you removed any data points as outliers in the other sections, remove them in this sections homework too.

\begin{enumerate}
\def\labelenumi{\arabic{enumi}.}
\tightlist
\item
  When an anthropologist finds skeletal remains, they need to figure out the height of the person. The height of a person (in cm) and the length of their metacarpal bone one (in cm) were collected and are in table \#10.1.5 ("Prediction of height," 2013).
\end{enumerate}

\begin{enumerate}
\def\labelenumi{\alph{enumi}.}
\item
  Test at the 1\% level for a positive correlation between length of metacarpal bone one and height of a person.
\item
  Find the standard error of the estimate.
\item
  Compute a 99\% prediction interval for height of a person with a metacarpal length of 44 cm.
\end{enumerate}

\begin{enumerate}
\def\labelenumi{\arabic{enumi}.}
\setcounter{enumi}{1}
\tightlist
\item
  Table \#10.1.6 contains the value of the house and the amount of rental income in a year that the house brings in ("Capital and rental," 2013).
\end{enumerate}

\begin{enumerate}
\def\labelenumi{\alph{enumi}.}
\item
  Test at the 5\% level for a positive correlation between house value and rental amount.
\item
  Find the standard error of the estimate.
\item
  Compute a 95\% prediction interval for the rental income on a house worth \$230,000.
\end{enumerate}

\begin{enumerate}
\def\labelenumi{\arabic{enumi}.}
\setcounter{enumi}{2}
\tightlist
\item
  The World Bank collects information on the life expectancy of a person in each country ("Life expectancy at," 2013) and the fertility rate per woman in the country ("Fertility rate," 2013). The data for 24 randomly selected countries for the year 2011 are in table \#10.1.7.
\end{enumerate}

\begin{enumerate}
\def\labelenumi{\alph{enumi}.}
\item
  Test at the 1\% level for a negative correlation between fertility rate and life expectancy.
\item
  Find the standard error of the estimate.
\item
  Compute a 99\% prediction interval for the life expectancy for a country that has a fertility rate of 2.7.
\end{enumerate}

\begin{enumerate}
\def\labelenumi{\arabic{enumi}.}
\setcounter{enumi}{3}
\tightlist
\item
  The World Bank collected data on the percentage of GDP that a country spends on health expenditures ("Health expenditure," 2013) and also the percentage of women receiving prenatal care ("Pregnant woman receiving," 2013). The data for the countries where this information is available for the year 2011 are in table \#10.1.8.
\end{enumerate}

\begin{enumerate}
\def\labelenumi{\alph{enumi}.}
\item
  Test at the 5\% level for a correlation between percentage spent on health expenditure and the percentage of women receiving prenatal care.
\item
  Find the standard error of the estimate.
\item
  Compute a 95\% prediction interval for the percentage of woman receiving prenatal care for a country that spends 5.0 \% of GDP on health expenditure.
\end{enumerate}

\begin{enumerate}
\def\labelenumi{\arabic{enumi}.}
\setcounter{enumi}{4}
\tightlist
\item
  The height and weight of baseball players are in table \#10.1.9 ("MLB heightsweights," 2013).
\end{enumerate}

\begin{enumerate}
\def\labelenumi{\alph{enumi}.}
\item
  Test at the 5\% level for a positive correlation between height and weight of baseball players.
\item
  Find the standard error of the estimate.
\item
  Compute a 95\% prediction interval for the weight of a baseball player that is 75 inches tall.
\end{enumerate}

\begin{enumerate}
\def\labelenumi{\arabic{enumi}.}
\setcounter{enumi}{5}
\tightlist
\item
  Different species have different body weights and brain weights are in table \#10.1.10. ("Brain2bodyweight," 2013).
\end{enumerate}

\begin{enumerate}
\def\labelenumi{\alph{enumi}.}
\item
  Test at the 1\% level for a positive correlation between body weights and brain weights.
\item
  Find the standard error of the estimate.
\item
  Compute a 99\% prediction interval for the brain weight for a species that has a body weight of 62 kg.
\end{enumerate}

\begin{enumerate}
\def\labelenumi{\arabic{enumi}.}
\setcounter{enumi}{6}
\tightlist
\item
  A random sample of beef hotdogs was taken and the amount of sodium (in mg) and calories were measured. ("Data hotdogs," 2013) The data are in table \#10.1.11.
\end{enumerate}

\begin{enumerate}
\def\labelenumi{\alph{enumi}.}
\item
  Test at the 5\% level for a correlation between amount of calories and amount of sodium.
\item
  Find the standard error of the estimate.
\item
  Compute a 95\% prediction interval for the amount of sodium a beef hotdog has if it is 170 calories.
\end{enumerate}

\begin{enumerate}
\def\labelenumi{\arabic{enumi}.}
\setcounter{enumi}{7}
\tightlist
\item
  Per capita income in 1960 dollars for European countries and the percent of the labor force that works in agriculture in 1960 are in table \#10.1.12 ("OECD economic development," 2013).
\end{enumerate}

\begin{enumerate}
\def\labelenumi{\alph{enumi}.}
\item
  Test at the 5\% level for a negative correlation between percent of labor force in agriculture and per capita income.
\item
  Find the standard error of the estimate.
\item
  Compute a 90\% prediction interval for the per capita income in a country that has 21 percent of labor in agriculture.
\end{enumerate}

\begin{enumerate}
\def\labelenumi{\arabic{enumi}.}
\setcounter{enumi}{8}
\tightlist
\item
  Cigarette smoking and cancer have been linked. The number of deaths per one hundred thousand from bladder cancer and the number of cigarettes sold per capita in 1960 are in table \#10.1.13 ("Smoking and cancer," 2013).
\end{enumerate}

\begin{enumerate}
\def\labelenumi{\alph{enumi}.}
\item
  Test at the 1\% level for a positive correlation between cigarette smoking and deaths of bladder cancer.
\item
  Find the standard error of the estimate.
\item
  Compute a 99\% prediction interval for the number of deaths from bladder cancer when the cigarette sales were 20 per capita.
\end{enumerate}

\begin{enumerate}
\def\labelenumi{\arabic{enumi}.}
\setcounter{enumi}{9}
\tightlist
\item
  The weight of a car can influence the mileage that the car can obtain. A random sample of cars weights and mileage was collected and are in table \#10.1.14 ("Passenger car mileage," 2013).
\end{enumerate}

\begin{enumerate}
\def\labelenumi{\alph{enumi}.}
\item
  Test at the 5\% level for a negative correlation between the weight of cars and mileage.
\item
  Find the standard error of the estimate.
\item
  Compute a 95\% prediction interval for the mileage on a car that weighs 3800 pounds.
\end{enumerate}

Data Source:

\emph{Brain2bodyweight}. (2013, November 16). Retrieved from
\url{http://wiki.stat.ucla.edu/socr/index.php/SOCR_Data_Brain2BodyWeight}

\emph{Calories in beer, beer alcohol, beer carbohydrates}. (2011, October
25). Retrieved from
\href{http://www.beer100.com/beercalories.htm}{www.beer100.com/beercalories.htm}

\emph{Capital and rental values of Auckland properties}. (2013, September
26). Retrieved from \url{http://www.statsci.org/data/oz/rentcap.html}

\emph{Data hotdogs}. (2013, November 16). Retrieved from
\url{http://wiki.stat.ucla.edu/socr/index.php/SOCR_012708_ID_Data_HotDogs}

\emph{Fertility rate}. (2013, October 14). Retrieved from
\url{http://data.worldbank.org/indicator/SP.DYN.TFRT.IN}

\emph{Health expenditure}. (2013, October 14). Retrieved from
\url{http://data.worldbank.org/indicator/SH.XPD.TOTL.ZS}

\emph{Life expectancy at birth}. (2013, October 14). Retrieved from
\url{http://data.worldbank.org/indicator/SP.DYN.LE00.IN}

\emph{MLB heightsweights}. (2013, November 16). Retrieved from
\url{http://wiki.stat.ucla.edu/socr/index.php/SOCR_Data_MLB_HeightsWeights}

\emph{OECD economic development}. (2013, December 04). Retrieved from
\url{http://lib.stat.cmu.edu/DASL/Datafiles/oecdat.html}

\emph{Passenger car mileage}. (2013, December 04). Retrieved from
\url{http://lib.stat.cmu.edu/DASL/Datafiles/carmpgdat.html}

\emph{Prediction of height from metacarpal bone length}. (2013, September
26). Retrieved from \url{http://www.statsci.org/data/general/stature.html}

\emph{Pregnant woman receiving prenatal care}. (2013, October 14). Retrieved
from \url{http://data.worldbank.org/indicator/SH.STA.ANVC.ZS}

\emph{Smoking and cancer}. (2013, December 04). Retrieved from
\url{http://lib.stat.cmu.edu/DASL/Datafiles/cigcancerdat.html}

\hypertarget{chi-square-and-anova-tests}{%
\chapter{Chi-Square and ANOVA Tests}\label{chi-square-and-anova-tests}}

This chapter presents material on three more hypothesis tests. One is used to determine significant relationship between two qualitative variables, the second is used to determine if the sample data has a particular distribution, and the last is used to determine significant relationships between means of 3 or more samples.

\hypertarget{chi-square-test-for-independence}{%
\section{Chi-Square Test for Independence}\label{chi-square-test-for-independence}}

Remember, qualitative data is where you collect data on individuals that are categories or names. Then you would count how many of the individuals had particular qualities. An example is that there is a theory that there is a relationship between breastfeeding and autism. To determine if there is a relationship, researchers could collect the time period that a mother breastfed her child and if that child was diagnosed with autism. Then you would have a table containing this information. Now you want to know if each cell is independent of each other cell. Remember, independence says that one event does not affect another event. Here it means that having autism is independent of being breastfed. What you really want is to see if they are not independent. In other words, does one affect the other? If you were to do a hypothesis test, this is your alternative hypothesis and the null hypothesis is that they are independent. There is a hypothesis test for this and it is called the \textbf{Chi-Square Test for Independence}. Technically it should be called the Chi-Square Test for Dependence, but for historical reasons it is known as the test for independence. Just as with previous hypothesis tests, all the steps are the same except for the assumptions and the test statistic.

\textbf{Hypothesis Test for Chi-Square Test}

\begin{enumerate}
\def\labelenumi{\arabic{enumi}.}
\tightlist
\item
  State the null and alternative hypotheses and the level of significance
\end{enumerate}

\begin{itemize}
\tightlist
\item
  the two variables are independent (this means that the one variable is not affected by the other)
\item
  the two variables are dependent (this means that the one variable is \textgreater{} affected by the other)
\item
  Also, state your level here.
\end{itemize}

\begin{enumerate}
\def\labelenumi{\arabic{enumi}.}
\setcounter{enumi}{1}
\tightlist
\item
  State and check the assumptions for the hypothesis test
\end{enumerate}

\begin{enumerate}
\def\labelenumi{\alph{enumi}.}
\item
  A random sample is taken.
\item
  Expected frequencies for each cell are greater than or equal to 5 (The expected frequencies, \emph{E}, will be calculated later, and this assumption means ).
\end{enumerate}

\begin{enumerate}
\def\labelenumi{\arabic{enumi}.}
\setcounter{enumi}{2}
\tightlist
\item
  Find the test statistic and p-value
\end{enumerate}

\begin{quote}
Finding the test statistic involves several steps. First the data is collected and counted, and then it is organized into a table (in a table each entry is called a cell). These values are known as the observed frequencies, which the symbol for an observed frequency is \emph{O}. Each table is made up of rows and columns. Then each row is totaled to give a row total and each column is totaled to give a column total.

The null hypothesis is that the variables are independent. Using the multiplication rule for independent events you can calculate the probability of being one value of the first variable, \emph{A}, and one value of the second variable, \emph{B} (the probability of a particular cell ). Remember in a hypothesis test, you assume that is true, the two variables are assumed to be independent.

Now you want to find out how many individuals you expect to be in a certain cell. To find the expected frequencies, you just need to multiply the probability of that cell times the total number of individuals. Do not round the expected frequencies.

If the variables are independent the expected frequencies and the observed frequencies should be the same. The test statistic here will involve looking at the difference between the expected frequency and the observed frequency for each cell. Then you want to find the ``total difference'' of all of these differences. The larger the total, the smaller the chances that you could find that test statistic given that the assumption of independence is true. That means that the assumption of independence is not true. How do you find the test statistic? First find the differences between the observed and expected frequencies. Because some of these differences will be positive and some will be negative, you need to square these differences. These squares could be large just because the frequencies are large, you need to divide by the expected frequencies to scale them. Then finally add up all of these fractional values. This is the test statistic.
\end{quote}

\textbf{Test Statistic:}

\begin{quote}
The symbol for Chi-Square is

where \emph{O} is the observed frequency and \emph{E} is the expected frequency
\end{quote}

\textbf{\\
}

\textbf{Distribution of Chi-Square}

\begin{quote}
has different curves depending on the degrees of freedom. It is skewed
to the right for small degrees of freedom and gets more symmetric as
the degrees of freedom increases (see figure \#11.1.1). Since the test
statistic involves squaring the differences, the test statistics are
all positive. A chi-squared test for independence is always right
tailed.

\textbf{Figure \#11.1.1: Chi-Square Distribution}

\includegraphics[width=3.77778in,height=2.45833in]{media/image12.png}
\end{quote}

p-value:

\begin{quote}
Using the TI-83/84:

Using R:

Where the degrees of freedom is
\end{quote}

\begin{enumerate}
\def\labelenumi{\arabic{enumi}.}
\setcounter{enumi}{3}
\tightlist
\item
  Conclusion
\end{enumerate}

\begin{quote}
This is where you write reject or fail to reject . The rule is: if the p-value \textless{} , then reject . If the p-value , then fail to reject
\end{quote}

\begin{enumerate}
\def\labelenumi{\arabic{enumi}.}
\setcounter{enumi}{4}
\tightlist
\item
  Interpretation
\end{enumerate}

\begin{quote}
This is where you interpret in real world terms the conclusion to the test. The conclusion for a hypothesis test is that you either have enough evidence to show is true, or you do not have enough evidence to show is true.
\end{quote}

\textbf{Example \#11.1.1: Hypothesis Test with Chi-Square Test Using Formula}

\begin{quote}
Is there a relationship between autism and breastfeeding? To determine if there is, a researcher asked mothers of autistic and non-autistic children to say what time period they breastfed their children. The data is in table \#11.1.1 (Schultz, Klonoff-Cohen, Wingard, Askhoomoff, Macera, Ji \& Bacher, 2006). Do the data provide enough evidence to show that that breastfeeding and autism are independent? Test at the 1\% level.
\end{quote}

\textbf{\\
}

\begin{quote}
\textbf{Table \#11.1.1: Autism Versus Breastfeeding}
\end{quote}

\begin{longtable}[]{@{}llllll@{}}
\toprule
Autism & Breast Feeding Timelines & Row Total & & &\tabularnewline
\midrule
\endhead
& None & Less than 2 months & 2 to 6 months & More than 6 months &\tabularnewline
Yes & 241 & 198 & 164 & 215 & 818\tabularnewline
No & 20 & 25 & 27 & 44 & 116\tabularnewline
Column Total & 261 & 223 & 191 & 259 & 934\tabularnewline
\bottomrule
\end{longtable}

\begin{quote}
\textbf{Solution:}
\end{quote}

\begin{enumerate}
\def\labelenumi{\arabic{enumi}.}
\tightlist
\item
  State the null and alternative hypotheses and the level of significance
\end{enumerate}

\begin{quote}
Breastfeeding and autism are independent

Breastfeeding and autism are dependent
\end{quote}

\begin{enumerate}
\def\labelenumi{\arabic{enumi}.}
\setcounter{enumi}{1}
\tightlist
\item
  State and check the assumptions for the hypothesis test
\end{enumerate}

\begin{enumerate}
\def\labelenumi{\alph{enumi}.}
\item
  A random sample of breastfeeding time frames and autism incidence was taken.
\item
  Expected frequencies for each cell are greater than or equal to 5 (i.e. {[}MISSING{]} ). See step 3. All expected frequencies are more than 5.
\end{enumerate}

\begin{enumerate}
\def\labelenumi{\arabic{enumi}.}
\setcounter{enumi}{2}
\tightlist
\item
  Find the test statistic and p-value
\end{enumerate}

Test statistic:

First find the expected frequencies for each cell.

\begin{quote}
Others are done similarly. It is easier to do the calculations for the test statistic with a table, the others are in table \#11.1.2 along with the calculation for the test statistic. (Note: the column of should add to 0 or close to 0.)
\end{quote}

\textbf{\\
}

\begin{quote}
\textbf{Table \#11.1.2: Calculations for Chi-Square Test Statistic}
\end{quote}

\begin{longtable}[]{@{}lllll@{}}
\toprule
\emph{O} & \emph{E} & & &\tabularnewline
\midrule
\endhead
241 & 228.585 & 12.415 & 154.132225 & 0.674288448\tabularnewline
198 & 195.304 & 2.696 & 7.268416 & 0.03721591\tabularnewline
164 & 167.278 & -3.278 & 10.745284 & 0.064236086\tabularnewline
215 & 226.833 & -11.833 & 140.019889 & 0.617281828\tabularnewline
20 & 32.4154 & -12.4154 & 154.1421572 & 4.755213792\tabularnewline
25 & 27.6959 & -2.6959 & 7.26787681 & 0.262417066\tabularnewline
27 & 23.7216 & 3.2784 & 10.74790656 & 0.453085229\tabularnewline
44 & 32.167 & 11.833 & 140.019889 & 4.352904809\tabularnewline
Total & & 0.0001 & & 11.2166432\tabularnewline
\bottomrule
\end{longtable}

\begin{quote}
The test statistic formula is , which is the total of the last column in table \#11.1.2.
\end{quote}

p-value:

\begin{quote}
Using TI-83/84:

Using R:
\end{quote}

\begin{enumerate}
\def\labelenumi{\arabic{enumi}.}
\setcounter{enumi}{3}
\tightlist
\item
  Conclusion
\end{enumerate}

\begin{quote}
Fail to reject since the p-value is more than 0.01.
\end{quote}

\begin{enumerate}
\def\labelenumi{\arabic{enumi}.}
\setcounter{enumi}{4}
\tightlist
\item
  Interpretation
\end{enumerate}

\begin{quote}
There is not enough evidence to show that breastfeeding and autism are dependent. This means that you cannot say that the whether a child is breastfed or not will indicate if that the child will be diagnosed with autism.
\end{quote}

\textbf{Example \#11.1.2: Hypothesis Test with Chi-Square Test Using
Technology}

\begin{quote}
Is there a relationship between autism and breastfeeding? To determine if there is, a researcher asked mothers of autistic and non-autistic children to say what time period they breastfed their children. The data is in table \#11.1.1 (Schultz, Klonoff-Cohen, Wingard, Askhoomoff, Macera, Ji \& Bacher, 2006). Do the data provide enough evidence to show that that breastfeeding and autism are independent? Test at the 1\% level.

\textbf{Solution:}
\end{quote}

\begin{enumerate}
\def\labelenumi{\arabic{enumi}.}
\tightlist
\item
  State the null and alternative hypotheses and the level of significance
\end{enumerate}

\begin{quote}
Breastfeeding and autism are independent

Breastfeeding and autism are dependent
\end{quote}

\begin{enumerate}
\def\labelenumi{\arabic{enumi}.}
\setcounter{enumi}{1}
\tightlist
\item
  State and check the assumptions for the hypothesis test
\end{enumerate}

\begin{enumerate}
\def\labelenumi{\alph{enumi}.}
\item
  A random sample of breastfeeding time frames and autism incidence was taken.
\item
  Expected frequencies for each cell are greater than or equal to 5 (ie. ). See step 3. All expected frequencies are more than 5.
\end{enumerate}

\begin{enumerate}
\def\labelenumi{\arabic{enumi}.}
\setcounter{enumi}{2}
\tightlist
\item
  Find the test statistic and p-value
\end{enumerate}

Test statistic:

\begin{quote}
To use the TI-83/84 calculator to compute the test statistic, you must first put the data into the calculator. However, this process is different than for other hypothesis tests. You need to put the data in as a matrix instead of in the list. Go into the MATRX menu then move over to EDIT and choose 1:\[A\]. This will allow you to type the table into the calculator. Figure \#11.1.2 shows what you will see on your calculator when you choose 1:\[A\] from the EDIT menu.

\textbf{Figure \#11.1.2: Matrix Edit Menu on TI-83/84}

\includegraphics[width=2.75in,height=1.86111in]{media/image46.png}

The table has 2 rows and 4 columns (don't include the row total column and the column total row in your count). You need to tell the calculator that you have a 2 by 4. The 1 X1 (you might have another size in your matrix, but it doesn't matter because you will change it) on the calculator is the size of the matrix. So type 2 ENTER and 4 ENTER and the calculator will make a matrix of the correct size. See figure \#11.1.3.

\textbf{Figure \#11.1.3: Matrix Setup for Table}

\includegraphics[width=2.75in,height=1.86111in]{media/image47.png}

Now type the table in by pressing ENTER after each cell value. Figure \#11.1.4 contains the complete table typed in. Once you have the data in, press QUIT.

\textbf{Figure \#11.1.4: Data Typed into Matrix}

\includegraphics[width=2.75in,height=1.86111in]{media/image48.png}

To run the test on the calculator, go into STAT, then move over to TEST and choose -Test from the list. The setup for the test is in figure \#11.1.5.

\textbf{Figure \#11.1.5: Setup for Chi-Square Test on TI-83/84}

\includegraphics[width=2.75in,height=1.86111in]{media/image50.png}

Once you press ENTER on Calculate you will see the results in figure \#11.1.6.

\textbf{Figure \#11.1.6: Results for Chi-Square Test on TI-83/84}

\includegraphics[width=2.75in,height=1.86111in]{media/image51.png}

The test statistic is and the p-value is . Notice that the calculator calculates the expected values for you and places them in matrix B. To review the expected values, go into MATRX and choose 2:\[B\]. Figure \#11.1.7 shows the output. Press the right arrows to see the entire matrix.

\textbf{Figure \#11.1.7: Expected Frequency for Chi-Square Test on TI-83/84}

\includegraphics[width=2.75in,height=1.86111in]{media/image54.png}

\includegraphics[width=2.75in,height=1.86111in]{media/image55.png}

To compute the test statistic and p-value with R,

row1 = c(data from row 1 separated by commas)

row2 = c(data from row 2 separated by commas)

keep going until you have all of your rows typed in.

data.table = rbind(row1, row2, \ldots{}) -- makes the data into a table.
You can call it what ever you want. It does not have to be data.table.

data.table -- use if you want to look at the table

chisq.test(data.table) -- calculates the chi-squared test for
independence

chisq.test(data.table)\$expected -- let's you see the expected values

For this example, the commands would be

row1 = c(241, 198, 164, 215)

row2 = c(20, 25, 27, 44)

data.table = rbind(row1, row2)

data.table

Output:

\[,1\] \[,2\] \[,3\] \[,4\]

row1 241 198 164 215

row2 20 25 27 44

chisq.test(data.table)

Output:

Pearson's Chi-squared test

data: data.table

X-squared = 11.217, df = 3, p-value = 0.01061

chisq.test(data.table)\$expected

Output:

\[,1\] \[,2\] \[,3\] \[,4\]

row1 228.58458 195.30407 167.27837 226.83298

row2 32.41542 27.69593 23.72163 32.16702

The test statistic is and the p-value is .
\end{quote}

\begin{enumerate}
\def\labelenumi{\arabic{enumi}.}
\setcounter{enumi}{3}
\tightlist
\item
  Conclusion
\end{enumerate}

\begin{quote}
Fail to reject since the p-value is more than 0.01.
\end{quote}

\begin{enumerate}
\def\labelenumi{\arabic{enumi}.}
\setcounter{enumi}{4}
\tightlist
\item
  Interpretation
\end{enumerate}

\begin{quote}
There is not enough evidence to show that breastfeeding and autism are dependent. This means that you cannot say that the whether a child is breastfed or not will indicate if that the child will be diagnosed with autism.
\end{quote}

\textbf{Example \#11.1.3: Hypothesis Test with Chi-Square Test Using Formula}

\begin{quote}
The World Health Organization (WHO) keeps track of how many incidents of leprosy there are in the world. Using the WHO regions and the World Banks income groups, one can ask if an income level and a WHO region are dependent on each other in terms of predicting where the disease is. Data on leprosy cases in different countries was collected for the year 2011 and a summary is presented in table \#11.1.3 ("Leprosy: Number of," 2013). Is there evidence to show that income level and WHO region are independent when dealing with the disease of leprosy? Test at the 5\% level.

\textbf{Table \#11.1.3: Number of Leprosy Cases}
\end{quote}

\begin{longtable}[]{@{}llllll@{}}
\toprule
WHO Region & World Bank Income Group & Row Total & & &\tabularnewline
\midrule
\endhead
& High Income & Upper Middle Income & Lower Middle Income & Low Income &\tabularnewline
Americas & 174 & 36028 & 615 & 0 & 36817\tabularnewline
Eastern Mediterranean & 54 & 6 & 1883 & 604 & 2547\tabularnewline
Europe & 10 & 0 & 0 & 0 & 10\tabularnewline
Western Pacific & 26 & 216 & 3689 & 1155 & 5086\tabularnewline
Africa & 0 & 39 & 1986 & 15928 & 17953\tabularnewline
South-East Asia & 0 & 0 & 149896 & 10236 & 160132\tabularnewline
Column Total & 264 & 36289 & 158069 & 27923 & 222545\tabularnewline
\bottomrule
\end{longtable}

\begin{quote}
\textbf{Solution:}
\end{quote}

\begin{enumerate}
\def\labelenumi{\arabic{enumi}.}
\tightlist
\item
  State the null and alternative hypotheses and the level of
  significance
\end{enumerate}

\begin{quote}
WHO region and Income Level when dealing with the disease of leprosy are independent

WHO region and Income Level when dealing with the disease of leprosy are dependent
\end{quote}

\begin{enumerate}
\def\labelenumi{\arabic{enumi}.}
\setcounter{enumi}{1}
\tightlist
\item
  State and check the assumptions for the hypothesis test
\end{enumerate}

\begin{enumerate}
\def\labelenumi{\alph{enumi}.}
\item
  A random sample of incidence of leprosy was taken from different countries and the income level and WHO region was taken.
\item
  Expected frequencies for each cell are greater than or equal to 5 (ie. ). See step 3. There are actually 4 expected frequencies that are less than 5, and the results of the test may not be valid. If you look at the expected frequencies you will notice that they are all in Europe. This is because Europe didn't have many cases in 2011.
\end{enumerate}

\begin{enumerate}
\def\labelenumi{\arabic{enumi}.}
\setcounter{enumi}{2}
\tightlist
\item
  Find the test statistic and p-value
\end{enumerate}

Test statistic:

First find the expected frequencies for each cell.

\begin{quote}
Others are done similarly. It is easier to do the calculations for the test statistic with a table, and the others are in table \#11.1.4 along with the calculation for the test statistic.

\textbf{Table \#11.1.4: Calculations for Chi-Square Test Statistic}
\end{quote}

\begin{longtable}[]{@{}lllll@{}}
\toprule
\emph{O} & \emph{E} & & &\tabularnewline
\midrule
\endhead
174 & 43.675 & 130.325 & 16984.564 & 388.8838719\tabularnewline
54 & 3.021 & 50.979 & 2598.813 & 860.1218328\tabularnewline
10 & 0.012 & 9.988 & 99.763 & 8409.746711\tabularnewline
26 & 6.033 & 19.967 & 398.665 & 66.07628214\tabularnewline
0 & 21.297 & -21.297 & 453.572 & 21.29722977\tabularnewline
0 & 189.961 & -189.961 & 36085.143 & 189.9608978\tabularnewline
36028 & 6003.514 & 30024.486 & 901469735.315 & 150157.0038\tabularnewline
6 & 415.323 & -409.323 & 167545.414 & 403.4097962\tabularnewline
0 & 1.631 & -1.631 & 2.659 & 1.6306365\tabularnewline
216 & 829.342 & -613.342 & 376188.071 & 453.5983897\tabularnewline
39 & 2927.482 & -2888.482 & 8343326.585 & 2850.001268\tabularnewline
0 & 26111.708 & -26111.708 & 681821316.065 & 26111.70841\tabularnewline
615 & 26150.335 & -25535.335 & 652053349.724 & 24934.7988\tabularnewline
1883 & 1809.080 & 73.920 & 5464.144 & 3.020398811\tabularnewline
0 & 7.103 & -7.103 & 50.450 & 7.1027882\tabularnewline
3689 & 3612.478 & 76.522 & 5855.604 & 1.620938405\tabularnewline
1986 & 12751.636 & -10765.636 & 115898911.071 & 9088.944681\tabularnewline
149896 & 113738.368 & 36157.632 & 1307374351.380 & 11494.57632\tabularnewline
0 & 4619.475 & -4619.475 & 21339550.402 & 4619.475122\tabularnewline
604 & 319.575 & 284.425 & 80897.421 & 253.1404187\tabularnewline
0 & 1.255 & -1.255 & 1.574 & 1.25471253\tabularnewline
1155 & 638.147 & 516.853 & 267137.238 & 418.6140882\tabularnewline
15928 & 2252.585 & 13675.415 & 187016964.340 & 83023.25138\tabularnewline
10236 & 20091.963 & -9855.963 & 97140000.472 & 4834.769106\tabularnewline
Total & & 0.000 & & 328594.008\tabularnewline
\bottomrule
\end{longtable}

\begin{quote}
The test statistic formula is , which is the total of the last column in table \#11.1.2.
\end{quote}

p-value:

\begin{quote}
Using the TI-83/84:

Using R:
\end{quote}

\begin{enumerate}
\def\labelenumi{\arabic{enumi}.}
\setcounter{enumi}{3}
\tightlist
\item
  Conclusion
\end{enumerate}

\begin{quote}
Reject since the p-value is less than 0.05.
\end{quote}

\begin{enumerate}
\def\labelenumi{\arabic{enumi}.}
\setcounter{enumi}{4}
\tightlist
\item
  Interpretation
\end{enumerate}

\begin{quote}
There is enough evidence to show that WHO region and income level are dependent when dealing with the disease of leprosy. WHO can decide how to focus their efforts based on region and income level. Do remember though that the results may not be valid due to the expected frequencies not all be more than 5.
\end{quote}

\textbf{Example \#11.1.4: Hypothesis Test with Chi-Square Test Using Technology}

\begin{quote}
The World Health Organization (WHO) keeps track of how many incidents of leprosy there are in the world. Using the WHO regions and the World Banks income groups, one can ask if an income level and a WHO region are dependent on each other in terms of predicting where the disease is. Data on leprosy cases in different countries was collected for the year 2011 and a summary is presented in table \#11.1.3 ("Leprosy: Number of," 2013). Is there evidence to show that income level and WHO region are independent when dealing with the disease of leprosy? Test at the 5\% level.

\textbf{Solution:}
\end{quote}

\begin{enumerate}
\def\labelenumi{\arabic{enumi}.}
\tightlist
\item
  State the null and alternative hypotheses and the level of
  significance
\end{enumerate}

\begin{quote}
WHO region and Income Level when dealing with the disease of leprosy are independent

WHO region and Income Level when dealing with the disease of leprosy are dependent
\end{quote}

\begin{enumerate}
\def\labelenumi{\arabic{enumi}.}
\setcounter{enumi}{1}
\tightlist
\item
  State and check the assumptions for the hypothesis test
\end{enumerate}

\begin{enumerate}
\def\labelenumi{\alph{enumi}.}
\item
  A random sample of incidence of leprosy was taken from different countries and the income level and WHO region was taken.
\item
  Expected frequencies for each cell are greater than or equal to 5 (ie. ). See step 3. There are actually 4 expected frequencies that are less than 5, and the results of the test may not be valid. If you look at the expected frequencies you will notice that they are all in Europe. This is because Europe didn't have many cases in 2011.
\end{enumerate}

\begin{enumerate}
\def\labelenumi{\arabic{enumi}.}
\setcounter{enumi}{2}
\tightlist
\item
  Find the test statistic and p-value
\end{enumerate}

Test statistic:

\begin{quote}
Using the TI-83/84. See example \#11.1.2 for the process of doing the test on the calculator. Remember, you need to put the data in as a matrix instead of in the list.
\end{quote}

\textbf{\\
}

\textbf{Figure \#11.1.8: Setup for Matrix on TI-83/84}

\begin{quote}
\includegraphics[width=2.75in,height=1.86111in]{media/image80.png}

\includegraphics[width=2.75in,height=1.86111in]{media/image81.png}
\end{quote}

\textbf{Figure \#11.1.9: Results for Chi-Square Test on TI-83/84}

\begin{quote}
\includegraphics[width=2.75in,height=1.86111in]{media/image82.png}

\textbf{Figure \#11.1.10: Expected Frequency for Chi-Square Test on
TI-83/84}

\includegraphics[width=2.75in,height=1.86111in]{media/image84.png}

Press the right arrow to look at the other expected frequencies.
\end{quote}

p-value:

\begin{quote}
Using R:

row1=c(174, 36028, 615, 0)

row2=c(54, 6, 1883, 604)

row3=c(10, 0, 0, 0)

row4=c(26, 216, 3689, 1155)

row5=c(0, 39, 1986, 15928)

row6=c(0, 0, 149896, 10236)

chisq.test(data.table)

Pearson's Chi-squared test

data: data.table

X-squared = 328590, df = 15, p-value \textless{} 2.2e-16

Warning message:

In chisq.test(data.table) : Chi-squared approximation may be incorrect

chisq.test(data.table)\$expected

\[,1\] \[,2\] \[,3\] \[,4\]

row1 43.67515783 6003.514404 2.615034e+04 4619.475122

row2 3.02144735 415.323117 1.809080e+03 319.575281

row3 0.01186277 1.630637 7.102788e+00 1.254713

row4 6.03340448 829.341724 3.612478e+03 638.146793

row5 21.29722977 2927.481709 1.275164e+04 2252.585405

row6 189.96089780 26111.708410 1.137384e+05 20091.962686

Warning message:

In chisq.test(data.table) : Chi-squared approximation may be incorrect

and p-value =
\end{quote}

\begin{enumerate}
\def\labelenumi{\arabic{enumi}.}
\setcounter{enumi}{3}
\tightlist
\item
  Conclusion
\end{enumerate}

\begin{quote}
Reject since the p-value is less than 0.05.
\end{quote}

\begin{enumerate}
\def\labelenumi{\arabic{enumi}.}
\setcounter{enumi}{4}
\tightlist
\item
  Interpretation
\end{enumerate}

\begin{quote}
There is enough evidence to show that WHO region and income level are dependent when dealing with the disease of leprosy. WHO can decide how to focus their efforts based on region and income level. Do remember though that the results may not be valid due to the expected frequencies not all be more than 5.
\end{quote}

\hypertarget{homework-32}{%
\subsection{Homework}\label{homework-32}}

In each problem show all steps of the hypothesis test. If some of the assumptions are not met, note that the results of the test may not be correct and then continue the process of the hypothesis test.

\begin{enumerate}
\def\labelenumi{\arabic{enumi}.}
\item
  The number of people who survived the Titanic based on class and sex is in table \#11.1.5 ("Encyclopedia Titanica," 2013). Is there enough evidence to show that the class and the sex of a person who survived the Titanic are independent? Test at the 5\% level.

  \textbf{Table \#11.1.5: Surviving the Titanic}
\end{enumerate}

\begin{longtable}[]{@{}llll@{}}
\toprule
Class & Sex & Total &\tabularnewline
\midrule
\endhead
& Female & Male &\tabularnewline
1st & 134 & 59 & 193\tabularnewline
2nd & 94 & 25 & 119\tabularnewline
3rd & 80 & 58 & 138\tabularnewline
Total & 308 & 142 & 450\tabularnewline
\bottomrule
\end{longtable}

\begin{enumerate}
\def\labelenumi{\arabic{enumi}.}
\setcounter{enumi}{1}
\item
  Researchers watched groups of dolphins off the coast of Ireland in 1998 to determine what activities the dolphins partake in at certain times of the day ("Activities of dolphin," 2013). The numbers in table \#11.1.6 represent the number of groups of dolphins that were partaking in an activity at certain times of days. Is there enough evidence to show that the activity and the time period are independent for dolphins? Test at the 1\% level.

  \textbf{Table \#11.1.6: Dolphin Activity}
\end{enumerate}

\begin{longtable}[]{@{}llllll@{}}
\toprule
\begin{minipage}[b]{0.17\columnwidth}\raggedright
Activity\strut
\end{minipage} & \begin{minipage}[b]{0.12\columnwidth}\raggedright
Period\strut
\end{minipage} & \begin{minipage}[b]{0.09\columnwidth}\raggedright
Row

Total\strut
\end{minipage} & \begin{minipage}[b]{0.14\columnwidth}\raggedright
\strut
\end{minipage} & \begin{minipage}[b]{0.12\columnwidth}\raggedright
\strut
\end{minipage} & \begin{minipage}[b]{0.07\columnwidth}\raggedright
\strut
\end{minipage}\tabularnewline
\midrule
\endhead
\begin{minipage}[t]{0.17\columnwidth}\raggedright
\strut
\end{minipage} & \begin{minipage}[t]{0.12\columnwidth}\raggedright
Morning\strut
\end{minipage} & \begin{minipage}[t]{0.09\columnwidth}\raggedright
Noon\strut
\end{minipage} & \begin{minipage}[t]{0.14\columnwidth}\raggedright
Afternoon\strut
\end{minipage} & \begin{minipage}[t]{0.12\columnwidth}\raggedright
Evening\strut
\end{minipage} & \begin{minipage}[t]{0.07\columnwidth}\raggedright
\strut
\end{minipage}\tabularnewline
\begin{minipage}[t]{0.17\columnwidth}\raggedright
Travel\strut
\end{minipage} & \begin{minipage}[t]{0.12\columnwidth}\raggedright
6\strut
\end{minipage} & \begin{minipage}[t]{0.09\columnwidth}\raggedright
6\strut
\end{minipage} & \begin{minipage}[t]{0.14\columnwidth}\raggedright
14\strut
\end{minipage} & \begin{minipage}[t]{0.12\columnwidth}\raggedright
13\strut
\end{minipage} & \begin{minipage}[t]{0.07\columnwidth}\raggedright
39\strut
\end{minipage}\tabularnewline
\begin{minipage}[t]{0.17\columnwidth}\raggedright
Feed\strut
\end{minipage} & \begin{minipage}[t]{0.12\columnwidth}\raggedright
28\strut
\end{minipage} & \begin{minipage}[t]{0.09\columnwidth}\raggedright
4\strut
\end{minipage} & \begin{minipage}[t]{0.14\columnwidth}\raggedright
0\strut
\end{minipage} & \begin{minipage}[t]{0.12\columnwidth}\raggedright
56\strut
\end{minipage} & \begin{minipage}[t]{0.07\columnwidth}\raggedright
88\strut
\end{minipage}\tabularnewline
\begin{minipage}[t]{0.17\columnwidth}\raggedright
Social\strut
\end{minipage} & \begin{minipage}[t]{0.12\columnwidth}\raggedright
38\strut
\end{minipage} & \begin{minipage}[t]{0.09\columnwidth}\raggedright
5\strut
\end{minipage} & \begin{minipage}[t]{0.14\columnwidth}\raggedright
9\strut
\end{minipage} & \begin{minipage}[t]{0.12\columnwidth}\raggedright
10\strut
\end{minipage} & \begin{minipage}[t]{0.07\columnwidth}\raggedright
62\strut
\end{minipage}\tabularnewline
\begin{minipage}[t]{0.17\columnwidth}\raggedright
Column Total\strut
\end{minipage} & \begin{minipage}[t]{0.12\columnwidth}\raggedright
72\strut
\end{minipage} & \begin{minipage}[t]{0.09\columnwidth}\raggedright
15\strut
\end{minipage} & \begin{minipage}[t]{0.14\columnwidth}\raggedright
23\strut
\end{minipage} & \begin{minipage}[t]{0.12\columnwidth}\raggedright
79\strut
\end{minipage} & \begin{minipage}[t]{0.07\columnwidth}\raggedright
189\strut
\end{minipage}\tabularnewline
\bottomrule
\end{longtable}

\begin{enumerate}
\def\labelenumi{\arabic{enumi}.}
\setcounter{enumi}{2}
\tightlist
\item
  Is there a relationship between autism and what an infant is fed? To determine if there is, a researcher asked mothers of autistic and non-autistic children to say what they fed their infant. The data is in table \#11.1.7 (Schultz, Klonoff-Cohen, Wingard, Askhoomoff, Macera, Ji \& Bacher, 2006). Do the data provide enough evidence to show that that what an infant is fed and autism are independent? Test at the1\% level.
\end{enumerate}

\begin{quote}
\textbf{Table \#11.1.7: Autism Versus Breastfeeding}
\end{quote}

\begin{longtable}[]{@{}lllll@{}}
\toprule
Autism & Feeding & Row Total & &\tabularnewline
\midrule
\endhead
& Brest-feeding & Formula with DHA/ARA & Formula without DHA/ARA &\tabularnewline
Yes & 12 & 39 & 65 & 116\tabularnewline
No & 6 & 22 & 10 & 38\tabularnewline
Column Total & 18 & 61 & 75 & 154\tabularnewline
\bottomrule
\end{longtable}

\begin{enumerate}
\def\labelenumi{\arabic{enumi}.}
\setcounter{enumi}{3}
\item
  A person's educational attainment and age group was collected by the U.S. Census Bureau in 1984 to see if age group and educational attainment are related. The counts in thousands are in table \#11.1.8 ("Education by age," 2013). Do the data show that educational attainment and age are independent? Test at the 5\% level.

  \textbf{Table \#11.1.8: Educational Attainment and Age Group}
\end{enumerate}

\begin{longtable}[]{@{}lllllll@{}}
\toprule
Education & Age Group & Row Total & & & &\tabularnewline
\midrule
\endhead
& 25-34 & 35-44 & 45-54 & 55-64 & \textgreater{}64 &\tabularnewline
Did not complete HS & 5416 & 5030 & 5777 & 7606 & 13746 & 37575\tabularnewline
Competed HS & 16431 & 1855 & 9435 & 8795 & 7558 & 44074\tabularnewline
College 1-3 years & 8555 & 5576 & 3124 & 2524 & 2503 & 22282\tabularnewline
College 4 or more years & 9771 & 7596 & 3904 & 3109 & 2483 & 26863\tabularnewline
Column Total & 40173 & 20057 & 22240 & 22034 & 26290 & 130794\tabularnewline
\bottomrule
\end{longtable}

\begin{enumerate}
\def\labelenumi{\arabic{enumi}.}
\setcounter{enumi}{4}
\tightlist
\item
  Students at multiple grade schools were asked what their personal goal (get good grades, be popular, be good at sports) was and how important good grades were to them (1 very important and 4 least important). The data is in table \#11.1.9 ("Popular kids datafile," 2013). Do the data provide enough evidence to show that goal attainment and importance of grades are independent? Test at the 5\% level.
\end{enumerate}

\begin{quote}
\textbf{Table \#11.1.9: Personal Goal and Importance of Grades}
\end{quote}

\begin{longtable}[]{@{}llllll@{}}
\toprule
Goal & Grades Importance Rating & Row Total & & &\tabularnewline
\midrule
\endhead
& 1 & 2 & 3 & 4 &\tabularnewline
Grades & 70 & 66 & 55 & 56 & 247\tabularnewline
Popular & 14 & 33 & 45 & 49 & 141\tabularnewline
Sports & 10 & 24 & 33 & 23 & 90\tabularnewline
Column Total & 94 & 123 & 133 & 128 & 478\tabularnewline
\bottomrule
\end{longtable}

\begin{enumerate}
\def\labelenumi{\arabic{enumi}.}
\setcounter{enumi}{5}
\tightlist
\item
  Students at multiple grade schools were asked what their personal goal (get good grades, be popular, be good at sports) was and how important being good at sports were to them (1 very important and 4 least important). The data is in table \#11.1.10 ("Popular kids datafile," 2013). Do the data provide enough evidence to show that goal attainment and importance of sports are independent? Test at the 5\% level.
\end{enumerate}

\begin{quote}
\textbf{Table \#11.1.10: Personal Goal and Importance of Sports}
\end{quote}

\begin{longtable}[]{@{}llllll@{}}
\toprule
Goal & Sports Importance Rating & Row Total & & &\tabularnewline
\midrule
\endhead
& 1 & 2 & 3 & 4 &\tabularnewline
Grades & 83 & 81 & 55 & 28 & 247\tabularnewline
Popular & 32 & 49 & 43 & 17 & 141\tabularnewline
Sports & 50 & 24 & 14 & 2 & 90\tabularnewline
Column Total & 165 & 154 & 112 & 47 & 478\tabularnewline
\bottomrule
\end{longtable}

\begin{enumerate}
\def\labelenumi{\arabic{enumi}.}
\setcounter{enumi}{6}
\tightlist
\item
  Students at multiple grade schools were asked what their personal goal (get good grades, be popular, be good at sports) was and how important having good looks were to them (1 very important and 4 least important). The data is in table \#11.1.11 ("Popular kids datafile," 2013). Do the data provide enough evidence to show that goal attainment and importance of looks are independent? Test at the 5\% level.
\end{enumerate}

\begin{quote}
\textbf{Table \#11.1.11: Personal Goal and Importance of Looks}
\end{quote}

\begin{longtable}[]{@{}llllll@{}}
\toprule
Goal & Looks Importance Rating & Row Total & & &\tabularnewline
\midrule
\endhead
& 1 & 2 & 3 & 4 &\tabularnewline
Grades & 80 & 66 & 66 & 35 & 247\tabularnewline
Popular & 81 & 30 & 18 & 12 & 141\tabularnewline
Sports & 24 & 30 & 17 & 19 & 90\tabularnewline
Column Total & 185 & 126 & 101 & 66 & 478\tabularnewline
\bottomrule
\end{longtable}

\begin{enumerate}
\def\labelenumi{\arabic{enumi}.}
\setcounter{enumi}{7}
\tightlist
\item
  Students at multiple grade schools were asked what their personal goal (get good grades, be popular, be good at sports) was and how important having money were to them (1 very important and 4 least important). The data is in table \#11.1.12 ("Popular kids datafile," 2013). Do the data provide enough evidence to show that goal attainment and importance of money are independent? Test at the 5\% level.
\end{enumerate}

\begin{quote}
\textbf{Table \#11.1.12: Personal Goal and Importance of Money}
\end{quote}

\begin{longtable}[]{@{}llllll@{}}
\toprule
Goal & Money Importance Rating & Row Total & & &\tabularnewline
\midrule
\endhead
& 1 & 2 & 3 & 4 &\tabularnewline
Grades & 14 & 34 & 71 & 128 & 247\tabularnewline
Popular & 14 & 29 & 35 & 63 & 141\tabularnewline
Sports & 6 & 12 & 26 & 46 & 90\tabularnewline
Column Total & 34 & 75 & 132 & 237 & 478\tabularnewline
\bottomrule
\end{longtable}

\textbf{\\
}

\hypertarget{chi-square-goodness-of-fit}{%
\section{Chi-Square Goodness of Fit}\label{chi-square-goodness-of-fit}}

In probability, you calculated probabilities using both experimental and theoretical methods. There are times when it is important to determine how well the experimental values match the theoretical values. An example of this is if you wish to verify if a die is fair. To determine if observed values fit the expected values, you want to see if the difference between observed values and expected values is large enough to say that the test statistic is unlikely to happen if you assume that the observed values fit the expected values. The test statistic in this case is also the chi-square. The process is the same as for the chi-square test for independence.

\textbf{Hypothesis Test for Goodness of Fit Test}

\begin{enumerate}
\def\labelenumi{\arabic{enumi}.}
\tightlist
\item
  State the null and alternative hypotheses and the level of significance
\end{enumerate}

\begin{quote}
The data are consistent with a specific distribution

The data are not consistent with a specific distribution

Also, state your level here.
\end{quote}

\begin{enumerate}
\def\labelenumi{\arabic{enumi}.}
\setcounter{enumi}{1}
\tightlist
\item
  State and check the assumptions for the hypothesis test
\end{enumerate}

\begin{enumerate}
\def\labelenumi{\alph{enumi}.}
\item
  A random sample is taken.
\item
  Expected frequencies for each cell are greater than or equal to 5 (The expected frequencies, \emph{E}, will be calculated later, and this assumption means ).
\end{enumerate}

\begin{enumerate}
\def\labelenumi{\arabic{enumi}.}
\setcounter{enumi}{2}
\tightlist
\item
  Find the test statistic and p-value
\end{enumerate}

\begin{quote}
Finding the test statistic involves several steps. First the data is collected and counted, and then it is organized into a table (in a table each entry is called a cell). These values are known as the observed frequencies, which the symbol for an observed frequency is \emph{O}. The table is made up of \emph{k} entries. The total number of observed frequencies is \emph{n}. The expected frequencies are calculated by multiplying the probability of each entry, \emph{p}, times \emph{n}.
\end{quote}

\textbf{Test Statistic:}

\begin{quote}
where \emph{O} is the observed frequency and \emph{E} is the expected frequency

Again, the test statistic involves squaring the differences, so the test statistics are all positive. Thus a chi-squared test for goodness of fit is always right tailed.
\end{quote}

p-value:

\begin{quote}
Using the TI-83/84:

Using R:

Where the degrees of freedom is
\end{quote}

\begin{enumerate}
\def\labelenumi{\arabic{enumi}.}
\setcounter{enumi}{3}
\tightlist
\item
  Conclusion
\end{enumerate}

\begin{quote}
This is where you write reject or fail to reject . The rule is: if the p-value \textless{} , then reject . If the p-value , then fail to reject
\end{quote}

\begin{enumerate}
\def\labelenumi{\arabic{enumi}.}
\setcounter{enumi}{4}
\tightlist
\item
  Interpretation
\end{enumerate}

\begin{quote}
This is where you interpret in real world terms the conclusion to the test. The conclusion for a hypothesis test is that you either have enough evidence to show is true, or you do not have enough evidence to show is true.
\end{quote}

\textbf{Example \#11.2.1: Goodness of Fit Test Using the Formula}

\begin{quote}
Suppose you have a die that you are curious if it is fair or not. If it is fair then the proportion for each value should be the same. You need to find the observed frequencies and to accomplish this you roll the die 500 times and count how often each side comes up. The data is in table \#11.2.1. Do the data show that the die is fair? Test at the 5\% level.

\textbf{Table \#11.2.1: Observed Frequencies of Die}
\end{quote}

\begin{longtable}[]{@{}llllllll@{}}
\toprule
Die values & 1 & 2 & 3 & 4 & 5 & 6 & Total\tabularnewline
\midrule
\endhead
Observed Frequency & 78 & 87 & 87 & 76 & 85 & 87 & 100\tabularnewline
\bottomrule
\end{longtable}

\begin{quote}
\textbf{Solution:}
\end{quote}

\begin{enumerate}
\def\labelenumi{\arabic{enumi}.}
\tightlist
\item
  State the null and alternative hypotheses and the level of significance
\end{enumerate}

\begin{quote}
The observed frequencies are consistent with the distribution for fair die (the die is fair)

The observed frequencies are not consistent with the distribution for fair die (the die is not fair)
\end{quote}

\begin{enumerate}
\def\labelenumi{\arabic{enumi}.}
\setcounter{enumi}{1}
\tightlist
\item
  State and check the assumptions for the hypothesis test
\end{enumerate}

\begin{enumerate}
\def\labelenumi{\alph{enumi}.}
\item
  A random sample is taken since each throw of a die is a random event.
\item
  Expected frequencies for each cell are greater than or equal to 5. See step 3.
\end{enumerate}

\begin{enumerate}
\def\labelenumi{\arabic{enumi}.}
\setcounter{enumi}{2}
\tightlist
\item
  Find the test statistic and p-value
\end{enumerate}

\begin{quote}
First you need to find the probability of rolling each side of the die. The sample space for rolling a die is \{1, 2, 3, 4, 5, 6\}. Since you are assuming that the die is fair, then {[}MISSING{]}.
\end{quote}

\begin{quote}
Now you can find the expected frequency for each side of the die. Since all the probabilities are the same, then each expected frequency is the same.
\end{quote}

\begin{quote}
Test Statistic:

It is easier to calculate the test statistic using a table.

\textbf{Table \#11.2.2: Calculation of the Chi-Square Test Statistic}
\end{quote}

\begin{longtable}[]{@{}lllll@{}}
\toprule
\emph{O} & \emph{E} & & &\tabularnewline
\midrule
\endhead
78 & 83.33 & -5.33 & 28.4089 & 0.340920437\tabularnewline
87 & 83.33 & 3.67 & 13.4689 & 0.161633265\tabularnewline
87 & 83.33 & 3.67 & 13.4689 & 0.161633265\tabularnewline
76 & 83.33 & -7.33 & 53.7289 & 0.644772591\tabularnewline
85 & 83.33 & 1.67 & 2.7889 & 0.033468139\tabularnewline
87 & 83.33 & 3.67 & 13.4689 & 0.161633265\tabularnewline
Total & & 0.02 & & 1.504060962\tabularnewline
\bottomrule
\end{longtable}

The test statistic is 1.504060962

The degrees of freedom are

Using TI-83/84:

Using R:

\begin{enumerate}
\def\labelenumi{\arabic{enumi}.}
\setcounter{enumi}{3}
\tightlist
\item
  Conclusion
\end{enumerate}

\begin{quote}
Fail to reject since the p-value is greater than 0.05.
\end{quote}

\begin{enumerate}
\def\labelenumi{\arabic{enumi}.}
\setcounter{enumi}{4}
\tightlist
\item
  Interpretation
\end{enumerate}

\begin{quote}
There is not enough evidence to show that the die is not consistent with the distribution for a fair die. There is not enough evidence to show that the die is not fair.
\end{quote}

\textbf{Example \#11.2.2: Goodness of Fit Test Using Technology}

\begin{quote}
Suppose you have a die that you are curious if it is fair or not. If it is fair then the proportion for each value should be the same. You need to find the observed frequencies and to accomplish this you roll the die 500 times and count how often each side comes up. The data is in table \#11.2.1. Do the data show that the die is fair? Test at the 5\% level.

\textbf{Solution:}
\end{quote}

\begin{enumerate}
\def\labelenumi{\arabic{enumi}.}
\tightlist
\item
  State the null and alternative hypotheses and the level of significance
\end{enumerate}

\begin{quote}
The observed frequencies are consistent with the distribution for fair die (the die is fair)

The observed frequencies are not consistent with the distribution for fair die (the die is not fair)
\end{quote}

\begin{enumerate}
\def\labelenumi{\arabic{enumi}.}
\setcounter{enumi}{1}
\tightlist
\item
  State and check the assumptions for the hypothesis test
\end{enumerate}

\begin{enumerate}
\def\labelenumi{\alph{enumi}.}
\item
  A random sample is taken since each throw of a die is a random
  event.
\item
  Expected frequencies for each cell are greater than or equal to 5.
  See step 3.
\end{enumerate}

\begin{enumerate}
\def\labelenumi{\arabic{enumi}.}
\setcounter{enumi}{2}
\tightlist
\item
  Find the test statistic and p-value
\end{enumerate}

\begin{quote}
Using the TI-83/84 calculator:

\textbf{Using the TI-83:}

To use the TI-83 calculator to compute the test statistic, you must first put the data into the calculator. Type the observed frequencies into L1 and the expected frequencies into L2. Then you will need to go to L3, arrow up onto the name, and type in . Now you use 1-Var Stats L3 to find the total. See figure \#11.2.1 for the initial setup, figure \#11.2.2 for the results of that calculation, and figure \#11.2.3 for the result of the 1-Var Stats L3.
\end{quote}

\begin{quote}
\textbf{Figure \#11.2.1: Input into TI-83}

\includegraphics[width=2.75in,height=1.86111in]{media/image124.png}

\textbf{Figure \#11.2.2: Result for L3 on TI-83}

\includegraphics[width=2.75in,height=1.86111in]{media/image125.png}

\textbf{Figure \#11.2.3: 1-Var Stats L3 Result on TI-83}

\includegraphics[width=2.75in,height=1.86111in]{media/image126.png}

The total is the chi-square value, .

The p-value is found using , where the degrees of freedom is .

\textbf{Using the TI-84:}

To run the test on the TI-84, type the observed frequencies into L1 and the expected frequencies into L2, then go into STAT, move over to TEST and choose GOF-Test from the list. The setup for the test is in figure \#11.2.4.
\end{quote}

\textbf{Figure \#11.2.4: Setup for Chi-Square Goodness of Fit Test on TI-84}

\begin{quote}
\includegraphics[width=2.75in,height=1.86111in]{media/image131.png}

Once you press ENTER on Calculate you will see the results in figure \#11.2.5.

\textbf{Figure \#11.2.5: Results for Chi-Square Test on TI-83/84}

\includegraphics[width=2.75in,height=1.86111in]{media/image132.png}
\end{quote}

The test statistic is 1.504060962

The

\begin{quote}
The CNTRB represent the for each die value. You can see the values by pressing the right arrow.

Using R:

Type in the observed frequencies. Call it something like observed.

observed\textless{}- c(type in data with commas in between)

Type in the probabilities that you are comparing to the observed frequencies. Call it something like null.probs.

null.probs \textless{}- c(type in probabilities with commas in between)

chisq.test(observed, p=null.probs) -- the command for the hypothesis test

For this example (Note since you are looking to see if the die is fair, then the probability of each side of a fair die coming up is 1/6.)

observed\textless{}-c(78, 87, 87, 76, 85, 87)

null.probs\textless{}-c(1/6, 1/6, 1/6, 1/6, 1/6, 1/6)

chisq.test(observed, p=null.probs)

Output:

Chi-squared test for given probabilities

data: observed

X-squared = 1.504, df = 5, p-value = 0.9126

The test statistic is and the p-value = 0.9126.
\end{quote}

\begin{enumerate}
\def\labelenumi{\arabic{enumi}.}
\setcounter{enumi}{3}
\tightlist
\item
  Conclusion
\end{enumerate}

\begin{quote}
Fail to reject since the p-value is greater than 0.05.
\end{quote}

\begin{enumerate}
\def\labelenumi{\arabic{enumi}.}
\setcounter{enumi}{4}
\tightlist
\item
  Interpretation
\end{enumerate}

\begin{quote}
There is not enough evidence to show that the die is not consistent with the distribution for a fair die. There is not enough evidence to show that the die is not fair.
\end{quote}

\hypertarget{homework-33}{%
\subsection{Homework}\label{homework-33}}

In each problem show all steps of the hypothesis test. If some of the assumptions are not met, note that the results of the test may not be correct and then continue the process of the hypothesis test.

\begin{enumerate}
\def\labelenumi{\arabic{enumi}.}
\tightlist
\item
  According to the M\&M candy company, the expected proportion can be found in Table \#11.2.3. In addition, the table contains the number of M\&M's of each color that were found in a case of candy (Madison, 2013). At the 5\% level, do the observed frequencies support the claim of M\&M?
\end{enumerate}

\textbf{Table \#11.2.3: M\&M Observed and Proportions}

\begin{longtable}[]{@{}llllllll@{}}
\toprule
& Blue & Brown & Green & Orange & Red & Yellow & Total\tabularnewline
\midrule
\endhead
Observed Frequencies & 481 & 371 & 483 & 544 & 372 & 369 & 2620\tabularnewline
Expected Proportion & 0.24 & 0.13 & 0.16 & 0.20 & 0.13 & 0.14 &\tabularnewline
\bottomrule
\end{longtable}

\begin{enumerate}
\def\labelenumi{\arabic{enumi}.}
\setcounter{enumi}{1}
\tightlist
\item
  Eyeglassomatic manufactures eyeglasses for different retailers. They test to see how many defective lenses they made the time period of January 1 to March 31. Table \#11.2.4 gives the defect and the number of defects.
\end{enumerate}

\textbf{Table \#11.2.4: Number of Defective Lenses}

\begin{longtable}[]{@{}ll@{}}
\toprule
Defect type & Number of defects\tabularnewline
\midrule
\endhead
Scratch & 5865\tabularnewline
Right shaped -- small & 4613\tabularnewline
Flaked & 1992\tabularnewline
Wrong axis & 1838\tabularnewline
Chamfer wrong & 1596\tabularnewline
Crazing, cracks & 1546\tabularnewline
Wrong shape & 1485\tabularnewline
Wrong PD & 1398\tabularnewline
Spots and bubbles & 1371\tabularnewline
Wrong height & 1130\tabularnewline
Right shape -- big & 1105\tabularnewline
Lost in lab & 976\tabularnewline
Spots/bubble -- intern & 976\tabularnewline
\bottomrule
\end{longtable}

Do the data support the notion that each defect type occurs in the same proportion? Test at the 10\% level.

\begin{enumerate}
\def\labelenumi{\arabic{enumi}.}
\setcounter{enumi}{2}
\item
  On occasion, medical studies need to model the proportion of the population that has a disease and compare that to observed frequencies of the disease actually occurring. Suppose the end-stage renal failure in south-west Wales was collected for different age groups. Do the data in table 11.2.5 show that the observed frequencies are in agreement with proportion of people in each age group (Boyle, Flowerdew \& Williams, 1997)? Test at the 1\% level.

  \textbf{Table \#11.2.5: Renal Failure Frequencies}
\end{enumerate}

\begin{longtable}[]{@{}lllllll@{}}
\toprule
Age Group & 16-29 & 30-44 & 45-59 & 60-75 & 75+ & Total\tabularnewline
\midrule
\endhead
Observed Frequency & 32 & 66 & 132 & 218 & 91 & 539\tabularnewline
Expected Proportion & 0.23 & 0.25 & 0.22 & 0.21 & 0.09 &\tabularnewline
\bottomrule
\end{longtable}

\begin{enumerate}
\def\labelenumi{\arabic{enumi}.}
\setcounter{enumi}{3}
\item
  In Africa in 2011, the number of deaths of a female from cardiovascular disease for different age groups are in table \#11.2.6 ("Global health observatory," 2013). In addition, the proportion of deaths of females from all causes for the same age groups are also in table \#11.2.6. Do the data show that the death from cardiovascular disease are in the same proportion as all deaths for the different age groups? Test at the 5\% level.

  \textbf{Table \#11.2.6: Deaths of Females for Different Age Groups}
\end{enumerate}

\begin{longtable}[]{@{}llllll@{}}
\toprule
Age & 5-14 & 15-29 & 30-49 & 50-69 & Total\tabularnewline
\midrule
\endhead
Cardiovascular Frequency & 8 & 16 & 56 & 433 & 513\tabularnewline
All Cause Proportion & 0.10 & 0.12 & 0.26 & 0.52 &\tabularnewline
\bottomrule
\end{longtable}

\begin{enumerate}
\def\labelenumi{\arabic{enumi}.}
\setcounter{enumi}{4}
\item
  In Australia in 1995, there was a question of whether indigenous people are more likely to die in prison than non-indigenous people. To figure out, the data in table 11.2.7 was collected. ("Aboriginal deaths in," 2013). Do the data show that indigenous people die in the same proportion as non-indigenous people? Test at the 1\% level.

  \textbf{Table \#11.2.7: Death of Prisoners}

  \begin{longtable}[]{@{}rlll@{}}
  \toprule
  Pris & oner Dies Pris & oner Did Not Die Tota & l\tabularnewline
  \midrule
  \endhead
  Indigenous Prisoner Frequency & 17 & 2890 & 2907\tabularnewline
  Frequency of Non-Indigenous Prisoner & 42 & 14459 & 14501\tabularnewline
  \bottomrule
  \end{longtable}
\item
  A project conducted by the Australian Federal Office of Road Safety asked people many questions about their cars. One question was the reason that a person chooses a given car, and that data is in table \#11.2.8 ("Car preferences," 2013).
\end{enumerate}

\begin{quote}
\textbf{Table \#11.2.8: Reason for Choosing a Car}
\end{quote}

\begin{longtable}[]{@{}llllll@{}}
\toprule
Safety & Reliability & Cost & Performance & Comfort & Looks\tabularnewline
\midrule
\endhead
84 & 62 & 46 & 34 & 47 & 27\tabularnewline
\bottomrule
\end{longtable}

\begin{quote}
Do the data show that the frequencies observed substantiate the claim that the reasons for choosing a car are equally likely? Test at the 5\% level.
\end{quote}

\hypertarget{analysis-of-variance-anova}{%
\section{Analysis of Variance (ANOVA)}\label{analysis-of-variance-anova}}

There are times where you want to compare three or more population means. One idea is to just test different combinations of two means. The problem with that is that your chance for a type I error increases. Instead you need a process for analyzing all of them at the same time. This process is known as \textbf{analysis of variance (ANOVA)}. The test statistic for the ANOVA is fairly complicated, you will want to use technology to find the test statistic and p-value. The test statistic is distributed as an F-distribution, which is skewed right and depends on degrees of freedom. Since you will use technology to find these, the distribution and the test statistic will not be presented. Remember, all hypothesis tests are the same process. Note that to obtain a statistically significant result there need only be a difference between any two of the \emph{k} means.

Before conducting the hypothesis test, it is helpful to look at the means and standard deviations for each data set. If the sample means with consideration of the sample standard deviations are different, it may mean that some of the population means are different. However, do realize that if they are different, it doesn't provide enough evidence to show the population means are different. Calculating the sample statistics just gives you an idea that conducting the hypothesis test is a good idea.

\textbf{Hypothesis test using ANOVA to compare \emph{k} means}

\begin{enumerate}
\def\labelenumi{\arabic{enumi}.}
\item
  State the random variables and the parameters in words
\item
  State the null and alternative hypotheses and the level of significance
\end{enumerate}

\begin{quote}
Also, state your level here.
\end{quote}

\begin{enumerate}
\def\labelenumi{\arabic{enumi}.}
\setcounter{enumi}{2}
\tightlist
\item
  State and check the assumptions for the hypothesis test
\end{enumerate}

\begin{enumerate}
\def\labelenumi{\alph{enumi}.}
\item
  A random sample of size is taken from each population.
\item
  All the samples are independent of each other.
\item
  Each population is normally distributed. The ANOVA test is fairly robust to the assumption especially if the sample sizes are fairly close to each other. Unless the populations are really not normally distributed and the sample sizes are close to each other, then this is a loose assumption.
\item
  The population variances are all equal. If the sample sizes are close to each other, then this is a loose assumption.
\end{enumerate}

\begin{enumerate}
\def\labelenumi{\arabic{enumi}.}
\setcounter{enumi}{3}
\tightlist
\item
  Find the test statistic and p-value
\end{enumerate}

\begin{quote}
The test statistic is , where is the mean square between the groups (or factors), and is the mean square within the groups. The degrees of freedom between the groups is and the degrees of freedom within the groups is . To find all of the values, use technology such as the TI-83/84 calculator or R.

The test statistic, \emph{F}, is distributed as an F-distribution, where both degrees of freedom are needed in this distribution. The p-value is also calculated by the calculator or R.
\end{quote}

\begin{enumerate}
\def\labelenumi{\arabic{enumi}.}
\setcounter{enumi}{4}
\tightlist
\item
  Conclusion
\end{enumerate}

\begin{quote}
This is where you write reject or fail to reject . The rule is: if the p-value \textless{} , then reject . If the p-value , then fail to reject
\end{quote}

\begin{enumerate}
\def\labelenumi{\arabic{enumi}.}
\setcounter{enumi}{5}
\tightlist
\item
  Interpretation
\end{enumerate}

\begin{quote}
This is where you interpret in real world terms the conclusion to the test. The conclusion for a hypothesis test is that you either have enough evidence to show is true, or you do not have enough evidence to show is true.
\end{quote}

If you do in fact reject , then you know that at least two of the means are different. The next question you might ask is which are different? You can look at the sample means, but realize that these only give a preliminary result. To actually determine which means are different, you need to conduct other tests. Some of these tests are the range test, multiple comparison tests, Duncan test, Student-Newman-Keuls test, Tukey test, Scheffé test, Dunnett test, least significant different test, and the Bonferroni test. There is no consensus on which test to use. These tests are available in statistical computer packages such as Minitab and SPSS.

\textbf{Example \#11.3.1: Hypothesis Test Involving Several Means}

\begin{quote}
Cancer is a terrible disease. Surviving may depend on the type of cancer the person has. To see if the mean survival time for several types of cancer are different, data was collected on the survival time in days of patients with one of these cancer in advanced stage. The data is in table \#11.3.1 ("Cancer survival story," 2013). (Please realize that this data is from 1978. There have been many advances in cancer treatment, so do not use this data as an indication of survival rates from these cancers.) Do the data indicate that at least two of the mean survival time for these types of cancer are not all equal? Test at the 1\% level.

\textbf{Table \#11.3.1: Survival Times in Days of Five Cancer Types}
\end{quote}

\begin{longtable}[]{@{}lllll@{}}
\toprule
Stomach & Bronchus & Colon & Ovary & Breast\tabularnewline
\midrule
\endhead
124 & 81 & 248 & 1234 & 1235\tabularnewline
42 & 461 & 377 & 89 & 24\tabularnewline
25 & 20 & 189 & 201 & 1581\tabularnewline
45 & 450 & 1843 & 356 & 1166\tabularnewline
412 & 246 & 180 & 2970 & 40\tabularnewline
51 & 166 & 537 & 456 & 727\tabularnewline
1112 & 63 & 519 & & 3808\tabularnewline
46 & 64 & 455 & & 791\tabularnewline
103 & 155 & 406 & & 1804\tabularnewline
876 & 859 & 365 & & 3460\tabularnewline
146 & 151 & 942 & & 719\tabularnewline
340 & 166 & 776 & &\tabularnewline
396 & 37 & 372 & &\tabularnewline
& 223 & 163 & &\tabularnewline
& 138 & 101 & &\tabularnewline
& 72 & 20 & &\tabularnewline
& 245 & 283 & &\tabularnewline
\bottomrule
\end{longtable}

\begin{quote}
\textbf{Solution:}
\end{quote}

\begin{enumerate}
\def\labelenumi{\arabic{enumi}.}
\tightlist
\item
  State the random variables and the parameters in words
\end{enumerate}

\begin{quote}
Now before conducting the hypothesis test, look at the means and standard deviations.
\end{quote}

\begin{quote}
There appears to be a difference between at least two of the means, but realize that the standard deviations are very different. The difference you see may not be significant.

Notice the sample sizes are not the same. The sample sizes are
\end{quote}

\begin{enumerate}
\def\labelenumi{\arabic{enumi}.}
\setcounter{enumi}{1}
\item
  State the null and alternative hypotheses and the level of significance
\item
  State and check the assumptions for the hypothesis test
\end{enumerate}

\begin{enumerate}
\def\labelenumi{\alph{enumi}.}
\item
  A random sample of 13 survival times from stomach cancer was taken. A random sample of 17 survival times from bronchus cancer was taken. A random sample of 17 survival times from colon cancer was taken. A random sample of 6 survival times from ovarian cancer was taken. A random sample of 11 survival times from breast cancer was taken. These statements may not be true. This information was not shared as to whether the samples were random or not but it may be safe to assume that.
\item
  Since the individuals have different cancers, then the samples are independent.
\item
  Population of all survival times from stomach cancer is normally distributed. Population of all survival times from bronchus cancer is normally distributed. Population of all survival times from colon cancer is normally distributed. Population of all survival times from ovarian cancer is normally distributed. Population of all survival times from breast cancer is normally distributed. Looking at the histograms, box plots and normal quantile plots for each sample, it appears that none of the populations are normally distributed. The sample sizes are somewhat different for the problem. This assumption may not be true.
\item
  The population variances are all equal. The sample standard deviations are approximately 346.3, 209.9, 427.2, 1098.6, and 1239.0 respectively. This assumption does not appear to be met, since the sample standard deviations are very different. The sample sizes are somewhat different for the problem. This assumption may not be true.
\end{enumerate}

\begin{enumerate}
\def\labelenumi{\arabic{enumi}.}
\setcounter{enumi}{3}
\tightlist
\item
  Find the test statistic and p-value
\end{enumerate}

\begin{quote}
To find the test statistic and p-value using the TI-83/84, type each data set into L1 through L5. Then go into STAT and over to TESTS and choose ANOVA(. Then type in L1,L2,L3,L4,L5 and press enter. You will get the results of the ANOVA test.
\end{quote}

\textbf{\\
}

\begin{quote}
\textbf{Figure \#11.3.1: Setup for ANOVA on TI-83/84}

\includegraphics[width=2.75in,height=1.86111in]{media/image172.png}

\includegraphics[width=2.75in,height=1.86111in]{media/image173.png}

\textbf{Figure \#11.3.2: Results of ANOVA on TI-83/84}

\includegraphics[width=2.75in,height=1.86111in]{media/image174.png}

\includegraphics[width=2.75in,height=1.86111in]{media/image175.png}

The test statistic is and .

Just so you know, the Factor information is between the groups and the Error is within the groups. So and .

To find the test statistic and p-value on R:

The commands would be:

variable=c(type in all data values with commas in between) -- this is the response variable

factor=c(rep("factor 1", number of data values for factor 1), rep("factor 2", number of data values for factor 2), etc) -- this separates the data into the different factors that the measurements were based on.

data\_name = data.frame(variable, factor) -- this puts the data into one variable. data\_name is the name you give this variable

aov(variable \textasciitilde{} factor, data = data name) -- runs the ANOVA analysis

For this example, the commands would be:

time=c(124, 42, 25, 45, 412, 51, 1112, 46, 103, 876, 146, 340, 396,
81, 461, 20, 450, 246, 166, 63, 64, 155, 859, 151, 166, 37, 223, 138,
72, 245, 248, 377, 189, 1843, 180, 537, 519, 455, 406, 365, 942, 776,
372, 163, 101, 20, 283, 1234, 89, 201, 356, 2970, 456, 1235, 24, 1581,
1166, 40, 727, 3808, 791, 1804, 3460, 719)

factor=c(rep("Stomach", 13), rep("Bronchus", 17), rep("Colon",
17), rep("Ovary", 6), rep("Breast", 11))

survival=data.frame(time, factor)

results=aov(time\textasciitilde{}factor, data=survival)

summary(results)

Df Sum Sq Mean Sq F value Pr(\textgreater{}F)

factor 4 11535761 2883940 6.433 0.000229 ***

Residuals 59 26448144 448274

-\/-\/-

Signif. codes: 0 `***' 0.001 `**' 0.01 `*' 0.05 `.' 0.1 ' ' 1

The test statistic is F = 6.433 and the p-value = 0.000229.
\end{quote}

\begin{enumerate}
\def\labelenumi{\arabic{enumi}.}
\setcounter{enumi}{4}
\tightlist
\item
  Conclusion
\end{enumerate}

\begin{quote}
Reject since the p-value is less than 0.01.
\end{quote}

\begin{enumerate}
\def\labelenumi{\arabic{enumi}.}
\setcounter{enumi}{5}
\tightlist
\item
  Interpretation
\end{enumerate}

\begin{quote}
There is evidence to show that at least two of the mean survival times from different cancers are not equal.

By examination of the means, it appears that the mean survival time for breast cancer is different from the mean survival times for both stomach and bronchus cancers. It may also be different for the mean survival time for colon cancer. The others may not be different enough to actually say for sure.
\end{quote}

\textbf{\\
}

\hypertarget{homework-34}{%
\subsection{Homework}\label{homework-34}}

In each problem show all steps of the hypothesis test. If some of the assumptions are not met, note that the results of the test may not be correct and then continue the process of the hypothesis test.

\begin{enumerate}
\def\labelenumi{\arabic{enumi}.}
\tightlist
\item
  Cuckoo birds are in the habit of laying their eggs in other birds' nest. The other birds adopt and hatch the eggs. The lengths (in cm) of cuckoo birds' eggs in the other species nests were measured and are in table \#11.3.2 ("Cuckoo eggs in," 2013). Do the data show that the mean length of cuckoo bird's eggs is not all the same when put into different nests? Test at the 5\% level.
\end{enumerate}

\begin{quote}
\textbf{Table \#11.3.2: Lengths of Cuckoo Bird Eggs in Different Species
Nests}
\end{quote}

\begin{longtable}[]{@{}lllllll@{}}
\toprule
Meadow Pipit & Tree Pipit & Hedge Sparrow & Robin & Pied Wagtail & Wren &\tabularnewline
\midrule
\endhead
19.65 & 22.25 & 21.05 & 20.85 & 21.05 & 21.05 & 19.85\tabularnewline
20.05 & 22.45 & 21.85 & 21.65 & 21.85 & 21.85 & 20.05\tabularnewline
20.65 & 22.45 & 22.05 & 22.05 & 22.05 & 21.85 & 20.25\tabularnewline
20.85 & 22.45 & 22.45 & 22.85 & 22.05 & 21.85 & 20.85\tabularnewline
21.65 & 22.65 & 22.65 & 23.05 & 22.05 & 22.05 & 20.85\tabularnewline
21.65 & 22.65 & 23.25 & 23.05 & 22.25 & 22.45 & 20.85\tabularnewline
21.65 & 22.85 & 23.25 & 23.05 & 22.45 & 22.65 & 21.05\tabularnewline
21.85 & 22.85 & 23.25 & 23.05 & 22.45 & 23.05 & 21.05\tabularnewline
21.85 & 22.85 & 23.45 & 23.45 & 22.65 & 23.05 & 21.05\tabularnewline
21.85 & 22.85 & 23.45 & 23.85 & 23.05 & 23.25 & 21.25\tabularnewline
22.05 & 23.05 & 23.65 & 23.85 & 23.05 & 23.45 & 21.45\tabularnewline
22.05 & 23.25 & 23.85 & 23.85 & 23.05 & 24.05 & 22.05\tabularnewline
22.05 & 23.25 & 24.05 & 24.05 & 23.05 & 24.05 & 22.05\tabularnewline
22.05 & 23.45 & 24.05 & 25.05 & 23.05 & 24.05 & 22.05\tabularnewline
22.05 & 23.65 & 24.05 & & 23.25 & 24.85 & 22.25\tabularnewline
22.05 & 23.85 & & & 23.85 & &\tabularnewline
22.05 & 24.25 & & & & &\tabularnewline
22.05 & 24.45 & & & & &\tabularnewline
22.05 & 22.25 & & & & &\tabularnewline
22.05 & 22.25 & & & & &\tabularnewline
22.25 & 22.25 & & & & &\tabularnewline
22.25 & 22.25 & & & & &\tabularnewline
22.25 & & & & & &\tabularnewline
\bottomrule
\end{longtable}

\begin{enumerate}
\def\labelenumi{\arabic{enumi}.}
\setcounter{enumi}{1}
\tightlist
\item
  Levi-Strauss Co manufactures clothing. The quality control department measures weekly values of different suppliers for the percentage difference of waste between the layout on the computer and the actual waste when the clothing is made (called run-up). The data is in table \#11.3.3, and there are some negative values because sometimes the supplier is able to layout the pattern better than the computer ("Waste run up," 2013). Do the data show that there is a difference between some of the suppliers? Test at the 1\% level.
\end{enumerate}

\begin{quote}
\textbf{Table \#11.3.3: Run-ups for Different Plants Making Levi Strauss
Clothing}
\end{quote}

\begin{longtable}[]{@{}lllll@{}}
\toprule
Plant 1 & Plant 2 & Plant 3 & Plant 4 & Plant 5\tabularnewline
\midrule
\endhead
1.2 & 16.4 & 12.1 & 11.5 & 24\tabularnewline
10.1 & -6 & 9.7 & 10.2 & -3.7\tabularnewline
-2 & -11.6 & 7.4 & 3.8 & 8.2\tabularnewline
1.5 & -1.3 & -2.1 & 8.3 & 9.2\tabularnewline
-3 & 4 & 10.1 & 6.6 & -9.3\tabularnewline
-0.7 & 17 & 4.7 & 10.2 & 8\tabularnewline
3.2 & 3.8 & 4.6 & 8.8 & 15.8\tabularnewline
2.7 & 4.3 & 3.9 & 2.7 & 22.3\tabularnewline
-3.2 & 10.4 & 3.6 & 5.1 & 3.1\tabularnewline
-1.7 & 4.2 & 9.6 & 11.2 & 16.8\tabularnewline
2.4 & 8.5 & 9.8 & 5.9 & 11.3\tabularnewline
0.3 & 6.3 & 6.5 & 13 & 12.3\tabularnewline
3.5 & 9 & 5.7 & 6.8 & 16.9\tabularnewline
-0.8 & 7.1 & 5.1 & 14.5 &\tabularnewline
19.4 & 4.3 & 3.4 & 5.2 &\tabularnewline
2.8 & 19.7 & -0.8 & 7.3 &\tabularnewline
13 & 3 & -3.9 & 7.1 &\tabularnewline
42.7 & 7.6 & 0.9 & 3.4 &\tabularnewline
1.4 & 70.2 & 1.5 & 0.7 &\tabularnewline
3 & 8.5 & & &\tabularnewline
2.4 & 6 & & &\tabularnewline
1.3 & 2.9 & & &\tabularnewline
\bottomrule
\end{longtable}

\begin{enumerate}
\def\labelenumi{\arabic{enumi}.}
\setcounter{enumi}{2}
\tightlist
\item
  Several magazines were grouped into three categories based on what level of education of their readers the magazines are geared towards: high, medium, or low level. Then random samples of the magazines were selected to determine the number of three-plus-syllable words were in the advertising copy, and the data is in table \#11.3.4 ("Magazine ads readability," 2013). Is there enough evidence to show that the mean number of three-plus-syllable words in advertising copy is different for at least two of the education levels? Test at the 5\% level.
\end{enumerate}

\begin{quote}
\textbf{Table \#11.3.4: Number of Three Plus Syllable Words in Advertising
Copy}
\end{quote}

\begin{longtable}[]{@{}lll@{}}
\toprule
High Education & Medium Education & Low Education\tabularnewline
\midrule
\endhead
34 & 13 & 7\tabularnewline
21 & 22 & 7\tabularnewline
37 & 25 & 7\tabularnewline
31 & 3 & 7\tabularnewline
10 & 5 & 7\tabularnewline
24 & 2 & 7\tabularnewline
39 & 9 & 8\tabularnewline
10 & 3 & 8\tabularnewline
17 & 0 & 8\tabularnewline
18 & 4 & 8\tabularnewline
32 & 29 & 8\tabularnewline
17 & 26 & 8\tabularnewline
3 & 5 & 9\tabularnewline
10 & 5 & 9\tabularnewline
6 & 24 & 9\tabularnewline
5 & 15 & 9\tabularnewline
6 & 3 & 9\tabularnewline
6 & 8 & 9\tabularnewline
\bottomrule
\end{longtable}

\begin{enumerate}
\def\labelenumi{\arabic{enumi}.}
\setcounter{enumi}{3}
\tightlist
\item
  A study was undertaken to see how accurate food labeling for calories on food that is considered reduced calorie. The group measured the amount of calories for each item of food and then found the percent difference between measured and labeled food, . The group also looked at food that was nationally advertised, regionally distributed, or locally prepared. The data is in table \#11.3.5 ("Calories datafile," 2013). Do the data indicate that at least two of the mean percent differences between the three groups are different? Test at the 10\% level.
\end{enumerate}

\begin{quote}
\textbf{Table \#11.3.5: Percent Differences Between Measured and Labeled
Food}
\end{quote}

\begin{longtable}[]{@{}lll@{}}
\toprule
National Advertised & Regionally Distributed & Locally Prepared\tabularnewline
\midrule
\endhead
2 & 41 & 15\tabularnewline
-28 & 46 & 60\tabularnewline
-6 & 2 & 250\tabularnewline
8 & 25 & 145\tabularnewline
6 & 39 & 6\tabularnewline
-1 & 16.5 & 80\tabularnewline
10 & 17 & 95\tabularnewline
13 & 28 & 3\tabularnewline
15 & -3 &\tabularnewline
-4 & 14 &\tabularnewline
-4 & 34 &\tabularnewline
-18 & 42 &\tabularnewline
10 & &\tabularnewline
5 & &\tabularnewline
3 & &\tabularnewline
-7 & &\tabularnewline
3 & &\tabularnewline
-0.5 & &\tabularnewline
-10 & &\tabularnewline
6 & &\tabularnewline
\bottomrule
\end{longtable}

\begin{enumerate}
\def\labelenumi{\arabic{enumi}.}
\setcounter{enumi}{4}
\tightlist
\item
  The amount of sodium (in mg) in different types of hotdogs is in table \#11.3.6 ("Hot dogs story," 2013). Is there sufficient evidence to show that the mean amount of sodium in the types of hotdogs are not all equal? Test at the 5\% level.
\end{enumerate}

\begin{quote}
\textbf{Table \#11.3.6: Amount of Sodium (in mg) in Beef, Meat, and Poultry \textgreater{} Hotdogs}
\end{quote}

\begin{longtable}[]{@{}lll@{}}
\toprule
Beef & Meat & Poultry\tabularnewline
\midrule
\endhead
495 & 458 & 430\tabularnewline
477 & 506 & 375\tabularnewline
425 & 473 & 396\tabularnewline
322 & 545 & 383\tabularnewline
482 & 496 & 387\tabularnewline
587 & 360 & 542\tabularnewline
370 & 387 & 359\tabularnewline
322 & 386 & 357\tabularnewline
479 & 507 & 528\tabularnewline
375 & 393 & 513\tabularnewline
330 & 405 & 426\tabularnewline
300 & 372 & 513\tabularnewline
386 & 144 & 358\tabularnewline
401 & 511 & 581\tabularnewline
645 & 405 & 588\tabularnewline
440 & 428 & 522\tabularnewline
317 & 339 & 545\tabularnewline
319 & &\tabularnewline
298 & &\tabularnewline
253 & &\tabularnewline
\bottomrule
\end{longtable}

Data Source:

\emph{Aboriginal deaths in custody}. (2013, September 26). Retrieved from
\url{http://www.statsci.org/data/oz/custody.html}

\emph{Activities of dolphin groups}. (2013, September 26). Retrieved from
\url{http://www.statsci.org/data/general/dolpacti.html}

Boyle, P., Flowerdew, R., \& Williams, A. (1997). Evaluating the goodness
of fit in models of sparse medical data: A simulation approach.
\emph{International Journal of Epidemiology}, \emph{26}(3), 651-656. Retrieved
from \url{http://ije.oxfordjournals.org/content/26/3/651.full.pdf} html

\emph{Calories datafile}. (2013, December 07). Retrieved from
\url{http://lib.stat.cmu.edu/DASL/Datafiles/Calories.html}

\emph{Cancer survival story}. (2013, December 04). Retrieved from
\url{http://lib.stat.cmu.edu/DASL/Stories/CancerSurvival.html}

\emph{Car preferences}. (2013, September 26). Retrieved from
\url{http://www.statsci.org/data/oz/carprefs.html}

\emph{Cuckoo eggs in nest of other birds}. (2013, December 04). Retrieved
from \url{http://lib.stat.cmu.edu/DASL/Stories/cuckoo.html}

\emph{Education by age datafile}. (2013, December 05). Retrieved from
\url{http://lib.stat.cmu.edu/DASL/Datafiles/Educationbyage.html}

\emph{Encyclopedia Titanica}. (2013, November 09). Retrieved from
\url{http://www.encyclopedia-titanica.org/}

\emph{Global health observatory data respository}. (2013, October 09).
Retrieved from
\url{http://apps.who.int/gho/athena/data/download.xsl?format=xml\&target=GHO/MORT/_400\&profile=excel\&filter=AGEGROUP:YEARS05-14;AGEGROUP:YEARS15-29;AGEGROUP:YEARS30-49;AGEGROUP:YEARS50-69;AGEGROUP:YEARS70}
;MGHEREG:REG6\_AFR;GHECAUSES:*;SEX:*

\emph{Hot dogs story}. (2013, November 16). Retrieved from
\url{http://lib.stat.cmu.edu/DASL/Stories/Hotdogs.html}

\emph{Leprosy: Number of reported cases by country}. (2013, September 04).
Retrieved from \url{http://apps.who.int/gho/data/node.main.A1639}

Madison, J. (2013, October 15). \emph{M\&M's color distribution analysis}.
Retrieved from
\url{http://joshmadison.com/2007/12/02/mms-color-distribution-analysis/}

\emph{Magazine ads readability}. (2013, December 04). Retrieved from
\url{http://lib.stat.cmu.edu/DASL/Datafiles/magadsdat.html}

\emph{Popular kids datafile}. (2013, December 05). Retrieved from
\url{http://lib.stat.cmu.edu/DASL/Datafiles/PopularKids.html}

Schultz, S. T., Klonoff-Cohen, H. S., Wingard, D. L., Askhoomoff, N. A.,
Macera, C. A., Ji, M., \& Bacher, C. (2006). Breastfeeding, infant
formula supplementation, and autistic disorder: the results of a parent
survey. \emph{International Breastfeeding Journal}, \emph{1}(16), doi:
10.1186/1746-4358-1-16

\emph{Waste run up}. (2013, December 04). Retrieved from
\url{http://lib.stat.cmu.edu/DASL/Stories/wasterunup.html}

\bibliography{book.bib}


\end{document}
